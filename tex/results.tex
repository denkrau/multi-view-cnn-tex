\chapter{Results}
\label{sec:results}
In this chapter, the performance of the network architecture of this work, its components and results of classifications are going to be presented.
Mainly it is focused on the results of the grouping mechanism for analyzing the information content of views of the same objects but with different color marks.
Furthermore, the impact of different color marks is examined.
First, the overall performance of the network is discussed.
Then, the grouping mechanism is evaluated with respect to the information content of views, followed by an examination of misclassifications.
There are several networks trained with an increasing number of classes of combinations of category and color classes for being able to compare and dedicate possible occurring effects.
It starts with multi-views of only a single category, in particular, bathtubs, with first 3 different color marks.
Those related color classes include the raw object, a green color mark and a red one.
This is further increased to 6 color classes containing a green and red color mark, two green color marks, and two red color marks.
Finally, four category classes in total are classified including additional dressers, monitors and sofas.
Here an identical process of adding color marks is performed in the same order as before.
This leads to a total of 9 networks, where the basic architecture stays the same and only the number of outputs or classes, respectively, changes.
However, for testing different hyperparameters, more networks have been trained on the dataset with bathtubs and 3 color classes.
In the following the syntax \emph{\#categories-\#colors} refers to the corresponding trained network where each number corresponds to the number of  classifications of the class type.
That means, the 4-3 network, for example, classifies a combination of four category classes and three color classes.
Furthermore, the 0-3 network classifies only color classes of bathtubs, while the 4-0 network classifies only category classes with blank objects.
The classes that each network predict are summarized in \tabref{tab:network-classes}.
Every model is trained for 20 epochs on the training set with a batch size of 8.
Each batch element contains 12 rendered views of an object.
The initial learning rate is set to $0.0001$.
Furthermore, the dropout probability of layer 6 and 7 is specified as $0.5$ during training, which matches the AlexNet configuration.
Otherwise this layer is ignored, hence, is assigned a dropout probability of $0.0$.
Any dataset samples presented or predicted belong to the test set, thus, on data the network is not trained on.
This means, the training and test set contain the same classes, either category or color, but different data samples.
Each network is only trained once due to a lack of time, hence, all presented results refer to a particular training process and are not averages of several ones.
\begin{table}[]
\centering
\caption[Classification classes of each network]{Classification classes of each network. The notation follows \emph{\#categories-\#colors}.}
\label{tab:network-classes}
\begin{tabular}{l|ccccccccc}
               & \multicolumn{1}{l}{0-3} & \multicolumn{1}{l}{0-4} & \multicolumn{1}{l}{0-5} & \multicolumn{1}{l}{0-6} & \multicolumn{1}{l}{4-0} & \multicolumn{1}{l}{4-3} & \multicolumn{1}{l}{4-4} & \multicolumn{1}{l}{4-5} & \multicolumn{1}{l}{4-6} \\ \hline
Blank          & x                       & x                       & x                       & x                       &                         & x                       & x                       & x                       & x                       \\
Green          & x                       & x                       & x                       & x                       &                         & x                       & x                       & x                       & x                       \\
Red            & x                       & x                       & x                       & x                       &                         & x                       & x                       & x                       & x                       \\
Green-Red      &                         & x                       & x                       & x                       &                         &                         & x                       & x                       & x                       \\
Green-Green    &                         &                         & x                       & x                       &                         &                         &                         & x                       & x                       \\
Red-Red        &                         &                         &                         & x                       &                         &                         &                         &                         & x                       \\ \hline
Category classes & 0                       & 0                       & 0                       & 0                       & 4                       & 4                       & 4                       & 4                       & 4                      
\end{tabular}
\end{table}

\section{Overall Performance}
\label{sec:results-overall}
\section{View to Group Classification}
\label{sec:results-grouping}
Because the grouping mechanism supplies the core functionality of the network architecture it is evaluated first.
Even if the overall performance would yield satisfiable results, but the grouping mechanism would fail the original intention, it would need to be revised.
In contrast, if the results of the network are not satisfiable, the grouping algorithm could be the cause.

The easiest case for evaluation is a single object category with three material features.
Here the network only needs to find views with a material for assigning them a high discrimination score.
It is supposed, that those views have the highest scores of all views and, hence, are members of a group with a high weight.
Views where no material is seen, should have a score close to zero, because the final prediction cannot rely on them at all.
Thus, this configuration is the most interpretable one.
The group dividing for the 0-3 network is shown in \figref{fig:grouping-0-3}.
Each number below a view refers to its score.
The text above views shows the group index with its corresponding weight.
All views appear in a ascending order by their score.
Hence, all subsequent views are part of a group until another group is mentioned.
\begin{figure}
	\centering
	\begin{subfigure}{\textwidth}
		\includegraphics[trim=10 20 10 20, clip]{images/mn-sl-0-3-20/bathtub_0107_0_grouping.png}
		\caption{Blank}
		\label{fig:grouping-0-3-blank}
	\end{subfigure}
	\begin{subfigure}{\textwidth}
		\includegraphics[trim=10 20 10 20, clip]{images/mn-sl-0-3-20/bathtub_0107_1_grouping.png}
		\caption{Green Material}
		\label{fig:grouping-0-3-green}
	\end{subfigure}
	\begin{subfigure}{\textwidth}
		\includegraphics[trim=10 20 10 20, clip]{images/mn-sl-0-3-20/bathtub_0107_2_grouping.png}
		\caption{Red Material}
		\label{fig:grouping-0-3-red}
	\end{subfigure}
	\caption[Grouping in 0-3 Network]{Grouping in 0-3 Network}
	\label{fig:grouping-0-3}
\end{figure}
In \figref{fig:grouping-0-3-blank} a blank object is classified.
Hence, every views is similar discriminative, due to no available colored material.
Thus, the view scores are almost identical.
Those little changes presumable depend on a different weight initialization and would even out after more training epochs.
Although the scores are very low, the views are fully taken into account because they all belong to the same group with a weight of 1.
However, in this particular case a normalization of the group weight is not necessary.
Without one the group weight would be $w = 0.0079$, thus, decreasing the shape descriptor enormously, but the network would learn that a descriptor close to 0 represents a blank object.
The decision rule would be, if the descriptor represents no feature, the objects shows no feature.
With the classification of more categories, tough, this is not possible anymore, because a very small descriptor cannot just represent any blank object, but the object category class.
If there are two objects, for example, and only one view each shows a different feature with a small discrimination score, they would be divided into two categories.
Without a normalization all views would pretty much account to the same amount to the shape descriptor.
With normalization, however, the group with one view is much higher weighted than the not discriminative views.
Hence, normalization kind of removes noise, i.e. not discriminative views, that could influence the prediction unfavorably.
\figref{fig:grouping-0-3-green} and \figref{fig:grouping-0-3-red} show the expected result.
The views showing the material feature are by far the top rated views referring to their score.
It looks like, that the network prefers the slightly tilted vertical edge with a feature to its right for recognizing material features.
This exact edge is not visible in the first view showing a feature, due to the change in perspective.
Due to the mesh representation of objects, all material features are triangles.
Perhaps the dataset contains more features following this shape than in rotated ones, hence, the network focuses on that correlation.
Moreover, both figures show exactly the same order.
This shows, that the weights for each color channel are optimized in the same direction.
However, the views with the green material are in a closer range compared to the ones with the red material.
The latter differ extremely.
The least discriminative view with a material is closer to the not discriminative views than the discriminative ones.

%It is supposed, that similar views like opposite perspectives of a symmetrical object are divided into the same group, because they contain similar features.
%For example, the views of the left and right side of a car look almost identical.
%Both have the same contours but in a mirrored direction.
\section{Misclassified Predictions}
\label{sec:results-predictions}