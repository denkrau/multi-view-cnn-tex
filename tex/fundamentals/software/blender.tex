\subsection{Blender}
\label{sec:software-blender}
Blender \cite{blender} is a free and open source 3D creation suite to model, texture and animate objects.
Furthermore, it supports importing existing models and manipulating them.
Additionally, an API interface is provided, that can be used with the programming language Python, to control every function of Blender.
This eases repetitive tasks tremendously.

%Every object in Blender has its own coordinate system.
%Hence, a way of expressing points of one coordinate system in another would be useful.
%It is necessary to define a world coordinate system $\mathcal{W}$ with the origin $\vec{o}_\mathcal{W} = (0,0,0)^T$ and the rotation $\vec{r}_{\mathcal{W}} = (0,0,0)^T$, where each element of the latter represents a rotation around the $x$-, $y$- or $z$-axis, respectively, in radians.
%This system contains every other system.
%Now another coordinate system $\mathcal{L}$ is created, of course, inside the world coordinate system.
%However, $\mathcal{L}$ can be translated and rotated in comparison to $\mathcal{W}$.
%Hence, every coordinate system has a rotation matrix $\vec{R}$ in Euler representation and a translation vector $\vec{t}$ that stores how they are rotated and translated to every other coordinate system.
%Considering $\mathcal{L}$ and $\mathcal{W}$ yields
%\begin{align}
%	\vec{R}_{\mathcal{L} \rightarrow \mathcal{W}} &= \vec{R}_{\mathcal{W} \rightarrow \mathcal{L}}^{-1} \\
%	\vec{t}_{\mathcal{L} \rightarrow \mathcal{W}} &= - \vec{t}_{\mathcal{W} \rightarrow \mathcal{L}}
%\end{align}
%as properties, where the subscript indicates the transfer.
%Transferring the coordinates of an arbitrary local point $\vec{x}_{\mathcal{L}}$ into corresponding coordinates $\vec{x}_{\mathcal{W}}$ of the reference coordinate system is done by using
%\begin{equation}
%	\vec{x}_{\mathcal{W}} = \vec{R}_{\mathcal{L} \rightarrow \mathcal{W}} \cdot \vec{x}_{\mathcal{L}} + \vec{t}_{\mathcal{L} \rightarrow \mathcal{W}}
%\end{equation}
%as the general expression.