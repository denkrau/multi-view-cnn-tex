\subsubsection{Activation Functions}
\label{sec:improving-performance-activation-functions}
The activation functions of a neural network affect its performance and convergence as well.
Common activation functions are shown in \figref{fig:activation-functions}.
The most common ones are going to be examined.
The sigmoid function recently has fallen out of favor.
One disadvantage is its saturation which leads to a very small gradient.
If this small gradient is often multiplied because of several layers, the gradient gets very small.
This vanishing gradient leads to very small parameter updates.
Furthermore, a well-suited weight initialization is required.
If they are too large, related neurons become saturated and the network will barely learn.
Hence, in both cases, convergence takes a very long time.
Another disadvantage is, that outputs are not zero-centered.
That means, that, for example, if all data is positive, all related gradients point into the same, either positive or negative, direction.
This leads to an undesired zigzag pattern of the parameter updates for several samples.
However, the use of mini-batches smooths out this effect.
The tanh activation function has a zero-centered output, though, the saturation of the output remains.
The ReLU activation function accelerates the convergence process of stochastic gradient descent compared to sigmoid and tanh.
According to \textit{Krizhevsky et al.}, it is six times faster on the CIFAR10 dataset \cite{Krizhevsky2009LearningML} compared to tanh activation functions \cite{Krizhevsky:2012:ICD:2999134.2999257}.
This is due to its linear, non-saturation form.
Another advantage is its simple and light computation.
Furthermore, it makes the activations sparse from the perspective of a neural network.
This means not all neurons are active due to the ReLU being zero for input values below zero.
Hence, the overall network is lighter.
However, its property of the evaluation to zero is also a disadvantage.
This results in a gradient of zero as well and therefore in no related parameter updates.
If the weights are initialized badly or an unfavorable update is applied, for example, due to a too large learning rate, a ReLU unit can die, because its evaluation and the gradient are zero for all further computations.
In this case, a unit is very unlikely to recover.
A solution forms the leaky ReLU, which adds a slight slope of like $\lambda = 0.01$ to the horizontal line of 0, preventing the gradient to become zero.
Hence, the unit can recover.
However, the results of this approach are very inconsistent.

Taking all that information into account yields the recommendation of the ReLU activation function, if the weights and learning rate are chosen carefully
Additionally, the fraction of dead units should be monitored.
If the number is still concerning, leaky ReLU activations should be applied.
The tanh activation should be generally preferred over the sigmoid one.

\begin{figure}
	\setlength\figureheight{.3\textwidth}
	\setlength\figurewidth{.5\textwidth}
	\centering
	\begin{subfigure}{.5\textwidth}
		\centering
		\input{images/threshold-activation.tikz}
		\begin{equation*}
		\phi(x) =
		\begin{cases}
		1 & x \geq \text{Bias $b$} \\
		0 & x < \text{Bias $b$}
		\end{cases}
		\end{equation*}
		\caption{Threshold}
		\label{fig:threshold-activation}
	\end{subfigure}%
	\hfill
	\begin{subfigure}{.5\textwidth}
		\centering
		% This file was created by matplotlib2tikz v0.7.3.
\begin{tikzpicture}

\begin{axis}[
height=\figureheight,
tick align=outside,
tick pos=left,
width=\figurewidth,
x grid style={white!90.01960784313725!black},
xlabel={\(\displaystyle x\)},
xmajorgrids,
xmin=-10.999, xmax=10.9789999999996,
xtick style={color=black},
y grid style={white!90.01960784313725!black},
ylabel={\(\displaystyle \phi(x)\)},
ymajorgrids,
ymin=-0.05, ymax=1.05,
ytick style={color=black},
ytick={-0.2,0,0.2,0.4,0.6,0.8,1,1.2},
yticklabels={,0.0,0.2,0.4,0.6,0.8,1.0,}
]
\addplot [semithick, green!50.0!black]
table {%
-10 0
-9.98 0
-9.96 0
-9.94 0
-9.92 0
-9.9 0
-9.88 0
-9.86 0
-9.84 0
-9.82 0
-9.8 0
-9.78 0
-9.76000000000001 0
-9.74000000000001 0
-9.72000000000001 0
-9.70000000000001 0
-9.68000000000001 0
-9.66000000000001 0
-9.64000000000001 0
-9.62000000000001 0
-9.60000000000001 0
-9.58000000000001 0
-9.56000000000001 0
-9.54000000000001 0
-9.52000000000001 0
-9.50000000000001 0
-9.48000000000001 0
-9.46000000000001 0
-9.44000000000001 0
-9.42000000000001 0
-9.40000000000001 0
-9.38000000000001 0
-9.36000000000001 0
-9.34000000000001 0
-9.32000000000001 0
-9.30000000000001 0
-9.28000000000002 0
-9.26000000000002 0
-9.24000000000002 0
-9.22000000000002 0
-9.20000000000002 0
-9.18000000000002 0
-9.16000000000002 0
-9.14000000000002 0
-9.12000000000002 0
-9.10000000000002 0
-9.08000000000002 0
-9.06000000000002 0
-9.04000000000002 0
-9.02000000000002 0
-9.00000000000002 0
-8.98000000000002 0
-8.96000000000002 0
-8.94000000000002 0
-8.92000000000002 0
-8.90000000000002 0
-8.88000000000002 0
-8.86000000000002 0
-8.84000000000002 0
-8.82000000000003 0
-8.80000000000003 0
-8.78000000000003 0
-8.76000000000003 0
-8.74000000000003 0
-8.72000000000003 0
-8.70000000000003 0
-8.68000000000003 0
-8.66000000000003 0
-8.64000000000003 0
-8.62000000000003 0
-8.60000000000003 0
-8.58000000000003 0
-8.56000000000003 0
-8.54000000000003 0
-8.52000000000003 0
-8.50000000000003 0
-8.48000000000003 0
-8.46000000000003 0
-8.44000000000003 0
-8.42000000000003 0
-8.40000000000003 0
-8.38000000000003 0
-8.36000000000003 0
-8.34000000000004 0
-8.32000000000004 0
-8.30000000000004 0
-8.28000000000004 0
-8.26000000000004 0
-8.24000000000004 0
-8.22000000000004 0
-8.20000000000004 0
-8.18000000000004 0
-8.16000000000004 0
-8.14000000000004 0
-8.12000000000004 0
-8.10000000000004 0
-8.08000000000004 0
-8.06000000000004 0
-8.04000000000004 0
-8.02000000000004 0
-8.00000000000004 0
-7.98000000000004 0
-7.96000000000004 0
-7.94000000000004 0
-7.92000000000004 0
-7.90000000000004 0
-7.88000000000005 0
-7.86000000000005 0
-7.84000000000005 0
-7.82000000000005 0
-7.80000000000005 0
-7.78000000000005 0
-7.76000000000005 0
-7.74000000000005 0
-7.72000000000005 0
-7.70000000000005 0
-7.68000000000005 0
-7.66000000000005 0
-7.64000000000005 0
-7.62000000000005 0
-7.60000000000005 0
-7.58000000000005 0
-7.56000000000005 0
-7.54000000000005 0
-7.52000000000005 0
-7.50000000000005 0
-7.48000000000005 0
-7.46000000000005 0
-7.44000000000005 0
-7.42000000000005 0
-7.40000000000006 0
-7.38000000000006 0
-7.36000000000006 0
-7.34000000000006 0
-7.32000000000006 0
-7.30000000000006 0
-7.28000000000006 0
-7.26000000000006 0
-7.24000000000006 0
-7.22000000000006 0
-7.20000000000006 0
-7.18000000000006 0
-7.16000000000006 0
-7.14000000000006 0
-7.12000000000006 0
-7.10000000000006 0
-7.08000000000006 0
-7.06000000000006 0
-7.04000000000006 0
-7.02000000000006 0
-7.00000000000006 0
-6.98000000000006 0
-6.96000000000006 0
-6.94000000000007 0
-6.92000000000007 0
-6.90000000000007 0
-6.88000000000007 0
-6.86000000000007 0
-6.84000000000007 0
-6.82000000000007 0
-6.80000000000007 0
-6.78000000000007 0
-6.76000000000007 0
-6.74000000000007 0
-6.72000000000007 0
-6.70000000000007 0
-6.68000000000007 0
-6.66000000000007 0
-6.64000000000007 0
-6.62000000000007 0
-6.60000000000007 0
-6.58000000000007 0
-6.56000000000007 0
-6.54000000000007 0
-6.52000000000007 0
-6.50000000000007 0
-6.48000000000008 0
-6.46000000000008 0
-6.44000000000008 0
-6.42000000000008 0
-6.40000000000008 0
-6.38000000000008 0
-6.36000000000008 0
-6.34000000000008 0
-6.32000000000008 0
-6.30000000000008 0
-6.28000000000008 0
-6.26000000000008 0
-6.24000000000008 0
-6.22000000000008 0
-6.20000000000008 0
-6.18000000000008 0
-6.16000000000008 0
-6.14000000000008 0
-6.12000000000008 0
-6.10000000000008 0
-6.08000000000008 0
-6.06000000000008 0
-6.04000000000008 0
-6.02000000000008 0
-6.00000000000009 0
-5.98000000000009 0
-5.96000000000009 0
-5.94000000000009 0
-5.92000000000009 0
-5.90000000000009 0
-5.88000000000009 0
-5.86000000000009 0
-5.84000000000009 0
-5.82000000000009 0
-5.80000000000009 0
-5.78000000000009 0
-5.76000000000009 0
-5.74000000000009 0
-5.72000000000009 0
-5.70000000000009 0
-5.68000000000009 0
-5.66000000000009 0
-5.64000000000009 0
-5.62000000000009 0
-5.60000000000009 0
-5.58000000000009 0
-5.56000000000009 0
-5.5400000000001 0
-5.5200000000001 0
-5.5000000000001 0
-5.4800000000001 0
-5.4600000000001 0
-5.4400000000001 0
-5.4200000000001 0
-5.4000000000001 0
-5.3800000000001 0
-5.3600000000001 0
-5.3400000000001 0
-5.3200000000001 0
-5.3000000000001 0
-5.2800000000001 0
-5.2600000000001 0
-5.2400000000001 0
-5.2200000000001 0
-5.2000000000001 0
-5.1800000000001 0
-5.1600000000001 0
-5.1400000000001 0
-5.1200000000001 0
-5.1000000000001 0
-5.0800000000001 0
-5.06000000000011 0
-5.04000000000011 0
-5.02000000000011 0
-5.00000000000011 0
-4.98000000000011 0
-4.96000000000011 0
-4.94000000000011 0
-4.92000000000011 0
-4.90000000000011 0
-4.88000000000011 0
-4.86000000000011 0
-4.84000000000011 0
-4.82000000000011 0
-4.80000000000011 0
-4.78000000000011 0
-4.76000000000011 0
-4.74000000000011 0
-4.72000000000011 0
-4.70000000000011 0
-4.68000000000011 0
-4.66000000000011 0
-4.64000000000011 0
-4.62000000000011 0
-4.60000000000012 0
-4.58000000000012 0
-4.56000000000012 0
-4.54000000000012 0
-4.52000000000012 0
-4.50000000000012 0
-4.48000000000012 0
-4.46000000000012 0
-4.44000000000012 0
-4.42000000000012 0
-4.40000000000012 0
-4.38000000000012 0
-4.36000000000012 0
-4.34000000000012 0
-4.32000000000012 0
-4.30000000000012 0
-4.28000000000012 0
-4.26000000000012 0
-4.24000000000012 0
-4.22000000000012 0
-4.20000000000012 0
-4.18000000000012 0
-4.16000000000012 0
-4.14000000000012 0
-4.12000000000013 0
-4.10000000000013 0
-4.08000000000013 0
-4.06000000000013 0
-4.04000000000013 0
-4.02000000000013 0
-4.00000000000013 0
-3.98000000000013 0
-3.96000000000013 0
-3.94000000000013 0
-3.92000000000013 0
-3.90000000000013 0
-3.88000000000013 0
-3.86000000000013 0
-3.84000000000013 0
-3.82000000000013 0
-3.80000000000013 0
-3.78000000000013 0
-3.76000000000013 0
-3.74000000000013 0
-3.72000000000013 0
-3.70000000000013 0
-3.68000000000013 0
-3.66000000000014 0
-3.64000000000014 0
-3.62000000000014 0
-3.60000000000014 0
-3.58000000000014 0
-3.56000000000014 0
-3.54000000000014 0
-3.52000000000014 0
-3.50000000000014 0
-3.48000000000014 0
-3.46000000000014 0
-3.44000000000014 0
-3.42000000000014 0
-3.40000000000014 0
-3.38000000000014 0
-3.36000000000014 0
-3.34000000000014 0
-3.32000000000014 0
-3.30000000000014 0
-3.28000000000014 0
-3.26000000000014 0
-3.24000000000014 0
-3.22000000000014 0
-3.20000000000014 0
-3.18000000000015 0
-3.16000000000015 0
-3.14000000000015 0
-3.12000000000015 0
-3.10000000000015 0
-3.08000000000015 0
-3.06000000000015 0
-3.04000000000015 0
-3.02000000000015 0
-3.00000000000015 0
-2.98000000000015 0
-2.96000000000015 0
-2.94000000000015 0
-2.92000000000015 0
-2.90000000000015 0
-2.88000000000015 0
-2.86000000000015 0
-2.84000000000015 0
-2.82000000000015 0
-2.80000000000015 0
-2.78000000000015 0
-2.76000000000015 0
-2.74000000000015 0
-2.72000000000016 0
-2.70000000000016 0
-2.68000000000016 0
-2.66000000000016 0
-2.64000000000016 0
-2.62000000000016 0
-2.60000000000016 0
-2.58000000000016 0
-2.56000000000016 0
-2.54000000000016 0
-2.52000000000016 0
-2.50000000000016 0
-2.48000000000016 0
-2.46000000000016 0
-2.44000000000016 0
-2.42000000000016 0
-2.40000000000016 0
-2.38000000000016 0
-2.36000000000016 0
-2.34000000000016 0
-2.32000000000016 0
-2.30000000000016 0
-2.28000000000016 0
-2.26000000000016 0
-2.24000000000017 0
-2.22000000000017 0
-2.20000000000017 0
-2.18000000000017 0
-2.16000000000017 0
-2.14000000000017 0
-2.12000000000017 0
-2.10000000000017 0
-2.08000000000017 0
-2.06000000000017 0
-2.04000000000017 0
-2.02000000000017 0
-2.00000000000017 0
-1.98000000000017 0
-1.96000000000017 0
-1.94000000000017 0
-1.92000000000017 0
-1.90000000000017 0
-1.88000000000017 0
-1.86000000000017 0
-1.84000000000017 0
-1.82000000000017 0
-1.80000000000017 0
-1.78000000000018 0
-1.76000000000018 0
-1.74000000000018 0
-1.72000000000018 0
-1.70000000000018 0
-1.68000000000018 0
-1.66000000000018 0
-1.64000000000018 0
-1.62000000000018 0
-1.60000000000018 0
-1.58000000000018 0
-1.56000000000018 0
-1.54000000000018 0
-1.52000000000018 0
-1.50000000000018 0
-1.48000000000018 0
-1.46000000000018 0
-1.44000000000018 0
-1.42000000000018 0
-1.40000000000018 0
-1.38000000000018 0
-1.36000000000018 0
-1.34000000000018 0
-1.32000000000019 0
-1.30000000000019 0
-1.28000000000019 0
-1.26000000000019 0
-1.24000000000019 0
-1.22000000000019 0
-1.20000000000019 0
-1.18000000000019 0
-1.16000000000019 0
-1.14000000000019 0
-1.12000000000019 0
-1.10000000000019 0
-1.08000000000019 0
-1.06000000000019 0
-1.04000000000019 0
-1.02000000000019 0
-1.00000000000019 0
-0.980000000000192 0
-0.960000000000193 0
-0.940000000000193 0
-0.920000000000194 0
-0.900000000000194 0
-0.880000000000194 0
-0.860000000000195 0
-0.840000000000195 0
-0.820000000000196 0
-0.800000000000196 0
-0.780000000000197 0
-0.760000000000197 0
-0.740000000000197 0
-0.720000000000198 0
-0.700000000000198 0
-0.680000000000199 0
-0.660000000000199 0
-0.6400000000002 0
-0.6200000000002 0
-0.6000000000002 0
-0.580000000000201 0
-0.560000000000201 0
-0.540000000000202 0
-0.520000000000202 0
-0.500000000000203 0
-0.480000000000203 0
-0.460000000000203 0
-0.440000000000204 0
-0.420000000000204 0
-0.400000000000205 0
-0.380000000000205 0
-0.360000000000205 0
-0.340000000000206 0
-0.320000000000206 0
-0.300000000000207 0
-0.280000000000207 0
-0.260000000000208 0
-0.240000000000208 0
-0.220000000000208 0
-0.200000000000209 0
-0.180000000000209 0
-0.16000000000021 0
-0.14000000000021 0
-0.120000000000211 0
-0.100000000000211 0
-0.0800000000002115 0
-0.0600000000002119 0
-0.0400000000002123 0
-0.0200000000002127 0
-2.1316282072803e-13 0
0.0199999999997864 1
0.039999999999786 1
0.0599999999997856 1
0.0799999999997851 1
0.0999999999997847 1
0.119999999999784 1
0.139999999999784 1
0.159999999999783 1
0.179999999999783 1
0.199999999999783 1
0.219999999999782 1
0.239999999999782 1
0.259999999999781 1
0.279999999999781 1
0.29999999999978 1
0.31999999999978 1
0.33999999999978 1
0.359999999999779 1
0.379999999999779 1
0.399999999999778 1
0.419999999999778 1
0.439999999999777 1
0.459999999999777 1
0.479999999999777 1
0.499999999999776 1
0.519999999999776 1
0.539999999999775 1
0.559999999999775 1
0.579999999999774 1
0.599999999999774 1
0.619999999999774 1
0.639999999999773 1
0.659999999999773 1
0.679999999999772 1
0.699999999999772 1
0.719999999999771 1
0.739999999999771 1
0.759999999999771 1
0.77999999999977 1
0.79999999999977 1
0.819999999999769 1
0.839999999999769 1
0.859999999999769 1
0.879999999999768 1
0.899999999999768 1
0.919999999999767 1
0.939999999999767 1
0.959999999999766 1
0.979999999999766 1
0.999999999999766 1
1.01999999999977 1
1.03999999999976 1
1.05999999999976 1
1.07999999999976 1
1.09999999999976 1
1.11999999999976 1
1.13999999999976 1
1.15999999999976 1
1.17999999999976 1
1.19999999999976 1
1.21999999999976 1
1.23999999999976 1
1.25999999999976 1
1.27999999999976 1
1.29999999999976 1
1.31999999999976 1
1.33999999999976 1
1.35999999999976 1
1.37999999999976 1
1.39999999999976 1
1.41999999999976 1
1.43999999999976 1
1.45999999999976 1
1.47999999999976 1
1.49999999999975 1
1.51999999999975 1
1.53999999999975 1
1.55999999999975 1
1.57999999999975 1
1.59999999999975 1
1.61999999999975 1
1.63999999999975 1
1.65999999999975 1
1.67999999999975 1
1.69999999999975 1
1.71999999999975 1
1.73999999999975 1
1.75999999999975 1
1.77999999999975 1
1.79999999999975 1
1.81999999999975 1
1.83999999999975 1
1.85999999999975 1
1.87999999999975 1
1.89999999999975 1
1.91999999999975 1
1.93999999999975 1
1.95999999999975 1
1.97999999999974 1
1.99999999999974 1
2.01999999999974 1
2.03999999999974 1
2.05999999999974 1
2.07999999999974 1
2.09999999999974 1
2.11999999999974 1
2.13999999999974 1
2.15999999999974 1
2.17999999999974 1
2.19999999999974 1
2.21999999999974 1
2.23999999999974 1
2.25999999999974 1
2.27999999999974 1
2.29999999999974 1
2.31999999999974 1
2.33999999999974 1
2.35999999999974 1
2.37999999999974 1
2.39999999999974 1
2.41999999999974 1
2.43999999999973 1
2.45999999999973 1
2.47999999999973 1
2.49999999999973 1
2.51999999999973 1
2.53999999999973 1
2.55999999999973 1
2.57999999999973 1
2.59999999999973 1
2.61999999999973 1
2.63999999999973 1
2.65999999999973 1
2.67999999999973 1
2.69999999999973 1
2.71999999999973 1
2.73999999999973 1
2.75999999999973 1
2.77999999999973 1
2.79999999999973 1
2.81999999999973 1
2.83999999999973 1
2.85999999999973 1
2.87999999999973 1
2.89999999999973 1
2.91999999999972 1
2.93999999999972 1
2.95999999999972 1
2.97999999999972 1
2.99999999999972 1
3.01999999999972 1
3.03999999999972 1
3.05999999999972 1
3.07999999999972 1
3.09999999999972 1
3.11999999999972 1
3.13999999999972 1
3.15999999999972 1
3.17999999999972 1
3.19999999999972 1
3.21999999999972 1
3.23999999999972 1
3.25999999999972 1
3.27999999999972 1
3.29999999999972 1
3.31999999999972 1
3.33999999999972 1
3.35999999999972 1
3.37999999999971 1
3.39999999999971 1
3.41999999999971 1
3.43999999999971 1
3.45999999999971 1
3.47999999999971 1
3.49999999999971 1
3.51999999999971 1
3.53999999999971 1
3.55999999999971 1
3.57999999999971 1
3.59999999999971 1
3.61999999999971 1
3.63999999999971 1
3.65999999999971 1
3.67999999999971 1
3.69999999999971 1
3.71999999999971 1
3.73999999999971 1
3.75999999999971 1
3.77999999999971 1
3.79999999999971 1
3.81999999999971 1
3.8399999999997 1
3.8599999999997 1
3.8799999999997 1
3.8999999999997 1
3.9199999999997 1
3.9399999999997 1
3.9599999999997 1
3.9799999999997 1
3.9999999999997 1
4.0199999999997 1
4.0399999999997 1
4.0599999999997 1
4.0799999999997 1
4.0999999999997 1
4.1199999999997 1
4.1399999999997 1
4.1599999999997 1
4.1799999999997 1
4.1999999999997 1
4.2199999999997 1
4.2399999999997 1
4.2599999999997 1
4.2799999999997 1
4.2999999999997 1
4.31999999999969 1
4.33999999999969 1
4.35999999999969 1
4.37999999999969 1
4.39999999999969 1
4.41999999999969 1
4.43999999999969 1
4.45999999999969 1
4.47999999999969 1
4.49999999999969 1
4.51999999999969 1
4.53999999999969 1
4.55999999999969 1
4.57999999999969 1
4.59999999999969 1
4.61999999999969 1
4.63999999999969 1
4.65999999999969 1
4.67999999999969 1
4.69999999999969 1
4.71999999999969 1
4.73999999999969 1
4.75999999999969 1
4.77999999999968 1
4.79999999999968 1
4.81999999999968 1
4.83999999999968 1
4.85999999999968 1
4.87999999999968 1
4.89999999999968 1
4.91999999999968 1
4.93999999999968 1
4.95999999999968 1
4.97999999999968 1
4.99999999999968 1
5.01999999999968 1
5.03999999999968 1
5.05999999999968 1
5.07999999999968 1
5.09999999999968 1
5.11999999999968 1
5.13999999999968 1
5.15999999999968 1
5.17999999999968 1
5.19999999999968 1
5.21999999999968 1
5.23999999999968 1
5.25999999999967 1
5.27999999999967 1
5.29999999999967 1
5.31999999999967 1
5.33999999999967 1
5.35999999999967 1
5.37999999999967 1
5.39999999999967 1
5.41999999999967 1
5.43999999999967 1
5.45999999999967 1
5.47999999999967 1
5.49999999999967 1
5.51999999999967 1
5.53999999999967 1
5.55999999999967 1
5.57999999999967 1
5.59999999999967 1
5.61999999999967 1
5.63999999999967 1
5.65999999999967 1
5.67999999999967 1
5.69999999999967 1
5.71999999999966 1
5.73999999999966 1
5.75999999999966 1
5.77999999999966 1
5.79999999999966 1
5.81999999999966 1
5.83999999999966 1
5.85999999999966 1
5.87999999999966 1
5.89999999999966 1
5.91999999999966 1
5.93999999999966 1
5.95999999999966 1
5.97999999999966 1
5.99999999999966 1
6.01999999999966 1
6.03999999999966 1
6.05999999999966 1
6.07999999999966 1
6.09999999999966 1
6.11999999999966 1
6.13999999999966 1
6.15999999999966 1
6.17999999999966 1
6.19999999999965 1
6.21999999999965 1
6.23999999999965 1
6.25999999999965 1
6.27999999999965 1
6.29999999999965 1
6.31999999999965 1
6.33999999999965 1
6.35999999999965 1
6.37999999999965 1
6.39999999999965 1
6.41999999999965 1
6.43999999999965 1
6.45999999999965 1
6.47999999999965 1
6.49999999999965 1
6.51999999999965 1
6.53999999999965 1
6.55999999999965 1
6.57999999999965 1
6.59999999999965 1
6.61999999999965 1
6.63999999999965 1
6.65999999999964 1
6.67999999999964 1
6.69999999999964 1
6.71999999999964 1
6.73999999999964 1
6.75999999999964 1
6.77999999999964 1
6.79999999999964 1
6.81999999999964 1
6.83999999999964 1
6.85999999999964 1
6.87999999999964 1
6.89999999999964 1
6.91999999999964 1
6.93999999999964 1
6.95999999999964 1
6.97999999999964 1
6.99999999999964 1
7.01999999999964 1
7.03999999999964 1
7.05999999999964 1
7.07999999999964 1
7.09999999999964 1
7.11999999999964 1
7.13999999999963 1
7.15999999999963 1
7.17999999999963 1
7.19999999999963 1
7.21999999999963 1
7.23999999999963 1
7.25999999999963 1
7.27999999999963 1
7.29999999999963 1
7.31999999999963 1
7.33999999999963 1
7.35999999999963 1
7.37999999999963 1
7.39999999999963 1
7.41999999999963 1
7.43999999999963 1
7.45999999999963 1
7.47999999999963 1
7.49999999999963 1
7.51999999999963 1
7.53999999999963 1
7.55999999999963 1
7.57999999999963 1
7.59999999999962 1
7.61999999999962 1
7.63999999999962 1
7.65999999999962 1
7.67999999999962 1
7.69999999999962 1
7.71999999999962 1
7.73999999999962 1
7.75999999999962 1
7.77999999999962 1
7.79999999999962 1
7.81999999999962 1
7.83999999999962 1
7.85999999999962 1
7.87999999999962 1
7.89999999999962 1
7.91999999999962 1
7.93999999999962 1
7.95999999999962 1
7.97999999999962 1
7.99999999999962 1
8.01999999999962 1
8.03999999999962 1
8.05999999999962 1
8.07999999999961 1
8.09999999999961 1
8.11999999999961 1
8.13999999999961 1
8.15999999999961 1
8.17999999999961 1
8.19999999999961 1
8.21999999999961 1
8.23999999999961 1
8.25999999999961 1
8.27999999999961 1
8.29999999999961 1
8.31999999999961 1
8.33999999999961 1
8.35999999999961 1
8.37999999999961 1
8.39999999999961 1
8.41999999999961 1
8.43999999999961 1
8.45999999999961 1
8.47999999999961 1
8.49999999999961 1
8.51999999999961 1
8.5399999999996 1
8.5599999999996 1
8.5799999999996 1
8.5999999999996 1
8.6199999999996 1
8.6399999999996 1
8.6599999999996 1
8.6799999999996 1
8.6999999999996 1
8.7199999999996 1
8.7399999999996 1
8.7599999999996 1
8.7799999999996 1
8.7999999999996 1
8.8199999999996 1
8.8399999999996 1
8.8599999999996 1
8.8799999999996 1
8.8999999999996 1
8.9199999999996 1
8.9399999999996 1
8.9599999999996 1
8.9799999999996 1
8.99999999999959 1
9.01999999999959 1
9.03999999999959 1
9.05999999999959 1
9.07999999999959 1
9.09999999999959 1
9.11999999999959 1
9.13999999999959 1
9.15999999999959 1
9.17999999999959 1
9.19999999999959 1
9.21999999999959 1
9.23999999999959 1
9.25999999999959 1
9.27999999999959 1
9.29999999999959 1
9.31999999999959 1
9.33999999999959 1
9.35999999999959 1
9.37999999999959 1
9.39999999999959 1
9.41999999999959 1
9.43999999999959 1
9.45999999999959 1
9.47999999999958 1
9.49999999999958 1
9.51999999999958 1
9.53999999999958 1
9.55999999999958 1
9.57999999999958 1
9.59999999999958 1
9.61999999999958 1
9.63999999999958 1
9.65999999999958 1
9.67999999999958 1
9.69999999999958 1
9.71999999999958 1
9.73999999999958 1
9.75999999999958 1
9.77999999999958 1
9.79999999999958 1
9.81999999999958 1
9.83999999999958 1
9.85999999999958 1
9.87999999999958 1
9.89999999999958 1
9.91999999999958 1
9.93999999999957 1
9.95999999999957 1
9.97999999999957 1
};
\end{axis}

\end{tikzpicture}
		\begin{equation*}
		\phi(x) =
		\begin{cases}
		1 & x \geq 0 \\
		0 & x < 0
		\end{cases}
		\end{equation*}
		\caption{Heaviside}
		\label{fig:heaviside-activation}
	\end{subfigure}
	
	\begin{subfigure}{.5\textwidth}
		\centering
		\input{images/relu-activation.tikz}
		\begin{equation*}
		\phi(x) =
		\begin{cases}
		x & x \geq 0 \\
		0 & x < 0
		\end{cases}
		\end{equation*}
		\caption{Rectified Linear Unit}
		\label{fig:relu-activation}
	\end{subfigure}%
	\hfill
	\begin{subfigure}{.5\textwidth}
		\centering
		% This file was created by matplotlib2tikz v0.7.3.
\begin{tikzpicture}

\begin{axis}[
height=\figureheight,
tick align=outside,
tick pos=left,
width=\figurewidth,
x grid style={white!90.01960784313725!black},
xlabel={\(\displaystyle x\)},
xmajorgrids,
xmin=-10.999, xmax=10.9789999999996,
xtick style={color=black},
y grid style={white!90.01960784313725!black},
ylabel={\(\displaystyle \phi(x)\)},
ymajorgrids,
ymin=-1.54899999999998, ymax=10.5289999999996,
ytick style={color=black}
]
\addplot [semithick, green!50.0!black]
table {%
-10 -1
-9.98 -0.998
-9.96 -0.996
-9.94 -0.994
-9.92 -0.992
-9.9 -0.99
-9.88 -0.988
-9.86 -0.986
-9.84 -0.984
-9.82 -0.982
-9.8 -0.98
-9.78 -0.978000000000001
-9.76000000000001 -0.976000000000001
-9.74000000000001 -0.974000000000001
-9.72000000000001 -0.972000000000001
-9.70000000000001 -0.970000000000001
-9.68000000000001 -0.968000000000001
-9.66000000000001 -0.966000000000001
-9.64000000000001 -0.964000000000001
-9.62000000000001 -0.962000000000001
-9.60000000000001 -0.960000000000001
-9.58000000000001 -0.958000000000001
-9.56000000000001 -0.956000000000001
-9.54000000000001 -0.954000000000001
-9.52000000000001 -0.952000000000001
-9.50000000000001 -0.950000000000001
-9.48000000000001 -0.948000000000001
-9.46000000000001 -0.946000000000001
-9.44000000000001 -0.944000000000001
-9.42000000000001 -0.942000000000001
-9.40000000000001 -0.940000000000001
-9.38000000000001 -0.938000000000001
-9.36000000000001 -0.936000000000001
-9.34000000000001 -0.934000000000001
-9.32000000000001 -0.932000000000001
-9.30000000000001 -0.930000000000001
-9.28000000000002 -0.928000000000002
-9.26000000000002 -0.926000000000002
-9.24000000000002 -0.924000000000002
-9.22000000000002 -0.922000000000002
-9.20000000000002 -0.920000000000002
-9.18000000000002 -0.918000000000002
-9.16000000000002 -0.916000000000002
-9.14000000000002 -0.914000000000002
-9.12000000000002 -0.912000000000002
-9.10000000000002 -0.910000000000002
-9.08000000000002 -0.908000000000002
-9.06000000000002 -0.906000000000002
-9.04000000000002 -0.904000000000002
-9.02000000000002 -0.902000000000002
-9.00000000000002 -0.900000000000002
-8.98000000000002 -0.898000000000002
-8.96000000000002 -0.896000000000002
-8.94000000000002 -0.894000000000002
-8.92000000000002 -0.892000000000002
-8.90000000000002 -0.890000000000002
-8.88000000000002 -0.888000000000002
-8.86000000000002 -0.886000000000002
-8.84000000000002 -0.884000000000003
-8.82000000000003 -0.882000000000003
-8.80000000000003 -0.880000000000003
-8.78000000000003 -0.878000000000003
-8.76000000000003 -0.876000000000003
-8.74000000000003 -0.874000000000003
-8.72000000000003 -0.872000000000003
-8.70000000000003 -0.870000000000003
-8.68000000000003 -0.868000000000003
-8.66000000000003 -0.866000000000003
-8.64000000000003 -0.864000000000003
-8.62000000000003 -0.862000000000003
-8.60000000000003 -0.860000000000003
-8.58000000000003 -0.858000000000003
-8.56000000000003 -0.856000000000003
-8.54000000000003 -0.854000000000003
-8.52000000000003 -0.852000000000003
-8.50000000000003 -0.850000000000003
-8.48000000000003 -0.848000000000003
-8.46000000000003 -0.846000000000003
-8.44000000000003 -0.844000000000003
-8.42000000000003 -0.842000000000003
-8.40000000000003 -0.840000000000003
-8.38000000000003 -0.838000000000004
-8.36000000000003 -0.836000000000004
-8.34000000000004 -0.834000000000004
-8.32000000000004 -0.832000000000004
-8.30000000000004 -0.830000000000004
-8.28000000000004 -0.828000000000004
-8.26000000000004 -0.826000000000004
-8.24000000000004 -0.824000000000004
-8.22000000000004 -0.822000000000004
-8.20000000000004 -0.820000000000004
-8.18000000000004 -0.818000000000004
-8.16000000000004 -0.816000000000004
-8.14000000000004 -0.814000000000004
-8.12000000000004 -0.812000000000004
-8.10000000000004 -0.810000000000004
-8.08000000000004 -0.808000000000004
-8.06000000000004 -0.806000000000004
-8.04000000000004 -0.804000000000004
-8.02000000000004 -0.802000000000004
-8.00000000000004 -0.800000000000004
-7.98000000000004 -0.798000000000004
-7.96000000000004 -0.796000000000004
-7.94000000000004 -0.794000000000004
-7.92000000000004 -0.792000000000004
-7.90000000000004 -0.790000000000004
-7.88000000000005 -0.788000000000005
-7.86000000000005 -0.786000000000005
-7.84000000000005 -0.784000000000005
-7.82000000000005 -0.782000000000005
-7.80000000000005 -0.780000000000005
-7.78000000000005 -0.778000000000005
-7.76000000000005 -0.776000000000005
-7.74000000000005 -0.774000000000005
-7.72000000000005 -0.772000000000005
-7.70000000000005 -0.770000000000005
-7.68000000000005 -0.768000000000005
-7.66000000000005 -0.766000000000005
-7.64000000000005 -0.764000000000005
-7.62000000000005 -0.762000000000005
-7.60000000000005 -0.760000000000005
-7.58000000000005 -0.758000000000005
-7.56000000000005 -0.756000000000005
-7.54000000000005 -0.754000000000005
-7.52000000000005 -0.752000000000005
-7.50000000000005 -0.750000000000005
-7.48000000000005 -0.748000000000005
-7.46000000000005 -0.746000000000005
-7.44000000000005 -0.744000000000006
-7.42000000000005 -0.742000000000006
-7.40000000000006 -0.740000000000006
-7.38000000000006 -0.738000000000006
-7.36000000000006 -0.736000000000006
-7.34000000000006 -0.734000000000006
-7.32000000000006 -0.732000000000006
-7.30000000000006 -0.730000000000006
-7.28000000000006 -0.728000000000006
-7.26000000000006 -0.726000000000006
-7.24000000000006 -0.724000000000006
-7.22000000000006 -0.722000000000006
-7.20000000000006 -0.720000000000006
-7.18000000000006 -0.718000000000006
-7.16000000000006 -0.716000000000006
-7.14000000000006 -0.714000000000006
-7.12000000000006 -0.712000000000006
-7.10000000000006 -0.710000000000006
-7.08000000000006 -0.708000000000006
-7.06000000000006 -0.706000000000006
-7.04000000000006 -0.704000000000006
-7.02000000000006 -0.702000000000006
-7.00000000000006 -0.700000000000006
-6.98000000000006 -0.698000000000007
-6.96000000000006 -0.696000000000007
-6.94000000000007 -0.694000000000007
-6.92000000000007 -0.692000000000007
-6.90000000000007 -0.690000000000007
-6.88000000000007 -0.688000000000007
-6.86000000000007 -0.686000000000007
-6.84000000000007 -0.684000000000007
-6.82000000000007 -0.682000000000007
-6.80000000000007 -0.680000000000007
-6.78000000000007 -0.678000000000007
-6.76000000000007 -0.676000000000007
-6.74000000000007 -0.674000000000007
-6.72000000000007 -0.672000000000007
-6.70000000000007 -0.670000000000007
-6.68000000000007 -0.668000000000007
-6.66000000000007 -0.666000000000007
-6.64000000000007 -0.664000000000007
-6.62000000000007 -0.662000000000007
-6.60000000000007 -0.660000000000007
-6.58000000000007 -0.658000000000007
-6.56000000000007 -0.656000000000007
-6.54000000000007 -0.654000000000007
-6.52000000000007 -0.652000000000007
-6.50000000000007 -0.650000000000007
-6.48000000000008 -0.648000000000008
-6.46000000000008 -0.646000000000008
-6.44000000000008 -0.644000000000008
-6.42000000000008 -0.642000000000008
-6.40000000000008 -0.640000000000008
-6.38000000000008 -0.638000000000008
-6.36000000000008 -0.636000000000008
-6.34000000000008 -0.634000000000008
-6.32000000000008 -0.632000000000008
-6.30000000000008 -0.630000000000008
-6.28000000000008 -0.628000000000008
-6.26000000000008 -0.626000000000008
-6.24000000000008 -0.624000000000008
-6.22000000000008 -0.622000000000008
-6.20000000000008 -0.620000000000008
-6.18000000000008 -0.618000000000008
-6.16000000000008 -0.616000000000008
-6.14000000000008 -0.614000000000008
-6.12000000000008 -0.612000000000008
-6.10000000000008 -0.610000000000008
-6.08000000000008 -0.608000000000008
-6.06000000000008 -0.606000000000008
-6.04000000000008 -0.604000000000009
-6.02000000000008 -0.602000000000009
-6.00000000000009 -0.600000000000009
-5.98000000000009 -0.598000000000009
-5.96000000000009 -0.596000000000009
-5.94000000000009 -0.594000000000009
-5.92000000000009 -0.592000000000009
-5.90000000000009 -0.590000000000009
-5.88000000000009 -0.588000000000009
-5.86000000000009 -0.586000000000009
-5.84000000000009 -0.584000000000009
-5.82000000000009 -0.582000000000009
-5.80000000000009 -0.580000000000009
-5.78000000000009 -0.578000000000009
-5.76000000000009 -0.576000000000009
-5.74000000000009 -0.574000000000009
-5.72000000000009 -0.572000000000009
-5.70000000000009 -0.570000000000009
-5.68000000000009 -0.568000000000009
-5.66000000000009 -0.566000000000009
-5.64000000000009 -0.564000000000009
-5.62000000000009 -0.562000000000009
-5.60000000000009 -0.560000000000009
-5.58000000000009 -0.558000000000009
-5.56000000000009 -0.556000000000009
-5.5400000000001 -0.554000000000009
-5.5200000000001 -0.55200000000001
-5.5000000000001 -0.55000000000001
-5.4800000000001 -0.54800000000001
-5.4600000000001 -0.54600000000001
-5.4400000000001 -0.54400000000001
-5.4200000000001 -0.54200000000001
-5.4000000000001 -0.54000000000001
-5.3800000000001 -0.53800000000001
-5.3600000000001 -0.53600000000001
-5.3400000000001 -0.53400000000001
-5.3200000000001 -0.53200000000001
-5.3000000000001 -0.53000000000001
-5.2800000000001 -0.52800000000001
-5.2600000000001 -0.52600000000001
-5.2400000000001 -0.52400000000001
-5.2200000000001 -0.52200000000001
-5.2000000000001 -0.52000000000001
-5.1800000000001 -0.51800000000001
-5.1600000000001 -0.51600000000001
-5.1400000000001 -0.51400000000001
-5.1200000000001 -0.51200000000001
-5.1000000000001 -0.51000000000001
-5.0800000000001 -0.508000000000011
-5.06000000000011 -0.506000000000011
-5.04000000000011 -0.504000000000011
-5.02000000000011 -0.502000000000011
-5.00000000000011 -0.500000000000011
-4.98000000000011 -0.498000000000011
-4.96000000000011 -0.496000000000011
-4.94000000000011 -0.494000000000011
-4.92000000000011 -0.492000000000011
-4.90000000000011 -0.490000000000011
-4.88000000000011 -0.488000000000011
-4.86000000000011 -0.486000000000011
-4.84000000000011 -0.484000000000011
-4.82000000000011 -0.482000000000011
-4.80000000000011 -0.480000000000011
-4.78000000000011 -0.478000000000011
-4.76000000000011 -0.476000000000011
-4.74000000000011 -0.474000000000011
-4.72000000000011 -0.472000000000011
-4.70000000000011 -0.470000000000011
-4.68000000000011 -0.468000000000011
-4.66000000000011 -0.466000000000011
-4.64000000000011 -0.464000000000011
-4.62000000000011 -0.462000000000012
-4.60000000000012 -0.460000000000012
-4.58000000000012 -0.458000000000012
-4.56000000000012 -0.456000000000012
-4.54000000000012 -0.454000000000012
-4.52000000000012 -0.452000000000012
-4.50000000000012 -0.450000000000012
-4.48000000000012 -0.448000000000012
-4.46000000000012 -0.446000000000012
-4.44000000000012 -0.444000000000012
-4.42000000000012 -0.442000000000012
-4.40000000000012 -0.440000000000012
-4.38000000000012 -0.438000000000012
-4.36000000000012 -0.436000000000012
-4.34000000000012 -0.434000000000012
-4.32000000000012 -0.432000000000012
-4.30000000000012 -0.430000000000012
-4.28000000000012 -0.428000000000012
-4.26000000000012 -0.426000000000012
-4.24000000000012 -0.424000000000012
-4.22000000000012 -0.422000000000012
-4.20000000000012 -0.420000000000012
-4.18000000000012 -0.418000000000012
-4.16000000000012 -0.416000000000012
-4.14000000000012 -0.414000000000013
-4.12000000000013 -0.412000000000013
-4.10000000000013 -0.410000000000013
-4.08000000000013 -0.408000000000013
-4.06000000000013 -0.406000000000013
-4.04000000000013 -0.404000000000013
-4.02000000000013 -0.402000000000013
-4.00000000000013 -0.400000000000013
-3.98000000000013 -0.398000000000013
-3.96000000000013 -0.396000000000013
-3.94000000000013 -0.394000000000013
-3.92000000000013 -0.392000000000013
-3.90000000000013 -0.390000000000013
-3.88000000000013 -0.388000000000013
-3.86000000000013 -0.386000000000013
-3.84000000000013 -0.384000000000013
-3.82000000000013 -0.382000000000013
-3.80000000000013 -0.380000000000013
-3.78000000000013 -0.378000000000013
-3.76000000000013 -0.376000000000013
-3.74000000000013 -0.374000000000013
-3.72000000000013 -0.372000000000013
-3.70000000000013 -0.370000000000013
-3.68000000000013 -0.368000000000013
-3.66000000000014 -0.366000000000014
-3.64000000000014 -0.364000000000014
-3.62000000000014 -0.362000000000014
-3.60000000000014 -0.360000000000014
-3.58000000000014 -0.358000000000014
-3.56000000000014 -0.356000000000014
-3.54000000000014 -0.354000000000014
-3.52000000000014 -0.352000000000014
-3.50000000000014 -0.350000000000014
-3.48000000000014 -0.348000000000014
-3.46000000000014 -0.346000000000014
-3.44000000000014 -0.344000000000014
-3.42000000000014 -0.342000000000014
-3.40000000000014 -0.340000000000014
-3.38000000000014 -0.338000000000014
-3.36000000000014 -0.336000000000014
-3.34000000000014 -0.334000000000014
-3.32000000000014 -0.332000000000014
-3.30000000000014 -0.330000000000014
-3.28000000000014 -0.328000000000014
-3.26000000000014 -0.326000000000014
-3.24000000000014 -0.324000000000014
-3.22000000000014 -0.322000000000014
-3.20000000000014 -0.320000000000014
-3.18000000000015 -0.318000000000015
-3.16000000000015 -0.316000000000015
-3.14000000000015 -0.314000000000015
-3.12000000000015 -0.312000000000015
-3.10000000000015 -0.310000000000015
-3.08000000000015 -0.308000000000015
-3.06000000000015 -0.306000000000015
-3.04000000000015 -0.304000000000015
-3.02000000000015 -0.302000000000015
-3.00000000000015 -0.300000000000015
-2.98000000000015 -0.298000000000015
-2.96000000000015 -0.296000000000015
-2.94000000000015 -0.294000000000015
-2.92000000000015 -0.292000000000015
-2.90000000000015 -0.290000000000015
-2.88000000000015 -0.288000000000015
-2.86000000000015 -0.286000000000015
-2.84000000000015 -0.284000000000015
-2.82000000000015 -0.282000000000015
-2.80000000000015 -0.280000000000015
-2.78000000000015 -0.278000000000015
-2.76000000000015 -0.276000000000015
-2.74000000000015 -0.274000000000016
-2.72000000000016 -0.272000000000016
-2.70000000000016 -0.270000000000016
-2.68000000000016 -0.268000000000016
-2.66000000000016 -0.266000000000016
-2.64000000000016 -0.264000000000016
-2.62000000000016 -0.262000000000016
-2.60000000000016 -0.260000000000016
-2.58000000000016 -0.258000000000016
-2.56000000000016 -0.256000000000016
-2.54000000000016 -0.254000000000016
-2.52000000000016 -0.252000000000016
-2.50000000000016 -0.250000000000016
-2.48000000000016 -0.248000000000016
-2.46000000000016 -0.246000000000016
-2.44000000000016 -0.244000000000016
-2.42000000000016 -0.242000000000016
-2.40000000000016 -0.240000000000016
-2.38000000000016 -0.238000000000016
-2.36000000000016 -0.236000000000016
-2.34000000000016 -0.234000000000016
-2.32000000000016 -0.232000000000016
-2.30000000000016 -0.230000000000016
-2.28000000000016 -0.228000000000016
-2.26000000000016 -0.226000000000017
-2.24000000000017 -0.224000000000017
-2.22000000000017 -0.222000000000017
-2.20000000000017 -0.220000000000017
-2.18000000000017 -0.218000000000017
-2.16000000000017 -0.216000000000017
-2.14000000000017 -0.214000000000017
-2.12000000000017 -0.212000000000017
-2.10000000000017 -0.210000000000017
-2.08000000000017 -0.208000000000017
-2.06000000000017 -0.206000000000017
-2.04000000000017 -0.204000000000017
-2.02000000000017 -0.202000000000017
-2.00000000000017 -0.200000000000017
-1.98000000000017 -0.198000000000017
-1.96000000000017 -0.196000000000017
-1.94000000000017 -0.194000000000017
-1.92000000000017 -0.192000000000017
-1.90000000000017 -0.190000000000017
-1.88000000000017 -0.188000000000017
-1.86000000000017 -0.186000000000017
-1.84000000000017 -0.184000000000017
-1.82000000000017 -0.182000000000017
-1.80000000000017 -0.180000000000017
-1.78000000000018 -0.178000000000018
-1.76000000000018 -0.176000000000018
-1.74000000000018 -0.174000000000018
-1.72000000000018 -0.172000000000018
-1.70000000000018 -0.170000000000018
-1.68000000000018 -0.168000000000018
-1.66000000000018 -0.166000000000018
-1.64000000000018 -0.164000000000018
-1.62000000000018 -0.162000000000018
-1.60000000000018 -0.160000000000018
-1.58000000000018 -0.158000000000018
-1.56000000000018 -0.156000000000018
-1.54000000000018 -0.154000000000018
-1.52000000000018 -0.152000000000018
-1.50000000000018 -0.150000000000018
-1.48000000000018 -0.148000000000018
-1.46000000000018 -0.146000000000018
-1.44000000000018 -0.144000000000018
-1.42000000000018 -0.142000000000018
-1.40000000000018 -0.140000000000018
-1.38000000000018 -0.138000000000018
-1.36000000000018 -0.136000000000018
-1.34000000000018 -0.134000000000018
-1.32000000000019 -0.132000000000019
-1.30000000000019 -0.130000000000019
-1.28000000000019 -0.128000000000019
-1.26000000000019 -0.126000000000019
-1.24000000000019 -0.124000000000019
-1.22000000000019 -0.122000000000019
-1.20000000000019 -0.120000000000019
-1.18000000000019 -0.118000000000019
-1.16000000000019 -0.116000000000019
-1.14000000000019 -0.114000000000019
-1.12000000000019 -0.112000000000019
-1.10000000000019 -0.110000000000019
-1.08000000000019 -0.108000000000019
-1.06000000000019 -0.106000000000019
-1.04000000000019 -0.104000000000019
-1.02000000000019 -0.102000000000019
-1.00000000000019 -0.100000000000019
-0.980000000000192 -0.0980000000000192
-0.960000000000193 -0.0960000000000193
-0.940000000000193 -0.0940000000000193
-0.920000000000194 -0.0920000000000194
-0.900000000000194 -0.0900000000000194
-0.880000000000194 -0.0880000000000195
-0.860000000000195 -0.0860000000000195
-0.840000000000195 -0.0840000000000195
-0.820000000000196 -0.0820000000000196
-0.800000000000196 -0.0800000000000196
-0.780000000000197 -0.0780000000000197
-0.760000000000197 -0.0760000000000197
-0.740000000000197 -0.0740000000000197
-0.720000000000198 -0.0720000000000198
-0.700000000000198 -0.0700000000000198
-0.680000000000199 -0.0680000000000199
-0.660000000000199 -0.0660000000000199
-0.6400000000002 -0.06400000000002
-0.6200000000002 -0.06200000000002
-0.6000000000002 -0.06000000000002
-0.580000000000201 -0.0580000000000201
-0.560000000000201 -0.0560000000000201
-0.540000000000202 -0.0540000000000202
-0.520000000000202 -0.0520000000000202
-0.500000000000203 -0.0500000000000203
-0.480000000000203 -0.0480000000000203
-0.460000000000203 -0.0460000000000203
-0.440000000000204 -0.0440000000000204
-0.420000000000204 -0.0420000000000204
-0.400000000000205 -0.0400000000000205
-0.380000000000205 -0.0380000000000205
-0.360000000000205 -0.0360000000000206
-0.340000000000206 -0.0340000000000206
-0.320000000000206 -0.0320000000000206
-0.300000000000207 -0.0300000000000207
-0.280000000000207 -0.0280000000000207
-0.260000000000208 -0.0260000000000208
-0.240000000000208 -0.0240000000000208
-0.220000000000208 -0.0220000000000209
-0.200000000000209 -0.0200000000000209
-0.180000000000209 -0.0180000000000209
-0.16000000000021 -0.016000000000021
-0.14000000000021 -0.014000000000021
-0.120000000000211 -0.0120000000000211
-0.100000000000211 -0.0100000000000211
-0.0800000000002115 -0.00800000000002115
-0.0600000000002119 -0.00600000000002119
-0.0400000000002123 -0.00400000000002123
-0.0200000000002127 -0.00200000000002127
-2.1316282072803e-13 -2.1316282072803e-14
0.0199999999997864 0.0199999999997864
0.039999999999786 0.039999999999786
0.0599999999997856 0.0599999999997856
0.0799999999997851 0.0799999999997851
0.0999999999997847 0.0999999999997847
0.119999999999784 0.119999999999784
0.139999999999784 0.139999999999784
0.159999999999783 0.159999999999783
0.179999999999783 0.179999999999783
0.199999999999783 0.199999999999783
0.219999999999782 0.219999999999782
0.239999999999782 0.239999999999782
0.259999999999781 0.259999999999781
0.279999999999781 0.279999999999781
0.29999999999978 0.29999999999978
0.31999999999978 0.31999999999978
0.33999999999978 0.33999999999978
0.359999999999779 0.359999999999779
0.379999999999779 0.379999999999779
0.399999999999778 0.399999999999778
0.419999999999778 0.419999999999778
0.439999999999777 0.439999999999777
0.459999999999777 0.459999999999777
0.479999999999777 0.479999999999777
0.499999999999776 0.499999999999776
0.519999999999776 0.519999999999776
0.539999999999775 0.539999999999775
0.559999999999775 0.559999999999775
0.579999999999774 0.579999999999774
0.599999999999774 0.599999999999774
0.619999999999774 0.619999999999774
0.639999999999773 0.639999999999773
0.659999999999773 0.659999999999773
0.679999999999772 0.679999999999772
0.699999999999772 0.699999999999772
0.719999999999771 0.719999999999771
0.739999999999771 0.739999999999771
0.759999999999771 0.759999999999771
0.77999999999977 0.77999999999977
0.79999999999977 0.79999999999977
0.819999999999769 0.819999999999769
0.839999999999769 0.839999999999769
0.859999999999769 0.859999999999769
0.879999999999768 0.879999999999768
0.899999999999768 0.899999999999768
0.919999999999767 0.919999999999767
0.939999999999767 0.939999999999767
0.959999999999766 0.959999999999766
0.979999999999766 0.979999999999766
0.999999999999766 0.999999999999766
1.01999999999977 1.01999999999977
1.03999999999976 1.03999999999976
1.05999999999976 1.05999999999976
1.07999999999976 1.07999999999976
1.09999999999976 1.09999999999976
1.11999999999976 1.11999999999976
1.13999999999976 1.13999999999976
1.15999999999976 1.15999999999976
1.17999999999976 1.17999999999976
1.19999999999976 1.19999999999976
1.21999999999976 1.21999999999976
1.23999999999976 1.23999999999976
1.25999999999976 1.25999999999976
1.27999999999976 1.27999999999976
1.29999999999976 1.29999999999976
1.31999999999976 1.31999999999976
1.33999999999976 1.33999999999976
1.35999999999976 1.35999999999976
1.37999999999976 1.37999999999976
1.39999999999976 1.39999999999976
1.41999999999976 1.41999999999976
1.43999999999976 1.43999999999976
1.45999999999976 1.45999999999976
1.47999999999976 1.47999999999976
1.49999999999975 1.49999999999975
1.51999999999975 1.51999999999975
1.53999999999975 1.53999999999975
1.55999999999975 1.55999999999975
1.57999999999975 1.57999999999975
1.59999999999975 1.59999999999975
1.61999999999975 1.61999999999975
1.63999999999975 1.63999999999975
1.65999999999975 1.65999999999975
1.67999999999975 1.67999999999975
1.69999999999975 1.69999999999975
1.71999999999975 1.71999999999975
1.73999999999975 1.73999999999975
1.75999999999975 1.75999999999975
1.77999999999975 1.77999999999975
1.79999999999975 1.79999999999975
1.81999999999975 1.81999999999975
1.83999999999975 1.83999999999975
1.85999999999975 1.85999999999975
1.87999999999975 1.87999999999975
1.89999999999975 1.89999999999975
1.91999999999975 1.91999999999975
1.93999999999975 1.93999999999975
1.95999999999975 1.95999999999975
1.97999999999974 1.97999999999974
1.99999999999974 1.99999999999974
2.01999999999974 2.01999999999974
2.03999999999974 2.03999999999974
2.05999999999974 2.05999999999974
2.07999999999974 2.07999999999974
2.09999999999974 2.09999999999974
2.11999999999974 2.11999999999974
2.13999999999974 2.13999999999974
2.15999999999974 2.15999999999974
2.17999999999974 2.17999999999974
2.19999999999974 2.19999999999974
2.21999999999974 2.21999999999974
2.23999999999974 2.23999999999974
2.25999999999974 2.25999999999974
2.27999999999974 2.27999999999974
2.29999999999974 2.29999999999974
2.31999999999974 2.31999999999974
2.33999999999974 2.33999999999974
2.35999999999974 2.35999999999974
2.37999999999974 2.37999999999974
2.39999999999974 2.39999999999974
2.41999999999974 2.41999999999974
2.43999999999973 2.43999999999973
2.45999999999973 2.45999999999973
2.47999999999973 2.47999999999973
2.49999999999973 2.49999999999973
2.51999999999973 2.51999999999973
2.53999999999973 2.53999999999973
2.55999999999973 2.55999999999973
2.57999999999973 2.57999999999973
2.59999999999973 2.59999999999973
2.61999999999973 2.61999999999973
2.63999999999973 2.63999999999973
2.65999999999973 2.65999999999973
2.67999999999973 2.67999999999973
2.69999999999973 2.69999999999973
2.71999999999973 2.71999999999973
2.73999999999973 2.73999999999973
2.75999999999973 2.75999999999973
2.77999999999973 2.77999999999973
2.79999999999973 2.79999999999973
2.81999999999973 2.81999999999973
2.83999999999973 2.83999999999973
2.85999999999973 2.85999999999973
2.87999999999973 2.87999999999973
2.89999999999973 2.89999999999973
2.91999999999972 2.91999999999972
2.93999999999972 2.93999999999972
2.95999999999972 2.95999999999972
2.97999999999972 2.97999999999972
2.99999999999972 2.99999999999972
3.01999999999972 3.01999999999972
3.03999999999972 3.03999999999972
3.05999999999972 3.05999999999972
3.07999999999972 3.07999999999972
3.09999999999972 3.09999999999972
3.11999999999972 3.11999999999972
3.13999999999972 3.13999999999972
3.15999999999972 3.15999999999972
3.17999999999972 3.17999999999972
3.19999999999972 3.19999999999972
3.21999999999972 3.21999999999972
3.23999999999972 3.23999999999972
3.25999999999972 3.25999999999972
3.27999999999972 3.27999999999972
3.29999999999972 3.29999999999972
3.31999999999972 3.31999999999972
3.33999999999972 3.33999999999972
3.35999999999972 3.35999999999972
3.37999999999971 3.37999999999971
3.39999999999971 3.39999999999971
3.41999999999971 3.41999999999971
3.43999999999971 3.43999999999971
3.45999999999971 3.45999999999971
3.47999999999971 3.47999999999971
3.49999999999971 3.49999999999971
3.51999999999971 3.51999999999971
3.53999999999971 3.53999999999971
3.55999999999971 3.55999999999971
3.57999999999971 3.57999999999971
3.59999999999971 3.59999999999971
3.61999999999971 3.61999999999971
3.63999999999971 3.63999999999971
3.65999999999971 3.65999999999971
3.67999999999971 3.67999999999971
3.69999999999971 3.69999999999971
3.71999999999971 3.71999999999971
3.73999999999971 3.73999999999971
3.75999999999971 3.75999999999971
3.77999999999971 3.77999999999971
3.79999999999971 3.79999999999971
3.81999999999971 3.81999999999971
3.8399999999997 3.8399999999997
3.8599999999997 3.8599999999997
3.8799999999997 3.8799999999997
3.8999999999997 3.8999999999997
3.9199999999997 3.9199999999997
3.9399999999997 3.9399999999997
3.9599999999997 3.9599999999997
3.9799999999997 3.9799999999997
3.9999999999997 3.9999999999997
4.0199999999997 4.0199999999997
4.0399999999997 4.0399999999997
4.0599999999997 4.0599999999997
4.0799999999997 4.0799999999997
4.0999999999997 4.0999999999997
4.1199999999997 4.1199999999997
4.1399999999997 4.1399999999997
4.1599999999997 4.1599999999997
4.1799999999997 4.1799999999997
4.1999999999997 4.1999999999997
4.2199999999997 4.2199999999997
4.2399999999997 4.2399999999997
4.2599999999997 4.2599999999997
4.2799999999997 4.2799999999997
4.2999999999997 4.2999999999997
4.31999999999969 4.31999999999969
4.33999999999969 4.33999999999969
4.35999999999969 4.35999999999969
4.37999999999969 4.37999999999969
4.39999999999969 4.39999999999969
4.41999999999969 4.41999999999969
4.43999999999969 4.43999999999969
4.45999999999969 4.45999999999969
4.47999999999969 4.47999999999969
4.49999999999969 4.49999999999969
4.51999999999969 4.51999999999969
4.53999999999969 4.53999999999969
4.55999999999969 4.55999999999969
4.57999999999969 4.57999999999969
4.59999999999969 4.59999999999969
4.61999999999969 4.61999999999969
4.63999999999969 4.63999999999969
4.65999999999969 4.65999999999969
4.67999999999969 4.67999999999969
4.69999999999969 4.69999999999969
4.71999999999969 4.71999999999969
4.73999999999969 4.73999999999969
4.75999999999969 4.75999999999969
4.77999999999968 4.77999999999968
4.79999999999968 4.79999999999968
4.81999999999968 4.81999999999968
4.83999999999968 4.83999999999968
4.85999999999968 4.85999999999968
4.87999999999968 4.87999999999968
4.89999999999968 4.89999999999968
4.91999999999968 4.91999999999968
4.93999999999968 4.93999999999968
4.95999999999968 4.95999999999968
4.97999999999968 4.97999999999968
4.99999999999968 4.99999999999968
5.01999999999968 5.01999999999968
5.03999999999968 5.03999999999968
5.05999999999968 5.05999999999968
5.07999999999968 5.07999999999968
5.09999999999968 5.09999999999968
5.11999999999968 5.11999999999968
5.13999999999968 5.13999999999968
5.15999999999968 5.15999999999968
5.17999999999968 5.17999999999968
5.19999999999968 5.19999999999968
5.21999999999968 5.21999999999968
5.23999999999968 5.23999999999968
5.25999999999967 5.25999999999967
5.27999999999967 5.27999999999967
5.29999999999967 5.29999999999967
5.31999999999967 5.31999999999967
5.33999999999967 5.33999999999967
5.35999999999967 5.35999999999967
5.37999999999967 5.37999999999967
5.39999999999967 5.39999999999967
5.41999999999967 5.41999999999967
5.43999999999967 5.43999999999967
5.45999999999967 5.45999999999967
5.47999999999967 5.47999999999967
5.49999999999967 5.49999999999967
5.51999999999967 5.51999999999967
5.53999999999967 5.53999999999967
5.55999999999967 5.55999999999967
5.57999999999967 5.57999999999967
5.59999999999967 5.59999999999967
5.61999999999967 5.61999999999967
5.63999999999967 5.63999999999967
5.65999999999967 5.65999999999967
5.67999999999967 5.67999999999967
5.69999999999967 5.69999999999967
5.71999999999966 5.71999999999966
5.73999999999966 5.73999999999966
5.75999999999966 5.75999999999966
5.77999999999966 5.77999999999966
5.79999999999966 5.79999999999966
5.81999999999966 5.81999999999966
5.83999999999966 5.83999999999966
5.85999999999966 5.85999999999966
5.87999999999966 5.87999999999966
5.89999999999966 5.89999999999966
5.91999999999966 5.91999999999966
5.93999999999966 5.93999999999966
5.95999999999966 5.95999999999966
5.97999999999966 5.97999999999966
5.99999999999966 5.99999999999966
6.01999999999966 6.01999999999966
6.03999999999966 6.03999999999966
6.05999999999966 6.05999999999966
6.07999999999966 6.07999999999966
6.09999999999966 6.09999999999966
6.11999999999966 6.11999999999966
6.13999999999966 6.13999999999966
6.15999999999966 6.15999999999966
6.17999999999966 6.17999999999966
6.19999999999965 6.19999999999965
6.21999999999965 6.21999999999965
6.23999999999965 6.23999999999965
6.25999999999965 6.25999999999965
6.27999999999965 6.27999999999965
6.29999999999965 6.29999999999965
6.31999999999965 6.31999999999965
6.33999999999965 6.33999999999965
6.35999999999965 6.35999999999965
6.37999999999965 6.37999999999965
6.39999999999965 6.39999999999965
6.41999999999965 6.41999999999965
6.43999999999965 6.43999999999965
6.45999999999965 6.45999999999965
6.47999999999965 6.47999999999965
6.49999999999965 6.49999999999965
6.51999999999965 6.51999999999965
6.53999999999965 6.53999999999965
6.55999999999965 6.55999999999965
6.57999999999965 6.57999999999965
6.59999999999965 6.59999999999965
6.61999999999965 6.61999999999965
6.63999999999965 6.63999999999965
6.65999999999964 6.65999999999964
6.67999999999964 6.67999999999964
6.69999999999964 6.69999999999964
6.71999999999964 6.71999999999964
6.73999999999964 6.73999999999964
6.75999999999964 6.75999999999964
6.77999999999964 6.77999999999964
6.79999999999964 6.79999999999964
6.81999999999964 6.81999999999964
6.83999999999964 6.83999999999964
6.85999999999964 6.85999999999964
6.87999999999964 6.87999999999964
6.89999999999964 6.89999999999964
6.91999999999964 6.91999999999964
6.93999999999964 6.93999999999964
6.95999999999964 6.95999999999964
6.97999999999964 6.97999999999964
6.99999999999964 6.99999999999964
7.01999999999964 7.01999999999964
7.03999999999964 7.03999999999964
7.05999999999964 7.05999999999964
7.07999999999964 7.07999999999964
7.09999999999964 7.09999999999964
7.11999999999964 7.11999999999964
7.13999999999963 7.13999999999963
7.15999999999963 7.15999999999963
7.17999999999963 7.17999999999963
7.19999999999963 7.19999999999963
7.21999999999963 7.21999999999963
7.23999999999963 7.23999999999963
7.25999999999963 7.25999999999963
7.27999999999963 7.27999999999963
7.29999999999963 7.29999999999963
7.31999999999963 7.31999999999963
7.33999999999963 7.33999999999963
7.35999999999963 7.35999999999963
7.37999999999963 7.37999999999963
7.39999999999963 7.39999999999963
7.41999999999963 7.41999999999963
7.43999999999963 7.43999999999963
7.45999999999963 7.45999999999963
7.47999999999963 7.47999999999963
7.49999999999963 7.49999999999963
7.51999999999963 7.51999999999963
7.53999999999963 7.53999999999963
7.55999999999963 7.55999999999963
7.57999999999963 7.57999999999963
7.59999999999962 7.59999999999962
7.61999999999962 7.61999999999962
7.63999999999962 7.63999999999962
7.65999999999962 7.65999999999962
7.67999999999962 7.67999999999962
7.69999999999962 7.69999999999962
7.71999999999962 7.71999999999962
7.73999999999962 7.73999999999962
7.75999999999962 7.75999999999962
7.77999999999962 7.77999999999962
7.79999999999962 7.79999999999962
7.81999999999962 7.81999999999962
7.83999999999962 7.83999999999962
7.85999999999962 7.85999999999962
7.87999999999962 7.87999999999962
7.89999999999962 7.89999999999962
7.91999999999962 7.91999999999962
7.93999999999962 7.93999999999962
7.95999999999962 7.95999999999962
7.97999999999962 7.97999999999962
7.99999999999962 7.99999999999962
8.01999999999962 8.01999999999962
8.03999999999962 8.03999999999962
8.05999999999962 8.05999999999962
8.07999999999961 8.07999999999961
8.09999999999961 8.09999999999961
8.11999999999961 8.11999999999961
8.13999999999961 8.13999999999961
8.15999999999961 8.15999999999961
8.17999999999961 8.17999999999961
8.19999999999961 8.19999999999961
8.21999999999961 8.21999999999961
8.23999999999961 8.23999999999961
8.25999999999961 8.25999999999961
8.27999999999961 8.27999999999961
8.29999999999961 8.29999999999961
8.31999999999961 8.31999999999961
8.33999999999961 8.33999999999961
8.35999999999961 8.35999999999961
8.37999999999961 8.37999999999961
8.39999999999961 8.39999999999961
8.41999999999961 8.41999999999961
8.43999999999961 8.43999999999961
8.45999999999961 8.45999999999961
8.47999999999961 8.47999999999961
8.49999999999961 8.49999999999961
8.51999999999961 8.51999999999961
8.5399999999996 8.5399999999996
8.5599999999996 8.5599999999996
8.5799999999996 8.5799999999996
8.5999999999996 8.5999999999996
8.6199999999996 8.6199999999996
8.6399999999996 8.6399999999996
8.6599999999996 8.6599999999996
8.6799999999996 8.6799999999996
8.6999999999996 8.6999999999996
8.7199999999996 8.7199999999996
8.7399999999996 8.7399999999996
8.7599999999996 8.7599999999996
8.7799999999996 8.7799999999996
8.7999999999996 8.7999999999996
8.8199999999996 8.8199999999996
8.8399999999996 8.8399999999996
8.8599999999996 8.8599999999996
8.8799999999996 8.8799999999996
8.8999999999996 8.8999999999996
8.9199999999996 8.9199999999996
8.9399999999996 8.9399999999996
8.9599999999996 8.9599999999996
8.9799999999996 8.9799999999996
8.99999999999959 8.99999999999959
9.01999999999959 9.01999999999959
9.03999999999959 9.03999999999959
9.05999999999959 9.05999999999959
9.07999999999959 9.07999999999959
9.09999999999959 9.09999999999959
9.11999999999959 9.11999999999959
9.13999999999959 9.13999999999959
9.15999999999959 9.15999999999959
9.17999999999959 9.17999999999959
9.19999999999959 9.19999999999959
9.21999999999959 9.21999999999959
9.23999999999959 9.23999999999959
9.25999999999959 9.25999999999959
9.27999999999959 9.27999999999959
9.29999999999959 9.29999999999959
9.31999999999959 9.31999999999959
9.33999999999959 9.33999999999959
9.35999999999959 9.35999999999959
9.37999999999959 9.37999999999959
9.39999999999959 9.39999999999959
9.41999999999959 9.41999999999959
9.43999999999959 9.43999999999959
9.45999999999959 9.45999999999959
9.47999999999958 9.47999999999958
9.49999999999958 9.49999999999958
9.51999999999958 9.51999999999958
9.53999999999958 9.53999999999958
9.55999999999958 9.55999999999958
9.57999999999958 9.57999999999958
9.59999999999958 9.59999999999958
9.61999999999958 9.61999999999958
9.63999999999958 9.63999999999958
9.65999999999958 9.65999999999958
9.67999999999958 9.67999999999958
9.69999999999958 9.69999999999958
9.71999999999958 9.71999999999958
9.73999999999958 9.73999999999958
9.75999999999958 9.75999999999958
9.77999999999958 9.77999999999958
9.79999999999958 9.79999999999958
9.81999999999958 9.81999999999958
9.83999999999958 9.83999999999958
9.85999999999958 9.85999999999958
9.87999999999958 9.87999999999958
9.89999999999958 9.89999999999958
9.91999999999958 9.91999999999958
9.93999999999957 9.93999999999957
9.95999999999957 9.95999999999957
9.97999999999957 9.97999999999957
};
\end{axis}

\end{tikzpicture}
		\begin{equation*}
		\phi(x) =
		\begin{cases}
		x & x \geq 0 \\
		x\cdot \text{Slope $\lambda$} & x < 0
		\end{cases}
		\end{equation*}
		\caption{Leaky Rectified Linear Unit}
		\label{fig:leakyrelu-activation}
	\end{subfigure}
	
	\begin{subfigure}{.5\textwidth}
		\centering
		% This file was created by matplotlib2tikz v0.7.3.
\begin{tikzpicture}

\begin{axis}[
height=\figureheight,
tick align=outside,
tick pos=left,
width=\figurewidth,
x grid style={white!90.01960784313725!black},
xlabel={\(\displaystyle x\)},
xmajorgrids,
xmin=-10.999, xmax=10.9789999999996,
xtick style={color=black},
y grid style={white!90.01960784313725!black},
ylabel={\(\displaystyle \phi(x)\)},
ymajorgrids,
ymin=-0.049950016491662, ymax=1.04994909943635,
ytick style={color=black},
ytick={-0.2,0,0.2,0.4,0.6,0.8,1,1.2},
yticklabels={,0.0,0.2,0.4,0.6,0.8,1.0,}
]
\addplot [semithick, green!50.0!black]
table {%
-10 4.53978687024344e-05
-9.98 4.63149240092682e-05
-9.96 4.72505033288101e-05
-9.94 4.82049808002045e-05
-9.92 4.91787381178213e-05
-9.9 5.01721646837641e-05
-9.88 5.11856577634538e-05
-9.86 5.22196226443509e-05
-9.84 5.32744727978785e-05
-9.82 5.43506300446111e-05
-9.8 5.54485247227947e-05
-9.78 5.65685958602659e-05
-9.76000000000001 5.77112913498368e-05
-9.74000000000001 5.88770681282178e-05
-9.72000000000001 6.00663923585472e-05
-9.70000000000001 6.1279739616602e-05
-9.68000000000001 6.2517595080763e-05
-9.66000000000001 6.37804537258095e-05
-9.64000000000001 6.50688205206224e-05
-9.62000000000001 6.63832106298714e-05
-9.60000000000001 6.77241496197696e-05
-9.58000000000001 6.90921736679753e-05
-9.56000000000001 7.04878297777246e-05
-9.54000000000001 7.19116759962808e-05
-9.52000000000001 7.33642816377872e-05
-9.50000000000001 7.48462275106104e-05
-9.48000000000001 7.63581061492668e-05
-9.46000000000001 7.79005220510224e-05
-9.44000000000001 7.9474091917261e-05
-9.42000000000001 8.10794448997163e-05
-9.40000000000001 8.27172228516654e-05
-9.38000000000001 8.4388080584184e-05
-9.36000000000001 8.60926861275648e-05
-9.34000000000001 8.78317209980021e-05
-9.32000000000001 8.96058804696499e-05
-9.30000000000001 9.14158738521601e-05
-9.28000000000002 9.32624247738116e-05
-9.26000000000002 9.51462714703421e-05
-9.24000000000002 9.70681670795981e-05
-9.22000000000002 9.90288799421182e-05
-9.20000000000002 0.000101029193907771
-9.18000000000002 0.000103069908648568
-9.16000000000002 0.000105151839977772
-9.14000000000002 0.000107275820175438
-9.12000000000002 0.000109442698320503
-9.10000000000002 0.000111653340629561
-9.08000000000002 0.00011390863080246
-9.06000000000002 0.000116209470374859
-9.04000000000002 0.000118556779077874
-9.02000000000002 0.00012095149520497
-9.00000000000002 0.000123394575986229
-8.98000000000002 0.000125886997970159
-8.96000000000002 0.000128429757413177
-8.94000000000002 0.000131023870676934
-8.92000000000002 0.000133670374633632
-8.90000000000002 0.000136370327079494
-8.88000000000002 0.000139124807156555
-8.86000000000002 0.000141934915782931
-8.84000000000002 0.000144801776091746
-8.82000000000003 0.000147726533878887
-8.80000000000003 0.000150710358059754
-8.78000000000003 0.000153754441135206
-8.76000000000003 0.000156859999666868
-8.74000000000003 0.000160028274761988
-8.72000000000003 0.000163260532568052
-8.70000000000003 0.000166558064777331
-8.68000000000003 0.000169922189141569
-8.66000000000003 0.000173354249997016
-8.64000000000003 0.000176855618800008
-8.62000000000003 0.000180427694673307
-8.60000000000003 0.000184071904963418
-8.58000000000003 0.000187789705809097
-8.56000000000003 0.000191582582721276
-8.54000000000003 0.000195452051174638
-8.52000000000003 0.000199399657211064
-8.50000000000003 0.0002034269780552
-8.48000000000003 0.000207535622742374
-8.46000000000003 0.000211727232759125
-8.44000000000003 0.000216003482696574
-8.42000000000003 0.000220366080916919
-8.40000000000003 0.000224816770233288
-8.38000000000003 0.000229357328603239
-8.36000000000003 0.000233989569836168
-8.34000000000004 0.000238715344314899
-8.32000000000004 0.000243536539731744
-8.30000000000004 0.000248455081839325
-8.28000000000004 0.000253472935216442
-8.26000000000004 0.000258592104049282
-8.24000000000004 0.000263814632928303
-8.22000000000004 0.000269142607661057
-8.20000000000004 0.000274578156101322
-8.18000000000004 0.000280123448994817
-8.16000000000004 0.000285780700841868
-8.14000000000004 0.000291552170777337
-8.12000000000004 0.000297440163468171
-8.10000000000004 0.000303447030028907
-8.08000000000004 0.000309575168955503
-8.06000000000004 0.00031582702707785
-8.04000000000004 0.00032220510053133
-8.02000000000004 0.000328711935747814
-8.00000000000004 0.000335350130466464
-7.98000000000004 0.000342122334764745
-7.96000000000004 0.000349031252110046
-7.94000000000004 0.000356079640432303
-7.92000000000004 0.000363270313218062
-7.90000000000004 0.00037060614062638
-7.88000000000005 0.000378090050627021
-7.86000000000005 0.00038572503016136
-7.84000000000005 0.000393514126326465
-7.82000000000005 0.000401460447582802
-7.80000000000005 0.000409567164986031
-7.78000000000005 0.000417837513443365
-7.76000000000005 0.00042627479299498
-7.74000000000005 0.000434882370120969
-7.72000000000005 0.000443663679074329
-7.70000000000005 0.000452622223240513
-7.68000000000005 0.000461761576524052
-7.66000000000005 0.000471085384762786
-7.64000000000005 0.000480597367170237
-7.62000000000005 0.000490301317806692
-7.60000000000005 0.000500201107079539
-7.58000000000005 0.000510300683273438
-7.56000000000005 0.000520604074110915
-7.54000000000005 0.000531115388343959
-7.52000000000005 0.000541838817377241
-7.50000000000005 0.00055277863692357
-7.48000000000005 0.000563939208692207
-7.46000000000005 0.000575324982110682
-7.44000000000005 0.000586940496080771
-7.42000000000005 0.000598790380769285
-7.40000000000006 0.000610879359434367
-7.38000000000006 0.000623212250287964
-7.36000000000006 0.000635793968395193
-7.34000000000006 0.000648629527611308
-7.32000000000006 0.000661724042557007
-7.30000000000006 0.000675082730632799
-7.28000000000006 0.000688710914073216
-7.26000000000006 0.000702614022041614
-7.24000000000006 0.000716797592766362
-7.22000000000006 0.00073126727571921
-7.20000000000006 0.000746028833836653
-7.18000000000006 0.000761088145785115
-7.16000000000006 0.000776451208270797
-7.14000000000006 0.000792124138395046
-7.12000000000006 0.000808113176056122
-7.10000000000006 0.000824424686398244
-7.08000000000006 0.000841065162308828
-7.06000000000006 0.000858041226964839
-7.04000000000006 0.000875359636429182
-7.02000000000006 0.000893027282298109
-7.00000000000006 0.000911051194400587
-6.98000000000006 0.000929438543550642
-6.96000000000006 0.000948196644353664
-6.94000000000007 0.00096733295806771
-6.92000000000007 0.000986855095520843
-6.90000000000007 0.00100677082008557
-6.88000000000007 0.00102708805071145
-6.86000000000007 0.00104781486501699
-6.84000000000007 0.00106895950244189
-6.82000000000007 0.0010905303674609
-6.80000000000007 0.00111253603286025
-6.78000000000007 0.00113498524307802
-6.76000000000007 0.00115788691760954
-6.74000000000007 0.00118125015447904
-6.72000000000007 0.00120508423377886
-6.70000000000007 0.00122939862127733
-6.68000000000007 0.0012542029720968
-6.66000000000007 0.00127950713446292
-6.64000000000007 0.00130532115352668
-6.62000000000007 0.00133165527526034
-6.60000000000007 0.00135851995042886
-6.58000000000007 0.00138592583863798
-6.56000000000007 0.00141388381246055
-6.54000000000007 0.00144240496164241
-6.52000000000007 0.00147150059738942
-6.50000000000007 0.00150118225673688
-6.48000000000008 0.00153146170700317
-6.46000000000008 0.00156235095032884
-6.44000000000008 0.00159386222830289
-6.42000000000008 0.00162600802667769
-6.40000000000008 0.00165880108017429
-6.38000000000008 0.00169225437737958
-6.36000000000008 0.00172638116573702
-6.34000000000008 0.00176119495663271
-6.32000000000008 0.0017967095305783
-6.30000000000008 0.00183293894249266
-6.28000000000008 0.0018698975270839
-6.26000000000008 0.00190759990433367
-6.24000000000008 0.0019460609850854
-6.22000000000008 0.00198529597673842
-6.20000000000008 0.00202532038904972
-6.18000000000008 0.00206615004004532
-6.16000000000008 0.00210780106204308
-6.14000000000008 0.00215028990778896
-6.12000000000008 0.00219363335670856
-6.10000000000008 0.00223784852127615
-6.08000000000008 0.00228295285350287
-6.06000000000008 0.00232896415154654
-6.04000000000008 0.00237590056644479
-6.02000000000008 0.00242378060897376
-6.00000000000009 0.00247262315663456
-5.98000000000009 0.00252244746076948
-5.96000000000009 0.00257327315381015
-5.94000000000009 0.00262512025665997
-5.92000000000009 0.00267800918621288
-5.90000000000009 0.00273196076301082
-5.88000000000009 0.00278699621904204
-5.86000000000009 0.00284313720568269
-5.84000000000009 0.00290040580178394
-5.82000000000009 0.00295882452190697
-5.80000000000009 0.00301841632470815
-5.78000000000009 0.00307920462147702
-5.76000000000009 0.00314121328482914
-5.74000000000009 0.00320446665755659
-5.72000000000009 0.00326898956163832
-5.70000000000009 0.00333480730741304
-5.68000000000009 0.00340194570291695
-5.66000000000009 0.00347043106338901
-5.64000000000009 0.00354029022094618
-5.62000000000009 0.00361155053443121
-5.60000000000009 0.00368423989943564
-5.58000000000009 0.00375838675850045
-5.56000000000009 0.00383402011149708
-5.5400000000001 0.0039111695261915
-5.5200000000001 0.00398986514899372
-5.5000000000001 0.00407013771589574
-5.4800000000001 0.00415201856360025
-5.4600000000001 0.00423553964084302
-5.4400000000001 0.0043207335199115
-5.4200000000001 0.00440763340836238
-5.4000000000001 0.00449627316094074
-5.3800000000001 0.00458668729170357
-5.3600000000001 0.00467891098635024
-5.3400000000001 0.00477298011476273
-5.3200000000001 0.00486893124375816
-5.3000000000001 0.00496680165005647
-5.2800000000001 0.00506662933346575
-5.2600000000001 0.00516845303028805
-5.2400000000001 0.00527231222694814
-5.2200000000001 0.00537824717384806
-5.2000000000001 0.00548629889944985
-5.1800000000001 0.00559650922458923
-5.1600000000001 0.00570892077702276
-5.1400000000001 0.00582357700621092
-5.1200000000001 0.00594052219833976
-5.1000000000001 0.00605980149158349
-5.0800000000001 0.0061814608916105
-5.06000000000011 0.0063055472873352
-5.04000000000011 0.00643210846691796
-5.02000000000011 0.00656119313401553
-5.00000000000011 0.00669285092428415
-4.98000000000011 0.00682713242213744
-4.96000000000011 0.00696408917776135
-4.94000000000011 0.00710377372438802
-4.92000000000011 0.00724623959583065
-4.90000000000011 0.00739154134428118
-4.88000000000011 0.00753973455837258
-4.86000000000011 0.0076908758815075
-4.84000000000011 0.00784502303045478
-4.82000000000011 0.00800223481421537
-4.80000000000011 0.008162571153159
-4.78000000000011 0.00832609309843287
-4.76000000000011 0.00849286285164341
-4.74000000000011 0.0086629437848122
-4.72000000000011 0.00883640046060673
-4.70000000000011 0.00901329865284682
-4.68000000000011 0.00919370536728706
-4.66000000000011 0.0093776888626758
-4.64000000000011 0.00956531867209059
-4.62000000000011 0.00975666562455025
-4.60000000000012 0.00995180186690319
-4.58000000000012 0.0101508008859916
-4.56000000000012 0.0103537375310906
-4.54000000000012 0.0105606880366219
-4.52000000000012 0.0107717300451404
-4.50000000000012 0.0109869426305919
-4.48000000000012 0.0112064063218416
-4.46000000000012 0.0114302031264694
-4.44000000000012 0.0116584165548317
-4.42000000000012 0.0118911316443856
-4.40000000000012 0.0121284349842728
-4.38000000000012 0.012370414740161
-4.36000000000012 0.0126171606793372
-4.34000000000012 0.012868764196051
-4.32000000000012 0.0131253183371012
-4.30000000000012 0.0133869178276632
-4.28000000000012 0.0136536590973492
-4.26000000000012 0.0139256403064987
-4.24000000000012 0.0142029613726894
-4.22000000000012 0.0144857239974651
-4.20000000000012 0.0147740316932713
-4.18000000000012 0.015067989810591
-4.16000000000012 0.0153677055652737
-4.14000000000012 0.0156732880660466
-4.12000000000013 0.0159848483422006
-4.10000000000013 0.0163024993714389
-4.08000000000013 0.0166263561078796
-4.06000000000013 0.0169565355101987
-4.04000000000013 0.0172931565699033
-4.02000000000013 0.0176363403397205
-4.00000000000013 0.0179862099620893
-3.98000000000013 0.018342890697741
-3.96000000000013 0.0187065099543522
-3.94000000000013 0.0190771973152557
-3.92000000000013 0.0194550845681906
-3.90000000000013 0.019840305734075
-3.88000000000013 0.0202329970957828
-3.86000000000013 0.0206332972269036
-3.84000000000013 0.0210413470204656
-3.82000000000013 0.0214572897176008
-3.80000000000013 0.0218812709361276
-3.78000000000013 0.0223134386990293
-3.76000000000013 0.022753943462801
-3.74000000000013 0.0232029381456411
-3.72000000000013 0.0236605781554581
-3.70000000000013 0.024127021417666
-3.68000000000013 0.0246024284027362
-3.66000000000014 0.0250869621534765
-3.64000000000014 0.0255807883120043
-3.62000000000014 0.0260840751463795
-3.60000000000014 0.0265969935768623
-3.58000000000014 0.0271197172017594
-3.56000000000014 0.0276524223228194
-3.54000000000014 0.0281952879701388
-3.52000000000014 0.0287484959265361
-3.50000000000014 0.0293122307513524
-3.48000000000014 0.0298866798036322
-3.46000000000014 0.03047203326464
-3.44000000000014 0.0310684841596632
-3.42000000000014 0.0316762283790521
-3.40000000000014 0.0322954646984461
-3.38000000000014 0.0329263947981318
-3.36000000000014 0.0335692232814779
-3.34000000000014 0.0342241576923912
-3.32000000000014 0.034891408531732
-3.30000000000014 0.0355711892726313
-3.28000000000014 0.0362637163746433
-3.26000000000014 0.0369692092966719
-3.24000000000014 0.0376878905086007
-3.22000000000014 0.03841998550156
-3.20000000000014 0.0391657227967589
-3.18000000000015 0.0399253339528082
-3.16000000000015 0.0406990535714608
-3.14000000000015 0.04148711930169
-3.12000000000015 0.0422897718420279
-3.10000000000015 0.0431072549410801
-3.08000000000015 0.0439398153961351
-3.06000000000015 0.0447877030497804
-3.04000000000015 0.0456511707844373
-3.02000000000015 0.0465304745147249
-3.00000000000015 0.04742587317756
-2.98000000000015 0.0483376287198983
-2.96000000000015 0.0492660060840196
-2.94000000000015 0.0502112731902594
-2.92000000000015 0.0511737009170843
-2.90000000000015 0.0521535630784103
-2.88000000000015 0.0531511363980561
-2.86000000000015 0.0541667004812283
-2.84000000000015 0.0552005377829263
-2.82000000000015 0.0562529335731592
-2.80000000000015 0.0573241758988604
-2.78000000000015 0.0584145555423878
-2.76000000000015 0.0595243659764929
-2.74000000000015 0.0606539033156433
-2.72000000000016 0.0618034662635796
-2.70000000000016 0.0629733560569873
-2.68000000000016 0.0641638764051646
-2.66000000000016 0.0653753334255631
-2.64000000000016 0.0666080355750809
-2.62000000000016 0.0678622935769845
-2.60000000000016 0.0691384203433367
-2.58000000000016 0.0704367308928067
-2.56000000000016 0.0717575422637408
-2.54000000000016 0.0731011734223674
-2.52000000000016 0.0744679451660171
-2.50000000000016 0.0758581800212323
-2.48000000000016 0.0772722021366485
-2.46000000000016 0.0787103371705236
-2.44000000000016 0.0801729121728004
-2.42000000000016 0.0816602554615825
-2.40000000000016 0.08317269649391
-2.38000000000016 0.0847105657307232
-2.36000000000016 0.0862741944959039
-2.34000000000016 0.0878639148292882
-2.32000000000016 0.0894800593335481
-2.30000000000016 0.0911229610148425
-2.28000000000016 0.0927929531171432
-2.26000000000016 0.0944903689501452
-2.24000000000017 0.0962155417106785
-2.22000000000017 0.0979688042975393
-2.20000000000017 0.0997504891196702
-2.18000000000017 0.101560927897621
-2.16000000000017 0.103400451458234
-2.14000000000017 0.105269389522494
-2.12000000000017 0.107168070486512
-2.10000000000017 0.109096821195597
-2.08000000000017 0.111055966711391
-2.06000000000017 0.113045830072062
-2.04000000000017 0.115066732045533
-2.02000000000017 0.117118990875763
-2.00000000000017 0.1192029220221
-1.98000000000017 0.121318837891719
-1.96000000000017 0.123467047565205
-1.94000000000017 0.125647856515327
-1.92000000000017 0.127861566319062
-1.90000000000017 0.130108474362978
-1.88000000000017 0.132388873542046
-1.86000000000017 0.134703051952008
-1.84000000000017 0.137051292575439
-1.82000000000017 0.139433872961629
-1.80000000000017 0.141851064900467
-1.78000000000018 0.144303134090497
-1.76000000000018 0.14679033980136
-1.74000000000018 0.149312934530821
-1.72000000000018 0.151871163656637
-1.70000000000018 0.154465265083512
-1.68000000000018 0.157095468885429
-1.66000000000018 0.159761996943645
-1.64000000000018 0.162465062580672
-1.62000000000018 0.16520487019059
-1.60000000000018 0.167981614866051
-1.58000000000018 0.170795482022349
-1.56000000000018 0.173646647018979
-1.54000000000018 0.176535274779091
-1.52000000000018 0.1794615194073
-1.50000000000018 0.182425523806329
-1.48000000000018 0.185427419292955
-1.46000000000018 0.188467325213792
-1.44000000000018 0.191545348561439
-1.42000000000018 0.194661583591549
-1.40000000000018 0.197816111441389
-1.38000000000018 0.2010089997505
-1.36000000000018 0.204240302284062
-1.34000000000018 0.207510058559605
-1.32000000000019 0.210818293477716
-1.30000000000019 0.21416501695741
-1.28000000000019 0.217550223576856
-1.26000000000019 0.220973892220156
-1.24000000000019 0.224435985730894
-1.22000000000019 0.227936450573183
-1.20000000000019 0.231475216500949
-1.18000000000019 0.235052196236201
-1.16000000000019 0.238667285157055
-1.14000000000019 0.24232036099626
-1.12000000000019 0.246011283551017
-1.10000000000019 0.249739894404847
-1.08000000000019 0.253506016662302
-1.06000000000019 0.257309454697278
-1.04000000000019 0.261149993915714
-1.02000000000019 0.265027400533444
-1.00000000000019 0.268941421369957
-0.980000000000192 0.272891783658832
-0.960000000000193 0.276878194875572
-0.940000000000193 0.280900342583577
-0.920000000000194 0.284957894298971
-0.900000000000194 0.289050497374956
-0.880000000000194 0.293177778906392
-0.860000000000195 0.297339345655228
-0.840000000000195 0.30153478399742
-0.820000000000196 0.305763659891928
-0.800000000000196 0.310025518872346
-0.780000000000197 0.314319886061704
-0.760000000000197 0.318646266210932
-0.740000000000197 0.323004143761434
-0.720000000000198 0.327392982932196
-0.700000000000198 0.33181222783179
-0.680000000000199 0.336261302595603
-0.660000000000199 0.34073961154857
-0.6400000000002 0.345246539393636
-0.6200000000002 0.349781451426127
-0.6000000000002 0.354343693774159
-0.580000000000201 0.358932593665137
-0.560000000000201 0.363547459718387
-0.540000000000202 0.368187582263851
-0.520000000000202 0.372852233686757
-0.500000000000203 0.377540668798098
-0.480000000000203 0.382252125230703
-0.460000000000203 0.386985823860616
-0.440000000000204 0.391740969253437
-0.420000000000204 0.396516750135225
-0.400000000000205 0.401312339887499
-0.380000000000205 0.406126897065808
-0.360000000000205 0.410959565941285
-0.340000000000206 0.415809477064543
-0.320000000000206 0.4206757478512
-0.300000000000207 0.42555748318829
-0.280000000000207 0.43045377606072
-0.260000000000208 0.43536370819692
-0.240000000000208 0.440286350732756
-0.220000000000208 0.445220764892734
-0.200000000000209 0.45016600268747
-0.180000000000209 0.455121107626368
-0.16000000000021 0.460085115444382
-0.14000000000021 0.465057054841733
-0.120000000000211 0.470035948235376
-0.100000000000211 0.475020812521007
-0.0800000000002115 0.480010659844365
-0.0600000000002119 0.485004498380537
-0.0400000000002123 0.490001333119982
-0.0200000000002127 0.495000166659947
-2.1316282072803e-13 0.499999999999947
0.0199999999997864 0.504999833339946
0.039999999999786 0.509998666879912
0.0599999999997856 0.514995501619356
0.0799999999997851 0.519989340155528
0.0999999999997847 0.524979187478886
0.119999999999784 0.529964051764518
0.139999999999784 0.534942945158161
0.159999999999783 0.539914884555512
0.179999999999783 0.544878892373526
0.199999999999783 0.549833997312424
0.219999999999782 0.554779235107161
0.239999999999782 0.559713649267139
0.259999999999781 0.564636291802975
0.279999999999781 0.569546223939175
0.29999999999978 0.574442516811605
0.31999999999978 0.579324252148696
0.33999999999978 0.584190522935354
0.359999999999779 0.589040434058612
0.379999999999779 0.593873102934089
0.399999999999778 0.598687660112399
0.419999999999778 0.603483249864673
0.439999999999777 0.608259030746461
0.459999999999777 0.613014176139283
0.479999999999777 0.617747874769196
0.499999999999776 0.622459331201802
0.519999999999776 0.627147766313143
0.539999999999775 0.631812417736049
0.559999999999775 0.636452540281514
0.579999999999774 0.641067406334765
0.599999999999774 0.645656306225744
0.619999999999774 0.650218548573775
0.639999999999773 0.654753460606268
0.659999999999773 0.659260388451334
0.679999999999772 0.663738697404302
0.699999999999772 0.668187772168116
0.719999999999771 0.67260701706771
0.739999999999771 0.676995856238473
0.759999999999771 0.681353733788976
0.77999999999977 0.685680113938204
0.79999999999977 0.689974481127563
0.819999999999769 0.694236340107982
0.839999999999769 0.69846521600249
0.859999999999769 0.702660654344683
0.879999999999768 0.70682222109352
0.899999999999768 0.710949502624956
0.919999999999767 0.715042105700942
0.939999999999767 0.719099657416337
0.959999999999766 0.723121805124343
0.979999999999766 0.727108216341083
0.999999999999766 0.731058578629959
1.01999999999977 0.734972599466473
1.03999999999976 0.738850006084204
1.05999999999976 0.742690545302641
1.07999999999976 0.746493983337617
1.09999999999976 0.750260105595073
1.11999999999976 0.753988716448904
1.13999999999976 0.757679639003661
1.15999999999976 0.761332714842867
1.17999999999976 0.764947803763722
1.19999999999976 0.768524783498975
1.21999999999976 0.772063549426742
1.23999999999976 0.775564014269032
1.25999999999976 0.779026107779771
1.27999999999976 0.782449776423072
1.29999999999976 0.785834983042518
1.31999999999976 0.789181706522213
1.33999999999976 0.792489941440325
1.35999999999976 0.795759697715869
1.37999999999976 0.798991000249432
1.39999999999976 0.802183888558543
1.41999999999976 0.805338416408384
1.43999999999976 0.808454651438495
1.45999999999976 0.811532674786143
1.47999999999976 0.814572580706981
1.49999999999975 0.817574476193607
1.51999999999975 0.820538480592637
1.53999999999975 0.823464725220848
1.55999999999975 0.82635335298096
1.57999999999975 0.829204517977591
1.59999999999975 0.83201838513389
1.61999999999975 0.834795129809351
1.63999999999975 0.83753493741927
1.65999999999975 0.840238003056298
1.67999999999975 0.842904531114514
1.69999999999975 0.845534734916433
1.71999999999975 0.848128836343309
1.73999999999975 0.850687065469124
1.75999999999975 0.853209660198586
1.77999999999975 0.85569686590945
1.79999999999975 0.858148935099482
1.81999999999975 0.86056612703832
1.83999999999975 0.86294870742451
1.85999999999975 0.865296948047942
1.87999999999975 0.867611126457906
1.89999999999975 0.869891525636973
1.91999999999975 0.87213843368089
1.93999999999975 0.874352143484626
1.95999999999975 0.876532952434748
1.97999999999974 0.878681162108236
1.99999999999974 0.880797077977856
2.01999999999974 0.882881009124193
2.03999999999974 0.884933267954424
2.05999999999974 0.886954169927895
2.07999999999974 0.888944033288567
2.09999999999974 0.890903178804362
2.11999999999974 0.892831929513447
2.13999999999974 0.894730610477466
2.15999999999974 0.896599548541726
2.17999999999974 0.89843907210234
2.19999999999974 0.900249510880291
2.21999999999974 0.902031195702423
2.23999999999974 0.903784458289285
2.25999999999974 0.905509631049818
2.27999999999974 0.907207046882821
2.29999999999974 0.908877038985122
2.31999999999974 0.910519940666417
2.33999999999974 0.912136085170678
2.35999999999974 0.913725805504063
2.37999999999974 0.915289434269244
2.39999999999974 0.916827303506057
2.41999999999974 0.918339744538386
2.43999999999973 0.919827087827168
2.45999999999973 0.921289662829445
2.47999999999973 0.922727797863321
2.49999999999973 0.924141819978738
2.51999999999973 0.925532054833954
2.53999999999973 0.926898826577604
2.55999999999973 0.928242457736231
2.57999999999973 0.929563269107165
2.59999999999973 0.930861579656636
2.61999999999973 0.932137706422989
2.63999999999973 0.933391964424893
2.65999999999973 0.934624666574411
2.67999999999973 0.93583612359481
2.69999999999973 0.937026643942988
2.71999999999973 0.938196533736396
2.73999999999973 0.939346096684332
2.75999999999973 0.940475634023483
2.77999999999973 0.941585444457589
2.79999999999973 0.942675824101116
2.81999999999973 0.943747066426818
2.83999999999973 0.944799462217051
2.85999999999973 0.94583329951875
2.87999999999973 0.946848863601923
2.89999999999973 0.947846436921569
2.91999999999972 0.948826299082895
2.93999999999972 0.94978872680972
2.95999999999972 0.95073399391596
2.97999999999972 0.951662371280082
2.99999999999972 0.952574126822421
3.01999999999972 0.953469525485256
3.03999999999972 0.954348829215544
3.05999999999972 0.955212296950201
3.07999999999972 0.956060184603847
3.09999999999972 0.956892745058902
3.11999999999972 0.957710228157955
3.13999999999972 0.958512880698293
3.15999999999972 0.959300946428522
3.17999999999972 0.960074666047176
3.19999999999972 0.960834277203225
3.21999999999972 0.961580014498424
3.23999999999972 0.962312109491384
3.25999999999972 0.963030790703313
3.27999999999972 0.963736283625342
3.29999999999972 0.964428810727354
3.31999999999972 0.965108591468254
3.33999999999972 0.965775842307595
3.35999999999972 0.966430776718508
3.37999999999971 0.967073605201855
3.39999999999971 0.967704535301541
3.41999999999971 0.968323771620935
3.43999999999971 0.968931515840324
3.45999999999971 0.969527966735347
3.47999999999971 0.970113320196356
3.49999999999971 0.970687769248635
3.51999999999971 0.971251504073452
3.53999999999971 0.971804712029849
3.55999999999971 0.972347577677169
3.57999999999971 0.972880282798229
3.59999999999971 0.973403006423127
3.61999999999971 0.97391592485361
3.63999999999971 0.974419211687985
3.65999999999971 0.974913037846513
3.67999999999971 0.975397571597253
3.69999999999971 0.975872978582324
3.71999999999971 0.976339421844532
3.73999999999971 0.976797061854349
3.75999999999971 0.97724605653719
3.77999999999971 0.977686561300961
3.79999999999971 0.978118729063863
3.81999999999971 0.97854271028239
3.8399999999997 0.978958652979526
3.8599999999997 0.979366702773088
3.8799999999997 0.979767002904209
3.8999999999997 0.980159694265917
3.9199999999997 0.980544915431801
3.9399999999997 0.980922802684736
3.9599999999997 0.98129349004564
3.9799999999997 0.981657109302251
3.9999999999997 0.982013790037903
4.0199999999997 0.982363659660272
4.0399999999997 0.982706843430089
4.0599999999997 0.983043464489794
4.0799999999997 0.983373643892113
4.0999999999997 0.983697500628554
4.1199999999997 0.984015151657793
4.1399999999997 0.984326711933947
4.1599999999997 0.98463229443472
4.1799999999997 0.984932010189403
4.1999999999997 0.985225968306723
4.2199999999997 0.985514276002529
4.2399999999997 0.985797038627305
4.2599999999997 0.986074359693495
4.2799999999997 0.986346340902645
4.2999999999997 0.986613082172331
4.31999999999969 0.986874681662893
4.33999999999969 0.987131235803944
4.35999999999969 0.987382839320657
4.37999999999969 0.987629585259834
4.39999999999969 0.987871565015722
4.41999999999969 0.988108868355609
4.43999999999969 0.988341583445163
4.45999999999969 0.988569796873526
4.47999999999969 0.988793593678154
4.49999999999969 0.989013057369403
4.51999999999969 0.989228269954855
4.53999999999969 0.989439311963374
4.55999999999969 0.989646262468905
4.57999999999969 0.989849199114004
4.59999999999969 0.990048198133093
4.61999999999969 0.990243334375446
4.63999999999969 0.990434681327905
4.65999999999969 0.99062231113732
4.67999999999969 0.990806294632709
4.69999999999969 0.990986701347149
4.71999999999969 0.99116359953939
4.73999999999969 0.991337056215184
4.75999999999969 0.991507137148353
4.77999999999968 0.991673906901564
4.79999999999968 0.991837428846837
4.81999999999968 0.991997765185781
4.83999999999968 0.992154976969542
4.85999999999968 0.992309124118489
4.87999999999968 0.992460265441624
4.89999999999968 0.992608458655716
4.91999999999968 0.992753760404166
4.93999999999968 0.992896226275609
4.95999999999968 0.993035910822236
4.97999999999968 0.99317286757786
4.99999999999968 0.993307149075713
5.01999999999968 0.993438806865982
5.03999999999968 0.993567891533079
5.05999999999968 0.993694452712662
5.07999999999968 0.993818539108387
5.09999999999968 0.993940198508414
5.11999999999968 0.994059477801658
5.13999999999968 0.994176422993787
5.15999999999968 0.994291079222975
5.17999999999968 0.994403490775408
5.19999999999968 0.994513701100548
5.21999999999968 0.99462175282615
5.23999999999968 0.99472768777305
5.25999999999967 0.99483154696971
5.27999999999967 0.994933370666532
5.29999999999967 0.995033198349941
5.31999999999967 0.99513106875624
5.33999999999967 0.995227019885235
5.35999999999967 0.995321089013648
5.37999999999967 0.995413312708294
5.39999999999967 0.995503726839057
5.41999999999967 0.995592366591636
5.43999999999967 0.995679266480087
5.45999999999967 0.995764460359155
5.47999999999967 0.995847981436398
5.49999999999967 0.995929862284103
5.51999999999967 0.996010134851005
5.53999999999967 0.996088830473807
5.55999999999967 0.996165979888501
5.57999999999967 0.996241613241498
5.59999999999967 0.996315760100563
5.61999999999967 0.996388449465567
5.63999999999967 0.996459709779052
5.65999999999967 0.996529568936609
5.67999999999967 0.996598054297082
5.69999999999967 0.996665192692586
5.71999999999966 0.99673101043836
5.73999999999966 0.996795533342442
5.75999999999966 0.996858786715169
5.77999999999966 0.996920795378522
5.79999999999966 0.996981583675291
5.81999999999966 0.997041175478092
5.83999999999966 0.997099594198215
5.85999999999966 0.997156862794316
5.87999999999966 0.997213003780957
5.89999999999966 0.997268039236988
5.91999999999966 0.997321990813786
5.93999999999966 0.997374879743339
5.95999999999966 0.997426726846189
5.97999999999966 0.99747755253923
5.99999999999966 0.997527376843364
6.01999999999966 0.997576219391025
6.03999999999966 0.997624099433554
6.05999999999966 0.997671035848453
6.07999999999966 0.997717047146496
6.09999999999966 0.997762151478723
6.11999999999966 0.997806366643291
6.13999999999966 0.99784971009221
6.15999999999966 0.997892198937956
6.17999999999966 0.997933849959954
6.19999999999965 0.997974679610949
6.21999999999965 0.998014704023261
6.23999999999965 0.998053939014914
6.25999999999965 0.998092400095666
6.27999999999965 0.998130102472915
6.29999999999965 0.998167061057507
6.31999999999965 0.998203290469421
6.33999999999965 0.998238805043366
6.35999999999965 0.998273618834262
6.37999999999965 0.99830774562262
6.39999999999965 0.998341198919825
6.41999999999965 0.998373991973322
6.43999999999965 0.998406137771696
6.45999999999965 0.99843764904967
6.47999999999965 0.998468538292996
6.49999999999965 0.998498817743263
6.51999999999965 0.99852849940261
6.53999999999965 0.998557595038357
6.55999999999965 0.998586116187539
6.57999999999965 0.998614074161361
6.59999999999965 0.99864148004957
6.61999999999965 0.998668344724739
6.63999999999965 0.998694678846473
6.65999999999964 0.998720492865537
6.67999999999964 0.998745797027903
6.69999999999964 0.998770601378722
6.71999999999964 0.998794915766221
6.73999999999964 0.99881874984552
6.75999999999964 0.99884211308239
6.77999999999964 0.998865014756922
6.79999999999964 0.998887463967139
6.81999999999964 0.998909469632539
6.83999999999964 0.998931040497558
6.85999999999964 0.998952185134983
6.87999999999964 0.998972911949288
6.89999999999964 0.998993229179914
6.91999999999964 0.999013144904479
6.93999999999964 0.999032667041932
6.95999999999964 0.999051803355646
6.97999999999964 0.999070561456449
6.99999999999964 0.999088948805599
7.01999999999964 0.999106972717702
7.03999999999964 0.99912464036357
7.05999999999964 0.999141958773035
7.07999999999964 0.999158934837691
7.09999999999964 0.999175575313601
7.11999999999964 0.999191886823943
7.13999999999963 0.999207875861605
7.15999999999963 0.999223548791729
7.17999999999963 0.999238911854215
7.19999999999963 0.999253971166163
7.21999999999963 0.999268732724281
7.23999999999963 0.999283202407233
7.25999999999963 0.999297385977958
7.27999999999963 0.999311289085926
7.29999999999963 0.999324917269367
7.31999999999963 0.999338275957443
7.33999999999963 0.999351370472388
7.35999999999963 0.999364206031605
7.37999999999963 0.999376787749712
7.39999999999963 0.999389120640565
7.41999999999963 0.99940120961923
7.43999999999963 0.999413059503919
7.45999999999963 0.999424675017889
7.47999999999963 0.999436060791308
7.49999999999963 0.999447221363076
7.51999999999963 0.999458161182622
7.53999999999963 0.999468884611656
7.55999999999963 0.999479395925889
7.57999999999963 0.999489699316726
7.59999999999962 0.99949979889292
7.61999999999962 0.999509698682193
7.63999999999962 0.99951940263283
7.65999999999962 0.999528914615237
7.67999999999962 0.999538238423476
7.69999999999962 0.999547377776759
7.71999999999962 0.999556336320926
7.73999999999962 0.999565117629879
7.75999999999962 0.999573725207005
7.77999999999962 0.999582162486557
7.79999999999962 0.999590432835014
7.81999999999962 0.999598539552417
7.83999999999962 0.999606485873673
7.85999999999962 0.999614274969838
7.87999999999962 0.999621909949373
7.89999999999962 0.999629393859374
7.91999999999962 0.999636729686782
7.93999999999962 0.999643920359568
7.95999999999962 0.99965096874789
7.97999999999962 0.999657877665235
7.99999999999962 0.999664649869533
8.01999999999962 0.999671288064252
8.03999999999962 0.999677794899469
8.05999999999962 0.999684172972922
8.07999999999961 0.999690424831044
8.09999999999961 0.999696552969971
8.11999999999961 0.999702559836532
8.13999999999961 0.999708447829222
8.15999999999961 0.999714219299158
8.17999999999961 0.999719876551005
8.19999999999961 0.999725421843899
8.21999999999961 0.999730857392339
8.23999999999961 0.999736185367072
8.25999999999961 0.999741407895951
8.27999999999961 0.999746527064784
8.29999999999961 0.999751544918161
8.31999999999961 0.999756463460268
8.33999999999961 0.999761284655685
8.35999999999961 0.999766010430164
8.37999999999961 0.999770642671397
8.39999999999961 0.999775183229767
8.41999999999961 0.999779633919083
8.43999999999961 0.999783996517303
8.45999999999961 0.999788272767241
8.47999999999961 0.999792464377258
8.49999999999961 0.999796573021945
8.51999999999961 0.999800600342789
8.5399999999996 0.999804547948825
8.5599999999996 0.999808417417279
8.5799999999996 0.999812210294191
8.5999999999996 0.999815928095037
8.6199999999996 0.999819572305327
8.6399999999996 0.9998231443812
8.6599999999996 0.999826645750003
8.6799999999996 0.999830077810858
8.6999999999996 0.999833441935223
8.7199999999996 0.999836739467432
8.7399999999996 0.999839971725238
8.7599999999996 0.999843140000333
8.7799999999996 0.999846245558865
8.7999999999996 0.99984928964194
8.8199999999996 0.999852273466121
8.8399999999996 0.999855198223908
8.8599999999996 0.999858065084217
8.8799999999996 0.999860875192843
8.8999999999996 0.99986362967292
8.9199999999996 0.999866329625366
8.9399999999996 0.999868976129323
8.9599999999996 0.999871570242587
8.9799999999996 0.99987411300203
8.99999999999959 0.999876605424014
9.01999999999959 0.999879048504795
9.03999999999959 0.999881443220922
9.05999999999959 0.999883790529625
9.07999999999959 0.999886091369197
9.09999999999959 0.99988834665937
9.11999999999959 0.99989055730168
9.13999999999959 0.999892724179824
9.15999999999959 0.999894848160022
9.17999999999959 0.999896930091351
9.19999999999959 0.999898970806092
9.21999999999959 0.999900971120058
9.23999999999959 0.99990293183292
9.25999999999959 0.99990485372853
9.27999999999959 0.999906737575226
9.29999999999959 0.999908584126148
9.31999999999959 0.99991039411953
9.33999999999959 0.999912168279002
9.35999999999959 0.999913907313872
9.37999999999959 0.999915611919416
9.39999999999959 0.999917282777148
9.41999999999959 0.9999189205551
9.43999999999959 0.999920525908083
9.45999999999959 0.999922099477949
9.47999999999958 0.999923641893851
9.49999999999958 0.999925153772489
9.51999999999958 0.999926635718362
9.53999999999958 0.999928088324004
9.55999999999958 0.999929512170222
9.57999999999958 0.999930907826332
9.59999999999958 0.99993227585038
9.61999999999958 0.99993361678937
9.63999999999958 0.999934931179479
9.65999999999958 0.999936219546274
9.67999999999958 0.999937482404919
9.69999999999958 0.999938720260383
9.71999999999958 0.999939933607641
9.73999999999958 0.999941122931872
9.75999999999958 0.99994228870865
9.77999999999958 0.99994343140414
9.79999999999958 0.999944551475277
9.81999999999958 0.999945649369955
9.83999999999958 0.999946725527202
9.85999999999958 0.999947780377356
9.87999999999958 0.999948814342237
9.89999999999958 0.999949827835316
9.91999999999958 0.999950821261882
9.93999999999957 0.9999517950192
9.95999999999957 0.999952749496671
9.97999999999957 0.999953685075991
};
\end{axis}

\end{tikzpicture}
		\begin{equation*}
		\phi(x) = \frac{1}{1+\exp(-x)}
		\end{equation*}
		\caption{Sigmoid}
		\label{fig:sigmoid-activation}
	\end{subfigure}
	\caption[Common activation functions]{Plots and equations of common used activation functions. Where the Bias $b$ is the threshold value and $\lambda$ adds a small slope. Usually, the latter is very small like $\lambda=0.01$.}
	\label{fig:activation-functions}
\end{figure}