\subsubsection{Weight Initialization}
\label{sec:training-weight-initialization}
Before the actual training starts, the parameters, the weights and biases, of the network need to be initialized.
If this is done right, i.e. the values are in a range that supports training, optimization will be achieved in lesser or least time.
In the other case, a converging to optimal values can be impossible.
Reasons for this are the exploding or vanishing of gradients during backpropagation\cite{Hochreiter1991}.
In the backward-pass the gradients are computed for every layer and are passed from end to beginning using the chain rule.
For example, the derivative of the sigmoid function as it can be seen in \figref{fig:sigmoid-derivative} is in the range of $(0, 0.25]$.
If this is multiplied several times, the gradients at the beginning are way smaller than at the end.
If the weights are too small or too large, this effect is intensified.
This is partly true for other activation functions like the ReLU as well.
But here the gradients can become very large too, if the weights are really large.
None of these scenarios is desirable, because the optimal weights are either not reached or skipped.
\begin{figure}
	\setlength\figureheight{.4\textwidth}
	\setlength\figurewidth{.7\textwidth}
	\centering
	% This file was created by matplotlib2tikz v0.7.3.
\begin{tikzpicture}

\begin{axis}[
height=\figureheight,
legend cell align={left},
legend style={at={(0.03,0.97)}, anchor=north west, draw=white!80.0!black},
tick align=outside,
tick pos=left,
width=\figurewidth,
x grid style={white!90.01960784313725!black},
xlabel={x},
xmajorgrids,
xmin=-10.9995, xmax=10.9894999999996,
xtick style={color=black},
y grid style={white!90.01960784313725!black},
ylabel={y},
ymajorgrids,
ymin=-0.0499500416966799, ymax=1.04994958340047,
ytick style={color=black},
ytick={-0.2,0,0.2,0.4,0.6,0.8,1,1.2},
yticklabels={,0.0,0.2,0.4,0.6,0.8,1.0,}
]
\addplot [semithick, green!50.0!black]
table {%
-10 4.53978687024344e-05
-9.99 4.58541039469413e-05
-9.98 4.63149240092682e-05
-9.97 4.67803749590633e-05
-9.96 4.72505033288101e-05
-9.95 4.77253561184756e-05
-9.94 4.82049808002045e-05
-9.93 4.86894253230617e-05
-9.92 4.91787381178213e-05
-9.91 4.96729681018042e-05
-9.9 5.01721646837641e-05
-9.89 5.06763777688226e-05
-9.88 5.11856577634538e-05
-9.87 5.17000555805188e-05
-9.86 5.22196226443509e-05
-9.85 5.27444108958916e-05
-9.84 5.32744727978785e-05
-9.83 5.38098613400846e-05
-9.82 5.43506300446111e-05
-9.81 5.48968329712322e-05
-9.8 5.54485247227947e-05
-9.79 5.60057604506708e-05
-9.78 5.65685958602659e-05
-9.77 5.71370872165821e-05
-9.76000000000001 5.77112913498368e-05
-9.75000000000001 5.82912656611383e-05
-9.74000000000001 5.88770681282178e-05
-9.73000000000001 5.94687573112193e-05
-9.72000000000001 6.00663923585472e-05
-9.71000000000001 6.06700330127733e-05
-9.70000000000001 6.1279739616602e-05
-9.69000000000001 6.18955731188962e-05
-9.68000000000001 6.2517595080763e-05
-9.67000000000001 6.31458676817011e-05
-9.66000000000001 6.37804537258095e-05
-9.65000000000001 6.44214166480583e-05
-9.64000000000001 6.50688205206224e-05
-9.63000000000001 6.57227300592795e-05
-9.62000000000001 6.63832106298714e-05
-9.61000000000001 6.705032825483e-05
-9.60000000000001 6.77241496197696e-05
-9.59000000000001 6.84047420801449e-05
-9.58000000000001 6.90921736679753e-05
-9.57000000000001 6.97865130986373e-05
-9.56000000000001 7.04878297777246e-05
-9.55000000000001 7.11961938079774e-05
-9.54000000000001 7.19116759962808e-05
-9.53000000000001 7.26343478607335e-05
-9.52000000000001 7.33642816377872e-05
-9.51000000000001 7.41015502894582e-05
-9.50000000000001 7.48462275106104e-05
-9.49000000000001 7.55983877363123e-05
-9.48000000000001 7.63581061492668e-05
-9.47000000000001 7.71254586873163e-05
-9.46000000000001 7.79005220510224e-05
-9.45000000000001 7.86833737113223e-05
-9.44000000000001 7.9474091917261e-05
-9.43000000000001 8.02727557038022e-05
-9.42000000000001 8.10794448997163e-05
-9.41000000000001 8.18942401355483e-05
-9.40000000000001 8.27172228516654e-05
-9.39000000000001 8.35484753063848e-05
-9.38000000000001 8.4388080584184e-05
-9.37000000000001 8.5236122603992e-05
-9.36000000000001 8.60926861275648e-05
-9.35000000000001 8.69578567679443e-05
-9.34000000000001 8.78317209980021e-05
-9.33000000000001 8.87143661590688e-05
-9.32000000000001 8.96058804696499e-05
-9.31000000000001 9.05063530342293e-05
-9.30000000000001 9.14158738521601e-05
-9.29000000000002 9.23345338266462e-05
-9.28000000000002 9.32624247738116e-05
-9.27000000000002 9.4199639431863e-05
-9.26000000000002 9.51462714703421e-05
-9.25000000000002 9.61024154994724e-05
-9.24000000000002 9.70681670795981e-05
-9.23000000000002 9.80436227307186e-05
-9.22000000000002 9.90288799421182e-05
-9.21000000000002 0.000100024037182092
-9.20000000000002 0.000101029193907771
-9.19000000000002 0.000102044450575041
-9.18000000000002 0.000103069908648568
-9.17000000000002 0.000104105670611915
-9.16000000000002 0.000105151839977772
-9.15000000000002 0.000106208521298275
-9.14000000000002 0.000107275820175438
-9.13000000000002 0.000108353843271689
-9.12000000000002 0.000109442698320503
-9.11000000000002 0.000110542494137153
-9.10000000000002 0.000111653340629561
-9.09000000000002 0.000112775348809258
-9.08000000000002 0.00011390863080246
-9.07000000000002 0.000115053299861247
-9.06000000000002 0.000116209470374859
-9.05000000000002 0.000117377257881102
-9.04000000000002 0.000118556779077874
-9.03000000000002 0.000119748151834796
-9.02000000000002 0.00012095149520497
-9.01000000000002 0.000122166929436849
-9.00000000000002 0.000123394575986229
-8.99000000000002 0.000124634557528356
-8.98000000000002 0.000125886997970159
-8.97000000000002 0.000127152022462606
-8.96000000000002 0.000128429757413177
-8.95000000000002 0.000129720330498472
-8.94000000000002 0.000131023870676934
-8.93000000000002 0.00013234050820171
-8.92000000000002 0.000133670374633632
-8.91000000000002 0.000135013602854333
-8.90000000000002 0.000136370327079494
-8.89000000000002 0.00013774068287222
-8.88000000000002 0.000139124807156555
-8.87000000000002 0.000140522838231126
-8.86000000000002 0.000141934915782931
-8.85000000000002 0.000143361180901257
-8.84000000000002 0.000144801776091746
-8.83000000000002 0.000146256845290592
-8.82000000000003 0.000147726533878887
-8.81000000000003 0.000149210988697109
-8.80000000000003 0.000150710358059754
-8.79000000000003 0.000152224791770114
-8.78000000000003 0.000153754441135206
-8.77000000000003 0.000155299458980844
-8.76000000000003 0.000156859999666868
-8.75000000000003 0.000158436219102522
-8.74000000000003 0.000160028274761988
-8.73000000000003 0.000161636325700073
-8.72000000000003 0.000163260532568052
-8.71000000000003 0.000164901057629677
-8.70000000000003 0.000166558064777331
-8.69000000000003 0.000168231719548363
-8.68000000000003 0.000169922189141569
-8.67000000000003 0.000171629642433847
-8.66000000000003 0.000173354249997016
-8.65000000000003 0.000175096184114805
-8.64000000000003 0.000176855618800008
-8.63000000000003 0.000178632729811813
-8.62000000000003 0.000180427694673307
-8.61000000000003 0.000182240692689147
-8.60000000000003 0.000184071904963418
-8.59000000000003 0.000185921514417664
-8.58000000000003 0.000187789705809097
-8.57000000000003 0.000189676665748995
-8.56000000000003 0.000191582582721276
-8.55000000000003 0.000193507647101265
-8.54000000000003 0.000195452051174638
-8.53000000000003 0.000197415989156572
-8.52000000000003 0.000199399657211064
-8.51000000000003 0.000201403253470467
-8.50000000000003 0.0002034269780552
-8.49000000000003 0.000205471033093667
-8.48000000000003 0.000207535622742374
-8.47000000000003 0.000209620953206242
-8.46000000000003 0.000211727232759125
-8.45000000000003 0.000213854671764536
-8.44000000000003 0.000216003482696574
-8.43000000000003 0.000218173880161067
-8.42000000000003 0.000220366080916919
-8.41000000000003 0.000222580303897672
-8.40000000000003 0.000224816770233288
-8.39000000000003 0.000227075703272142
-8.38000000000003 0.000229357328603239
-8.37000000000003 0.000231661874078649
-8.36000000000003 0.000233989569836168
-8.35000000000004 0.000236340648322206
-8.34000000000004 0.000238715344314899
-8.33000000000004 0.000241113894947457
-8.32000000000004 0.000243536539731744
-8.31000000000004 0.000245983520582085
-8.30000000000004 0.000248455081839325
-8.29000000000004 0.000250951470295114
-8.28000000000004 0.000253472935216442
-8.27000000000004 0.000256019728370415
-8.26000000000004 0.000258592104049282
-8.25000000000004 0.00026119031909571
-8.24000000000004 0.000263814632928303
-8.23000000000004 0.000266465307567389
-8.22000000000004 0.000269142607661057
-8.21000000000004 0.000271846800511449
-8.20000000000004 0.000274578156101322
-8.19000000000004 0.000277336947120871
-8.18000000000004 0.000280123448994817
-8.17000000000004 0.000282937939909772
-8.16000000000004 0.000285780700841868
-8.15000000000004 0.000288652015584664
-8.14000000000004 0.000291552170777337
-8.13000000000004 0.000294481455933143
-8.12000000000004 0.000297440163468171
-8.11000000000004 0.000300428588730376
-8.10000000000004 0.000303447030028907
-8.09000000000004 0.000306495788663722
-8.08000000000004 0.000309575168955503
-8.07000000000004 0.000312685478275861
-8.06000000000004 0.00031582702707785
-8.05000000000004 0.000319000128926778
-8.04000000000004 0.00032220510053133
-8.03000000000004 0.000325442261774996
-8.02000000000004 0.000328711935747814
-8.01000000000004 0.000332014448778428
-8.00000000000004 0.000335350130466464
-7.99000000000004 0.000338719313715231
-7.98000000000004 0.000342122334764745
-7.97000000000004 0.000345559533225079
-7.96000000000004 0.000349031252110046
-7.95000000000004 0.000352537837871223
-7.94000000000004 0.000356079640432303
-7.93000000000004 0.000359657013223794
-7.92000000000004 0.000363270313218062
-7.91000000000004 0.000366919900964721
-7.90000000000004 0.00037060614062638
-7.89000000000004 0.000374329400014733
-7.88000000000005 0.000378090050627021
-7.87000000000005 0.000381888467682847
-7.86000000000005 0.00038572503016136
-7.85000000000005 0.000389600120838808
-7.84000000000005 0.000393514126326465
-7.83000000000005 0.000397467437108928
-7.82000000000005 0.000401460447582802
-7.81000000000005 0.000405493556095768
-7.80000000000005 0.000409567164986031
-7.79000000000005 0.000413681680622168
-7.78000000000005 0.000417837513443365
-7.77000000000005 0.00042203507800006
-7.76000000000005 0.00042627479299498
-7.75000000000005 0.000430557081324594
-7.74000000000005 0.000434882370120969
-7.73000000000005 0.000439251090794044
-7.72000000000005 0.000443663679074329
-7.71000000000005 0.000448120575056012
-7.70000000000005 0.000452622223240513
-7.69000000000005 0.000457169072580449
-7.68000000000005 0.000461761576524052
-7.67000000000005 0.000466400193060015
-7.66000000000005 0.000471085384762786
-7.65000000000005 0.000475817618838312
-7.64000000000005 0.000480597367170237
-7.63000000000005 0.000485425106366547
-7.62000000000005 0.000490301317806692
-7.61000000000005 0.000495226487689162
-7.60000000000005 0.000500201107079539
-7.59000000000005 0.000505225671959023
-7.58000000000005 0.000510300683273438
-7.57000000000005 0.000515426646982725
-7.56000000000005 0.000520604074110915
-7.55000000000005 0.000525833480796612
-7.54000000000005 0.000531115388343959
-7.53000000000005 0.000536450323274114
-7.52000000000005 0.000541838817377241
-7.51000000000005 0.000547281407765
-7.50000000000005 0.00055277863692357
-7.49000000000005 0.000558331052767186
-7.48000000000005 0.000563939208692207
-7.47000000000005 0.000569603663631721
-7.46000000000005 0.000575324982110682
-7.45000000000005 0.0005811037343016
-7.44000000000005 0.000586940496080771
-7.43000000000005 0.000592835849085066
-7.42000000000005 0.000598790380769285
-7.41000000000006 0.000604804684464065
-7.40000000000006 0.000610879359434367
-7.39000000000006 0.000617015010938541
-7.38000000000006 0.000623212250287964
-7.37000000000006 0.000629471694907274
-7.36000000000006 0.000635793968395193
-7.35000000000006 0.000642179700585949
-7.34000000000006 0.000648629527611308
-7.33000000000006 0.000655144091963206
-7.32000000000006 0.000661724042557007
-7.31000000000006 0.000668370034795379
-7.30000000000006 0.000675082730632799
-7.29000000000006 0.000681862798640694
-7.28000000000006 0.000688710914073216
-7.27000000000006 0.000695627758933676
-7.26000000000006 0.000702614022041614
-7.25000000000006 0.000709670399100547
-7.24000000000006 0.000716797592766362
-7.23000000000006 0.000723996312716402
-7.22000000000006 0.00073126727571921
-7.21000000000006 0.000738611205704974
-7.20000000000006 0.000746028833836653
-7.19000000000006 0.000753520898581804
-7.18000000000006 0.000761088145785115
-7.17000000000006 0.000768731328741647
-7.16000000000006 0.000776451208270797
-7.15000000000006 0.000784248552790979
-7.14000000000006 0.000792124138395046
-7.13000000000006 0.000800078748926446
-7.12000000000006 0.000808113176056122
-7.11000000000006 0.000816228219360168
-7.10000000000006 0.000824424686398244
-7.09000000000006 0.000832703392792759
-7.08000000000006 0.000841065162308828
-7.07000000000006 0.000849510826935011
-7.06000000000006 0.000858041226964839
-7.05000000000006 0.000866657211079143
-7.04000000000006 0.000875359636429182
-7.03000000000006 0.000884149368720584
-7.02000000000006 0.000893027282298109
-7.01000000000006 0.000901994260231235
-7.00000000000006 0.000911051194400587
-6.99000000000006 0.000920198985585201
-6.98000000000006 0.000929438543550642
-6.97000000000006 0.000938770787137987
-6.96000000000006 0.000948196644353664
-6.95000000000007 0.000957717052460177
-6.94000000000007 0.00096733295806771
-6.93000000000007 0.000977045317226621
-6.92000000000007 0.000986855095520843
-6.91000000000007 0.000996763268162188
-6.90000000000007 0.00100677082008557
-6.89000000000007 0.00101687874604516
-6.88000000000007 0.00102708805071145
-6.87000000000007 0.00103739974876932
-6.86000000000007 0.00104781486501699
-6.85000000000007 0.00105833443446594
-6.84000000000007 0.00106895950244189
-6.83000000000007 0.00107969112468662
-6.82000000000007 0.0010905303674609
-6.81000000000007 0.00110147830764833
-6.80000000000007 0.00111253603286025
-6.79000000000007 0.00112370464154163
-6.78000000000007 0.00113498524307802
-6.77000000000007 0.00114637895790346
-6.76000000000007 0.00115788691760954
-6.75000000000007 0.00116951026505543
-6.74000000000007 0.00118125015447904
-6.73000000000007 0.00119310775160918
-6.72000000000007 0.00120508423377886
-6.71000000000007 0.00121718079003969
-6.70000000000007 0.00122939862127733
-6.69000000000007 0.00124173894032813
-6.68000000000007 0.0012542029720968
-6.67000000000007 0.00126679195367532
-6.66000000000007 0.00127950713446292
-6.65000000000007 0.00129234977628725
-6.64000000000007 0.00130532115352668
-6.63000000000007 0.00131842255323385
-6.62000000000007 0.00133165527526034
-6.61000000000007 0.00134502063238256
-6.60000000000007 0.00135851995042886
-6.59000000000007 0.00137215456840785
-6.58000000000007 0.00138592583863798
-6.57000000000007 0.0013998351268783
-6.56000000000007 0.00141388381246055
-6.55000000000007 0.00142807328842247
-6.54000000000007 0.00144240496164241
-6.53000000000007 0.00145688025297522
-6.52000000000007 0.00147150059738942
-6.51000000000007 0.00148626744410572
-6.50000000000007 0.00150118225673688
-6.49000000000007 0.00151624651342882
-6.48000000000008 0.00153146170700317
-6.47000000000008 0.00154682934510116
-6.46000000000008 0.00156235095032884
-6.45000000000008 0.00157802806040374
-6.44000000000008 0.00159386222830289
-6.43000000000008 0.00160985502241227
-6.42000000000008 0.00162600802667769
-6.41000000000008 0.00164232284075709
-6.40000000000008 0.00165880108017429
-6.39000000000008 0.00167544437647422
-6.38000000000008 0.00169225437737958
-6.37000000000008 0.00170923274694906
-6.36000000000008 0.00172638116573702
-6.35000000000008 0.00174370133095464
-6.34000000000008 0.00176119495663271
-6.33000000000008 0.00177886377378583
-6.32000000000008 0.0017967095305783
-6.31000000000008 0.00181473399249146
-6.30000000000008 0.00183293894249266
-6.29000000000008 0.00185132618120589
-6.28000000000008 0.0018698975270839
-6.27000000000008 0.00188865481658203
-6.26000000000008 0.00190759990433367
-6.25000000000008 0.00192673466332732
-6.24000000000008 0.0019460609850854
-6.23000000000008 0.00196558077984468
-6.22000000000008 0.00198529597673842
-6.21000000000008 0.00200520852398024
-6.20000000000008 0.00202532038904972
-6.19000000000008 0.00204563355887968
-6.18000000000008 0.00206615004004532
-6.17000000000008 0.00208687185895503
-6.16000000000008 0.00210780106204308
-6.15000000000008 0.00212893971596402
-6.14000000000008 0.00215028990778896
-6.13000000000008 0.00217185374520368
-6.12000000000008 0.00219363335670856
-6.11000000000008 0.00221563089182042
-6.10000000000008 0.00223784852127615
-6.09000000000008 0.00226028843723836
-6.08000000000008 0.00228295285350287
-6.07000000000008 0.00230584400570812
-6.06000000000008 0.00232896415154654
-6.05000000000008 0.00235231557097795
-6.04000000000008 0.00237590056644479
-6.03000000000008 0.00239972146308948
-6.02000000000008 0.00242378060897376
-6.01000000000009 0.00244808037529998
-6.00000000000009 0.00247262315663456
-5.99000000000009 0.00249741137113341
-5.98000000000009 0.00252244746076948
-5.97000000000009 0.00254773389156238
-5.96000000000009 0.00257327315381015
-5.95000000000009 0.00259906776232312
-5.94000000000009 0.00262512025665997
-5.93000000000009 0.00265143320136586
-5.92000000000009 0.00267800918621288
-5.91000000000009 0.00270485082644259
-5.90000000000009 0.00273196076301082
-5.89000000000009 0.00275934166283472
-5.88000000000009 0.00278699621904204
-5.87000000000009 0.00281492715122271
-5.86000000000009 0.00284313720568269
-5.85000000000009 0.00287162915570015
-5.84000000000009 0.00290040580178394
-5.83000000000009 0.00292946997193449
-5.82000000000009 0.00295882452190697
-5.81000000000009 0.00298847233547689
-5.80000000000009 0.00301841632470815
-5.79000000000009 0.00304865943022343
-5.78000000000009 0.00307920462147702
-5.77000000000009 0.00311005489703024
-5.76000000000009 0.00314121328482914
-5.75000000000009 0.0031726828424849
-5.74000000000009 0.00320446665755659
-5.73000000000009 0.00323656784783655
-5.72000000000009 0.00326898956163832
-5.71000000000009 0.00330173497808715
-5.70000000000009 0.00333480730741304
-5.69000000000009 0.00336820979124652
-5.68000000000009 0.00340194570291695
-5.67000000000009 0.00343601834775354
-5.66000000000009 0.00347043106338901
-5.65000000000009 0.00350518722006602
-5.64000000000009 0.00354029022094618
-5.63000000000009 0.00357574350242196
-5.62000000000009 0.00361155053443121
-5.61000000000009 0.00364771482077459
-5.60000000000009 0.00368423989943564
-5.59000000000009 0.00372112934290386
-5.58000000000009 0.00375838675850045
-5.57000000000009 0.003796015788707
-5.56000000000009 0.00383402011149708
-5.55000000000009 0.00387240344067067
-5.5400000000001 0.0039111695261915
-5.5300000000001 0.00395032215452744
-5.5200000000001 0.00398986514899372
-5.5100000000001 0.00402980237009923
-5.5000000000001 0.00407013771589574
-5.4900000000001 0.00411087512233024
-5.4800000000001 0.00415201856360025
-5.4700000000001 0.00419357205251221
-5.4600000000001 0.00423553964084302
-5.4500000000001 0.00427792541970456
-5.4400000000001 0.0043207335199115
-5.4300000000001 0.00436396811235212
-5.4200000000001 0.00440763340836238
-5.4100000000001 0.00445173366010317
-5.4000000000001 0.00449627316094074
-5.3900000000001 0.00454125624583043
-5.3800000000001 0.00458668729170357
-5.3700000000001 0.00463257071785773
-5.3600000000001 0.00467891098635024
-5.3500000000001 0.00472571260239501
-5.3400000000001 0.00477298011476273
-5.3300000000001 0.00482071811618437
-5.3200000000001 0.00486893124375816
-5.3100000000001 0.00491762417935984
-5.3000000000001 0.00496680165005647
-5.2900000000001 0.00501646842852355
-5.2800000000001 0.00506662933346575
-5.2700000000001 0.00511728923004097
-5.2600000000001 0.00516845303028805
-5.2500000000001 0.00522012569355787
-5.2400000000001 0.00527231222694814
-5.2300000000001 0.00532501768574163
-5.2200000000001 0.00537824717384806
-5.2100000000001 0.00543200584424957
-5.2000000000001 0.00548629889944985
-5.1900000000001 0.00554113159192684
-5.1800000000001 0.00559650922458923
-5.1700000000001 0.0056524371512365
-5.1600000000001 0.00570892077702276
-5.1500000000001 0.00576596555892431
-5.1400000000001 0.00582357700621092
-5.1300000000001 0.00588176068092085
-5.1200000000001 0.00594052219833976
-5.1100000000001 0.00599986722748326
-5.1000000000001 0.00605980149158349
-5.0900000000001 0.00612033076857929
-5.0800000000001 0.0061814608916105
-5.07000000000011 0.0062431977495159
-5.06000000000011 0.0063055472873352
-5.05000000000011 0.00636851550681488
-5.04000000000011 0.00643210846691796
-5.03000000000011 0.00649633228433775
-5.02000000000011 0.00656119313401553
-5.01000000000011 0.00662669724966224
-5.00000000000011 0.00669285092428415
-4.99000000000011 0.00675966051071253
-4.98000000000011 0.00682713242213744
-4.97000000000011 0.00689527313264541
-4.96000000000011 0.00696408917776135
-4.95000000000011 0.00703358715499441
-4.94000000000011 0.00710377372438802
-4.93000000000011 0.00717465560907397
-4.92000000000011 0.00724623959583065
-4.91000000000011 0.00731853253564544
-4.90000000000011 0.00739154134428118
-4.89000000000011 0.00746527300284688
-4.88000000000011 0.00753973455837258
-4.87000000000011 0.00761493312438833
-4.86000000000011 0.0076908758815075
-4.85000000000011 0.00776757007801415
-4.84000000000011 0.00784502303045478
-4.83000000000011 0.00792324212423412
-4.82000000000011 0.00800223481421537
-4.81000000000011 0.00808200862532451
-4.80000000000011 0.008162571153159
-4.79000000000011 0.00824393006460066
-4.78000000000011 0.00832609309843287
-4.77000000000011 0.00840906806596205
-4.76000000000011 0.00849286285164341
-4.75000000000011 0.00857748541371103
-4.74000000000011 0.0086629437848122
-4.73000000000011 0.00874924607264609
-4.72000000000011 0.00883640046060673
-4.71000000000011 0.00892441520843032
-4.70000000000011 0.00901329865284682
-4.69000000000011 0.00910305920823586
-4.68000000000011 0.00919370536728706
-4.67000000000011 0.00928524570166453
-4.66000000000011 0.0093776888626758
-4.65000000000011 0.00947104358194504
-4.64000000000011 0.00956531867209059
-4.63000000000011 0.00966052302740681
-4.62000000000011 0.00975666562455025
-4.61000000000011 0.00985375552323015
-4.60000000000012 0.00995180186690319
-4.59000000000012 0.0100508138834726
-4.58000000000012 0.0101508008859916
-4.57000000000012 0.0102517722733708
-4.56000000000012 0.0103537375310906
-4.55000000000012 0.0104567062319169
-4.54000000000012 0.0105606880366219
-4.53000000000012 0.0106656926947088
-4.52000000000012 0.0107717300451404
-4.51000000000012 0.0108788100170727
-4.50000000000012 0.0109869426305919
-4.49000000000012 0.0110961379974563
-4.48000000000012 0.0112064063218416
-4.47000000000012 0.011317757901091
-4.46000000000012 0.0114302031264694
-4.45000000000012 0.0115437524839209
-4.44000000000012 0.0116584165548317
-4.43000000000012 0.0117742060167958
-4.42000000000012 0.0118911316443856
-4.41000000000012 0.012009204309926
-4.40000000000012 0.0121284349842728
-4.39000000000012 0.0122488347375951
-4.38000000000012 0.012370414740161
-4.37000000000012 0.0124931862631281
-4.36000000000012 0.0126171606793372
-4.35000000000012 0.0127423494641101
-4.34000000000012 0.012868764196051
-4.33000000000012 0.0129964165578518
-4.32000000000012 0.0131253183371012
-4.31000000000012 0.0132554814270972
-4.30000000000012 0.0133869178276632
-4.29000000000012 0.013519639645968
-4.28000000000012 0.0136536590973492
-4.27000000000012 0.0137889885061399
-4.26000000000012 0.0139256403064987
-4.25000000000012 0.0140636270432438
-4.24000000000012 0.0142029613726894
-4.23000000000012 0.0143436560634864
-4.22000000000012 0.0144857239974651
-4.21000000000012 0.0146291781704825
-4.20000000000012 0.0147740316932713
-4.19000000000012 0.0149202977922924
-4.18000000000012 0.015067989810591
-4.17000000000012 0.0152171212086544
-4.16000000000012 0.0153677055652737
-4.15000000000012 0.015519756578407
-4.14000000000012 0.0156732880660466
-4.13000000000013 0.0158283139670879
-4.12000000000013 0.0159848483422006
-4.11000000000013 0.0161429053747032
-4.10000000000013 0.0163024993714389
-4.09000000000013 0.0164636447636542
-4.08000000000013 0.0166263561078796
-4.07000000000013 0.0167906480868117
-4.06000000000013 0.0169565355101987
-4.05000000000013 0.0171240333157256
-4.04000000000013 0.0172931565699033
-4.03000000000013 0.0174639204689575
-4.02000000000013 0.0176363403397205
-4.01000000000013 0.0178104316405232
-4.00000000000013 0.0179862099620893
-3.99000000000013 0.0181636910284302
-3.98000000000013 0.018342890697741
-3.97000000000013 0.0185238249632971
-3.96000000000013 0.0187065099543522
-3.95000000000013 0.0188909619370367
-3.94000000000013 0.0190771973152557
-3.93000000000013 0.0192652326315892
-3.92000000000013 0.0194550845681906
-3.91000000000013 0.0196467699476862
-3.90000000000013 0.019840305734075
-3.89000000000013 0.0200357090336271
-3.88000000000013 0.0202329970957828
-3.87000000000013 0.0204321873140504
-3.86000000000013 0.0206332972269036
-3.85000000000013 0.0208363445186778
-3.84000000000013 0.0210413470204656
-3.83000000000013 0.0212483227110108
-3.82000000000013 0.0214572897176008
-3.81000000000013 0.0216682663169577
-3.80000000000013 0.0218812709361276
-3.79000000000013 0.0220963221533673
-3.78000000000013 0.0223134386990293
-3.77000000000013 0.0225326394564444
-3.76000000000013 0.022753943462801
-3.75000000000013 0.0229773699100226
-3.74000000000013 0.0232029381456411
-3.73000000000013 0.0234306676736673
-3.72000000000013 0.0236605781554581
-3.71000000000013 0.0238926894105796
-3.70000000000013 0.024127021417666
-3.69000000000013 0.0243635943152746
-3.68000000000013 0.0246024284027362
-3.67000000000013 0.0248435441410004
-3.66000000000014 0.0250869621534765
-3.65000000000014 0.0253327032268684
-3.64000000000014 0.0255807883120043
-3.63000000000014 0.0258312385246606
-3.62000000000014 0.0260840751463795
-3.61000000000014 0.0263393196252799
-3.60000000000014 0.0265969935768623
-3.59000000000014 0.0268571187848061
-3.58000000000014 0.0271197172017594
-3.57000000000014 0.027384810950122
-3.56000000000014 0.0276524223228194
-3.55000000000014 0.0279225737840693
-3.54000000000014 0.0281952879701388
-3.53000000000014 0.0284705876900934
-3.52000000000014 0.0287484959265361
-3.51000000000014 0.0290290358363369
-3.50000000000014 0.0293122307513524
-3.49000000000014 0.0295981041791352
-3.48000000000014 0.0298866798036322
-3.47000000000014 0.0301779814858719
-3.46000000000014 0.03047203326464
-3.45000000000014 0.0307688593571438
-3.44000000000014 0.0310684841596632
-3.43000000000014 0.0313709322481897
-3.42000000000014 0.0316762283790521
-3.41000000000014 0.0319843974895289
-3.40000000000014 0.0322954646984461
-3.39000000000014 0.0326094553067611
-3.38000000000014 0.0329263947981318
-3.37000000000014 0.0332463088394696
-3.36000000000014 0.0335692232814779
-3.35000000000014 0.0338951641591735
-3.34000000000014 0.0342241576923912
-3.33000000000014 0.0345562302862718
-3.32000000000014 0.034891408531732
-3.31000000000014 0.0352297192059166
-3.30000000000014 0.0355711892726313
-3.29000000000014 0.0359158458827568
-3.28000000000014 0.0362637163746433
-3.27000000000014 0.0366148282744848
-3.26000000000014 0.0369692092966719
-3.25000000000014 0.0373268873441243
-3.24000000000014 0.0376878905086007
-3.23000000000014 0.0380522470709868
-3.22000000000014 0.03841998550156
-3.21000000000014 0.0387911344602306
-3.20000000000014 0.0391657227967589
-3.19000000000015 0.0395437795509474
-3.18000000000015 0.0399253339528082
-3.17000000000015 0.0403104154227036
-3.16000000000015 0.0406990535714608
-3.15000000000015 0.0410912782004593
-3.14000000000015 0.04148711930169
-3.13000000000015 0.0418866070577868
-3.12000000000015 0.0422897718420279
-3.11000000000015 0.0426966442183079
-3.10000000000015 0.0431072549410801
-3.09000000000015 0.0435216349552661
-3.08000000000015 0.0439398153961351
-3.07000000000015 0.0443618275891501
-3.06000000000015 0.0447877030497804
-3.05000000000015 0.0452174734832811
-3.04000000000015 0.0456511707844373
-3.03000000000015 0.0460888270372731
-3.02000000000015 0.0465304745147249
-3.01000000000015 0.0469761456782769
-3.00000000000015 0.04742587317756
-2.99000000000015 0.0478796898499121
-2.98000000000015 0.0483376287198983
-2.97000000000015 0.048799722998793
-2.96000000000015 0.0492660060840196
-2.95000000000015 0.0497365115585496
-2.94000000000015 0.0502112731902594
-2.93000000000015 0.0506903249312434
-2.92000000000015 0.0511737009170843
-2.91000000000015 0.0516614354660774
-2.90000000000015 0.0521535630784103
-2.89000000000015 0.0526501184352954
-2.88000000000015 0.0531511363980561
-2.87000000000015 0.0536566520071642
-2.86000000000015 0.0541667004812283
-2.85000000000015 0.0546813172159329
-2.84000000000015 0.0552005377829263
-2.83000000000015 0.0557243979286574
-2.82000000000015 0.0562529335731592
-2.81000000000015 0.0567861808087803
-2.80000000000015 0.0573241758988604
-2.79000000000015 0.0578669552763526
-2.78000000000015 0.0584145555423878
-2.77000000000015 0.0589670134647832
-2.76000000000015 0.0595243659764929
-2.75000000000015 0.0600866501739989
-2.74000000000015 0.0606539033156433
-2.73000000000015 0.0612261628198991
-2.72000000000016 0.0618034662635796
-2.71000000000016 0.0623858513799853
-2.70000000000016 0.0629733560569873
-2.69000000000016 0.063566018335046
-2.68000000000016 0.0641638764051646
-2.67000000000016 0.0647669686067761
-2.66000000000016 0.0653753334255631
-2.65000000000016 0.0659890094912091
-2.64000000000016 0.0666080355750809
-2.63000000000016 0.0672324505878402
-2.62000000000016 0.0678622935769845
-2.61000000000016 0.068497603724316
-2.60000000000016 0.0691384203433367
-2.59000000000016 0.0697847828765698
-2.58000000000016 0.0704367308928067
-2.57000000000016 0.0710943040842767
-2.56000000000016 0.0717575422637408
-2.55000000000016 0.072426485361507
-2.54000000000016 0.0731011734223674
-2.53000000000016 0.0737816466024544
-2.52000000000016 0.0744679451660171
-2.51000000000016 0.0751601094821155
-2.50000000000016 0.0758581800212323
-2.49000000000016 0.0765621973518009
-2.48000000000016 0.0772722021366485
-2.47000000000016 0.0779882351293548
-2.46000000000016 0.0787103371705236
-2.45000000000016 0.0794385491839666
-2.44000000000016 0.0801729121728004
-2.43000000000016 0.0809134672154534
-2.42000000000016 0.0816602554615825
-2.41000000000016 0.0824133181279006
-2.40000000000016 0.08317269649391
-2.39000000000016 0.0839384318975457
-2.38000000000016 0.0847105657307232
-2.37000000000016 0.0854891394347938
-2.36000000000016 0.0862741944959039
-2.35000000000016 0.0870657724402583
-2.34000000000016 0.0878639148292882
-2.33000000000016 0.0886686632547201
-2.32000000000016 0.0894800593335481
-2.31000000000016 0.0902981447029064
-2.30000000000016 0.0911229610148425
-2.29000000000016 0.0919545499309906
-2.28000000000016 0.0927929531171432
-2.27000000000016 0.0936382122377217
-2.26000000000016 0.0944903689501452
-2.25000000000017 0.0953494648990953
-2.24000000000017 0.0962155417106785
-2.23000000000017 0.0970886409864844
-2.22000000000017 0.0979688042975393
-2.21000000000017 0.0988560731781546
-2.20000000000017 0.0997504891196702
-2.19000000000017 0.100652093564092
-2.18000000000017 0.101560927897621
-2.17000000000017 0.102477033444082
-2.16000000000017 0.103400451458234
-2.15000000000017 0.104331223118986
-2.14000000000017 0.105269389522494
-2.13000000000017 0.10621499167516
-2.12000000000017 0.107168070486512
-2.11000000000017 0.108128666761987
-2.10000000000017 0.109096821195597
-2.09000000000017 0.110072574362486
-2.08000000000017 0.111055966711391
-2.07000000000017 0.112047038556973
-2.06000000000017 0.113045830072062
-2.05000000000017 0.114052381279774
-2.04000000000017 0.115066732045533
-2.03000000000017 0.116088922068977
-2.02000000000017 0.117118990875763
-2.01000000000017 0.118156977809252
-2.00000000000017 0.1192029220221
-1.99000000000017 0.120256862467733
-1.98000000000017 0.121318837891719
-1.97000000000017 0.12238888682303
-1.96000000000017 0.123467047565205
-1.95000000000017 0.124553358187398
-1.94000000000017 0.125647856515327
-1.93000000000017 0.126750580122122
-1.92000000000017 0.127861566319062
-1.91000000000017 0.128980852146216
-1.90000000000017 0.130108474362978
-1.89000000000017 0.131244469438504
-1.88000000000017 0.132388873542046
-1.87000000000017 0.133541722533192
-1.86000000000017 0.134703051952008
-1.85000000000017 0.135872897009074
-1.84000000000017 0.137051292575439
-1.83000000000017 0.138238273172473
-1.82000000000017 0.139433872961629
-1.81000000000017 0.140638125734113
-1.80000000000017 0.141851064900467
-1.79000000000018 0.143072723480059
-1.78000000000018 0.144303134090497
-1.77000000000018 0.145542328936943
-1.76000000000018 0.14679033980136
-1.75000000000018 0.148047198031667
-1.74000000000018 0.149312934530821
-1.73000000000018 0.150587579745821
-1.72000000000018 0.151871163656637
-1.71000000000018 0.153163715765063
-1.70000000000018 0.154465265083512
-1.69000000000018 0.155775840123724
-1.68000000000018 0.157095468885429
-1.67000000000018 0.158424178844932
-1.66000000000018 0.159761996943645
-1.65000000000018 0.161108949576561
-1.64000000000018 0.162465062580672
-1.63000000000018 0.163830361223334
-1.62000000000018 0.16520487019059
-1.61000000000018 0.166588613575435
-1.60000000000018 0.167981614866051
-1.59000000000018 0.169383896933993
-1.58000000000018 0.170795482022349
-1.57000000000018 0.172216391733852
-1.56000000000018 0.173646647018979
-1.55000000000018 0.175086268164014
-1.54000000000018 0.176535274779091
-1.53000000000018 0.17799368578622
-1.52000000000018 0.1794615194073
-1.51000000000018 0.180938793152115
-1.50000000000018 0.182425523806329
-1.49000000000018 0.183921727419477
-1.48000000000018 0.185427419292955
-1.47000000000018 0.186942613968019
-1.46000000000018 0.188467325213792
-1.45000000000018 0.190001566015285
-1.44000000000018 0.191545348561439
-1.43000000000018 0.193098684233188
-1.42000000000018 0.194661583591549
-1.41000000000018 0.19623405636575
-1.40000000000018 0.197816111441389
-1.39000000000018 0.199407756848639
-1.38000000000018 0.2010089997505
-1.37000000000018 0.202619846431101
-1.36000000000018 0.204240302284062
-1.35000000000018 0.205870371800917
-1.34000000000018 0.207510058559605
-1.33000000000018 0.209159365213033
-1.32000000000019 0.210818293477716
-1.31000000000019 0.212486844122508
-1.30000000000019 0.21416501695741
-1.29000000000019 0.215852810822483
-1.28000000000019 0.217550223576856
-1.27000000000019 0.21925725208784
-1.26000000000019 0.220973892220156
-1.25000000000019 0.222700138825277
-1.24000000000019 0.224435985730894
-1.23000000000019 0.226181425730513
-1.22000000000019 0.227936450573183
-1.21000000000019 0.229701050953365
-1.20000000000019 0.231475216500949
-1.19000000000019 0.233258935771424
-1.18000000000019 0.235052196236201
-1.17000000000019 0.236854984273111
-1.16000000000019 0.238667285157055
-1.15000000000019 0.240489083050854
-1.14000000000019 0.24232036099626
-1.13000000000019 0.244161100905168
-1.12000000000019 0.246011283551017
-1.11000000000019 0.247870888560395
-1.10000000000019 0.249739894404847
-1.09000000000019 0.2516182783929
-1.08000000000019 0.253506016662302
-1.07000000000019 0.255403084172488
-1.06000000000019 0.257309454697278
-1.05000000000019 0.259225100817809
-1.04000000000019 0.261149993915714
-1.03000000000019 0.263084104166543
-1.02000000000019 0.265027400533444
-1.01000000000019 0.266979850761105
-1.00000000000019 0.268941421369957
-0.990000000000192 0.270912077650656
-0.980000000000192 0.272891783658832
-0.970000000000192 0.274880502210139
-0.960000000000193 0.276878194875572
-0.950000000000193 0.278884821977098
-0.940000000000193 0.280900342583577
-0.930000000000193 0.282924714506988
-0.920000000000194 0.284957894298971
-0.910000000000194 0.286999837247679
-0.900000000000194 0.289050497374956
-0.890000000000194 0.29110982743384
-0.880000000000194 0.293177778906392
-0.870000000000195 0.295254302001868
-0.860000000000195 0.297339345655228
-0.850000000000195 0.299432857525986
-0.840000000000195 0.30153478399742
-0.830000000000195 0.303645070176125
-0.820000000000196 0.305763659891928
-0.810000000000196 0.30789049569817
-0.800000000000196 0.310025518872346
-0.790000000000196 0.312168669417117
-0.780000000000197 0.314319886061704
-0.770000000000197 0.316479106263642
-0.760000000000197 0.318646266210932
-0.750000000000197 0.320821300824564
-0.740000000000197 0.323004143761434
-0.730000000000198 0.325194727417643
-0.720000000000198 0.327392982932196
-0.710000000000198 0.329598840191088
-0.700000000000198 0.33181222783179
-0.690000000000198 0.334033073248136
-0.680000000000199 0.336261302595603
-0.670000000000199 0.338496840797003
-0.660000000000199 0.34073961154857
-0.650000000000199 0.342989537326456
-0.6400000000002 0.345246539393636
-0.6300000000002 0.34751053780721
-0.6200000000002 0.349781451426127
-0.6100000000002 0.352059197919304
-0.6000000000002 0.354343693774159
-0.590000000000201 0.356634854305552
-0.580000000000201 0.358932593665137
-0.570000000000201 0.361236824851112
-0.560000000000201 0.363547459718387
-0.550000000000201 0.365864408989153
-0.540000000000202 0.368187582263851
-0.530000000000202 0.370516888032558
-0.520000000000202 0.372852233686757
-0.510000000000202 0.375193525531523
-0.500000000000203 0.377540668798098
-0.490000000000203 0.379893567656862
-0.480000000000203 0.382252125230703
-0.470000000000203 0.38461624360877
-0.460000000000203 0.386985823860616
-0.450000000000204 0.38936076605073
-0.440000000000204 0.391740969253437
-0.430000000000204 0.394126331568191
-0.420000000000204 0.396516750135225
-0.410000000000204 0.398912121151581
-0.400000000000205 0.401312339887499
-0.390000000000205 0.403717300703163
-0.380000000000205 0.406126897065808
-0.370000000000205 0.40854102156717
-0.360000000000205 0.410959565941285
-0.350000000000206 0.41338242108262
-0.340000000000206 0.415809477064543
-0.330000000000206 0.418240623158114
-0.320000000000206 0.4206757478512
-0.310000000000207 0.423114738867903
-0.300000000000207 0.42555748318829
-0.290000000000207 0.428003867068431
-0.280000000000207 0.43045377606072
-0.270000000000207 0.432907095034495
-0.260000000000208 0.43536370819692
-0.250000000000208 0.437823499114151
-0.240000000000208 0.440286350732756
-0.230000000000208 0.442752145401393
-0.220000000000208 0.445220764892734
-0.210000000000209 0.447692090425623
-0.200000000000209 0.45016600268747
-0.190000000000209 0.452642381856859
-0.180000000000209 0.455121107626368
-0.17000000000021 0.457602059225597
-0.16000000000021 0.460085115444382
-0.15000000000021 0.462570154656198
-0.14000000000021 0.465057054841733
-0.13000000000021 0.467545693612629
-0.120000000000211 0.470035948235376
-0.110000000000211 0.472527695655354
-0.100000000000211 0.475020812521007
-0.0900000000002112 0.477515175208147
-0.0800000000002115 0.480010659844365
-0.0700000000002117 0.482507142333557
-0.0600000000002119 0.485004498380537
-0.0500000000002121 0.487502603515737
-0.0400000000002123 0.490001333119982
-0.0300000000002125 0.492500562449326
-0.0200000000002127 0.495000166659947
-0.0100000000002129 0.497500020833072
-2.1316282072803e-13 0.499999999999947
0.00999999999978662 0.502499979166822
0.0199999999997864 0.504999833339946
0.0299999999997862 0.507499437550567
0.039999999999786 0.509998666879912
0.0499999999997858 0.512497396484157
0.0599999999997856 0.514995501619356
0.0699999999997853 0.517492857666336
0.0799999999997851 0.519989340155528
0.0899999999997849 0.522484824791747
0.0999999999997847 0.524979187478886
0.109999999999784 0.52747230434454
0.119999999999784 0.529964051764518
0.129999999999784 0.532454306387265
0.139999999999784 0.534942945158161
0.149999999999784 0.537429845343696
0.159999999999783 0.539914884555512
0.169999999999783 0.542397940774297
0.179999999999783 0.544878892373526
0.189999999999783 0.547357618143035
0.199999999999783 0.549833997312424
0.209999999999782 0.552307909574271
0.219999999999782 0.554779235107161
0.229999999999782 0.557247854598502
0.239999999999782 0.559713649267139
0.249999999999782 0.562176500885744
0.259999999999781 0.564636291802975
0.269999999999781 0.567092904965401
0.279999999999781 0.569546223939175
0.289999999999781 0.571996132931465
0.29999999999978 0.574442516811605
0.30999999999978 0.576885261131993
0.31999999999978 0.579324252148696
0.32999999999978 0.581759376841783
0.33999999999978 0.584190522935354
0.349999999999779 0.586617578917277
0.359999999999779 0.589040434058612
0.369999999999779 0.591458978432727
0.379999999999779 0.593873102934089
0.389999999999779 0.596282699296735
0.399999999999778 0.598687660112399
0.409999999999778 0.601087878848316
0.419999999999778 0.603483249864673
0.429999999999778 0.605873668431708
0.439999999999777 0.608259030746461
0.449999999999777 0.610639233949169
0.459999999999777 0.613014176139283
0.469999999999777 0.615383756391129
0.479999999999777 0.617747874769196
0.489999999999776 0.620106432343037
0.499999999999776 0.622459331201802
0.509999999999776 0.624806474468377
0.519999999999776 0.627147766313143
0.529999999999776 0.629483111967343
0.539999999999775 0.631812417736049
0.549999999999775 0.634135591010749
0.559999999999775 0.636452540281514
0.569999999999775 0.63876317514879
0.579999999999774 0.641067406334765
0.589999999999774 0.64336514569435
0.599999999999774 0.645656306225744
0.609999999999774 0.647940802080599
0.619999999999774 0.650218548573775
0.629999999999773 0.652489462192693
0.639999999999773 0.654753460606268
0.649999999999773 0.657010462673448
0.659999999999773 0.659260388451334
0.669999999999773 0.661503159202901
0.679999999999772 0.663738697404302
0.689999999999772 0.66596692675177
0.699999999999772 0.668187772168116
0.709999999999772 0.670401159808818
0.719999999999771 0.67260701706771
0.729999999999771 0.674805272582263
0.739999999999771 0.676995856238473
0.749999999999771 0.679178699175343
0.759999999999771 0.681353733788976
0.76999999999977 0.683520893736266
0.77999999999977 0.685680113938204
0.78999999999977 0.687831330582791
0.79999999999977 0.689974481127563
0.80999999999977 0.692109504301739
0.819999999999769 0.694236340107982
0.829999999999769 0.696354929823785
0.839999999999769 0.69846521600249
0.849999999999769 0.700567142473924
0.859999999999769 0.702660654344683
0.869999999999768 0.704745697998043
0.879999999999768 0.70682222109352
0.889999999999768 0.708890172566072
0.899999999999768 0.710949502624956
0.909999999999767 0.713000162752234
0.919999999999767 0.715042105700942
0.929999999999767 0.717075285492925
0.939999999999767 0.719099657416337
0.949999999999767 0.721115178022816
0.959999999999766 0.723121805124343
0.969999999999766 0.725119497789776
0.979999999999766 0.727108216341083
0.989999999999766 0.72908792234926
0.999999999999766 0.731058578629959
1.00999999999977 0.733020149238812
1.01999999999977 0.734972599466473
1.02999999999976 0.736915895833375
1.03999999999976 0.738850006084204
1.04999999999976 0.740774899182109
1.05999999999976 0.742690545302641
1.06999999999976 0.744596915827431
1.07999999999976 0.746493983337617
1.08999999999976 0.74838172160702
1.09999999999976 0.750260105595073
1.10999999999976 0.752129111439526
1.11999999999976 0.753988716448904
1.12999999999976 0.755838899094753
1.13999999999976 0.757679639003661
1.14999999999976 0.759510916949068
1.15999999999976 0.761332714842867
1.16999999999976 0.763145015726812
1.17999999999976 0.764947803763722
1.18999999999976 0.7667410642285
1.19999999999976 0.768524783498975
1.20999999999976 0.770298949046559
1.21999999999976 0.772063549426742
1.22999999999976 0.773818574269412
1.23999999999976 0.775564014269032
1.24999999999976 0.77729986117465
1.25999999999976 0.779026107779771
1.26999999999976 0.780742747912087
1.27999999999976 0.782449776423072
1.28999999999976 0.784147189177445
1.29999999999976 0.785834983042518
1.30999999999976 0.78751315587742
1.31999999999976 0.789181706522213
1.32999999999976 0.790840634786897
1.33999999999976 0.792489941440325
1.34999999999976 0.794129628199013
1.35999999999976 0.795759697715869
1.36999999999976 0.797380153568831
1.37999999999976 0.798991000249432
1.38999999999976 0.800592243151293
1.39999999999976 0.802183888558543
1.40999999999976 0.803765943634183
1.41999999999976 0.805338416408384
1.42999999999976 0.806901315766746
1.43999999999976 0.808454651438495
1.44999999999976 0.809998433984649
1.45999999999976 0.811532674786143
1.46999999999976 0.813057386031916
1.47999999999976 0.814572580706981
1.48999999999976 0.816078272580459
1.49999999999975 0.817574476193607
1.50999999999975 0.819061206847822
1.51999999999975 0.820538480592637
1.52999999999975 0.822006314213718
1.53999999999975 0.823464725220848
1.54999999999975 0.824913731835925
1.55999999999975 0.82635335298096
1.56999999999975 0.827783608266087
1.57999999999975 0.829204517977591
1.58999999999975 0.830616103065947
1.59999999999975 0.83201838513389
1.60999999999975 0.833411386424506
1.61999999999975 0.834795129809351
1.62999999999975 0.836169638776607
1.63999999999975 0.83753493741927
1.64999999999975 0.838891050423381
1.65999999999975 0.840238003056298
1.66999999999975 0.841575821155011
1.67999999999975 0.842904531114514
1.68999999999975 0.84422415987622
1.69999999999975 0.845534734916433
1.70999999999975 0.846836284234881
1.71999999999975 0.848128836343309
1.72999999999975 0.849412420254124
1.73999999999975 0.850687065469124
1.74999999999975 0.851952801968279
1.75999999999975 0.853209660198586
1.76999999999975 0.854457671063004
1.77999999999975 0.85569686590945
1.78999999999975 0.856927276519888
1.79999999999975 0.858148935099482
1.80999999999975 0.859361874265835
1.81999999999975 0.86056612703832
1.82999999999975 0.861761726827476
1.83999999999975 0.86294870742451
1.84999999999975 0.864127102990876
1.85999999999975 0.865296948047942
1.86999999999975 0.866458277466758
1.87999999999975 0.867611126457906
1.88999999999975 0.868755530561448
1.89999999999975 0.869891525636973
1.90999999999975 0.871019147853736
1.91999999999975 0.87213843368089
1.92999999999975 0.873249419877831
1.93999999999975 0.874352143484626
1.94999999999975 0.875446641812556
1.95999999999975 0.876532952434748
1.96999999999974 0.877611113176924
1.97999999999974 0.878681162108236
1.98999999999974 0.879743137532222
1.99999999999974 0.880797077977856
2.00999999999974 0.881843022190704
2.01999999999974 0.882881009124193
2.02999999999974 0.883911077930979
2.03999999999974 0.884933267954424
2.04999999999974 0.885947618720183
2.05999999999974 0.886954169927895
2.06999999999974 0.887952961442984
2.07999999999974 0.888944033288567
2.08999999999974 0.889927425637472
2.09999999999974 0.890903178804362
2.10999999999974 0.891871333237972
2.11999999999974 0.892831929513447
2.12999999999974 0.8937850083248
2.13999999999974 0.894730610477466
2.14999999999974 0.895668776880974
2.15999999999974 0.896599548541726
2.16999999999974 0.897522966555879
2.17999999999974 0.89843907210234
2.18999999999974 0.89934790643587
2.19999999999974 0.900249510880291
2.20999999999974 0.901143926821807
2.21999999999974 0.902031195702423
2.22999999999974 0.902911359013478
2.23999999999974 0.903784458289285
2.24999999999974 0.904650535100868
2.25999999999974 0.905509631049818
2.26999999999974 0.906361787762242
2.27999999999974 0.907207046882821
2.28999999999974 0.908045450068974
2.29999999999974 0.908877038985122
2.30999999999974 0.909701855297058
2.31999999999974 0.910519940666417
2.32999999999974 0.911331336745245
2.33999999999974 0.912136085170678
2.34999999999974 0.912934227559708
2.35999999999974 0.913725805504063
2.36999999999974 0.914510860565173
2.37999999999974 0.915289434269244
2.38999999999974 0.916061568102422
2.39999999999974 0.916827303506057
2.40999999999974 0.917586681872067
2.41999999999974 0.918339744538386
2.42999999999974 0.919086532784515
2.43999999999973 0.919827087827168
2.44999999999973 0.920561450816002
2.45999999999973 0.921289662829445
2.46999999999973 0.922011764870615
2.47999999999973 0.922727797863321
2.48999999999973 0.923437802648169
2.49999999999973 0.924141819978738
2.50999999999973 0.924839890517855
2.51999999999973 0.925532054833954
2.52999999999973 0.926218353397516
2.53999999999973 0.926898826577604
2.54999999999973 0.927573514638464
2.55999999999973 0.928242457736231
2.56999999999973 0.928905695915695
2.57999999999973 0.929563269107165
2.58999999999973 0.930215217123402
2.59999999999973 0.930861579656636
2.60999999999973 0.931502396275657
2.61999999999973 0.932137706422989
2.62999999999973 0.932767549412133
2.63999999999973 0.933391964424893
2.64999999999973 0.934010990508765
2.65999999999973 0.934624666574411
2.66999999999973 0.935233031393198
2.67999999999973 0.93583612359481
2.68999999999973 0.936433981664929
2.69999999999973 0.937026643942988
2.70999999999973 0.93761414861999
2.71999999999973 0.938196533736396
2.72999999999973 0.938773837180076
2.73999999999973 0.939346096684332
2.74999999999973 0.939913349825977
2.75999999999973 0.940475634023483
2.76999999999973 0.941032986535193
2.77999999999973 0.941585444457589
2.78999999999973 0.942133044723624
2.79999999999973 0.942675824101116
2.80999999999973 0.943213819191197
2.81999999999973 0.943747066426818
2.82999999999973 0.94427560207132
2.83999999999973 0.944799462217051
2.84999999999973 0.945318682784045
2.85999999999973 0.94583329951875
2.86999999999973 0.946343347992814
2.87999999999973 0.946848863601923
2.88999999999973 0.947349881564683
2.89999999999973 0.947846436921569
2.90999999999972 0.948338564533902
2.91999999999972 0.948826299082895
2.92999999999972 0.949309675068736
2.93999999999972 0.94978872680972
2.94999999999972 0.95026348844143
2.95999999999972 0.95073399391596
2.96999999999972 0.951200277001187
2.97999999999972 0.951662371280082
2.98999999999972 0.952120310150069
2.99999999999972 0.952574126822421
3.00999999999972 0.953023854321704
3.01999999999972 0.953469525485256
3.02999999999972 0.953911172962708
3.03999999999972 0.954348829215544
3.04999999999972 0.9547825265167
3.05999999999972 0.955212296950201
3.06999999999972 0.955638172410832
3.07999999999972 0.956060184603847
3.08999999999972 0.956478365044716
3.09999999999972 0.956892745058902
3.10999999999972 0.957303355781675
3.11999999999972 0.957710228157955
3.12999999999972 0.958113392942196
3.13999999999972 0.958512880698293
3.14999999999972 0.958908721799524
3.15999999999972 0.959300946428522
3.16999999999972 0.95968958457728
3.17999999999972 0.960074666047176
3.18999999999972 0.960456220449036
3.19999999999972 0.960834277203225
3.20999999999972 0.961208865539753
3.21999999999972 0.961580014498424
3.22999999999972 0.961947752928997
3.23999999999972 0.962312109491384
3.24999999999972 0.96267311265586
3.25999999999972 0.963030790703313
3.26999999999972 0.9633851717255
3.27999999999972 0.963736283625342
3.28999999999972 0.964084154117228
3.29999999999972 0.964428810727354
3.30999999999972 0.964770280794069
3.31999999999972 0.965108591468254
3.32999999999972 0.965443769713714
3.33999999999972 0.965775842307595
3.34999999999972 0.966104835840813
3.35999999999972 0.966430776718508
3.36999999999972 0.966753691160517
3.37999999999971 0.967073605201855
3.38999999999971 0.967390544693225
3.39999999999971 0.967704535301541
3.40999999999971 0.968015602510458
3.41999999999971 0.968323771620935
3.42999999999971 0.968629067751797
3.43999999999971 0.968931515840324
3.44999999999971 0.969231140642843
3.45999999999971 0.969527966735347
3.46999999999971 0.969822018514116
3.47999999999971 0.970113320196356
3.48999999999971 0.970401895820852
3.49999999999971 0.970687769248635
3.50999999999971 0.970970964163651
3.51999999999971 0.971251504073452
3.52999999999971 0.971529412309895
3.53999999999971 0.971804712029849
3.54999999999971 0.972077426215919
3.55999999999971 0.972347577677169
3.56999999999971 0.972615189049867
3.57999999999971 0.972880282798229
3.58999999999971 0.973142881215183
3.59999999999971 0.973403006423127
3.60999999999971 0.973660680374709
3.61999999999971 0.97391592485361
3.62999999999971 0.974168761475329
3.63999999999971 0.974419211687985
3.64999999999971 0.974667296773121
3.65999999999971 0.974913037846513
3.66999999999971 0.975156455858989
3.67999999999971 0.975397571597253
3.68999999999971 0.975636405684715
3.69999999999971 0.975872978582324
3.70999999999971 0.97610731058941
3.71999999999971 0.976339421844532
3.72999999999971 0.976569332326323
3.73999999999971 0.976797061854349
3.74999999999971 0.977022630089968
3.75999999999971 0.97724605653719
3.76999999999971 0.977467360543546
3.77999999999971 0.977686561300961
3.78999999999971 0.977903677846624
3.79999999999971 0.978118729063863
3.80999999999971 0.978331733683033
3.81999999999971 0.97854271028239
3.82999999999971 0.97875167728898
3.8399999999997 0.978958652979526
3.8499999999997 0.979163655481314
3.8599999999997 0.979366702773088
3.8699999999997 0.979567812685941
3.8799999999997 0.979767002904209
3.8899999999997 0.979964290966365
3.8999999999997 0.980159694265917
3.9099999999997 0.980353230052306
3.9199999999997 0.980544915431801
3.9299999999997 0.980734767368403
3.9399999999997 0.980922802684736
3.9499999999997 0.981109038062955
3.9599999999997 0.98129349004564
3.9699999999997 0.981476175036695
3.9799999999997 0.981657109302251
3.9899999999997 0.981836308971562
3.9999999999997 0.982013790037903
4.0099999999997 0.982189568359469
4.0199999999997 0.982363659660272
4.0299999999997 0.982536079531035
4.0399999999997 0.982706843430089
4.0499999999997 0.982875966684267
4.0599999999997 0.983043464489794
4.0699999999997 0.983209351913181
4.0799999999997 0.983373643892113
4.0899999999997 0.983536355236339
4.0999999999997 0.983697500628554
4.1099999999997 0.98385709462529
4.1199999999997 0.984015151657793
4.1299999999997 0.984171686032906
4.1399999999997 0.984326711933947
4.1499999999997 0.984480243421587
4.1599999999997 0.98463229443472
4.1699999999997 0.984782878791339
4.1799999999997 0.984932010189403
4.1899999999997 0.985079702207701
4.1999999999997 0.985225968306723
4.2099999999997 0.985370821829511
4.2199999999997 0.985514276002529
4.2299999999997 0.985656343936508
4.2399999999997 0.985797038627305
4.2499999999997 0.98593637295675
4.2599999999997 0.986074359693495
4.2699999999997 0.986211011493854
4.2799999999997 0.986346340902645
4.2899999999997 0.986480360354026
4.2999999999997 0.986613082172331
4.30999999999969 0.986744518572897
4.31999999999969 0.986874681662893
4.32999999999969 0.987003583442143
4.33999999999969 0.987131235803944
4.34999999999969 0.987257650535885
4.35999999999969 0.987382839320657
4.36999999999969 0.987506813736867
4.37999999999969 0.987629585259834
4.38999999999969 0.9877511652624
4.39999999999969 0.987871565015722
4.40999999999969 0.987990795690069
4.41999999999969 0.988108868355609
4.42999999999969 0.988225793983199
4.43999999999969 0.988341583445163
4.44999999999969 0.988456247516074
4.45999999999969 0.988569796873526
4.46999999999969 0.988682242098904
4.47999999999969 0.988793593678154
4.48999999999969 0.988903862002539
4.49999999999969 0.989013057369403
4.50999999999969 0.989121189982923
4.51999999999969 0.989228269954855
4.52999999999969 0.989334307305287
4.53999999999969 0.989439311963374
4.54999999999969 0.989543293768079
4.55999999999969 0.989646262468905
4.56999999999969 0.989748227726625
4.57999999999969 0.989849199114004
4.58999999999969 0.989949186116523
4.59999999999969 0.990048198133093
4.60999999999969 0.990146244476766
4.61999999999969 0.990243334375446
4.62999999999969 0.990339476972589
4.63999999999969 0.990434681327905
4.64999999999969 0.990528956418051
4.65999999999969 0.99062231113732
4.66999999999969 0.990714754298331
4.67999999999969 0.990806294632709
4.68999999999969 0.99089694079176
4.69999999999969 0.990986701347149
4.70999999999969 0.991075584791566
4.71999999999969 0.99116359953939
4.72999999999969 0.99125075392735
4.73999999999969 0.991337056215184
4.74999999999969 0.991422514586285
4.75999999999969 0.991507137148353
4.76999999999969 0.991590931934034
4.77999999999968 0.991673906901564
4.78999999999968 0.991756069935396
4.79999999999968 0.991837428846837
4.80999999999968 0.991917991374672
4.81999999999968 0.991997765185781
4.82999999999968 0.992076757875763
4.83999999999968 0.992154976969542
4.84999999999968 0.992232429921983
4.85999999999968 0.992309124118489
4.86999999999968 0.992385066875608
4.87999999999968 0.992460265441624
4.88999999999968 0.99253472699715
4.89999999999968 0.992608458655716
4.90999999999968 0.992681467464351
4.91999999999968 0.992753760404166
4.92999999999968 0.992825344390923
4.93999999999968 0.992896226275609
4.94999999999968 0.992966412845003
4.95999999999968 0.993035910822236
4.96999999999968 0.993104726867352
4.97999999999968 0.99317286757786
4.98999999999968 0.993240339489285
4.99999999999968 0.993307149075713
5.00999999999968 0.993373302750335
5.01999999999968 0.993438806865982
5.02999999999968 0.99350366771566
5.03999999999968 0.993567891533079
5.04999999999968 0.993631484493182
5.05999999999968 0.993694452712662
5.06999999999968 0.993756802250481
5.07999999999968 0.993818539108387
5.08999999999968 0.993879669231418
5.09999999999968 0.993940198508414
5.10999999999968 0.994000132772514
5.11999999999968 0.994059477801658
5.12999999999968 0.994118239319077
5.13999999999968 0.994176422993787
5.14999999999968 0.994234034441073
5.15999999999968 0.994291079222975
5.16999999999968 0.994347562848761
5.17999999999968 0.994403490775408
5.18999999999968 0.994458868408071
5.19999999999968 0.994513701100548
5.20999999999968 0.994567994155748
5.21999999999968 0.99462175282615
5.22999999999968 0.994674982314256
5.23999999999968 0.99472768777305
5.24999999999967 0.99477987430644
5.25999999999967 0.99483154696971
5.26999999999967 0.994882710769957
5.27999999999967 0.994933370666532
5.28999999999967 0.994983531571474
5.29999999999967 0.995033198349941
5.30999999999967 0.995082375820638
5.31999999999967 0.99513106875624
5.32999999999967 0.995179281883814
5.33999999999967 0.995227019885235
5.34999999999967 0.995274287397603
5.35999999999967 0.995321089013648
5.36999999999967 0.99536742928214
5.37999999999967 0.995413312708294
5.38999999999967 0.995458743754168
5.39999999999967 0.995503726839057
5.40999999999967 0.995548266339895
5.41999999999967 0.995592366591636
5.42999999999967 0.995636031887646
5.43999999999967 0.995679266480087
5.44999999999967 0.995722074580294
5.45999999999967 0.995764460359155
5.46999999999967 0.995806427947486
5.47999999999967 0.995847981436398
5.48999999999967 0.995889124877668
5.49999999999967 0.995929862284103
5.50999999999967 0.995970197629899
5.51999999999967 0.996010134851005
5.52999999999967 0.996049677845471
5.53999999999967 0.996088830473807
5.54999999999967 0.996127596559328
5.55999999999967 0.996165979888501
5.56999999999967 0.996203984211291
5.57999999999967 0.996241613241498
5.58999999999967 0.996278870657095
5.59999999999967 0.996315760100563
5.60999999999967 0.996352285179224
5.61999999999967 0.996388449465567
5.62999999999967 0.996424256497577
5.63999999999967 0.996459709779052
5.64999999999967 0.996494812779932
5.65999999999967 0.996529568936609
5.66999999999967 0.996563981652245
5.67999999999967 0.996598054297082
5.68999999999967 0.996631790208752
5.69999999999967 0.996665192692586
5.70999999999967 0.996698265021911
5.71999999999966 0.99673101043836
5.72999999999966 0.996763432152162
5.73999999999966 0.996795533342442
5.74999999999966 0.996827317157514
5.75999999999966 0.996858786715169
5.76999999999966 0.996889945102969
5.77999999999966 0.996920795378522
5.78999999999966 0.996951340569775
5.79999999999966 0.996981583675291
5.80999999999966 0.997011527664522
5.81999999999966 0.997041175478092
5.82999999999966 0.997070530028064
5.83999999999966 0.997099594198215
5.84999999999966 0.997128370844299
5.85999999999966 0.997156862794316
5.86999999999966 0.997185072848776
5.87999999999966 0.997213003780957
5.88999999999966 0.997240658337164
5.89999999999966 0.997268039236988
5.90999999999966 0.997295149173556
5.91999999999966 0.997321990813786
5.92999999999966 0.997348566798633
5.93999999999966 0.997374879743339
5.94999999999966 0.997400932237676
5.95999999999966 0.997426726846189
5.96999999999966 0.997452266108437
5.97999999999966 0.99747755253923
5.98999999999966 0.997502588628866
5.99999999999966 0.997527376843364
6.00999999999966 0.997551919624699
6.01999999999966 0.997576219391025
6.02999999999966 0.99760027853691
6.03999999999966 0.997624099433554
6.04999999999966 0.997647684429021
6.05999999999966 0.997671035848453
6.06999999999966 0.997694155994291
6.07999999999966 0.997717047146496
6.08999999999966 0.997739711562761
6.09999999999966 0.997762151478723
6.10999999999966 0.997784369108179
6.11999999999966 0.997806366643291
6.12999999999965 0.997828146254795
6.13999999999966 0.99784971009221
6.14999999999966 0.997871060284035
6.15999999999966 0.997892198937956
6.16999999999965 0.997913128141044
6.17999999999966 0.997933849959954
6.18999999999966 0.997954366441119
6.19999999999965 0.997974679610949
6.20999999999965 0.997994791476019
6.21999999999965 0.998014704023261
6.22999999999966 0.998034419220154
6.23999999999965 0.998053939014914
6.24999999999965 0.998073265336672
6.25999999999965 0.998092400095666
6.26999999999965 0.998111345183417
6.27999999999965 0.998130102472915
6.28999999999965 0.998148673818793
6.29999999999965 0.998167061057507
6.30999999999965 0.998185266007508
6.31999999999965 0.998203290469421
6.32999999999965 0.998221136226213
6.33999999999965 0.998238805043366
6.34999999999965 0.998256298669045
6.35999999999965 0.998273618834262
6.36999999999965 0.99829076725305
6.37999999999965 0.99830774562262
6.38999999999965 0.998324555623525
6.39999999999965 0.998341198919825
6.40999999999965 0.998357677159242
6.41999999999965 0.998373991973322
6.42999999999965 0.998390144977587
6.43999999999965 0.998406137771696
6.44999999999965 0.998421971939595
6.45999999999965 0.99843764904967
6.46999999999965 0.998453170654898
6.47999999999965 0.998468538292996
6.48999999999965 0.99848375348657
6.49999999999965 0.998498817743263
6.50999999999965 0.998513732555894
6.51999999999965 0.99852849940261
6.52999999999965 0.998543119747024
6.53999999999965 0.998557595038357
6.54999999999965 0.998571926711577
6.55999999999965 0.998586116187539
6.56999999999965 0.998600164873121
6.57999999999965 0.998614074161361
6.58999999999965 0.998627845431592
6.59999999999965 0.99864148004957
6.60999999999964 0.998654979367617
6.61999999999965 0.998668344724739
6.62999999999965 0.998681577446766
6.63999999999965 0.998694678846473
6.64999999999964 0.998707650223712
6.65999999999964 0.998720492865537
6.66999999999965 0.998733208046324
6.67999999999964 0.998745797027903
6.68999999999964 0.998758261059671
6.69999999999964 0.998770601378722
6.70999999999965 0.99878281920996
6.71999999999964 0.998794915766221
6.72999999999964 0.99880689224839
6.73999999999964 0.99881874984552
6.74999999999964 0.998830489734944
6.75999999999964 0.99884211308239
6.76999999999964 0.998853621042096
6.77999999999964 0.998865014756922
6.78999999999964 0.998876295358458
6.79999999999964 0.998887463967139
6.80999999999964 0.998898521692351
6.81999999999964 0.998909469632539
6.82999999999964 0.998920308875313
6.83999999999964 0.998931040497558
6.84999999999964 0.998941665565534
6.85999999999964 0.998952185134983
6.86999999999964 0.99896260025123
6.87999999999964 0.998972911949288
6.88999999999964 0.998983121253954
6.89999999999964 0.998993229179914
6.90999999999964 0.999003236731837
6.91999999999964 0.999013144904479
6.92999999999964 0.999022954682773
6.93999999999964 0.999032667041932
6.94999999999964 0.999042282947539
6.95999999999964 0.999051803355646
6.96999999999964 0.999061229212862
6.97999999999964 0.999070561456449
6.98999999999964 0.999079801014414
6.99999999999964 0.999088948805599
7.00999999999964 0.999098005739768
7.01999999999964 0.999106972717702
7.02999999999964 0.999115850631279
7.03999999999964 0.99912464036357
7.04999999999963 0.999133342788921
7.05999999999964 0.999141958773035
7.06999999999964 0.999150489173065
7.07999999999964 0.999158934837691
7.08999999999963 0.999167296607207
7.09999999999964 0.999175575313601
7.10999999999964 0.99918377178064
7.11999999999964 0.999191886823943
7.12999999999963 0.999199921251073
7.13999999999963 0.999207875861605
7.14999999999964 0.999215751447209
7.15999999999963 0.999223548791729
7.16999999999963 0.999231268671258
7.17999999999963 0.999238911854215
7.18999999999964 0.999246479101418
7.19999999999963 0.999253971166163
7.20999999999963 0.999261388794295
7.21999999999963 0.999268732724281
7.22999999999963 0.999276003687283
7.23999999999963 0.999283202407233
7.24999999999963 0.999290329600899
7.25999999999963 0.999297385977958
7.26999999999963 0.999304372241066
7.27999999999963 0.999311289085926
7.28999999999963 0.999318137201359
7.29999999999963 0.999324917269367
7.30999999999963 0.999331629965204
7.31999999999963 0.999338275957443
7.32999999999963 0.999344855908037
7.33999999999963 0.999351370472388
7.34999999999963 0.999357820299414
7.35999999999963 0.999364206031605
7.36999999999963 0.999370528305093
7.37999999999963 0.999376787749712
7.38999999999963 0.999382984989061
7.39999999999963 0.999389120640565
7.40999999999963 0.999395195315536
7.41999999999963 0.99940120961923
7.42999999999963 0.999407164150915
7.43999999999963 0.999413059503919
7.44999999999963 0.999418896265698
7.45999999999963 0.999424675017889
7.46999999999963 0.999430396336368
7.47999999999963 0.999436060791308
7.48999999999963 0.999441668947233
7.49999999999963 0.999447221363076
7.50999999999963 0.999452718592235
7.51999999999963 0.999458161182622
7.52999999999962 0.999463549676726
7.53999999999963 0.999468884611656
7.54999999999963 0.999474166519203
7.55999999999963 0.999479395925889
7.56999999999962 0.999484573353017
7.57999999999963 0.999489699316726
7.58999999999963 0.999494774328041
7.59999999999962 0.99949979889292
7.60999999999962 0.999504773512311
7.61999999999962 0.999509698682193
7.62999999999963 0.999514574893633
7.63999999999962 0.99951940263283
7.64999999999962 0.999524182381162
7.65999999999962 0.999528914615237
7.66999999999963 0.99953359980694
7.67999999999962 0.999538238423476
7.68999999999962 0.999542830927419
7.69999999999962 0.999547377776759
7.70999999999962 0.999551879424944
7.71999999999962 0.999556336320926
7.72999999999962 0.999560748909206
7.73999999999962 0.999565117629879
7.74999999999962 0.999569442918675
7.75999999999962 0.999573725207005
7.76999999999962 0.999577964922
7.77999999999962 0.999582162486557
7.78999999999962 0.999586318319378
7.79999999999962 0.999590432835014
7.80999999999962 0.999594506443904
7.81999999999962 0.999598539552417
7.82999999999962 0.999602532562891
7.83999999999962 0.999606485873673
7.84999999999962 0.999610399879161
7.85999999999962 0.999614274969838
7.86999999999962 0.999618111532317
7.87999999999962 0.999621909949373
7.88999999999962 0.999625670599985
7.89999999999962 0.999629393859374
7.90999999999962 0.999633080099035
7.91999999999962 0.999636729686782
7.92999999999962 0.999640342986776
7.93999999999962 0.999643920359568
7.94999999999962 0.999647462162129
7.95999999999962 0.99965096874789
7.96999999999962 0.999654440466775
7.97999999999962 0.999657877665235
7.98999999999962 0.999661280686285
7.99999999999962 0.999664649869533
8.00999999999961 0.999667985551221
8.01999999999962 0.999671288064252
8.02999999999962 0.999674557738225
8.03999999999962 0.999677794899469
8.04999999999961 0.999680999871073
8.05999999999962 0.999684172972922
8.06999999999962 0.999687314521724
8.07999999999961 0.999690424831044
8.08999999999961 0.999693504211336
8.09999999999961 0.999696552969971
8.10999999999962 0.999699571411269
8.11999999999961 0.999702559836532
8.12999999999961 0.999705518544067
8.13999999999961 0.999708447829222
8.14999999999961 0.999711347984415
8.15999999999961 0.999714219299158
8.16999999999961 0.99971706206009
8.17999999999961 0.999719876551005
8.18999999999961 0.999722663052879
8.19999999999961 0.999725421843899
8.20999999999961 0.999728153199488
8.21999999999961 0.999730857392339
8.22999999999961 0.999733534692432
8.23999999999961 0.999736185367072
8.24999999999961 0.999738809680904
8.25999999999961 0.999741407895951
8.26999999999961 0.999743980271629
8.27999999999961 0.999746527064784
8.28999999999961 0.999749048529705
8.29999999999961 0.999751544918161
8.30999999999961 0.999754016479418
8.31999999999961 0.999756463460268
8.32999999999961 0.999758886105052
8.33999999999961 0.999761284655685
8.34999999999961 0.999763659351678
8.35999999999961 0.999766010430164
8.36999999999961 0.999768338125921
8.37999999999961 0.999770642671397
8.38999999999961 0.999772924296728
8.39999999999961 0.999775183229767
8.40999999999961 0.999777419696102
8.41999999999961 0.999779633919083
8.42999999999961 0.999781826119839
8.43999999999961 0.999783996517303
8.4499999999996 0.999786145328235
8.45999999999961 0.999788272767241
8.46999999999961 0.999790379046794
8.47999999999961 0.999792464377258
8.4899999999996 0.999794528966906
8.49999999999961 0.999796573021945
8.50999999999961 0.999798596746529
8.51999999999961 0.999800600342789
8.5299999999996 0.999802584010843
8.5399999999996 0.999804547948825
8.54999999999961 0.999806492352899
8.5599999999996 0.999808417417279
8.5699999999996 0.999810323334251
8.5799999999996 0.999812210294191
8.58999999999961 0.999814078485582
8.5999999999996 0.999815928095037
8.6099999999996 0.999817759307311
8.6199999999996 0.999819572305327
8.6299999999996 0.999821367270188
8.6399999999996 0.9998231443812
8.6499999999996 0.999824903815885
8.6599999999996 0.999826645750003
8.6699999999996 0.999828370357566
8.6799999999996 0.999830077810858
8.6899999999996 0.999831768280452
8.6999999999996 0.999833441935223
8.7099999999996 0.99983509894237
8.7199999999996 0.999836739467432
8.7299999999996 0.9998383636743
8.7399999999996 0.999839971725238
8.7499999999996 0.999841563780898
8.7599999999996 0.999843140000333
8.7699999999996 0.999844700541019
8.7799999999996 0.999846245558865
8.7899999999996 0.99984777520823
8.7999999999996 0.99984928964194
8.8099999999996 0.999850789011303
8.8199999999996 0.999852273466121
8.8299999999996 0.999853743154709
8.8399999999996 0.999855198223908
8.8499999999996 0.999856638819099
8.8599999999996 0.999858065084217
8.8699999999996 0.999859477161769
8.8799999999996 0.999860875192843
8.8899999999996 0.999862259317128
8.8999999999996 0.99986362967292
8.9099999999996 0.999864986397146
8.9199999999996 0.999866329625366
8.92999999999959 0.999867659491798
8.9399999999996 0.999868976129323
8.9499999999996 0.999870279669502
8.9599999999996 0.999871570242587
8.96999999999959 0.999872847977537
8.9799999999996 0.99987411300203
8.9899999999996 0.999875365442471
8.99999999999959 0.999876605424014
9.00999999999959 0.999877833070563
9.01999999999959 0.999879048504795
9.0299999999996 0.999880251848165
9.03999999999959 0.999881443220922
9.04999999999959 0.999882622742119
9.05999999999959 0.999883790529625
9.0699999999996 0.999884946700139
9.07999999999959 0.999886091369197
9.08999999999959 0.999887224651191
9.09999999999959 0.99988834665937
9.10999999999959 0.999889457505863
9.11999999999959 0.99989055730168
9.12999999999959 0.999891646156728
9.13999999999959 0.999892724179824
9.14999999999959 0.999893791478702
9.15999999999959 0.999894848160022
9.16999999999959 0.999895894329388
9.17999999999959 0.999896930091351
9.18999999999959 0.999897955549425
9.19999999999959 0.999898970806092
9.20999999999959 0.999899975962818
9.21999999999959 0.999900971120058
9.22999999999959 0.999901956377269
9.23999999999959 0.99990293183292
9.24999999999959 0.9999038975845
9.25999999999959 0.99990485372853
9.26999999999959 0.999905800360568
9.27999999999959 0.999906737575226
9.28999999999959 0.999907665466173
9.29999999999959 0.999908584126148
9.30999999999959 0.999909493646966
9.31999999999959 0.99991039411953
9.32999999999959 0.999911285633841
9.33999999999959 0.999912168279002
9.34999999999959 0.999913042143232
9.35999999999959 0.999913907313872
9.36999999999959 0.999914763877396
9.37999999999959 0.999915611919416
9.38999999999959 0.999916451524694
9.39999999999959 0.999917282777148
9.40999999999958 0.999918105759865
9.41999999999959 0.9999189205551
9.42999999999959 0.999919727244296
9.43999999999959 0.999920525908083
9.44999999999958 0.999921316626289
9.45999999999959 0.999922099477949
9.46999999999959 0.999922874541313
9.47999999999958 0.999923641893851
9.48999999999958 0.999924401612264
9.49999999999958 0.999925153772489
9.50999999999959 0.99992589844971
9.51999999999958 0.999926635718362
9.52999999999958 0.999927365652139
9.53999999999958 0.999928088324004
9.54999999999959 0.999928803806192
9.55999999999958 0.999929512170222
9.56999999999958 0.999930213486901
9.57999999999958 0.999930907826332
9.58999999999958 0.99993159525792
9.59999999999958 0.99993227585038
9.60999999999958 0.999932949671745
9.61999999999958 0.99993361678937
9.62999999999958 0.999934277269941
9.63999999999958 0.999934931179479
9.64999999999958 0.999935578583352
9.65999999999958 0.999936219546274
9.66999999999958 0.999936854132318
9.67999999999958 0.999937482404919
9.68999999999958 0.999938104426881
9.69999999999958 0.999938720260383
9.70999999999958 0.999939329966987
9.71999999999958 0.999939933607641
9.72999999999958 0.999940531242689
9.73999999999958 0.999941122931872
9.74999999999958 0.999941708734339
9.75999999999958 0.99994228870865
9.76999999999958 0.999942862912783
9.77999999999958 0.99994343140414
9.78999999999958 0.999943994239549
9.79999999999958 0.999944551475277
9.80999999999958 0.999945103167029
9.81999999999958 0.999945649369955
9.82999999999958 0.99994619013866
9.83999999999958 0.999946725527202
9.84999999999958 0.999947255589104
9.85999999999958 0.999947780377356
9.86999999999958 0.999948299944419
9.87999999999958 0.999948814342237
9.88999999999957 0.999949323622231
9.89999999999958 0.999949827835316
9.90999999999958 0.999950327031898
9.91999999999958 0.999950821261882
9.92999999999957 0.999951310574677
9.93999999999957 0.9999517950192
9.94999999999958 0.999952274643882
9.95999999999957 0.999952749496671
9.96999999999957 0.999953219625041
9.97999999999957 0.999953685075991
9.98999999999958 0.999954145896053
};
\addlegendentry{Sigmoid}
\addplot [semithick, red]
table {%
-10 4.53958077359517e-05
-9.99 4.58520013480925e-05
-9.98 4.63127789370822e-05
-9.97 4.6778186555582e-05
-9.96 4.72482707187453e-05
-9.95 4.77230784088589e-05
-9.94 4.82026570800306e-05
-9.93 4.86870546629234e-05
-9.92 4.91763195695385e-05
-9.91 4.96705006980442e-05
-9.9 5.01696474376551e-05
-9.89 5.06738096735589e-05
-9.88 5.11830377918931e-05
-9.87 5.16973826847718e-05
-9.86 5.22168957553618e-05
-9.85 5.27416289230109e-05
-9.84 5.32716346284266e-05
-9.83 5.38069658389072e-05
-9.82 5.43476760536248e-05
-9.81 5.48938193089619e-05
-9.8 5.54454501839008e-05
-9.79 5.60026238054671e-05
-9.78 5.65653958542283e-05
-9.77 5.71338225698465e-05
-9.76000000000001 5.77079607566876e-05
-9.75000000000001 5.8287867789486e-05
-9.74000000000001 5.88736016190664e-05
-9.73000000000001 5.94652207781231e-05
-9.72000000000001 6.00627843870562e-05
-9.71000000000001 6.06663521598676e-05
-9.70000000000001 6.12759844101146e-05
-9.69000000000001 6.18917420569245e-05
-9.68000000000001 6.25136866310683e-05
-9.67000000000001 6.31418802810959e-05
-9.66000000000001 6.37763857795321e-05
-9.65000000000001 6.44172665291353e-05
-9.64000000000001 6.50645865692184e-05
-9.63000000000001 6.57184105820331e-05
-9.62000000000001 6.63788038992178e-05
-9.61000000000001 6.70458325083109e-05
-9.60000000000001 6.77195630593279e-05
-9.59000000000001 6.84000628714059e-05
-9.58000000000001 6.90873999395131e-05
-9.57000000000001 6.97816429412268e-05
-9.56000000000001 7.04828612435778e-05
-9.55000000000001 7.11911249099647e-05
-9.54000000000001 7.19065047071362e-05
-9.53000000000001 7.26290721122443e-05
-9.52000000000001 7.3358899319967e-05
-9.51000000000001 7.40960592497029e-05
-9.50000000000001 7.48406255528379e-05
-9.49000000000001 7.5592672620084e-05
-9.48000000000001 7.63522755888921e-05
-9.47000000000001 7.71195103509386e-05
-9.46000000000001 7.78944535596866e-05
-9.45000000000001 7.86771826380237e-05
-9.44000000000001 7.94677757859749e-05
-9.43000000000001 8.02663119884939e-05
-9.42000000000001 8.1072871023331e-05
-9.41000000000001 8.18875334689809e-05
-9.40000000000001 8.27103807127091e-05
-9.39000000000001 8.35414949586588e-05
-9.38000000000001 8.43809592360393e-05
-9.37000000000001 8.52288574073954e-05
-9.36000000000001 8.60852741769601e-05
-9.35000000000001 8.69502950990906e-05
-9.34000000000001 8.78240065867886e-05
-9.33000000000001 8.87064959203058e-05
-9.32000000000001 8.95978512558352e-05
-9.31000000000001 9.04981616342897e-05
-9.30000000000001 9.1407516990168e-05
-9.29000000000002 9.23260081605092e-05
-9.28000000000002 9.32537268939369e-05
-9.27000000000002 9.41907658597939e-05
-9.26000000000002 9.51372186573674e-05
-9.25000000000002 9.60931798252076e-05
-9.24000000000002 9.70587448505379e-05
-9.23000000000002 9.80340101787604e-05
-9.22000000000002 9.90190732230556e-05
-9.21000000000002 0.000100014032374078
-9.20000000000002 0.000101018987009749
-9.19000000000002 0.000102034037505148
-9.18000000000002 0.000103059285242499
-9.17000000000002 0.000104094832621261
-9.16000000000002 0.000105140783068321
-9.15000000000002 0.000106197241048278
-9.14000000000002 0.000107264312073844
-9.13000000000002 0.000108342102716337
-9.12000000000002 0.000109430720616288
-9.11000000000002 0.000110530274494143
-9.10000000000002 0.000111640874161087
-9.09000000000002 0.000112762630529959
-9.08000000000002 0.000113895655626288
-9.07000000000002 0.000115040062599438
-9.06000000000002 0.000116195965733854
-9.05000000000002 0.000117363480460435
-9.04000000000002 0.000118542723368009
-9.03000000000002 0.000119733812214928
-9.02000000000002 0.000120936865940777
-9.01000000000002 0.000122152004678201
-9.00000000000002 0.000123379349764846
-8.99000000000002 0.000124619023755426
-8.98000000000002 0.000125871150433901
-8.97000000000002 0.00012713585482579
-8.96000000000002 0.000128413263210588
-8.95000000000002 0.000129703503134327
-8.94000000000002 0.000131006703422247
-8.93000000000002 0.000132322994191599
-8.92000000000002 0.000133652506864577
-8.91000000000002 0.000134995374181377
-8.90000000000002 0.000136351730213386
-8.89000000000002 0.000137721710376502
-8.88000000000002 0.000139105451444588
-8.87000000000002 0.000140503091563061
-8.86000000000002 0.000141914770262612
-8.85000000000002 0.000143340628473068
-8.84000000000002 0.000144780808537387
-8.83000000000002 0.000146235454225797
-8.82000000000003 0.000147704710750075
-8.81000000000003 0.000149188724777961
-8.80000000000003 0.000150687644447727
-8.79000000000003 0.000152201619382885
-8.78000000000003 0.000153730800707037
-8.77000000000003 0.000155275341058884
-8.76000000000003 0.000156835394607372
-8.75000000000003 0.000158411117066998
-8.74000000000003 0.000160002665713265
-8.73000000000003 0.000161610199398287
-8.72000000000003 0.000163233878566558
-8.71000000000003 0.000164873865270869
-8.70000000000003 0.000166530323188389
-8.69000000000003 0.000168203417636901
-8.68000000000003 0.000169893315591207
-8.67000000000003 0.000171600185699685
-8.66000000000003 0.000173324198301024
-8.65000000000003 0.000175065525441113
-8.64000000000003 0.000176824340890107
-8.63000000000003 0.000178600820159653
-8.62000000000003 0.000180395140520302
-8.61000000000003 0.000182207481019075
-8.60000000000003 0.000184038022497222
-8.59000000000003 0.000185886947608141
-8.58000000000003 0.000187754440835489
-8.57000000000003 0.000189640688511465
-8.56000000000003 0.000191545878835274
-8.55000000000003 0.000193470201891778
-8.54000000000003 0.00019541384967033
-8.53000000000003 0.000197377016083797
-8.52000000000003 0.000199359896987769
-8.51000000000003 0.000201362690199959
-8.50000000000003 0.000203385595519799
-8.49000000000003 0.000205428814748227
-8.48000000000003 0.000207492551707667
-8.47000000000003 0.000209577012262219
-8.46000000000003 0.000211682404338033
-8.45000000000003 0.0002138089379439
-8.44000000000003 0.000215956825192037
-8.43000000000003 0.000218126280319083
-8.42000000000003 0.0002203175197073
-8.41000000000003 0.000222530761905989
-8.40000000000003 0.00022476622765311
-8.39000000000003 0.000227024139897126
-8.38000000000003 0.000229304723819055
-8.37000000000003 0.000231608206854748
-8.36000000000003 0.000233934818717376
-8.35000000000004 0.000236284791420156
-8.34000000000004 0.000238658359299287
-8.33000000000004 0.00024105575903712
-8.32000000000004 0.000243477229685559
-8.31000000000004 0.000245923012689687
-8.30000000000004 0.000248393351911634
-8.29000000000004 0.000250888493654671
-8.28000000000004 0.000253408686687554
-8.27000000000004 0.0002559541822691
-8.26000000000004 0.000258525234173006
-8.25000000000004 0.00026112209871292
-8.24000000000004 0.000263745034767756
-8.23000000000004 0.000266394303807253
-8.22000000000004 0.000269070169917799
-8.21000000000004 0.000271772899828501
-8.20000000000004 0.000274502762937514
-8.19000000000004 0.000277260031338632
-8.18000000000004 0.00028004497984814
-8.17000000000004 0.000282857886031932
-8.16000000000004 0.000285699030232894
-8.15000000000004 0.000288568695598563
-8.14000000000004 0.000291467168109052
-8.13000000000004 0.000294394736605255
-8.12000000000004 0.000297351692817327
-8.11000000000004 0.000300338331393449
-8.10000000000004 0.000303354949928873
-8.09000000000004 0.000306401848995254
-8.08000000000004 0.000309479332170269
-8.07000000000004 0.000312587706067536
-8.06000000000004 0.000315727280366817
-8.05000000000004 0.000318898367844523
-8.04000000000004 0.000322101284404522
-8.03000000000004 0.000325336349109247
-8.02000000000004 0.000328603884211111
-8.01000000000004 0.00033190421518423
-8.00000000000004 0.00033523767075646
-7.99000000000004 0.000338604582941748
-7.98000000000004 0.000342005287072801
-7.97000000000004 0.000345440121834076
-7.96000000000004 0.000348909429295097
-7.95000000000004 0.000352413554944092
-7.94000000000004 0.000355952847721973
-7.93000000000004 0.000359527660056633
-7.92000000000004 0.000363138347897596
-7.91000000000004 0.000366785270750997
-7.90000000000004 0.00037046879171491
-7.89000000000004 0.000374189277515018
-7.88000000000005 0.000377947098540638
-7.87000000000005 0.000381742628881097
-7.86000000000005 0.000385576246362467
-7.85000000000005 0.000389448332584651
-7.84000000000005 0.000393359272958846
-7.83000000000005 0.000397309456745366
-7.82000000000005 0.000401299277091829
-7.81000000000005 0.000405329131071733
-7.80000000000005 0.000409399419723396
-7.79000000000005 0.000413510548089285
-7.78000000000005 0.000417662925255724
-7.77000000000005 0.000421856964392997
-7.76000000000005 0.000426093082795837
-7.75000000000005 0.000430371701924316
-7.74000000000005 0.000434693247445127
-7.73000000000005 0.000439058149273281
-7.72000000000005 0.000443466841614199
-7.71000000000005 0.000447919763006224
-7.70000000000005 0.000452417356363542
-7.69000000000005 0.000456960069019525
-7.68000000000005 0.000461548352770498
-7.67000000000005 0.000466182663919928
-7.66000000000005 0.000470863463323048
-7.65000000000005 0.000475591216431915
-7.64000000000005 0.000480366393340906
-7.63000000000005 0.000485189468832656
-7.62000000000005 0.000490060922424449
-7.61000000000005 0.000494981238415053
-7.60000000000005 0.000499950905932015
-7.59000000000005 0.000504970418979416
-7.58000000000005 0.000510040276486089
-7.57000000000005 0.000515160982354305
-7.56000000000005 0.000520333045508935
-7.55000000000005 0.000525556979947086
-7.54000000000005 0.000530833304788223
-7.53000000000005 0.000536162544324773
-7.52000000000005 0.000541545228073224
-7.51000000000005 0.000546981890825715
-7.50000000000005 0.000552473072702131
-7.49000000000005 0.000558019319202702
-7.48000000000005 0.000563621181261107
-7.47000000000005 0.000569279215298098
-7.46000000000005 0.000574993983275642
-7.45000000000005 0.000580766052751581
-7.44000000000005 0.000586595996934831
-7.43000000000005 0.000592484394741106
-7.42000000000005 0.000598431830849183
-7.41000000000006 0.000604438895757715
-7.40000000000006 0.000610506185842584
-7.39000000000006 0.000616634303414818
-7.38000000000006 0.000622823856779055
-7.37000000000006 0.000629075460292585
-7.36000000000006 0.000635389734424945
-7.35000000000006 0.000641767305818105
-7.34000000000006 0.000648208807347219
-7.33000000000006 0.000654714878181972
-7.32000000000006 0.000661286163848509
-7.31000000000006 0.000667923316291966
-7.30000000000006 0.000674626993939601
-7.29000000000006 0.000681397861764524
-7.28000000000006 0.000688236591350053
-7.27000000000006 0.000695143860954677
-7.26000000000006 0.000702120355577645
-7.25000000000006 0.000709166767025187
-7.24000000000006 0.000716283793977366
-7.23000000000006 0.000723472142055575
-7.22000000000006 0.000730732523890672
-7.21000000000006 0.000738065659191781
-7.20000000000006 0.000745472274815737
-7.19000000000006 0.000752953104837204
-7.18000000000006 0.00076050889061946
-7.17000000000006 0.000768140380885858
-7.16000000000006 0.000775848331791971
-7.15000000000006 0.000783633506998424
-7.14000000000006 0.000791496677744418
-7.13000000000006 0.000799438622921963
-7.12000000000006 0.000807460129150807
-7.11000000000006 0.000815561990854088
-7.10000000000006 0.000823745010334701
-7.09000000000006 0.000832009997852391
-7.08000000000006 0.000840357771701579
-7.07000000000006 0.000848789158289931
-7.06000000000006 0.000857304992217667
-7.05000000000006 0.000865906116357628
-7.04000000000006 0.000874593381936093
-7.03000000000006 0.000883367648614375
-7.02000000000006 0.00089222978457118
-7.01000000000006 0.000901180666585745
-7.00000000000006 0.000910221180121768
-6.99000000000006 0.000919352219412129
-6.98000000000006 0.000928574687544405
-6.97000000000006 0.000937889496547203
-6.96000000000006 0.000947297567477301
-6.95000000000007 0.000956799830507604
-6.94000000000007 0.000966397225015946
-6.93000000000007 0.000976090699674707
-6.92000000000007 0.000985881212541288
-6.91000000000007 0.000995769731149431
-6.90000000000007 0.00100575723260139
-6.89000000000007 0.001015844703661
-6.88000000000007 0.00102603314084754
-6.87000000000007 0.00103632355053058
-6.86000000000007 0.00104671694902564
-6.85000000000007 0.00105721436269077
-6.84000000000007 0.00106781682802403
-6.83000000000007 0.0010785253917619
-6.82000000000007 0.00108934111097854
-6.81000000000007 0.00110026505318611
-6.80000000000007 0.00111129829643583
-6.79000000000007 0.00112244192942021
-6.78000000000007 0.00113369705157601
-6.77000000000007 0.00114506477318833
-6.76000000000007 0.00115654621549556
-6.75000000000007 0.00116814251079536
-6.74000000000007 0.00117985480255159
-6.73000000000007 0.00119168424550223
-6.72000000000007 0.00120363200576836
-6.71000000000007 0.00121569926096405
-6.70000000000007 0.00122788720030734
-6.69000000000007 0.0012401970247322
-6.68000000000007 0.00125262994700158
-6.67000000000007 0.00126518719182142
-6.66000000000007 0.00127786999595578
-6.65000000000007 0.00129067960834298
-6.64000000000007 0.00130361729021283
-6.63000000000007 0.00131668431520498
-6.62000000000007 0.00132988196948821
-6.61000000000007 0.00134321155188103
-6.60000000000007 0.00135667437397315
-6.59000000000007 0.00137027176024825
-6.58000000000007 0.00138400504820778
-6.57000000000007 0.00139787558849585
-6.56000000000007 0.00141188474502541
-6.55000000000007 0.00142603389510536
-6.54000000000007 0.00144032442956904
-6.53000000000007 0.00145475775290371
-6.52000000000007 0.0014693352833813
-6.51000000000007 0.00148405845319032
-6.50000000000007 0.00149892870856894
-6.49000000000007 0.00151394750993933
-6.48000000000008 0.00152911633204315
-6.47000000000008 0.0015444366640783
-6.46000000000008 0.00155991000983685
-6.45000000000008 0.00157553788784432
-6.44000000000008 0.00159132183150008
-6.43000000000008 0.00160726338921908
-6.42000000000008 0.00162336412457487
-6.41000000000008 0.00163962561644382
-6.40000000000008 0.00165604945915071
-6.39000000000008 0.00167263726261556
-6.38000000000008 0.00168939065250182
-6.37000000000008 0.00170631127036582
-6.36000000000008 0.00172340077380761
-6.35000000000008 0.00174066083662307
-6.34000000000008 0.00175809314895744
-6.33000000000008 0.00177569941746015
-6.32000000000008 0.00179348136544103
-6.31000000000008 0.00181144073302795
-6.30000000000008 0.00182957927732575
-6.29000000000008 0.00184789877257667
-6.28000000000008 0.00186640101032211
-6.27000000000008 0.00188508779956584
-6.26000000000008 0.00190396096693866
-6.25000000000008 0.00192302235686445
-6.24000000000008 0.00194227383172773
-6.23000000000008 0.00196171727204259
-6.22000000000008 0.00198135457662317
-6.21000000000008 0.0020011876627556
-6.20000000000008 0.00202121846637142
-6.19000000000008 0.00204144894222246
-6.18000000000008 0.00206188106405734
-6.17000000000008 0.00208251682479933
-6.16000000000008 0.00210335823672593
-6.15000000000008 0.00212440733164981
-6.14000000000008 0.00214566616110142
-6.13000000000008 0.00216713679651312
-6.12000000000008 0.0021888213294049
-6.11000000000008 0.00221072187157163
-6.10000000000008 0.00223284055527197
-6.09000000000008 0.00225517953341885
-6.08000000000008 0.00227774097977155
-6.07000000000008 0.00230052708912946
-6.06000000000008 0.00232354007752736
-6.05000000000008 0.00234678218243248
-6.04000000000008 0.00237025566294315
-6.03000000000008 0.00239396279998907
-6.02000000000008 0.00241790589653332
-6.01000000000009 0.00244208727777605
-6.00000000000009 0.00246650929135984
-5.99000000000009 0.00249117430757675
-5.98000000000009 0.00251608471957714
-5.97000000000009 0.00254124294358016
-5.96000000000009 0.00256665141908603
-5.95000000000009 0.00259231260908998
-5.94000000000009 0.00261822900029804
-5.93000000000009 0.00264440310334456
-5.92000000000009 0.00267083745301144
-5.91000000000009 0.00269753460844929
-5.90000000000009 0.00272449715340019
-5.89000000000009 0.00275172769642246
-5.88000000000009 0.00277922887111708
-5.87000000000009 0.00280700333635602
-5.86000000000009 0.00283505377651235
-5.85000000000009 0.00286338290169228
-5.84000000000009 0.00289199344796892
-5.83000000000009 0.00292088817761803
-5.82000000000009 0.00295006987935553
-5.81000000000009 0.00297954136857698
-5.80000000000009 0.00300930548759889
-5.79000000000009 0.00303936510590194
-5.78000000000009 0.0030697231203761
-5.77000000000009 0.0031003824555677
-5.76000000000009 0.00313134606392836
-5.75000000000009 0.0031626169260659
-5.74000000000009 0.0031941980509972
-5.73000000000009 0.0032260924764029
-5.72000000000009 0.00325830326888422
-5.71000000000009 0.00329083352422162
-5.70000000000009 0.00332368636763546
-5.69000000000009 0.00335686495404867
-5.68000000000009 0.00339037246835136
-5.67000000000009 0.00342421212566744
-5.66000000000009 0.00345838717162328
-5.65000000000009 0.0034929008826183
-5.64000000000009 0.00352775656609765
-5.63000000000009 0.00356295756082684
-5.62000000000009 0.00359850723716846
-5.61000000000009 0.00363440899736089
-5.60000000000009 0.00367066627579905
-5.59000000000009 0.00370728253931724
-5.58000000000009 0.00374426128747398
-5.57000000000009 0.00378160605283889
-5.56000000000009 0.00381932040128172
-5.55000000000009 0.00385740793226335
-5.5400000000001 0.00389587227912889
-5.5300000000001 0.00393471710940289
-5.5200000000001 0.00397394612508657
-5.5100000000001 0.00401356306295717
-5.5000000000001 0.00405357169486938
-5.4900000000001 0.00409397582805885
-5.4800000000001 0.00413477930544777
-5.4700000000001 0.0041759860059526
-5.4600000000001 0.00421759984479386
-5.4500000000001 0.00425962477380801
-5.4400000000001 0.00430206478176141
-5.4300000000001 0.00434492389466649
-5.4200000000001 0.00438820617609987
-5.4100000000001 0.00443191572752267
-5.4000000000001 0.00447605668860295
-5.3900000000001 0.00452063323754014
-5.3800000000001 0.00456564959139169
-5.3700000000001 0.00461111000640178
-5.3600000000001 0.00465701877833205
-5.3500000000001 0.00470338024279458
-5.3400000000001 0.00475019877558681
-5.3300000000001 0.00479747879302867
-5.3200000000001 0.00484522475230172
-5.3100000000001 0.00489344115179042
-5.3000000000001 0.00494213253142546
-5.2900000000001 0.00499130347302918
-5.2800000000001 0.00504095860066301
-5.2700000000001 0.00509110258097708
-5.2600000000001 0.00514174012356175
-5.2500000000001 0.00519287598130133
-5.2400000000001 0.00524451495072972
-5.2300000000001 0.00529666187238817
-5.2200000000001 0.00534932163118506
-5.2100000000001 0.00540249915675761
-5.2000000000001 0.00545619942383574
-5.1900000000001 0.00551042745260779
-5.1800000000001 0.00556518830908832
-5.1700000000001 0.00562048710548782
-5.1600000000001 0.00567632900058444
-5.1500000000001 0.00573271920009761
-5.1400000000001 0.00578966295706365
-5.1300000000001 0.00584716557221323
-5.1200000000001 0.00590523239435079
-5.1100000000001 0.00596386882073584
-5.1000000000001 0.00602308029746609
-5.0900000000001 0.00608287231986247
-5.0800000000001 0.00614325043285599
-5.07000000000011 0.00620422023137634
-5.06000000000011 0.00626578736074238
-5.05000000000011 0.00632795751705434
-5.04000000000011 0.00639073644758776
-5.03000000000011 0.00645412995118922
-5.02000000000011 0.00651814387867368
-5.01000000000011 0.00658278413322356
-5.00000000000011 0.00664805667078946
-4.99000000000011 0.00671396750049245
-4.98000000000011 0.00678052268502804
-4.97000000000011 0.00684772834107163
-4.96000000000011 0.00691559063968553
-4.95000000000011 0.00698411580672751
-4.94000000000011 0.00705331012326071
-4.93000000000011 0.00712317992596515
-4.92000000000011 0.00719373160755047
-4.91000000000011 0.00726497161717013
-4.90000000000011 0.00733690646083696
-4.89000000000011 0.00740954270183985
-4.88000000000011 0.00748288696116186
-4.87000000000011 0.00755694591789942
-4.86000000000011 0.00763172630968274
-4.85000000000011 0.00770723493309729
-4.84000000000011 0.00778347864410641
-4.83000000000011 0.00786046435847488
-4.82000000000011 0.00793819905219353
-4.81000000000011 0.00801668976190469
-4.80000000000011 0.00809594358532861
-4.79000000000011 0.00817596768169063
-4.78000000000011 0.00825676927214909
-4.77000000000011 0.00833835564022406
-4.76000000000011 0.00842073413222659
-4.75000000000011 0.00850391215768861
-4.74000000000011 0.00858789718979339
-4.73000000000011 0.00867269676580638
-4.72000000000011 0.00875831848750652
-4.71000000000011 0.00884477002161786
-4.70000000000011 0.00893205910024141
-4.69000000000011 0.00902019352128721
-4.68000000000011 0.00910918114890658
-4.67000000000011 0.00919902991392425
-4.66000000000011 0.00928974781427065
-4.65000000000011 0.00938134291541394
-4.64000000000011 0.00947382335079195
-4.63000000000011 0.00956719732224375
-4.62000000000011 0.00966147310044097
-4.61000000000011 0.00975665902531856
-4.60000000000012 0.00985276350650509
-4.59000000000012 0.0099497950237524
-4.58000000000012 0.0100477621273645
-4.57000000000012 0.0101466734386257
-4.56000000000012 0.0102465376502278
-4.55000000000012 0.0103473635266963
-4.54000000000012 0.0104491599048151
-4.53000000000012 0.0105519356940508
-4.52000000000012 0.010655699876975
-4.51000000000012 0.0107604615096851
-4.50000000000012 0.010866229722224
-4.49000000000012 0.0109730137189977
-4.48000000000012 0.0110808227791914
-4.47000000000012 0.0111896662571833
-4.46000000000012 0.011299553582957
-4.45000000000012 0.0114104942625109
-4.44000000000012 0.0115224978782658
-4.43000000000012 0.0116355740894699
-4.42000000000012 0.0117497326326015
-4.41000000000012 0.0118649833217684
-4.40000000000012 0.0119813360491051
-4.39000000000012 0.0120988007851661
-4.38000000000012 0.0122173875793174
-4.37000000000012 0.0123371065601229
-4.36000000000012 0.012457967935729
-4.35000000000012 0.0125799819942446
-4.34000000000012 0.0127031591041174
-4.33000000000012 0.0128275097145066
-4.32000000000012 0.012953044355651
-4.31000000000012 0.0130797736392331
-4.30000000000012 0.0132077082587386
-4.29000000000012 0.0133368589898112
-4.28000000000012 0.0134672366906026
-4.27000000000012 0.0135988523021174
-4.26000000000012 0.0137317168485527
-4.25000000000012 0.0138658414376323
-4.24000000000012 0.0140012372609353
-4.23000000000012 0.0141379155942188
-4.22000000000012 0.0142758877977344
-4.21000000000012 0.0144151653165388
-4.20000000000012 0.0145557596807975
-4.19000000000012 0.0146976825060817
-4.18000000000012 0.0148409454936589
-4.17000000000012 0.0149855604307756
-4.16000000000012 0.0151315391909327
-4.15000000000012 0.015278893734154
-4.14000000000012 0.0154276361072454
-4.13000000000013 0.0155777784440472
-4.12000000000013 0.0157293329656775
-4.11000000000013 0.0158823119807666
-4.10000000000013 0.0160367278856832
-4.09000000000013 0.0161925931647504
-4.08000000000013 0.0163499203904535
-4.07000000000013 0.0165087222236366
-4.06000000000013 0.01666901141369
-4.05000000000013 0.0168308007987275
-4.04000000000013 0.0169941033057521
-4.03000000000013 0.0171589319508114
-4.02000000000013 0.017325299839142
-4.01000000000013 0.0174932201653014
-4.00000000000013 0.0176627062132889
-3.99000000000013 0.017833771356654
-3.98000000000013 0.0180064290585917
-3.97000000000013 0.0181806928720262
-3.96000000000013 0.01835657643968
-3.95000000000013 0.0185340934941301
-3.94000000000013 0.0187132578578505
-3.93000000000013 0.01889408344324
-3.92000000000013 0.0190765842526351
-3.91000000000013 0.0192607743783089
-3.90000000000013 0.0194466680024534
-3.89000000000013 0.0196342793971469
-3.88000000000013 0.0198236229243049
-3.87000000000013 0.020014713035614
-3.86000000000013 0.0202075642724498
-3.85000000000013 0.0204021912657767
-3.84000000000013 0.0205986087360299
-3.83000000000013 0.0207968314929795
-3.82000000000013 0.0209968744355757
-3.81000000000013 0.0211987525517751
-3.80000000000013 0.0214024809183474
-3.79000000000013 0.0216080747006619
-3.78000000000013 0.021815549152454
-3.77000000000013 0.0220249196155703
-3.76000000000013 0.0222362015196927
-3.75000000000013 0.0224494103820406
-3.74000000000013 0.0226645618070506
-3.73000000000013 0.0228816714860334
-3.72000000000013 0.0231007551968076
-3.71000000000013 0.0233218288033092
-3.70000000000013 0.0235449082551775
-3.69000000000013 0.0237700095873154
-3.68000000000013 0.0239971489194244
-3.67000000000013 0.0242263424555146
-3.66000000000014 0.0244576064833865
-3.65000000000014 0.0246909573740878
-3.64000000000014 0.0249264115813407
-3.63000000000014 0.0251639856409427
-3.62000000000014 0.0254036961701375
-3.61000000000014 0.0256455598669572
-3.60000000000014 0.0258895935095347
-3.59000000000014 0.0261358139553849
-3.58000000000014 0.026384238140656
-3.57000000000014 0.0266348830793481
-3.56000000000014 0.0268877658624999
-3.55000000000014 0.0271429036573425
-3.54000000000014 0.0274003137064197
-3.53000000000014 0.0276600133266741
-3.52000000000014 0.0279220199084981
-3.51000000000014 0.0281863509147495
-3.50000000000014 0.0284530238797319
-3.49000000000014 0.0287220564081363
-3.48000000000014 0.0289934661739474
-3.47000000000014 0.0292672709193102
-3.46000000000014 0.0295434884533587
-3.45000000000014 0.0298221366510041
-3.44000000000014 0.030103233451684
-3.43000000000014 0.0303867968580691
-3.42000000000014 0.0306728449347302
-3.41000000000014 0.0309613958067607
-3.40000000000014 0.0312524676583575
-3.39000000000014 0.0315460787313575
-3.38000000000014 0.0318422473237293
-3.37000000000014 0.0321409917880202
-3.36000000000014 0.0324423305297562
-3.35000000000014 0.0327462820057962
-3.34000000000014 0.0330528647226375
-3.33000000000014 0.0333620972346739
-3.32000000000014 0.0336739981424038
-3.31000000000014 0.0339885860905889
-3.30000000000014 0.0343058797663619
-3.29000000000014 0.0346258978972828
-3.28000000000014 0.0349486592493428
-3.27000000000014 0.0352741826249148
-3.26000000000014 0.0356024868606508
-3.25000000000014 0.0359335908253234
-3.24000000000014 0.0362675134176124
-3.23000000000014 0.0366042735638354
-3.22000000000014 0.03694389021562
-3.21000000000014 0.037286382347519
-3.20000000000014 0.0376317689545663
-3.19000000000015 0.0379800690497735
-3.18000000000015 0.038331301661565
-3.17000000000015 0.0386854858311526
-3.16000000000015 0.0390426406098482
-3.15000000000015 0.0394027850563117
-3.14000000000015 0.0397659382337374
-3.13000000000015 0.0401321192069734
-3.12000000000015 0.0405013470395771
-3.11000000000015 0.0408736407908032
-3.10000000000015 0.0412490195125248
-3.09000000000015 0.0416275022460866
-3.08000000000015 0.0420091080190887
-3.07000000000015 0.0423938558421006
-3.06000000000015 0.0427817647053051
-3.05000000000015 0.0431728535750699
-3.04000000000015 0.0435671413904474
-3.03000000000015 0.0439646470596014
-3.02000000000015 0.0443653894561594
-3.01000000000015 0.0447693874154902
-3.00000000000015 0.045176659730906
-2.99000000000015 0.0455872251497883
-2.98000000000015 0.0460011023696356
-2.97000000000015 0.0464183100340341
-2.96000000000015 0.046838866728549
-2.95000000000015 0.0472627909765359
-2.94000000000015 0.0476901012348725
-2.93000000000015 0.0481208158896084
-2.92000000000015 0.0485549532515331
-2.91000000000015 0.0489925315516617
-2.90000000000015 0.0494335689366365
-2.89000000000015 0.0498780834640447
-2.88000000000015 0.0503260930976513
-2.87000000000015 0.0507776157025462
-2.86000000000015 0.0512326690402052
-2.85000000000015 0.0516912707634634
-2.84000000000015 0.0521534384114021
-2.83000000000015 0.052619189404146
-2.82000000000015 0.053088541037573
-2.81000000000015 0.0535615104779328
-2.80000000000015 0.054038114756377
-2.79000000000015 0.0545183707633972
-2.78000000000015 0.0550022952431731
-2.77000000000015 0.0554899047878273
-2.76000000000015 0.0559812158315894
-2.75000000000015 0.0564762446448664
-2.74000000000015 0.0569750073282199
-2.73000000000015 0.0574775198062503
-2.72000000000016 0.0579837978213862
-2.71000000000016 0.0584938569275797
-2.70000000000016 0.0590077124839072
-2.69000000000016 0.0595253796480746
-2.68000000000016 0.0600468733698273
-2.67000000000016 0.0605722083842649
-2.66000000000016 0.0611013992050595
-2.65000000000016 0.0616344601175782
-2.64000000000016 0.0621714051719097
-2.63000000000016 0.0627122481757938
-2.62000000000016 0.0632570026874557
-2.61000000000016 0.0638056820083426
-2.60000000000016 0.0643582991757648
-2.59000000000016 0.0649148669554398
-2.58000000000016 0.065475397833941
-2.57000000000016 0.0660399040110491
-2.56000000000016 0.0666083973920082
-2.55000000000016 0.0671808895796864
-2.54000000000016 0.0677573918666404
-2.53000000000016 0.0683379152270849
-2.52000000000016 0.0689224703087682
-2.51000000000016 0.0695110674247519
-2.50000000000016 0.0701037165450986
-2.49000000000016 0.0707004272884647
-2.48000000000016 0.0713012089136014
-2.47000000000016 0.0719060703107633
-2.46000000000016 0.0725150199930261
-2.45000000000016 0.0731280660875131
-2.44000000000016 0.0737452163265329
-2.43000000000016 0.0743664780386271
-2.42000000000016 0.0749918581395316
-2.41000000000016 0.07562136312305
-2.40000000000016 0.0762549990518419
-2.39000000000016 0.0768927715481268
-2.38000000000016 0.077534685784304
-2.37000000000016 0.0781807464734922
-2.36000000000016 0.0788309578599868
-2.35000000000016 0.0794853237096395
-2.34000000000016 0.0801438473001597
-2.33000000000016 0.0808065314113411
-2.32000000000016 0.0814733783152128
-2.31000000000016 0.0821443897661194
-2.30000000000016 0.08281956699073
-2.29000000000016 0.0834989106779796
-2.28000000000016 0.0841824209689428
-2.27000000000016 0.0848700974466451
-2.26000000000016 0.0855619391258106
-2.25000000000017 0.0862579444425515
-2.24000000000017 0.0869581112439992
-2.23000000000017 0.0876624367778819
-2.22000000000017 0.0883709176820497
-2.21000000000017 0.0890835499739499
-2.20000000000017 0.0898003290400568
-2.19000000000017 0.0905212496252571
-2.18000000000017 0.0912463058221956
-2.17000000000017 0.0919754910605825
-2.16000000000017 0.0927087980964675
-2.15000000000017 0.0934462190014821
-2.14000000000017 0.0941877451520554
-2.13000000000017 0.0949333672186054
-2.12000000000017 0.0956830751547101
-2.11000000000017 0.0964368581862625
-2.10000000000017 0.0971947048006126
-2.09000000000017 0.0979566027357014
-2.08000000000017 0.0987225389691893
-2.07000000000017 0.0994924997075856
-2.06000000000017 0.10026647037538
-2.05000000000017 0.101044435604187
-2.04000000000017 0.101826379221894
-2.03000000000017 0.10261228424184
-2.02000000000017 0.103402132852006
-2.01000000000017 0.104195906404236
-2.00000000000017 0.104993585403493
-1.99000000000017 0.10579514949715
-1.98000000000017 0.106600577464322
-1.97000000000017 0.10740984720525
-1.96000000000017 0.108222935730737
-1.95000000000017 0.10903981915164
-1.94000000000017 0.109860472668431
-1.93000000000017 0.110684870560827
-1.92000000000017 0.111512986177498
-1.91000000000017 0.112344791925852
-1.90000000000017 0.113180259261917
-1.89000000000017 0.114019358680309
-1.88000000000017 0.114862059704314
-1.87000000000017 0.11570833087606
-1.86000000000017 0.116558139746823
-1.85000000000017 0.117411452867435
-1.84000000000017 0.11826823577884
-1.83000000000017 0.119128453002766
-1.82000000000017 0.119992068032549
-1.81000000000017 0.120859043324109
-1.80000000000017 0.12172934028707
-1.79000000000018 0.122602919276058
-1.78000000000018 0.123479739582157
-1.77000000000018 0.124359759424554
-1.76000000000018 0.125242935942362
-1.75000000000018 0.12612922518664
-1.74000000000018 0.127018582112616
-1.73000000000018 0.127910960572117
-1.72000000000018 0.128806313306216
-1.71000000000018 0.129704591938102
-1.70000000000018 0.130605746966192
-1.69000000000018 0.131509727757472
-1.68000000000018 0.132416482541096
-1.67000000000018 0.133325958402241
-1.66000000000018 0.134238101276224
-1.65000000000018 0.135152855942898
-1.64000000000018 0.13607016602133
-1.63000000000018 0.136989973964766
-1.62000000000018 0.1379122210559
-1.61000000000018 0.138836847402449
-1.60000000000018 0.139763791933044
-1.59000000000018 0.140692992393448
-1.58000000000018 0.141624385343103
-1.57000000000018 0.142557906152025
-1.56000000000018 0.143493488998045
-1.55000000000018 0.144431066864413
-1.54000000000018 0.145370571537762
-1.53000000000018 0.146311933606456
-1.52000000000018 0.147255082459323
-1.51000000000018 0.148199946284771
-1.50000000000018 0.149146452070316
-1.49000000000018 0.150094525602513
-1.48000000000018 0.15104409146731
-1.47000000000018 0.151995073050823
-1.46000000000018 0.152947392540551
-1.45000000000018 0.153900970927024
-1.44000000000018 0.154855728005916
-1.43000000000018 0.1558115823806
-1.42000000000018 0.15676845146518
-1.41000000000018 0.157726251487994
-1.40000000000018 0.158684897495597
-1.39000000000018 0.159644303357233
-1.38000000000018 0.160604381769803
-1.37000000000018 0.161565044263338
-1.36000000000018 0.162526201206977
-1.35000000000018 0.163487761815469
-1.34000000000018 0.164449634156194
-1.33000000000018 0.165411725156714
-1.32000000000019 0.16637394061286
-1.31000000000019 0.167336185197365
-1.30000000000019 0.168298362469042
-1.29000000000019 0.169260374882516
-1.28000000000019 0.170222123798516
-1.27000000000019 0.171183509494729
-1.26000000000019 0.172144431177231
-1.25000000000019 0.173104786992479
-1.24000000000019 0.174064474039896
-1.23000000000019 0.175023388385026
-1.22000000000019 0.175981425073282
-1.21000000000019 0.176938478144285
-1.20000000000019 0.177894440646788
-1.19000000000019 0.178849204654206
-1.18000000000019 0.17980266128074
-1.17000000000019 0.180754700698095
-1.16000000000019 0.181705212152816
-1.15000000000019 0.182654083984214
-1.14000000000019 0.183601203642902
-1.13000000000019 0.184546457709944
-1.12000000000019 0.185489731916598
-1.11000000000019 0.186430911164675
-1.10000000000019 0.187369879547503
-1.09000000000019 0.188306520371493
-1.08000000000019 0.189240716178315
-1.07000000000019 0.190172348767669
-1.06000000000019 0.191101299220667
-1.05000000000019 0.192027447923806
-1.04000000000019 0.192950674593537
-1.03000000000019 0.19387085830143
-1.02000000000019 0.194787877499929
-1.01000000000019 0.195701610048683
-1.00000000000019 0.196611933241464
-0.990000000000192 0.197518723833661
-0.980000000000192 0.198421858070333
-0.970000000000192 0.199321211714841
-0.960000000000193 0.200216660078017
-0.950000000000193 0.2011080780479
-0.940000000000193 0.201995340120006
-0.930000000000193 0.202878320428127
-0.920000000000194 0.203756892775667
-0.910000000000194 0.204630930667485
-0.900000000000194 0.205500307342247
-0.890000000000194 0.20636489580528
-0.880000000000194 0.207224568861907
-0.870000000000195 0.208079199151258
-0.860000000000195 0.208928659180549
-0.850000000000195 0.209772821359809
-0.840000000000195 0.210611558037049
-0.830000000000195 0.211444741533861
-0.820000000000196 0.212272244181421
-0.810000000000196 0.213093938356905
-0.800000000000196 0.213909696520278
-0.790000000000196 0.214719391251464
-0.780000000000197 0.215522895287861
-0.770000000000197 0.216320081562208
-0.760000000000197 0.217110823240764
-0.750000000000197 0.217894993761799
-0.740000000000197 0.218672466874377
-0.730000000000198 0.219443116677408
-0.720000000000198 0.220206817658955
-0.710000000000198 0.220963444735778
-0.700000000000198 0.221712873293094
-0.690000000000198 0.222454979224541
-0.680000000000199 0.223189638972311
-0.670000000000199 0.223916729567451
-0.660000000000199 0.2246361286703
-0.650000000000199 0.22534771461104
-0.6400000000002 0.226051366430354
-0.6300000000002 0.226746963920154
-0.6200000000002 0.227434387664359
-0.6100000000002 0.22811351907972
-0.6000000000002 0.228784240456644
-0.590000000000201 0.22944643500001
-0.580000000000201 0.230099986869955
-0.570000000000201 0.230744781222599
-0.560000000000201 0.231380704250695
-0.550000000000201 0.232007643224171
-0.540000000000202 0.232625486530551
-0.530000000000202 0.233234123715227
-0.520000000000202 0.233833445521553
-0.510000000000202 0.234423343930749
-0.500000000000203 0.235003712201583
-0.490000000000203 0.235574444909803
-0.480000000000203 0.236135437987314
-0.470000000000203 0.236686588761049
-0.460000000000203 0.237227795991536
-0.450000000000204 0.237758959911119
-0.440000000000204 0.238279982261815
-0.430000000000204 0.238790766332791
-0.420000000000204 0.239291216997425
-0.410000000000204 0.239781240749927
-0.400000000000205 0.240260745741519
-0.390000000000205 0.240729641816115
-0.380000000000205 0.241187840545507
-0.370000000000205 0.241635255264023
-0.360000000000205 0.242071801102636
-0.350000000000206 0.242497395022491
-0.340000000000206 0.242911955847854
-0.330000000000206 0.243315404298426
-0.320000000000206 0.243707663021034
-0.310000000000207 0.244088656620649
-0.300000000000207 0.244458311690738
-0.290000000000207 0.2448165568429
-0.280000000000207 0.245163322735788
-0.270000000000207 0.24549854210329
-0.260000000000208 0.245822149781947
-0.250000000000208 0.246134082737592
-0.240000000000208 0.246434280091189
-0.230000000000208 0.246722683143857
-0.220000000000208 0.246999235401063
-0.210000000000209 0.247263882595959
-0.200000000000209 0.247516572711855
-0.190000000000209 0.247757256003808
-0.180000000000209 0.247985885019316
-0.17000000000021 0.24820241461809
-0.16000000000021 0.248406801990912
-0.15000000000021 0.248599006677539
-0.14000000000021 0.248778990583666
-0.13000000000021 0.248946717996915
-0.120000000000211 0.249102155601847
-0.110000000000211 0.249245272493995
-0.100000000000211 0.249376040192889
-0.0900000000002112 0.24949443265408
-0.0800000000002115 0.249600426280142
-0.0700000000002117 0.249693999930662
-0.0600000000002119 0.249775134931181
-0.0500000000002121 0.249843815081115
-0.0400000000002123 0.249900026660622
-0.0300000000002125 0.249943758436424
-0.0200000000002127 0.249975001666572
-0.0100000000002129 0.249993750104165
-2.1316282072803e-13 0.25
0.00999999999978662 0.249993750104165
0.0199999999997864 0.249975001666573
0.0299999999997862 0.249943758436425
0.039999999999786 0.249900026660625
0.0499999999997858 0.249843815081118
0.0599999999997856 0.249775134931184
0.0699999999997853 0.249693999930665
0.0799999999997851 0.249600426280147
0.0899999999997849 0.249494432654084
0.0999999999997847 0.249376040192895
0.109999999999784 0.249245272494001
0.119999999999784 0.249102155601853
0.129999999999784 0.248946717996922
0.139999999999784 0.248778990583674
0.149999999999784 0.248599006677547
0.159999999999783 0.24840680199092
0.169999999999783 0.248202414618099
0.179999999999783 0.247985885019325
0.189999999999783 0.247757256003818
0.199999999999783 0.247516572711865
0.209999999999782 0.24726388259597
0.219999999999782 0.246999235401074
0.229999999999782 0.246722683143869
0.239999999999782 0.246434280091201
0.249999999999782 0.246134082737605
0.259999999999781 0.245822149781961
0.269999999999781 0.245498542103304
0.279999999999781 0.245163322735802
0.289999999999781 0.244816556842915
0.29999999999978 0.244458311690754
0.30999999999978 0.244088656620665
0.31999999999978 0.24370766302105
0.32999999999978 0.243315404298443
0.33999999999978 0.242911955847872
0.349999999999779 0.242497395022509
0.359999999999779 0.242071801102654
0.369999999999779 0.241635255264042
0.379999999999779 0.241187840545526
0.389999999999779 0.240729641816135
0.399999999999778 0.24026074574154
0.409999999999778 0.239781240749948
0.419999999999778 0.239291216997446
0.429999999999778 0.238790766332813
0.439999999999777 0.238279982261837
0.449999999999777 0.237758959911141
0.459999999999777 0.237227795991559
0.469999999999777 0.236686588761073
0.479999999999777 0.236135437987338
0.489999999999776 0.235574444909827
0.499999999999776 0.235003712201607
0.509999999999776 0.234423343930774
0.519999999999776 0.233833445521578
0.529999999999776 0.233234123715253
0.539999999999775 0.232625486530577
0.549999999999775 0.232007643224197
0.559999999999775 0.231380704250722
0.569999999999775 0.230744781222626
0.579999999999774 0.230099986869982
0.589999999999774 0.229446435000038
0.599999999999774 0.228784240456672
0.609999999999774 0.228113519079749
0.619999999999774 0.227434387664388
0.629999999999773 0.226746963920183
0.639999999999773 0.226051366430384
0.649999999999773 0.22534771461107
0.659999999999773 0.22463612867033
0.669999999999773 0.223916729567482
0.679999999999772 0.223189638972342
0.689999999999772 0.222454979224573
0.699999999999772 0.221712873293126
0.709999999999772 0.22096344473581
0.719999999999771 0.220206817658987
0.729999999999771 0.219443116677441
0.739999999999771 0.21867246687441
0.749999999999771 0.217894993761832
0.759999999999771 0.217110823240797
0.76999999999977 0.216320081562242
0.77999999999977 0.215522895287895
0.78999999999977 0.214719391251498
0.79999999999977 0.213909696520313
0.80999999999977 0.21309393835694
0.819999999999769 0.212272244181457
0.829999999999769 0.211444741533896
0.839999999999769 0.210611558037085
0.849999999999769 0.209772821359844
0.859999999999769 0.208928659180585
0.869999999999768 0.208079199151294
0.879999999999768 0.207224568861943
0.889999999999768 0.206364895805317
0.899999999999768 0.205500307342284
0.909999999999767 0.204630930667522
0.919999999999767 0.203756892775705
0.929999999999767 0.202878320428165
0.939999999999767 0.201995340120044
0.949999999999767 0.201108078047938
0.959999999999766 0.200216660078055
0.969999999999766 0.199321211714879
0.979999999999766 0.198421858070372
0.989999999999766 0.197518723833699
0.999999999999766 0.196611933241503
1.00999999999977 0.195701610048722
1.01999999999977 0.194787877499968
1.02999999999976 0.19387085830147
1.03999999999976 0.192950674593576
1.04999999999976 0.192027447923845
1.05999999999976 0.191101299220707
1.06999999999976 0.190172348767709
1.07999999999976 0.189240716178354
1.08999999999976 0.188306520371533
1.09999999999976 0.187369879547543
1.10999999999976 0.186430911164715
1.11999999999976 0.185489731916638
1.12999999999976 0.184546457709985
1.13999999999976 0.183601203642943
1.14999999999976 0.182654083984254
1.15999999999976 0.181705212152857
1.16999999999976 0.180754700698136
1.17999999999976 0.17980266128078
1.18999999999976 0.178849204654247
1.19999999999976 0.177894440646829
1.20999999999976 0.176938478144325
1.21999999999976 0.175981425073323
1.22999999999976 0.175023388385067
1.23999999999976 0.174064474039937
1.24999999999976 0.17310478699252
1.25999999999976 0.172144431177272
1.26999999999976 0.17118350949477
1.27999999999976 0.170222123798557
1.28999999999976 0.169260374882557
1.29999999999976 0.168298362469083
1.30999999999976 0.167336185197406
1.31999999999976 0.166373940612901
1.32999999999976 0.165411725156755
1.33999999999976 0.164449634156236
1.34999999999976 0.16348776181551
1.35999999999976 0.162526201207018
1.36999999999976 0.161565044263379
1.37999999999976 0.160604381769844
1.38999999999976 0.159644303357274
1.39999999999976 0.158684897495638
1.40999999999976 0.157726251488035
1.41999999999976 0.15676845146522
1.42999999999976 0.15581158238064
1.43999999999976 0.154855728005957
1.44999999999976 0.153900970927065
1.45999999999976 0.152947392540591
1.46999999999976 0.151995073050864
1.47999999999976 0.15104409146735
1.48999999999976 0.150094525602553
1.49999999999975 0.149146452070356
1.50999999999975 0.148199946284812
1.51999999999975 0.147255082459364
1.52999999999975 0.146311933606497
1.53999999999975 0.145370571537802
1.54999999999975 0.144431066864453
1.55999999999975 0.143493488998085
1.56999999999975 0.142557906152064
1.57999999999975 0.141624385343142
1.58999999999975 0.140692992393487
1.59999999999975 0.139763791933084
1.60999999999975 0.138836847402489
1.61999999999975 0.13791222105594
1.62999999999975 0.136989973964806
1.63999999999975 0.136070166021369
1.64999999999975 0.135152855942937
1.65999999999975 0.134238101276263
1.66999999999975 0.13332595840228
1.67999999999975 0.132416482541135
1.68999999999975 0.131509727757511
1.69999999999975 0.130605746966231
1.70999999999975 0.129704591938141
1.71999999999975 0.128806313306254
1.72999999999975 0.127910960572155
1.73999999999975 0.127018582112654
1.74999999999975 0.126129225186677
1.75999999999975 0.125242935942399
1.76999999999975 0.124359759424592
1.77999999999975 0.123479739582195
1.78999999999975 0.122602919276095
1.79999999999975 0.121729340287107
1.80999999999975 0.120859043324146
1.81999999999975 0.119992068032586
1.82999999999975 0.119128453002803
1.83999999999975 0.118268235778877
1.84999999999975 0.117411452867472
1.85999999999975 0.116558139746859
1.86999999999975 0.115708330876096
1.87999999999975 0.11486205970435
1.88999999999975 0.114019358680345
1.89999999999975 0.113180259261952
1.90999999999975 0.112344791925888
1.91999999999975 0.111512986177534
1.92999999999975 0.110684870560863
1.93999999999975 0.109860472668466
1.94999999999975 0.109039819151674
1.95999999999975 0.108222935730771
1.96999999999974 0.107409847205284
1.97999999999974 0.106600577464356
1.98999999999974 0.105795149497184
1.99999999999974 0.104993585403527
2.00999999999974 0.10419590640427
2.01999999999974 0.10340213285204
2.02999999999974 0.102612284241874
2.03999999999974 0.101826379221928
2.04999999999974 0.10104443560422
2.05999999999974 0.100266470375414
2.06999999999974 0.0994924997076186
2.07999999999974 0.0987225389692221
2.08999999999974 0.0979566027357339
2.09999999999974 0.0971947048006449
2.10999999999974 0.0964368581862946
2.11999999999974 0.0956830751547421
2.12999999999974 0.0949333672186374
2.13999999999974 0.0941877451520872
2.14999999999974 0.0934462190015136
2.15999999999974 0.0927087980964988
2.16999999999974 0.0919754910606137
2.17999999999974 0.0912463058222266
2.18999999999974 0.0905212496252879
2.19999999999974 0.0898003290400874
2.20999999999974 0.0890835499739805
2.21999999999974 0.0883709176820801
2.22999999999974 0.0876624367779121
2.23999999999974 0.086958111244029
2.24999999999974 0.0862579444425812
2.25999999999974 0.0855619391258401
2.26999999999974 0.0848700974466744
2.27999999999974 0.0841824209689721
2.28999999999974 0.0834989106780086
2.29999999999974 0.0828195669907589
2.30999999999974 0.0821443897661481
2.31999999999974 0.0814733783152412
2.32999999999974 0.0808065314113695
2.33999999999974 0.0801438473001878
2.34999999999974 0.0794853237096675
2.35999999999974 0.0788309578600146
2.36999999999974 0.0781807464735197
2.37999999999974 0.0775346857843314
2.38999999999974 0.076892771548154
2.39999999999974 0.0762549990518691
2.40999999999974 0.075621363123077
2.41999999999974 0.0749918581395583
2.42999999999974 0.0743664780386537
2.43999999999973 0.0737452163265592
2.44999999999973 0.0731280660875394
2.45999999999973 0.0725150199930522
2.46999999999973 0.0719060703107891
2.47999999999973 0.071301208913627
2.48999999999973 0.0707004272884902
2.49999999999973 0.0701037165451241
2.50999999999973 0.0695110674247772
2.51999999999973 0.0689224703087932
2.52999999999973 0.0683379152271098
2.53999999999973 0.067757391866665
2.54999999999973 0.067180889579711
2.55999999999973 0.0666083973920326
2.56999999999973 0.0660399040110732
2.57999999999973 0.065475397833965
2.58999999999973 0.0649148669554637
2.59999999999973 0.0643582991757883
2.60999999999973 0.0638056820083661
2.61999999999973 0.063257002687479
2.62999999999973 0.062712248175817
2.63999999999973 0.0621714051719325
2.64999999999973 0.0616344601176011
2.65999999999973 0.0611013992050823
2.66999999999973 0.0605722083842873
2.67999999999973 0.0600468733698496
2.68999999999973 0.0595253796480968
2.69999999999973 0.0590077124839292
2.70999999999973 0.0584938569276016
2.71999999999973 0.0579837978214079
2.72999999999973 0.0574775198062718
2.73999999999973 0.0569750073282412
2.74999999999973 0.0564762446448875
2.75999999999973 0.0559812158316104
2.76999999999973 0.0554899047878481
2.77999999999973 0.0550022952431938
2.78999999999973 0.0545183707634179
2.79999999999973 0.0540381147563974
2.80999999999973 0.053561510477953
2.81999999999973 0.0530885410375931
2.82999999999973 0.052619189404166
2.83999999999973 0.0521534384114218
2.84999999999973 0.0516912707634829
2.85999999999973 0.0512326690402247
2.86999999999973 0.0507776157025656
2.87999999999973 0.0503260930976704
2.88999999999973 0.0498780834640637
2.89999999999973 0.0494335689366555
2.90999999999972 0.0489925315516805
2.91999999999972 0.0485549532515517
2.92999999999972 0.0481208158896268
2.93999999999972 0.0476901012348907
2.94999999999972 0.0472627909765541
2.95999999999972 0.0468388667285669
2.96999999999972 0.046418310034052
2.97999999999972 0.0460011023696533
2.98999999999972 0.0455872251498058
2.99999999999972 0.0451766597309236
3.00999999999972 0.0447693874155074
3.01999999999972 0.0443653894561766
3.02999999999972 0.0439646470596185
3.03999999999972 0.0435671413904641
3.04999999999972 0.0431728535750866
3.05999999999972 0.0427817647053217
3.06999999999972 0.0423938558421171
3.07999999999972 0.042009108019105
3.08999999999972 0.0416275022461027
3.09999999999972 0.0412490195125409
3.10999999999972 0.0408736407908192
3.11999999999972 0.0405013470395929
3.12999999999972 0.040132119206989
3.13999999999972 0.0397659382337529
3.14999999999972 0.0394027850563271
3.15999999999972 0.0390426406098635
3.16999999999972 0.0386854858311679
3.17999999999972 0.0383313016615799
3.18999999999972 0.0379800690497884
3.19999999999972 0.0376317689545812
3.20999999999972 0.0372863823475336
3.21999999999972 0.0369438902156345
3.22999999999972 0.0366042735638499
3.23999999999972 0.0362675134176267
3.24999999999972 0.0359335908253376
3.25999999999972 0.035602486860665
3.26999999999972 0.0352741826249287
3.27999999999972 0.0349486592493565
3.28999999999972 0.0346258978972966
3.29999999999972 0.0343058797663755
3.30999999999972 0.0339885860906023
3.31999999999972 0.0336739981424172
3.32999999999972 0.0333620972346871
3.33999999999972 0.0330528647226507
3.34999999999972 0.0327462820058092
3.35999999999972 0.0324423305297692
3.36999999999972 0.0321409917880329
3.37999999999971 0.031842247323742
3.38999999999971 0.0315460787313701
3.39999999999971 0.0312524676583699
3.40999999999971 0.0309613958067732
3.41999999999971 0.0306728449347424
3.42999999999971 0.0303867968580812
3.43999999999971 0.030103233451696
3.44999999999971 0.0298221366510161
3.45999999999971 0.0295434884533705
3.46999999999971 0.0292672709193219
3.47999999999971 0.0289934661739589
3.48999999999971 0.0287220564081479
3.49999999999971 0.0284530238797433
3.50999999999971 0.0281863509147609
3.51999999999971 0.0279220199085093
3.52999999999971 0.0276600133266854
3.53999999999971 0.0274003137064309
3.54999999999971 0.0271429036573535
3.55999999999971 0.0268877658625106
3.56999999999971 0.0266348830793588
3.57999999999971 0.0263842381406666
3.58999999999971 0.0261358139553953
3.59999999999971 0.025889593509545
3.60999999999971 0.0256455598669677
3.61999999999971 0.0254036961701478
3.62999999999971 0.0251639856409529
3.63999999999971 0.0249264115813508
3.64999999999971 0.0246909573740977
3.65999999999971 0.0244576064833965
3.66999999999971 0.0242263424555245
3.67999999999971 0.0239971489194342
3.68999999999971 0.023770009587325
3.69999999999971 0.0235449082551872
3.70999999999971 0.0233218288033187
3.71999999999971 0.023100755196817
3.72999999999971 0.0228816714860428
3.73999999999971 0.0226645618070599
3.74999999999971 0.0224494103820498
3.75999999999971 0.0222362015197017
3.76999999999971 0.0220249196155794
3.77999999999971 0.0218155491524629
3.78999999999971 0.0216080747006706
3.79999999999971 0.0214024809183561
3.80999999999971 0.0211987525517838
3.81999999999971 0.0209968744355842
3.82999999999971 0.0207968314929881
3.8399999999997 0.0205986087360384
3.8499999999997 0.020402191265785
3.8599999999997 0.020207564272458
3.8699999999997 0.0200147130356221
3.8799999999997 0.0198236229243129
3.8899999999997 0.019634279397155
3.8999999999997 0.0194466680024613
3.9099999999997 0.0192607743783168
3.9199999999997 0.0190765842526429
3.9299999999997 0.0188940834432477
3.9399999999997 0.0187132578578582
3.9499999999997 0.0185340934941377
3.9599999999997 0.0183565764396875
3.9699999999997 0.0181806928720338
3.9799999999997 0.0180064290585991
3.9899999999997 0.0178337713566614
3.9999999999997 0.0176627062132961
4.0099999999997 0.0174932201653087
4.0199999999997 0.0173252998391492
4.0299999999997 0.0171589319508185
4.0399999999997 0.0169941033057592
4.0499999999997 0.0168308007987345
4.0599999999997 0.0166690114136968
4.0699999999997 0.0165087222236435
4.0799999999997 0.0163499203904603
4.0899999999997 0.0161925931647571
4.0999999999997 0.0160367278856899
4.1099999999997 0.0158823119807731
4.1199999999997 0.0157293329656839
4.1299999999997 0.0155777784440536
4.1399999999997 0.0154276361072517
4.1499999999997 0.0152788937341602
4.1599999999997 0.0151315391909389
4.1699999999997 0.0149855604307817
4.1799999999997 0.0148409454936652
4.1899999999997 0.0146976825060877
4.1999999999997 0.0145557596808034
4.2099999999997 0.0144151653165447
4.2199999999997 0.0142758877977403
4.2299999999997 0.0141379155942247
4.2399999999997 0.0140012372609412
4.2499999999997 0.0138658414376381
4.2599999999997 0.0137317168485584
4.2699999999997 0.0135988523021231
4.2799999999997 0.0134672366906083
4.2899999999997 0.0133368589898167
4.2999999999997 0.013207708258744
4.30999999999969 0.0130797736392386
4.31999999999969 0.0129530443556563
4.32999999999969 0.012827509714512
4.33999999999969 0.0127031591041226
4.34999999999969 0.0125799819942498
4.35999999999969 0.0124579679357342
4.36999999999969 0.0123371065601281
4.37999999999969 0.0122173875793225
4.38999999999969 0.012098800785171
4.39999999999969 0.0119813360491101
4.40999999999969 0.0118649833217735
4.41999999999969 0.0117497326326065
4.42999999999969 0.0116355740894748
4.43999999999969 0.0115224978782706
4.44999999999969 0.0114104942625158
4.45999999999969 0.0112995535829618
4.46999999999969 0.011189666257188
4.47999999999969 0.011080822779196
4.48999999999969 0.0109730137190022
4.49999999999969 0.0108662297222286
4.50999999999969 0.0107604615096895
4.51999999999969 0.0106556998769793
4.52999999999969 0.0105519356940552
4.53999999999969 0.0104491599048195
4.54999999999969 0.0103473635267005
4.55999999999969 0.0102465376502322
4.56999999999969 0.0101466734386299
4.57999999999969 0.0100477621273688
4.58999999999969 0.00994979502375666
4.59999999999969 0.00985276350650918
4.60999999999969 0.00975665902532264
4.61999999999969 0.00966147310044499
4.62999999999969 0.00956719732224769
4.63999999999969 0.00947382335079603
4.64999999999969 0.00938134291541777
4.65999999999969 0.00928974781427464
4.66999999999969 0.0091990299139282
4.67999999999969 0.00910918114891047
4.68999999999969 0.00902019352129101
4.69999999999969 0.00893205910024524
4.70999999999969 0.00884477002162148
4.71999999999969 0.00875831848751013
4.72999999999969 0.00867269676580993
4.73999999999969 0.00858789718979692
4.74999999999969 0.00850391215769213
4.75999999999969 0.00842073413223019
4.76999999999969 0.0083383556402276
4.77999999999968 0.00825676927215263
4.78999999999968 0.00817596768169403
4.79999999999968 0.0080959435853321
4.80999999999968 0.00801668976190795
4.81999999999968 0.0079381990521969
4.82999999999968 0.00786046435847816
4.83999999999968 0.00778347864410967
4.84999999999968 0.00770723493310052
4.85999999999968 0.00763172630968583
4.86999999999968 0.0075569459179027
4.87999999999968 0.0074828869611651
4.88999999999968 0.00740954270184308
4.89999999999968 0.00733690646084006
4.90999999999968 0.00726497161717317
4.91999999999968 0.00719373160755348
4.92999999999968 0.00712317992596818
4.93999999999968 0.00705331012326373
4.94999999999968 0.00698411580673041
4.95999999999968 0.0069155906396884
4.96999999999968 0.00684772834107455
4.97999999999968 0.00678052268503073
4.98999999999968 0.00671396750049529
4.99999999999968 0.00664805667079222
5.00999999999968 0.0065827841332262
5.01999999999968 0.00651814387867637
5.02999999999968 0.00645412995119187
5.03999999999968 0.00639073644759043
5.04999999999968 0.00632795751705703
5.05999999999968 0.00626578736074504
5.06999999999968 0.00620422023137894
5.07999999999968 0.00614325043285855
5.08999999999968 0.00608287231986502
5.09999999999968 0.00602308029746849
5.10999999999968 0.00596386882073834
5.11999999999968 0.00590523239435329
5.12999999999968 0.00584716557221562
5.13999999999968 0.00578966295706609
5.14999999999968 0.00573271920009999
5.15999999999968 0.00567632900058688
5.16999999999968 0.00562048710549027
5.17999999999968 0.0055651883090906
5.18999999999968 0.00551042745261018
5.19999999999968 0.00545619942383816
5.20999999999968 0.00540249915675987
5.21999999999968 0.00534932163118735
5.22999999999968 0.0052966618723903
5.23999999999968 0.00524451495073201
5.24999999999967 0.00519287598130355
5.25999999999967 0.00514174012356403
5.26999999999967 0.00509110258097915
5.27999999999967 0.00504095860066526
5.28999999999967 0.00499130347303126
5.29999999999967 0.00494213253142756
5.30999999999967 0.00489344115179237
5.31999999999967 0.00484522475230374
5.32999999999967 0.00479747879303061
5.33999999999967 0.00475019877558874
5.34999999999967 0.00470338024279651
5.35999999999967 0.00465701877833411
5.36999999999967 0.00461111000640375
5.37999999999967 0.00456564959139363
5.38999999999967 0.00452063323754193
5.39999999999967 0.00447605668860488
5.40999999999967 0.00443191572752458
5.41999999999967 0.00438820617610172
5.42999999999967 0.00434492389466842
5.43999999999967 0.00430206478176336
5.44999999999967 0.00425962477380983
5.45999999999967 0.00421759984479561
5.46999999999967 0.00417598600595429
5.47999999999967 0.00413477930544954
5.48999999999967 0.00409397582806047
5.49999999999967 0.004053571694871
5.50999999999967 0.00401356306295902
5.51999999999967 0.00397394612508819
5.52999999999967 0.00393471710940461
5.53999999999967 0.00389587227913046
5.54999999999967 0.00385740793226508
5.55999999999967 0.00381932040128332
5.56999999999967 0.00378160605284057
5.57999999999967 0.00374426128747556
5.58999999999967 0.00370728253931873
5.59999999999967 0.00367066627580062
5.60999999999967 0.00363440899736237
5.61999999999967 0.00359850723716987
5.62999999999967 0.00356295756082828
5.63999999999967 0.00352775656609906
5.64999999999967 0.00349290088261982
5.65999999999967 0.00345838717162477
5.66999999999967 0.00342421212566887
5.67999999999967 0.00339037246835269
5.68999999999967 0.00335686495405011
5.69999999999967 0.00332368636763684
5.70999999999967 0.00329083352422308
5.71999999999966 0.00325830326888546
5.72999999999966 0.00322609247640433
5.73999999999966 0.00319419805099851
5.74999999999966 0.00316261692606728
5.75999999999966 0.00313134606392973
5.76999999999966 0.00310038245556887
5.77999999999966 0.00306972312037733
5.78999999999966 0.0030393651059032
5.79999999999966 0.00300930548760018
5.80999999999966 0.00297954136857814
5.81999999999966 0.00295006987935687
5.82999999999966 0.0029208881776193
5.83999999999966 0.00289199344797005
5.84999999999966 0.0028633829016934
5.85999999999966 0.00283505377651362
5.86999999999966 0.00280700333635719
5.87999999999966 0.00277922887111836
5.88999999999966 0.00275172769642372
5.89999999999966 0.00272449715340122
5.90999999999966 0.00269753460845047
5.91999999999966 0.00267083745301263
5.92999999999966 0.00264440310334554
5.93999999999966 0.00261822900029913
5.94999999999966 0.00259231260909095
5.95999999999966 0.00256665141908705
5.96999999999966 0.00254124294358126
5.97999999999966 0.00251608471957811
5.98999999999966 0.00249117430757779
5.99999999999966 0.00246650929136081
6.00999999999966 0.00244208727777704
6.01999999999966 0.00241790589653441
6.02999999999966 0.00239396279999
6.03999999999966 0.00237025566294402
6.04999999999966 0.00234678218243345
6.05999999999966 0.00232354007752829
6.06999999999966 0.00230052708913044
6.07999999999966 0.00227774097977256
6.08999999999966 0.00225517953341984
6.09999999999966 0.00223284055527288
6.10999999999966 0.00221072187157251
6.11999999999966 0.00218882132940574
6.12999999999965 0.00216713679651412
6.13999999999966 0.00214566616110222
6.14999999999966 0.00212440733165074
6.15999999999966 0.00210335823672685
6.16999999999965 0.00208251682480025
6.17999999999966 0.00206188106405831
6.18999999999966 0.00204144894222333
6.19999999999965 0.00202121846637226
6.20999999999965 0.00200118766275647
6.21999999999965 0.00198135457662399
6.22999999999966 0.00196171727204345
6.23999999999965 0.00194227383172864
6.24999999999965 0.00192302235686529
6.25999999999965 0.00190396096693941
6.26999999999965 0.00188508779956665
6.27999999999965 0.00186640101032298
6.28999999999965 0.00184789877257743
6.29999999999965 0.00182957927732657
6.30999999999965 0.00181144073302859
6.31999999999965 0.00179348136544181
6.32999999999965 0.00177569941746097
6.33999999999965 0.00175809314895824
6.34999999999965 0.0017406608366239
6.35999999999965 0.00172340077380845
6.36999999999965 0.00170631127036665
6.37999999999965 0.00168939065250242
6.38999999999965 0.00167263726261616
6.39999999999965 0.00165604945915132
6.40999999999965 0.00163962561644455
6.41999999999965 0.00162336412457563
6.42999999999965 0.00160726338921977
6.43999999999965 0.00159132183150079
6.44999999999965 0.0015755378878451
6.45999999999965 0.00155991000983764
6.46999999999965 0.001544436664079
6.47999999999965 0.00152911633204378
6.48999999999965 0.00151394750994003
6.49999999999965 0.00149892870856951
6.50999999999965 0.00148405845319088
6.51999999999965 0.00146933528338187
6.52999999999965 0.00145475775290435
6.53999999999965 0.00144032442956957
6.54999999999965 0.00142603389510587
6.55999999999965 0.00141188474502596
6.56999999999965 0.00139787558849647
6.57999999999965 0.00138400504820841
6.58999999999965 0.00137027176024884
6.59999999999965 0.00135667437397386
6.60999999999964 0.00134321155188175
6.61999999999965 0.00132988196948878
6.62999999999965 0.00131668431520546
6.63999999999965 0.00130361729021348
6.64999999999964 0.00129067960834351
6.65999999999964 0.0012778699959562
6.66999999999965 0.00126518719182184
6.67999999999964 0.00125262994700211
6.68999999999964 0.0012401970247326
6.69999999999964 0.0012278872003078
6.70999999999965 0.00121569926096454
6.71999999999964 0.0012036320057689
6.72999999999964 0.00119168424550268
6.73999999999964 0.00117985480255222
6.74999999999964 0.00116814251079589
6.75999999999964 0.00115654621549612
6.76999999999964 0.00114506477318891
6.77999999999964 0.00113369705157642
6.78999999999964 0.00112244192942068
6.79999999999964 0.00111129829643624
6.80999999999964 0.00110026505318651
6.81999999999964 0.00108934111097896
6.82999999999964 0.00107852539176225
6.83999999999964 0.00106781682802459
6.84999999999964 0.00105721436269117
6.85999999999964 0.00104671694902613
6.86999999999964 0.00103632355053096
6.87999999999964 0.00102603314084803
6.88999999999964 0.00101584470366145
6.89999999999964 0.00100575723260187
6.90999999999964 0.000995769731149801
6.91999999999964 0.000985881212541654
6.92999999999964 0.000976090699675061
6.93999999999964 0.000966397225016403
6.94999999999964 0.000956799830507931
6.95999999999964 0.000947297567477607
6.96999999999964 0.000937889496547659
6.97999999999964 0.000928574687544859
6.98999999999964 0.000919352219412634
6.99999999999964 0.000910221180122227
7.00999999999964 0.00090118066658614
7.01999999999964 0.000892229784571524
7.02999999999964 0.000883367648614734
7.03999999999964 0.000874593381936506
7.04999999999963 0.000865906116357964
7.05999999999964 0.000857304992218028
7.06999999999964 0.000848789158290326
7.07999999999964 0.000840357771702056
7.08999999999963 0.000832009997852697
7.09999999999964 0.000823745010334996
7.10999999999964 0.000815561990854368
7.11999999999964 0.000807460129151241
7.12999999999963 0.000799438622922362
7.13999999999963 0.000791496677744838
7.14999999999964 0.0007836335069988
7.15999999999963 0.000775848331792364
7.16999999999963 0.0007681403808861
7.17999999999963 0.000760508890619805
7.18999999999964 0.000752953104837561
7.19999999999963 0.000745472274816204
7.20999999999963 0.000738065659192093
7.21999999999963 0.000730732523890945
7.22999999999963 0.000723472142055863
7.23999999999963 0.000716283793977664
7.24999999999963 0.00070916676702561
7.25999999999963 0.000702120355578
7.26999999999963 0.000695143860954837
7.27999999999963 0.000688236591350356
7.28999999999963 0.000681397861764738
7.29999999999963 0.000674626993939793
7.30999999999963 0.000667923316292283
7.31999999999963 0.000661286163848878
7.32999999999963 0.000654714878182199
7.33999999999963 0.000648208807347598
7.34999999999963 0.000641767305818265
7.35999999999963 0.000635389734425167
7.36999999999963 0.000629075460292768
7.37999999999963 0.000622823856779365
7.38999999999963 0.00061663430341508
7.39999999999963 0.000610506185842819
7.40999999999963 0.000604438895758094
7.41999999999963 0.000598431830849496
7.42999999999963 0.000592484394741355
7.43999999999963 0.00058659599693505
7.44999999999963 0.000580766052751738
7.45999999999963 0.000574993983275951
7.46999999999963 0.000569279215298278
7.47999999999963 0.000563621181261365
7.48999999999963 0.000558019319202885
7.49999999999963 0.000552473072702384
7.50999999999963 0.000546981890825991
7.51999999999963 0.000541545228073554
7.52999999999962 0.000536162544324977
7.53999999999963 0.000530833304788425
7.54999999999963 0.000525556979947406
7.55999999999963 0.000520333045509161
7.56999999999962 0.000515160982354588
7.57999999999963 0.000510040276486163
7.58999999999963 0.000504970418979713
7.59999999999962 0.000499950905932188
7.60999999999962 0.000494981238415184
7.61999999999962 0.000490060922424675
7.62999999999963 0.000485189468832841
7.63999999999962 0.000480366393341101
7.64999999999962 0.000475591216432017
7.65999999999962 0.0004708634633233
7.66999999999963 0.000466182663920015
7.67999999999962 0.000461548352770668
7.68999999999962 0.000456960069019719
7.69999999999962 0.000452417356363746
7.70999999999962 0.000447919763006266
7.71999999999962 0.000443466841614316
7.72999999999962 0.000439058149273471
7.73999999999962 0.000434693247445395
7.74999999999962 0.000430371701924499
7.75999999999962 0.00042609308279592
7.76999999999962 0.000421856964393142
7.77999999999962 0.000417662925255846
7.78999999999962 0.000413510548089508
7.79999999999962 0.000409399419723667
7.80999999999962 0.000405329131071835
7.81999999999962 0.000401299277092073
7.82999999999962 0.00039730945674566
7.83999999999962 0.000393359272958981
7.84999999999962 0.000389448332584838
7.85999999999962 0.00038557624636265
7.86999999999962 0.000381742628881417
7.87999999999962 0.000377947098540783
7.88999999999962 0.000374189277515312
7.89999999999962 0.00037046879171498
7.90999999999962 0.0003667852707511
7.91999999999962 0.000363138347897679
7.92999999999962 0.000359527660056663
7.93999999999962 0.000355952847722052
7.94999999999962 0.000352413554944238
7.95999999999962 0.000348909429295216
7.96999999999962 0.000345440121834234
7.97999999999962 0.000342005287072993
7.98999999999962 0.000338604582941946
7.99999999999962 0.00033523767075659
8.00999999999961 0.000331904215184308
8.01999999999962 0.000328603884211313
8.02999999999962 0.000325336349109472
8.03999999999962 0.000322101284404683
8.04999999999961 0.000318898367844683
8.05999999999962 0.000315727280366964
8.06999999999962 0.000312587706067684
8.07999999999961 0.000309479332170344
8.08999999999961 0.00030640184899535
8.09999999999961 0.000303354949929014
8.10999999999962 0.000300338331393657
8.11999999999961 0.000297351692817377
8.12999999999961 0.000294394736605471
8.13999999999961 0.000291467168109302
8.14999999999961 0.00028856869559871
8.15999999999961 0.00028569903023298
8.16999999999961 0.000282857886032015
8.17999999999961 0.000280044979848296
8.18999999999961 0.000277260031338716
8.19999999999961 0.000274502762937627
8.20999999999961 0.000271772899828562
8.21999999999961 0.00026907016991794
8.22999999999961 0.000266394303807439
8.23999999999961 0.000263745034767914
8.24999999999961 0.000261122098713094
8.25999999999961 0.00025852523417305
8.26999999999961 0.000255954182269212
8.27999999999961 0.000253408686687605
8.28999999999961 0.000250888493654748
8.29999999999961 0.000248393351911769
8.30999999999961 0.000245923012689851
8.31999999999961 0.000243477229685674
8.32999999999961 0.000241055759037294
8.33999999999961 0.000238658359299465
8.34999999999961 0.000236284791420181
8.35999999999961 0.000233934818717432
8.36999999999961 0.000231608206854846
8.37999999999961 0.000229304723819224
8.38999999999961 0.000227024139897278
8.39999999999961 0.00022476622765316
8.40999999999961 0.000222530761906085
8.41999999999961 0.000220317519707293
8.42999999999961 0.000218126280319225
8.43999999999961 0.000215956825192032
8.4499999999996 0.000213808937944077
8.45999999999961 0.000211682404338215
8.46999999999961 0.00020957701226229
8.47999999999961 0.000207492551707743
8.4899999999996 0.000205428814748211
8.49999999999961 0.000203385595519798
8.50999999999961 0.000201362690200003
8.51999999999961 0.000199359896987756
8.5299999999996 0.000197377016083905
8.5399999999996 0.000195413849670463
8.54999999999961 0.000193470201891978
8.5599999999996 0.000191545878835448
8.5699999999996 0.000189640688511676
8.5799999999996 0.000187754440835519
8.58999999999961 0.000185886947608241
8.5999999999996 0.000184038022497193
8.6099999999996 0.000182207481019168
8.6199999999996 0.000180395140520297
8.6299999999996 0.000178600820159729
8.6399999999996 0.000176824340890308
8.6499999999996 0.00017506552544124
8.6599999999996 0.000173324198301101
8.6699999999996 0.000171600185699839
8.6799999999996 0.00016989331559144
8.6899999999996 0.000168203417637033
8.6999999999996 0.000166530323188336
8.7099999999996 0.000164873865270977
8.7199999999996 0.000163233878566605
8.7299999999996 0.000161610199398313
8.7399999999996 0.000160002665713302
8.7499999999996 0.000158411117066974
8.7599999999996 0.000156835394607474
8.7699999999996 0.000155275341058891
8.7799999999996 0.000153730800707021
8.7899999999996 0.00015220161938301
8.7999999999996 0.000150687644447887
8.8099999999996 0.000149188724777876
8.8199999999996 0.000147704710750144
8.8299999999996 0.000146235454225888
8.8399999999996 0.000144780808537415
8.8499999999996 0.000143340628473113
8.8599999999996 0.000141914770262641
8.8699999999996 0.000140503091563229
8.8799999999996 0.000139105451444643
8.8899999999996 0.000137721710376593
8.8999999999996 0.000136351730213475
8.9099999999996 0.000134995374181439
8.9199999999996 0.000133652506864688
8.92999999999959 0.000132322994191656
8.9399999999996 0.000131006703422407
8.9499999999996 0.000129703503134262
8.9599999999996 0.00012841326321075
8.96999999999959 0.000127135854825901
8.9799999999996 0.000125871150434079
8.9899999999996 0.000124619023755605
8.99999999999959 0.00012337934976493
9.00999999999959 0.000122152004678247
9.01999999999959 0.000120936865940887
9.0299999999996 0.000119733812215041
9.03999999999959 0.00011854272336804
9.04999999999959 0.000117363480460523
9.05999999999959 0.000116195965733824
9.0699999999996 0.00011504006259947
9.07999999999959 0.000113895655626346
9.08999999999959 0.00011276263052997
9.09999999999959 0.000111640874161096
9.10999999999959 0.000110530274494212
9.11999999999959 0.000109430720616256
9.12999999999959 0.000108342102716328
9.13999999999959 0.000107264312073962
9.14999999999959 0.000106197241048393
9.15999999999959 0.000105140783068273
9.16999999999959 0.000104094832621378
9.17999999999959 0.000103059285242432
9.18999999999959 0.000102034037505263
9.19999999999959 0.000101018987009732
9.20999999999959 0.000100014032374219
9.21999999999959 9.90190732231131e-05
9.22999999999959 9.80340101789563e-05
9.23999999999959 9.70587448506012e-05
9.24999999999959 9.60931798252456e-05
9.25999999999959 9.51372186573619e-05
9.26999999999959 9.41907658599552e-05
9.27999999999959 9.32537268939354e-05
9.28999999999959 9.23260081605953e-05
9.29999999999959 9.14075169902042e-05
9.30999999999959 9.04981616343746e-05
9.31999999999959 8.9597851255764e-05
9.32999999999959 8.87064959204408e-05
9.33999999999959 8.78240065868083e-05
9.34999999999959 8.69502950991903e-05
9.35999999999959 8.60852741770876e-05
9.36999999999959 8.522885740743e-05
9.37999999999959 8.43809592360521e-05
9.38999999999959 8.35414949586127e-05
9.39999999999959 8.27103807128462e-05
9.40999999999958 8.18875334689265e-05
9.41999999999959 8.10728710233823e-05
9.42999999999959 8.02663119884614e-05
9.43999999999959 7.94677757860448e-05
9.44999999999958 7.86771826380094e-05
9.45999999999959 7.78944535598083e-05
9.46999999999959 7.71195103509438e-05
9.47999999999958 7.63522755888799e-05
9.48999999999958 7.55926726199582e-05
9.49999999999958 7.48406255527552e-05
9.50999999999959 7.40960592498849e-05
9.51999999999958 7.33588993199131e-05
9.52999999999958 7.26290721122669e-05
9.53999999999958 7.1906504707039e-05
9.54999999999959 7.1191124910119e-05
9.55999999999958 7.04828612435522e-05
9.56999999999958 6.97816429411138e-05
9.57999999999958 6.90873999396661e-05
9.58999999999958 6.84000628714024e-05
9.59999999999958 6.77195630594216e-05
9.60999999999958 6.70458325084171e-05
9.61999999999958 6.63788038993631e-05
9.62999999999958 6.57184105822012e-05
9.63999999999958 6.50645865691934e-05
9.64999999999958 6.44172665291632e-05
9.65999999999958 6.3776385779515e-05
9.66999999999958 6.31418802810302e-05
9.67999999999958 6.25136866311074e-05
9.68999999999958 6.18917420570035e-05
9.69999999999958 6.12759844101834e-05
9.70999999999958 6.06663521600048e-05
9.71999999999958 6.00627843871799e-05
9.72999999999958 5.94652207780135e-05
9.73999999999958 5.88736016191976e-05
9.74999999999958 5.82878677894955e-05
9.75999999999958 5.77079607566442e-05
9.76999999999958 5.71338225699276e-05
9.77999999999958 5.65653958543025e-05
9.78999999999958 5.60026238055244e-05
9.79999999999958 5.54454501839403e-05
9.80999999999958 5.48938193090592e-05
9.81999999999958 5.43476760535666e-05
9.82999999999958 5.38069658388932e-05
9.83999999999958 5.32716346284521e-05
9.84999999999958 5.27416289230966e-05
9.85999999999958 5.22168957553559e-05
9.86999999999958 5.1697382684781e-05
9.87999999999958 5.11830377918478e-05
9.88999999999957 5.06738096735242e-05
9.89999999999958 5.01696474377281e-05
9.90999999999958 4.9670500698006e-05
9.91999999999958 4.91763195695448e-05
9.92999999999957 4.86870546629624e-05
9.93999999999957 4.82026570799855e-05
9.94999999999958 4.77230784087944e-05
9.95999999999957 4.72482707188124e-05
9.96999999999957 4.67781865556065e-05
9.97999999999957 4.63127789372303e-05
9.98999999999958 4.58520013482362e-05
};
\addlegendentry{Derivative of Sigmoid}
\end{axis}

\end{tikzpicture}
	\caption{Sigmoid Function and its Derivative}
	\label{fig:sigmoid-derivative}
\end{figure}
This will become more clear in \secref{sec:training-stochastic-gradient} when the expressions of backpropagation are presented.

If the weights are initialized with 0, every neuron would compute the same output.
This leads to an identical gradient for each one and therefore identical parameter updates.
All in all, this would reduce the network to a linear one.
Hence, a common initialization approach is using a Gaussian distribution like $N(\mu, \sigma^2) = N(0, 0.01)$.
However, this way the variance of this distribution of each neuron's output grows with the number of its inputs.
Therefore, a normalization of the variance of each neuron's output to 1 is performed.
This is done by scaling its weights by the square root of its number of inputs.
This can be derived with the $n$ inputs $\vec{x}$ and weights $\vec{w}$ by
\begin{align*}
	Var(z) &= Var \left( \sum_{i}^{n} w_i x_i \right) \\
	&= \sum_{i}^{n} Var \left( w_i x_i \right) \\
	&= \sum_{i}^{n} \left[ E(w_i)^2 \right] Var(x_i) + E \left[ (x_i) \right]^2 Var(w_i) + Var(x_i) Var(w_i) \\
	&= \sum_{i}^{n} Var(x_i) Var(w_i) \\
	&= (n Var(\vec{w})) Var(\vec{x}) \\
\end{align*}
where zero mean inputs and weights are assumed and an identically distribution of all $w_i$ and $x_i$.
Now, $z$ needs to have the same variance as all of its inputs $\vec{x}$, which yields $Var(w) = 1/n$ as every weights variance.
Hence,
\begin{equation}
	\vec{w} = \frac{N(0,1)}{\sqrt{n}}
\end{equation}
initializes the weights.
This is mostly universal, but must be used for $\tanh$ activation functions.
A similar analysis is done by \textit{Glorot and Bengio} \cite{Glorot10understandingthe} whose recommendation is
\begin{equation*}
	Var(\vec{w}) = \frac{2}{n_{in} + n_{out}}
\end{equation*}
where $n_{in}$ and $n_{out}$ is the number of neurons in the incoming and outgoing layer, respectively.
Their motivation is, that by doing the earlier variance calculations for the backpropagated signal, it turns out that
\begin{equation}
	Var(\vec{w}) = \frac{1}{n_{out}}
\end{equation}
is needed for keeping the variance of the input and output the same.
Because in general the constraint $n_{in} = n_{out}$ is not fulfilled, they make a compromise by taking the average.
Though, these initializations are not valid for, for example, ReLU units, due to their positive mean.
Fortunately, \textit{He et al.} \cite{DBLP:journals/corr/HeZR015} states the initialization
\begin{equation}
	\vec{w} = N(0,1) \cdot \sqrt{\frac{2}{n}}
\end{equation}
especially for ReLU neurons.