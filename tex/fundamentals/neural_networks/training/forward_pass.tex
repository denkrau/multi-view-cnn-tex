\subsubsection{Feed-Forward Pass}
\label{sec:training-forward-pass}
Let's recall the information, that are needed for the actual training step, i.e. the finding of optimal weights and biases.
Everything starts with a dataset $\mathbb{D}$ containing $m$ pairs of inputs and corresponding labels.
Performing a one-hot encoding on the labels and assuming in general flattened input matrices yields
\begin{equation}
	\label{eq:dataset-one-hot}
	\mathbb{D} =
	\begin{pmatrix}
		\vec{X} & \vec{Y}
	\end{pmatrix}
\end{equation}
where $\vec{X} \in \mathbb{R}^{n_x \times m}$ and $\vec{Y} \in \mathbb{R}^{n_y \times m}$ are representing each input and label as vectors, respectively, forming matrices.
Hereby, $n_x$ represents the size of each input and $n_y$ the number of categories.
Furthermore, there is a neural network with $L$ layers each containing an arbitrarily number of perceptrons.
Expressing the activation of every node with \thmref{eq:perceptron-sum} would get very confusing for a whole network.
Hence, a matrix notation is the way to go in the long run.
First, for every $j$-th node in the $l$-th layer its weights are summarized in a vector
\begin{equation}
	\label{eq:weights-vector}
	\vec{w}^{[l]}_j =
	\begin{pmatrix}
		w^{[l]}_{j,1} & w^{[l]}_{j,2} & \cdots & w^{[l]}_{j,n^{[l-1]}_h}
	\end{pmatrix}^T
\end{equation}
containing single weights, where the superscript in square brackets denotes the layer and the subscript denotes the edge of (target neuron, preceding neuron).
The number of hidden neurons in the $l$-th layer is represented by $n^{[l]}_h$.
These conventions are maintained for all parameters for the rest of this thesis.
The bias of the $j$-th neuron in the $l$-th layer is just a scalar denoted as $b^{[l]}_j$. 
Now, every weight vector and bias can be combined to a matrix and vector, respectively, for each layer.
This yields
\begin{subequations}
\label{eq:parameters}
	\begin{align}
		\vec{W}^{[l]} &=
		\begin{pmatrix}
			\vec{w}^{[l]}_1 & \vec{w}^{[l]}_2 & \cdots & \vec{w}^{[l]}_{n^{[l]}_h}
		\end{pmatrix}^T
		\label{eq:weights}
		\\
		\vec{b}^{[l]} &=
			\begin{pmatrix}
				b^{[l]}_1 & b^{[l]}_2 & \cdots & b^{[l]}_{n^{[l]}_h}
			\end{pmatrix}^T
		\label{eq:biases}
	\end{align}
\end{subequations}
where $\vec{W}^{[l]} \in \mathbb{R}^{n^{[l]}_h \times n^{[l-1]}_h}$ and $\vec{b}^{[l]} \in \mathbb{R}^{n^{[l]}_h}$.

Using these matrices and vectors data can easily be forwarded through the network by building up on \thmref{eq:perceptron-activation}.
The weighted sum of all neurons of the $l$-th layer is computed as
\begin{equation}
	\label{eq:weighted-sum}
	\vec{z}^{[l]} = \vec{W}^{[l]} \vec{a}^{[l-1]} + \vec{b}^{[l]}
\end{equation}
with the activations vector $\vec{a}$.
Putting this in an activation function yields
\begin{equation}
	\label{eq:activations}
	\vec{a}^{[l]} = \phi\left(\vec{z}^{[l]}\right)
\end{equation}
for the $l$-th layers activations.
Performing this for every layer results in 
\begin{equation}
	\label{eq:feedforward}
	\hat{\vec{y}}^{(i)} = f(\vec{x}^{(i)}, \vec{W}, \vec{b})
\end{equation}
as the network's prediction for the $i$-th data sample $\vec{x^}{(i)} \in \mathbb{R}^{n_y}$.
This superscript in round brackets is part of the used convention.

Let's assume that the result is already fed into a sigmoid function and therefore only contains values between 0 and 1.
This prediction needs to be compared with the ground-truth label $\vec{y}^{(i)}$ for checking how good the network performs and, hence, how well the parameters fit.
An example of how these vectors can look like is shown in \tabref{tab:prediction}.
\begin{table}[]
	\centering
	\caption[Example Comparison of One-Hot Encoded Ground Truth Label and Prediction]{Example comparison of an one-hot encoded ground truth label and a softmax prediction. However, the actual ground truth feature has the second smallest probability in the prediction. Therefore, the parameters need to be adjusted.}
	\label{tab:prediction}
	\begin{tabular}{ll}
		Ground-Truth & $\vec{y} = \begin{pmatrix} 0 & 1 & 0 & 0 & 0 \end{pmatrix}^T$                     \\
		Prediction   & $\hat{\vec{y}} = \begin{pmatrix} 0.54 & 0.28 & 0.2 & 0.63 & 0.96 \end{pmatrix}^T$
	\end{tabular}
\end{table}
It can be clearly seen, that the prediction is completely wrong.
The actual ground truth feature has the second smallest probability in the prediction.
In theory, an identical representation would be desirable.
However, in practical terms a slight deviation is normal.
Because finding optimal parameters is a optimization problem, a metric for the performance of the networks is served by a loss function that maps these parameters to a loss value.
The most common loss function for comparing two probability distributions of mutually exclusive classes is the cross entropy loss function.
It is defined as
\begin{equation}
	H(y,p) = -(y \log(p) + (1-y) \log(1-p))
\end{equation}
for a single output representing two classes and as
\begin{equation}
	\label{eq:cross-entropy-loss}
	H(\vec{y}, \vec{p}) = - \sum_{i}^{C} y_i \log (p_i)
\end{equation}
for multiclass classification, where $C$ is the number of classes, $i$ the moving index, $\vec{y}$ the ground truth vector and $\vec{p}$ the predicted probabilities\cite{murphy2013machine}.
Applying the previous notation to this yields
\begin{subequations}
	\begin{align}
		\vec{y} &= \vec{y}^{(i)} \\
		\vec{p} &= \hat{\vec{y}}^{(i)} \\
		C &= n_y
	\end{align}
\end{subequations}
as analogies.
Due to the one-hot encoding of the labels, only the positive class $C_p$ is taken into account in the loss computation.
Hence, \thmref{eq:cross-entropy-loss} is reduced to
\begin{equation}
	\label{eq:cross-entropy-loss-compact}
	H(\vec{y}, \vec{p}) = - y_p \log (p_p)
\end{equation}
where $y_p$ and $p_p$ denote the probability of the positive label and its corresponding prediction, respectively.
A visualization of this expression is shown in \figref{fig:cross-entropy}.
\begin{figure}
	\setlength\figureheight{.5\textwidth}
	\setlength\figurewidth{.8\textwidth}
	\centering
	% This file was created by matplotlib2tikz v0.7.3.
\begin{tikzpicture}

\begin{axis}[
height=\figureheight,
tick align=outside,
tick pos=left,
width=\figurewidth,
x grid style={white!90.01960784313725!black},
xlabel={Probability Difference},
xmajorgrids,
xmin=-0.04995, xmax=1.04895,
xtick style={color=black},
xtick={-0.2,0,0.2,0.4,0.6,0.8,1,1.2},
xticklabels={,0.0,0.2,0.4,0.6,0.8,1.0,},
y grid style={white!90.01960784313725!black},
ylabel={Cross-Entropy Loss},
ymajorgrids,
ymin=-0.345387763949107, ymax=7.25314304293124,
ytick style={color=black}
]
\addplot [semithick, green!50.0!black]
table {%
0 -0
0.001 0.00100050033358353
0.002 0.00200200267067308
0.003 0.00300450902029872
0.004 0.00400802139753882
0.005 0.00501254182354429
0.006 0.00601807232556302
0.007 0.00702461493696447
0.008 0.00803217169726427
0.009 0.00904074465214907
0.01 0.0100503358535015
0.011 0.0110609473594249
0.012 0.0120725812342692
0.013 0.0130852395486555
0.014 0.0140989243795016
0.015 0.0151136378100482
0.016 0.0161293819298836
0.017 0.0171461588349705
0.018 0.0181639706276712
0.019 0.019182819416774
0.02 0.0202027073175195
0.021 0.0212236364516267
0.022 0.0222456089473197
0.023 0.0232686269393543
0.024 0.0242926925690446
0.025 0.0253178079842899
0.026 0.026343975339602
0.027 0.027371196796132
0.028 0.028399474521698
0.029 0.0294288106908122
0.03 0.0304592074847086
0.031 0.0314906670913708
0.032 0.0325231917055601
0.033 0.0335567835288428
0.034 0.0345914447696191
0.035 0.0356271776431512
0.036 0.0366639843715915
0.037 0.0377018671840115
0.038 0.0387408283164306
0.039 0.0397808700118446
0.04 0.0408219945202552
0.041 0.0418642040986989
0.042 0.0429075010112765
0.043 0.0439518875291828
0.044 0.0449973659307358
0.045 0.0460439385014068
0.046 0.0470916075338506
0.047 0.048140375327935
0.048 0.0491902441907718
0.049 0.0502412164367468
0.05 0.0512932943875506
0.051 0.0523464803722092
0.052 0.0534007767271153
0.053 0.0544561857960589
0.054 0.0555127099302588
0.055 0.0565703514883944
0.056 0.0576291128366364
0.057 0.0586889963486796
0.058 0.059750004405774
0.059 0.0608121393967575
0.06 0.0618754037180875
0.061 0.0629397997738741
0.062 0.0640053299759124
0.063 0.0650719967437148
0.064 0.0661398025045451
0.065 0.06720874969345
0.066 0.0682788407532945
0.067 0.0693500781347932
0.068 0.0704224642965459
0.069 0.0714960017050699
0.07 0.0725706928348355
0.071 0.0736465401682985
0.072 0.0747235461959365
0.073 0.0758017134162818
0.074 0.0768810443359576
0.075 0.0779615414697118
0.076 0.0790432073404529
0.077 0.0801260444792849
0.078 0.0812100554255432
0.079 0.0822952427268302
0.08 0.083381608939051
0.081 0.08446915662645
0.082 0.0855578883616465
0.083 0.0866478067256722
0.084 0.0877389143080067
0.085 0.0888312137066157
0.086 0.089924707527987
0.087 0.0910193983871685
0.088 0.0921152889078056
0.089 0.0932123817221787
0.09 0.0943106794712413
0.091 0.0954101848046582
0.092 0.0965109003808437
0.093 0.0976128288670004
0.094 0.0987159729391577
0.095 0.0998203352822109
0.096 0.100925918589961
0.097 0.102032725565152
0.098 0.103140758919513
0.099 0.104250021373799
0.1 0.105360515657826
0.101 0.106472244510517
0.102 0.107585210679937
0.103 0.108699416923341
0.104 0.109814866007207
0.105 0.110931560707282
0.106 0.112049503808623
0.107 0.113168698105638
0.108 0.114289146402128
0.109 0.115410851511328
0.11 0.116533816255952
0.111 0.117658043468232
0.112 0.118783535989967
0.113 0.119910296672558
0.114 0.121038328377056
0.115 0.122167633974208
0.116 0.123298216344494
0.117 0.124430078378177
0.118 0.125563222975346
0.119 0.126697653045958
0.12 0.127833371509885
0.121 0.12897038129696
0.122 0.13010868534702
0.123 0.131248286609954
0.124 0.132389188045746
0.125 0.133531392624523
0.126 0.134674903326602
0.127 0.135819723142535
0.128 0.136965855073157
0.129 0.138113302129634
0.13 0.139262067333508
0.131 0.140412153716745
0.132 0.141563564321787
0.133 0.142716302201595
0.134 0.143870370419702
0.135 0.145025772050258
0.136 0.146182510178081
0.137 0.147340587898709
0.138 0.148500008318444
0.139 0.149660774554406
0.14 0.150822889734584
0.141 0.151986356997882
0.142 0.153151179494175
0.143 0.154317360384357
0.144 0.155484902840395
0.145 0.156653810045377
0.146 0.157824085193567
0.147 0.158995731490458
0.148 0.160168752152821
0.149 0.161343150408763
0.15 0.162518929497775
0.151 0.16369609267079
0.152 0.164874643190234
0.153 0.166054584330083
0.154 0.167235919375914
0.155 0.168418651624963
0.156 0.16960278438618
0.157 0.170788320980282
0.158 0.17197526473981
0.159 0.173163619009189
0.16 0.174353387144778
0.161 0.175544572514931
0.162 0.176737178500054
0.163 0.177931208492662
0.164 0.179126665897435
0.165 0.180323554131282
0.166 0.18152187662339
0.167 0.182721636815294
0.168 0.183922838160928
0.169 0.185125484126689
0.17 0.186329578191493
0.171 0.187535123846842
0.172 0.188742124596877
0.173 0.189950583958446
0.174 0.191160505461159
0.175 0.192371892647456
0.176 0.193584749072665
0.177 0.194799078305067
0.178 0.196014883925957
0.179 0.197232169529709
0.18 0.198450938723838
0.181 0.199671195129068
0.182 0.20089294237939
0.183 0.202116184122134
0.184 0.20334092401803
0.185 0.204567165741274
0.186 0.205794912979597
0.187 0.207024169434327
0.188 0.208254938820459
0.189 0.209487224866724
0.19 0.210721031315653
0.191 0.211956361923645
0.192 0.213193220461042
0.193 0.214431610712188
0.194 0.215671536475509
0.195 0.216913001563574
0.196 0.218156009803171
0.197 0.219400565035375
0.198 0.220646671115623
0.199 0.221894331913778
0.2 0.22314355131421
0.201 0.224394333215862
0.202 0.225646681532328
0.203 0.226900600191922
0.204 0.228156093137754
0.205 0.229413164327805
0.206 0.230671817735001
0.207 0.231932057347289
0.208 0.233193887167711
0.209 0.234457311214483
0.21 0.23572233352107
0.211 0.236988958136263
0.212 0.238257189124258
0.213 0.239527030564734
0.214 0.24079848655293
0.215 0.242071561199729
0.216 0.243346258631729
0.217 0.244622582991334
0.218 0.245900538436826
0.219 0.247180129142451
0.22 0.2484613592985
0.221 0.249744233111389
0.222 0.251028754803745
0.223 0.25231492861449
0.224 0.253602758798918
0.225 0.25489224962879
0.226 0.25618340539241
0.227 0.257476230394715
0.228 0.258770728957361
0.229 0.260066905418808
0.23 0.261364764134408
0.231 0.262664309476493
0.232 0.263965545834465
0.233 0.265268477614881
0.234 0.266573109241546
0.235 0.267879445155601
0.236 0.269187489815617
0.237 0.27049724769768
0.238 0.271808723295491
0.239 0.273121921120451
0.24 0.27443684570176
0.241 0.275753501586507
0.242 0.277071893339765
0.243 0.278392025544688
0.244 0.279713902802604
0.245 0.281037529733112
0.246 0.282362910974181
0.247 0.283690051182243
0.248 0.285018955032297
0.249 0.286349627218002
0.25 0.287682072451781
0.251 0.289016295464918
0.252 0.29035230100766
0.253 0.29169009384932
0.254 0.293029678778376
0.255 0.294371060602578
0.256 0.295714244149045
0.257 0.297059234264378
0.258 0.298406035814757
0.259 0.29975465368605
0.26 0.301105092783922
0.261 0.302457358033935
0.262 0.303811454381665
0.263 0.3051673867928
0.264 0.306525160253261
0.265 0.3078847797693
0.266 0.309246250367621
0.267 0.310609577095486
0.268 0.311974765020825
0.269 0.313341819232359
0.27 0.3147107448397
0.271 0.316081546973479
0.272 0.317454230785451
0.273 0.318828801448618
0.274 0.320205264157341
0.275 0.321583624127462
0.276 0.322963886596421
0.277 0.324346056823372
0.278 0.325730140089311
0.279 0.327116141697188
0.28 0.328504066972036
0.281 0.32989392126109
0.282 0.331285709933913
0.283 0.332679438382517
0.284 0.334075112021491
0.285 0.335472736288129
0.286 0.336872316642553
0.287 0.338273858567841
0.288 0.339677367570161
0.289 0.341082849178896
0.29 0.342490308946776
0.291 0.343899752450009
0.292 0.345311185288417
0.293 0.346724613085564
0.294 0.348140041488895
0.295 0.349557476169868
0.296 0.350976922824095
0.297 0.352398387171472
0.298 0.353821874956326
0.299 0.355247391947547
0.3 0.356674943938732
0.301 0.358104536748327
0.302 0.359536176219765
0.303 0.360969868221613
0.304 0.362405618647717
0.305 0.363843433417345
0.306 0.365283318475333
0.307 0.366725279792234
0.308 0.368169323364468
0.309 0.369615455214467
0.31 0.371063681390832
0.311 0.372514007968478
0.312 0.373966441048793
0.313 0.375420986759788
0.314 0.376877651256252
0.315 0.378336440719912
0.316 0.379797361359587
0.317 0.381260419411347
0.318 0.382725621138675
0.319 0.384192972832625
0.32 0.385662480811985
0.321 0.387134151423441
0.322 0.388607991041742
0.323 0.390084006069862
0.324 0.391562202939173
0.325 0.393042588109607
0.326 0.39452516806983
0.327 0.396009949337409
0.328 0.397496938458988
0.329 0.398986142010455
0.33 0.400477566597125
0.331 0.401971218853909
0.332 0.403467105445491
0.333 0.404965233066513
0.334 0.406465608441748
0.335 0.407968238326283
0.336 0.409473129505703
0.337 0.410980288796275
0.338 0.412489723045129
0.339 0.414001439130451
0.34 0.415515443961666
0.341 0.41703174447963
0.342 0.41855034765682
0.343 0.420071260497527
0.344 0.421594490038048
0.345 0.423120043346885
0.346 0.424647927524938
0.347 0.426178149705706
0.348 0.427710717055484
0.349 0.429245636773568
0.35 0.430782916092454
0.351 0.432322562278047
0.352 0.433864582629862
0.353 0.435408984481236
0.354 0.436955775199535
0.355 0.438504962186365
0.356 0.440056552877783
0.357 0.441610554744518
0.358 0.443166975292176
0.359 0.444725822061467
0.36 0.446287102628419
0.361 0.447850824604602
0.362 0.449416995637347
0.363 0.450985623409974
0.364 0.452556715642015
0.365 0.454130280089445
0.366 0.455706324544911
0.367 0.457284856837961
0.368 0.45886588483528
0.369 0.460449416440924
0.37 0.462035459596559
0.371 0.463624022281697
0.372 0.465215112513938
0.373 0.466808738349216
0.374 0.468404907882039
0.375 0.470003629245736
0.376 0.471604910612709
0.377 0.473208760194684
0.378 0.474815186242958
0.379 0.476424197048658
0.38 0.478035800943
0.381 0.479650006297541
0.382 0.481266821524446
0.383 0.482886255076749
0.384 0.484508315448617
0.385 0.486133011175619
0.386 0.487760350834995
0.387 0.489390343045926
0.388 0.491022996469811
0.389 0.492658319810542
0.39 0.49429632181478
0.391 0.49593701127224
0.392 0.49758039701597
0.393 0.499226487922639
0.394 0.500875292912823
0.395 0.502526820951296
0.396 0.504181081047322
0.397 0.505838082254952
0.398 0.507497833673316
0.399 0.509160344446929
0.4 0.510825623765991
0.401 0.512493680866688
0.402 0.514164525031505
0.403 0.515838165589535
0.404 0.517514611916787
0.405 0.519193873436507
0.406 0.520875959619492
0.407 0.522560879984412
0.408 0.524248644098131
0.409 0.525939261576039
0.41 0.527632742082372
0.411 0.52932909533055
0.412 0.53102833108351
0.413 0.532730459154041
0.414 0.534435489405125
0.415 0.536143431750281
0.416 0.53785429615391
0.417 0.539568092631645
0.418 0.541284831250699
0.419 0.543004522130226
0.42 0.544727175441672
0.421 0.546452801409142
0.422 0.548181410309759
0.423 0.549913012474038
0.424 0.551647618286246
0.425 0.553385238184787
0.426 0.555125882662571
0.427 0.556869562267398
0.428 0.558616287602339
0.429 0.560366069326127
0.43 0.562118918153541
0.431 0.563874844855806
0.432 0.565633860260986
0.433 0.567395975254385
0.434 0.569161200778954
0.435 0.570929547835696
0.436 0.572701027484078
0.437 0.574475650842447
0.438 0.576253429088446
0.439 0.578034373459441
0.44 0.579818495252942
0.441 0.581605805827038
0.442 0.583396316600826
0.443 0.585190039054853
0.444 0.586986984731555
0.445 0.588787165235703
0.446 0.590590592234853
0.447 0.592397277459802
0.448 0.594207232705042
0.449 0.596020469829223
0.45 0.59783700075562
0.451 0.599656837472607
0.452 0.601479992034121
0.453 0.603306476560156
0.454 0.605136303237232
0.455 0.606969484318893
0.456 0.608806032126194
0.457 0.610645959048202
0.458 0.612489277542491
0.459 0.614336000135656
0.46 0.616186139423817
0.461 0.61803970807314
0.462 0.619896718820353
0.463 0.621757184473273
0.464 0.623621117911335
0.465 0.625488532086131
0.466 0.627359440021942
0.467 0.629233854816293
0.468 0.631111789640493
0.469 0.632993257740198
0.47 0.63487827243597
0.471 0.636766847123838
0.472 0.638658995275876
0.473 0.640554730440775
0.474 0.642454066244427
0.475 0.644357016390513
0.476 0.646263594661095
0.477 0.648173814917214
0.478 0.650087691099498
0.479 0.65200523722877
0.48 0.653926467406664
0.481 0.655851395816248
0.482 0.657780036722654
0.483 0.659712404473708
0.484 0.661648513500574
0.485 0.663588378318401
0.486 0.665532013526972
0.487 0.667479433811368
0.488 0.669430653942629
0.489 0.671385688778433
0.49 0.673344553263766
0.491 0.675307262431614
0.492 0.677273831403655
0.493 0.679244275390954
0.494 0.681218609694672
0.495 0.683196849706777
0.496 0.685179010910768
0.497 0.687165108882398
0.498 0.689155159290408
0.499 0.691149177897272
0.5 0.693147180559945
0.501 0.695149183230618
0.502 0.697155201957484
0.503 0.699165252885508
0.504 0.70117935225721
0.505 0.703197516413447
0.506 0.705219761794215
0.507 0.707246104939447
0.508 0.709276562489829
0.509 0.711311151187616
0.51 0.713349887877465
0.511 0.715392789507265
0.512 0.71743987312899
0.513 0.719491155899547
0.514 0.721546655081643
0.515 0.723606388044654
0.516 0.725670372265505
0.517 0.727738625329564
0.518 0.729811164931537
0.519 0.731888008876376
0.52 0.7339691750802
0.521 0.736054681571222
0.522 0.738144546490681
0.523 0.740238788093796
0.524 0.742337424750717
0.525 0.744440474947496
0.526 0.746547957287061
0.527 0.748659890490204
0.528 0.750776293396582
0.529 0.752897184965719
0.53 0.755022584278033
0.531 0.757152510535858
0.532 0.75928698306449
0.533 0.76142602131324
0.534 0.763569644856491
0.535 0.765717873394781
0.536 0.767870726755882
0.537 0.770028224895903
0.538 0.772190387900398
0.539 0.774357235985489
0.54 0.776528789498996
0.541 0.778705068921592
0.542 0.780886094867952
0.543 0.783071888087932
0.544 0.785262469467751
0.545 0.787457860031187
0.546 0.789658080940789
0.547 0.791863153499103
0.548 0.794073099149906
0.549 0.796287939479459
0.55 0.798507696217772
0.551 0.800732391239883
0.552 0.802962046567152
0.553 0.805196684368568
0.554 0.807436326962073
0.555 0.809680996815897
0.556 0.811930716549912
0.557 0.814185508937001
0.558 0.816445396904439
0.559 0.818710403535291
0.56 0.82098055206983
0.561 0.823255865906966
0.562 0.825536368605691
0.563 0.827822083886547
0.564 0.830113035633103
0.565 0.832409247893453
0.566 0.834710744881732
0.567 0.837017550979647
0.568 0.839329690738027
0.569 0.841647188878389
0.57 0.843970070294529
0.571 0.84629836005412
0.572 0.84863208340034
0.573 0.850971265753513
0.574 0.853315932712767
0.575 0.85566611005772
0.576 0.858021823750179
0.577 0.860383099935859
0.578 0.862749964946125
0.579 0.865122445299756
0.58 0.867500567704723
0.581 0.869884359059999
0.582 0.872273846457381
0.583 0.874669057183336
0.584 0.877070018720874
0.585 0.879476758751439
0.586 0.881889305156823
0.587 0.884307686021104
0.588 0.886731929632611
0.589 0.889162064485902
0.59 0.891598119283784
0.591 0.894040122939335
0.592 0.896488104577975
0.593 0.898942093539542
0.594 0.901402119380404
0.595 0.903868211875598
0.596 0.906340401020987
0.597 0.908818717035454
0.598 0.911303190363116
0.599 0.913793851675568
0.6 0.916290731874155
0.601 0.918793862092274
0.602 0.921303273697699
0.603 0.923818998294947
0.604 0.926341067727656
0.605 0.928869514081015
0.606 0.931404369684203
0.607 0.933945667112876
0.608 0.936493439191674
0.609 0.939047718996771
0.61 0.941608539858445
0.611 0.944175935363691
0.612 0.946749939358864
0.613 0.949330585952355
0.614 0.951917909517306
0.615 0.954511944694353
0.616 0.95711272639441
0.617 0.959720289801491
0.618 0.962334670375562
0.619 0.964955903855436
0.62 0.967584026261706
0.621 0.970219073899711
0.622 0.972861083362549
0.623 0.975510091534126
0.624 0.978166135592243
0.625 0.980829253011726
0.626 0.983499481567605
0.627 0.986176859338322
0.628 0.988861424708991
0.629 0.991553216374702
0.63 0.994252273343867
0.631 0.99695863494161
0.632 0.999672340813206
0.633 1.00239343092757
0.634 1.00512194558077
0.635 1.00785792539965
0.636 1.0106014113454
0.637 1.01335244471729
0.638 1.01611106715637
0.639 1.01887732064926
0.64 1.02165124753198
0.641 1.02443289049386
0.642 1.02722229258144
0.643 1.0300194972025
0.644 1.03282454813011
0.645 1.03563748950672
0.646 1.03845836584836
0.647 1.04128722204884
0.648 1.04412410338404
0.649 1.04696905551627
0.65 1.04982212449868
0.651 1.05268335677971
0.652 1.05555279920766
0.653 1.05843049903528
0.654 1.06131650392441
0.655 1.06421086195078
0.656 1.06711362160874
0.657 1.0700248318162
0.658 1.07294454191953
0.659 1.07587280169862
0.66 1.07880966137193
0.661 1.08175517160169
0.662 1.08470938349912
0.663 1.08767234862978
0.664 1.09064411901893
0.665 1.09362474715707
0.666 1.09661428600544
0.667 1.09961278900169
0.668 1.10262031006565
0.669 1.10563690360507
0.67 1.10866262452161
0.671 1.11169752821677
0.672 1.11474167059799
0.673 1.11779510808488
0.674 1.12085789761543
0.675 1.1239300966524
0.676 1.12701176318981
0.677 1.13010295575948
0.678 1.13320373343773
0.679 1.13631415585212
0.68 1.13943428318837
0.681 1.14256417619729
0.682 1.14570389620196
0.683 1.14885350510486
0.684 1.15201306539523
0.685 1.1551826401565
0.686 1.15836229307388
0.687 1.16155208844198
0.688 1.16475209117265
0.689 1.1679623668029
0.69 1.17118298150295
0.691 1.17441400208439
0.692 1.17765549600856
0.693 1.18090753139494
0.694 1.18417017702976
0.695 1.18744350237473
0.696 1.19072757757592
0.697 1.19402247347277
0.698 1.19732826160727
0.699 1.20064501423326
0.7 1.20397280432594
0.701 1.20731170559145
0.702 1.21066179247673
0.703 1.21402314017944
0.704 1.21739582465808
0.705 1.22077992264232
0.706 1.22417551164346
0.707 1.22758266996507
0.708 1.23100147671386
0.709 1.23443201181064
0.71 1.23787435600162
0.711 1.2413285908697
0.712 1.24479479884619
0.713 1.24827306322252
0.714 1.25176346816228
0.715 1.25526609871349
0.716 1.25878104082093
0.717 1.2623083813389
0.718 1.26584820804402
0.719 1.26940060964839
0.72 1.27296567581289
0.721 1.27654349716077
0.722 1.2801341652915
0.723 1.2837377727948
0.724 1.28735441326499
0.725 1.29098418131557
0.726 1.29462717259407
0.727 1.29828348379718
0.728 1.30195321268614
0.729 1.30563645810244
0.73 1.30933331998376
0.731 1.3130438993803
0.732 1.31676829847128
0.733 1.32050662058189
0.734 1.32425897020044
0.735 1.32802545299591
0.736 1.33180617583582
0.737 1.33560124680437
0.738 1.33941077522104
0.739 1.34323487165944
0.74 1.34707364796661
0.741 1.3509272172826
0.742 1.35479569406052
0.743 1.35867919408692
0.744 1.36257783450257
0.745 1.36649173382371
0.746 1.3704210119636
0.747 1.37436579025462
0.748 1.37832619147071
0.749 1.38230233985035
0.75 1.38629436111989
0.751 1.39030238251743
0.752 1.39432653281715
0.753 1.39836694235416
0.754 1.40242374304977
0.755 1.40649706843741
0.756 1.41058705368894
0.757 1.41469383564159
0.758 1.41881755282545
0.759 1.42295834549148
0.76 1.42711635564015
0.761 1.43129172705063
0.762 1.43548460531066
0.763 1.43969513784701
0.764 1.44392347395653
0.765 1.44816976483798
0.766 1.45243416362444
0.767 1.45671682541644
0.768 1.46101790731583
0.769 1.46533756846034
0.77 1.46967597005894
0.771 1.4740332754279
0.772 1.4784096500277
0.773 1.48280526150073
0.774 1.48722027970985
0.775 1.49165487677772
0.776 1.4961092271271
0.777 1.50058350752202
0.778 1.50507789710986
0.779 1.50959257746438
0.78 1.51412773262978
0.781 1.51868354916564
0.782 1.52326021619305
0.783 1.52785792544168
0.784 1.53247687129797
0.785 1.53711725085447
0.786 1.54177926396029
0.787 1.54646311327271
0.788 1.55116900431012
0.789 1.55589714550607
0.79 1.56064774826467
0.791 1.56542102701733
0.792 1.57021719928082
0.793 1.57503648571677
0.794 1.57987911019256
0.795 1.58474529984373
0.796 1.58963528513792
0.797 1.59454929994035
0.798 1.59948758158093
0.799 1.60445037092306
0.8 1.6094379124341
0.801 1.61445045425764
0.802 1.6194882482876
0.803 1.62455155024415
0.804 1.62964061975162
0.805 1.63475572041839
0.806 1.63989711991881
0.807 1.64506509007725
0.808 1.65025990695436
0.809 1.65548185093551
0.81 1.66073120682165
0.811 1.66600826392249
0.812 1.67131331615219
0.813 1.67664666212755
0.814 1.68200860526894
0.815 1.68739945390381
0.816 1.69281952137315
0.817 1.69826912614072
0.818 1.70374859190534
0.819 1.70925824771631
0.82 1.71479842809193
0.821 1.72036947314138
0.822 1.72597172869005
0.823 1.73160554640831
0.824 1.73727128394399
0.825 1.74296930505862
0.826 1.74869997976761
0.827 1.75446368448436
0.828 1.76026080216868
0.829 1.76609172247948
0.83 1.77195684193188
0.831 1.77785656405906
0.832 1.78379129957888
0.833 1.78976146656538
0.834 1.79576749062559
0.835 1.80180980508156
0.836 1.80788885115794
0.837 1.81400507817537
0.838 1.82015894374975
0.839 1.82635091399767
0.84 1.83258146374831
0.841 1.83885107676191
0.842 1.84516024595517
0.843 1.85150947363383
0.844 1.8578992717326
0.845 1.86433016206289
0.846 1.87080267656851
0.847 1.8773173575897
0.848 1.88387475813586
0.849 1.89047544216721
0.85 1.89711998488588
0.851 1.90380897303668
0.852 1.91054300521802
0.853 1.9173226922034
0.854 1.9241486572738
0.855 1.93102153656156
0.856 1.93794197940614
0.857 1.94491064872223
0.858 1.95192822138088
0.859 1.95899538860397
0.86 1.96611285637283
0.861 1.97328134585145
0.862 1.98050159382493
0.863 1.98777435315401
0.864 1.99510039324608
0.865 2.00248050054371
0.866 2.00991547903123
0.867 2.01740615076038
0.868 2.02495335639577
0.869 2.03255795578099
0.87 2.04022082852655
0.871 2.04794287462046
0.872 2.05572501506252
0.873 2.06356819252355
0.874 2.07147337203066
0.875 2.07944154167984
0.876 2.0874737133771
0.877 2.09557092360972
0.878 2.10373423424888
0.879 2.1119647333854
0.88 2.12026353620009
0.881 2.12863178587061
0.882 2.13707065451647
0.883 2.14558134418438
0.884 2.15416508787577
0.885 2.16282315061889
0.886 2.17155683058764
0.887 2.1803674602698
0.888 2.18925640768704
0.889 2.1982250776698
0.89 2.20727491318972
0.891 2.21640739675299
0.892 2.22562405185792
0.893 2.23492644452023
0.894 2.24431618487007
0.895 2.25379492882461
0.896 2.26336437984076
0.897 2.2730262907525
0.898 2.28278246569787
0.899 2.29263476214088
0.9 2.30258509299405
0.901 2.31263542884755
0.902 2.32278780031157
0.903 2.33304430047875
0.904 2.3434070875143
0.905 2.3538783873816
0.906 2.36446049671213
0.907 2.37515578582888
0.908 2.3859667019331
0.909 2.39689577246529
0.91 2.40794560865187
0.911 2.41911890925
0.912 2.43041846450393
0.913 2.44184716032755
0.914 2.45340798272863
0.915 2.46510402249182
0.916 2.47693848013882
0.917 2.48891467118554
0.918 2.50103603171788
0.919 2.5133061243097
0.92 2.52572864430826
0.921 2.53830742651512
0.922 2.55104645229255
0.923 2.56394985712845
0.924 2.57702193869581
0.925 2.59026716544583
0.926 2.60369018577797
0.927 2.61729583783375
0.928 2.63108915996608
0.929 2.64507540194082
0.93 2.65926003693278
0.931 2.67364877438488
0.932 2.68824757380603
0.933 2.70306265959117
0.934 2.71810053695571
0.935 2.7333680090865
0.936 2.74887219562247
0.937 2.76462055259061
0.938 2.78062089393705
0.939 2.79688141480883
0.94 2.81341071676004
0.941 2.83021783507642
0.942 2.84731226843572
0.943 2.86470401114759
0.944 2.88240358824699
0.945 2.90042209374967
0.946 2.91877123241786
0.947 2.93746336543002
0.948 2.95651156040071
0.949 2.97592964625781
0.95 2.99573227355399
0.951 3.01593498087151
0.952 3.03655426807425
0.953 3.05760767727208
0.954 3.07911388249304
0.955 3.10109278921182
0.956 3.12356564506388
0.957 3.14655516328858
0.958 3.17008566069877
0.959 3.19418321227783
0.96 3.2188758248682
0.961 3.24419363285249
0.962 3.27016911925575
0.963 3.29683736633791
0.964 3.32423634052603
0.965 3.35240721749272
0.966 3.38139475436597
0.967 3.41124771751566
0.968 3.44201937618241
0.969 3.47376807449699
0.97 3.50655789731998
0.971 3.54045944899566
0.972 3.57555076880693
0.973 3.61191841297781
0.974 3.64965874096065
0.975 3.68887945411394
0.976 3.72970144863419
0.977 3.77226106305299
0.978 3.81671282562382
0.979 3.86323284125871
0.98 3.91202300542815
0.981 3.9633162998157
0.982 4.01738352108597
0.983 4.07454193492592
0.984 4.13516655674236
0.985 4.19970507787993
0.986 4.26869794936688
0.987 4.3428059215206
0.988 4.42284862919414
0.989 4.50986000618377
0.99 4.60517018598809
0.991 4.71053070164592
0.992 4.8283137373023
0.993 4.96184512992682
0.994 5.11599580975408
0.995 5.29831736654804
0.996 5.52146091786225
0.997 5.80914299031403
0.998 6.21460809842219
0.999 6.90775527898214
};
\end{axis}

\end{tikzpicture}
	\caption[Cross-Entropy Loss]{Cross-Entropy Loss. Large deviations in probability are strongly penalized.}
	\label{fig:cross-entropy}
\end{figure}
It can be seen, that large deviations in probability are strongly penalized.
In the range of small deviations the slope of the graph is small which leads to little changes in loss, if the probability difference changes only slightly.
Therefore, both kind of errors are penalized.
Using this loss function for computing a cost that averages all losses yields
\begin{equation}
	\label{eq:cross-entropy-cost}
	J(\vec{x}, \vec{W}, \vec{b}, \vec{y}) = J(\hat{\vec{y}}, \vec{y}) = \frac{1}{m} \sum_{i=1}^{m} H(\vec{y}^{(i)}, \hat{\vec{y}}^{(i)})
\end{equation}
as the cost function.
This function depends on all weights and biases as regression parameters, hence, it is highly dimensional.
Minimizing it with respect to the weights and biases yields optimal parameters.