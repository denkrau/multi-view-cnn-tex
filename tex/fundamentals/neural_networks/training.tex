\subsection{Training}
\label{sec:neural-networks-training}
Thus far optimal weights and biases were assumed in all examples.
But in practical terms they need to be found first.
This starts by generating and preparing a dataset from which the network can find correlations.
Based on this, the weights and biases are adapted step wise.
Each of these steps is covered in the following sections.

\subsubsection{Dataset}
\label{sec:dataset-generation}
The whole training process is based on the dataset from which the network is supposed to learn correlations of each input to label.
Usually, a dataset consists of input-label pairs, where the input is the data that is fed into the network and the label is the ground-truth of its class.
In the case of a classification task, the label represents the class.
However, there is no general rule for the amount of data.
It can be said, that more data is better for generalization, but processing too many samples can lead to overfitting the network to the shown data.
The latter means, that the network is trained too long or to intensive on the shown data.
The consequence is, that it adjusts its weights and biases to classify this data perfectly, but cannot reliably generalize to unknown data, because it slightly differs.
Basically, the amount of data depends on the objective of the network.
For classifying whether an image is black or white, only a few training samples would be needed, because this could be classified with the help of few filters, hence, few parameters that need to be adapted.
Is the objective classifying objects within images, it depends on the number of possible objects and their complexity.
If the objects are simple geometric shapes, then not as many samples are needed as if the objects are common objects like type of animals or cars.
There are several datasets available, that are already sorted and labeled, like the MNIST handwritten digits or the ImageNet dataset \cite{Russakovsky:2015:ILS:2846547.2846559}.
Available datasets are not limited to images but may include CAD models like the ModelNet dataset \cite{conf/cvpr/WuSKYZTX15} which is used for the network architecture going to be presented.

Usually, samples of a dataset are split into a training set, test set, and a validation set \cite{James2014}.
The first contains data, the network is trained on.
From this data correlations of each input to label are found.
After an arbitrary number of training steps where parameter changes are executed, the performance of the network is tested on the validation set.
This is data, the network is not trained on.
The objective here is to check if overfitting occurs.
If the loss of the training set decreases, the loss of the validation set has to decrease as well.
This shows that the network still learns and gets better.
If the loss of the training set decreases, but the loss of the validation set stays the same or increases, it is an indicator for overfitting.
This concept is also valid for the accuracy but reversed.
Both cases are visualized in \figref{fig:overfitting}.
The dotted line indicates when training should have been stopped.
\begin{figure}
	\setlength\figureheight{.4\textwidth}
	\setlength\figurewidth{.5\textwidth}
	\centering
	\begin{subfigure}{.5\textwidth}
		% This file was created by matplotlib2tikz v0.7.3.
\begin{tikzpicture}

\definecolor{color0}{rgb}{0.12156862745098,0.466666666666667,0.705882352941177}
\pgfplotsset{ticks=none}
\begin{axis}[
height=\figureheight,
legend cell align={left},
legend style={draw=white!80.0!black},
tick pos=left,
width=\figurewidth,
xlabel={Epoch},
xmin=-0.95, xmax=19.95,
ylabel={Loss},
ymin=0.00263310007998304, ymax=0.232912090270931
]
\draw[dashed] (4,0) -- (4,0.25);
\addplot [semithick, green!50.0!black]
table {%
0 0.22244486344407
1 0.0969903288186217
2 0.0692743199943254
3 0.0526620497363464
4 0.0414353723999035
5 0.0374175925930496
6 0.0302987794400736
7 0.0284669065409941
8 0.0235470414178155
9 0.0218413689432394
10 0.0205645219235344
11 0.0180700652724501
12 0.0201836266599646
13 0.0169590040850697
14 0.0164193118354549
15 0.0152169669780658
16 0.0139801482900892
17 0.0158254974882433
18 0.0135936839485778
19 0.0131003269068443
};
\addlegendentry{Training}
\addplot [semithick, color0]
table {%
0 0.122924613110349
1 0.0839405063420534
2 0.0768479639300611
3 0.0761227550969925
4 0.058712234780658
5 0.06148703338466
6 0.0699209525396516
7 0.0753328067764101
8 0.0766068423317171
9 0.0741938304939711
10 0.0692930655084725
11 0.0736947421684374
12 0.0767673894466832
13 0.0697559238091148
14 0.0750547711637986
15 0.0794915745678727
16 0.0847116109347335
17 0.0895691714388518
18 0.0857177854441012
19 0.0864813314799096
};
\addlegendentry{Validation}
\end{axis}

\end{tikzpicture}
		\caption[Loss]{Loss}
	\end{subfigure}%
	\begin{subfigure}{.5\textwidth}
		% This file was created by matplotlib2tikz v0.7.3.
\begin{tikzpicture}

\definecolor{color0}{rgb}{0.12156862745098,0.466666666666667,0.705882352941177}
\pgfplotsset{ticks=none}
\begin{axis}[
height=\figureheight,
legend cell align={left},
legend style={at={(0.97,0.03)}, anchor=south east, draw=white!80.0!black},
tick pos=left,
width=\figurewidth,
xlabel={Epoch},
xmin=-0.95, xmax=19.95,
ylabel={Accuracy},
ymin=0.931485, ymax=0.998915
]
\draw[dashed] (4,0) -- (4,1);
\addplot [semithick, green!50.0!black]
table {%
0 0.93455
1 0.970033333333333
2 0.9786
3 0.983083333333333
4 0.986416666666667
5 0.987733333333333
6 0.990233333333333
7 0.99035
8 0.992116666666667
9 0.992666666666667
10 0.992883333333333
11 0.9939
12 0.9933
13 0.99435
14 0.9946
15 0.995116666666667
16 0.9953
17 0.99505
18 0.99555
19 0.99585
};
\addlegendentry{Training}
\addplot [semithick, color0]
table {%
0 0.9608
1 0.9735
2 0.9754
3 0.9762
4 0.9823
5 0.9825
6 0.9818
7 0.9799
8 0.9806
9 0.9823
10 0.984
11 0.9823
12 0.9826
13 0.9839
14 0.9849
15 0.9838
16 0.982
17 0.9823
18 0.9829
19 0.9833
};
\addlegendentry{Validation}
\end{axis}

\end{tikzpicture}
		\caption[Accuracy]{Accuracy}
	\end{subfigure}
	\caption[Indicator of overfitting]{Indicator of overfitting. If loss or accuracy of the validation set do not follow the direction of the one of the training set the network does not generalize. The dashed line represents when training should have been stopped.}
	\label{fig:overfitting}
\end{figure}

The test set is data the network is not trained on as well.
It serves as a final performance check of the network to confirm its general accuracy.
It is bad practice but if no validation set is available, the testing set can be used.
How the dataset is split depends again on the number and complexity of samples and the objective.
However, an equal distribution of samples in terms of samples per class in each set should be minded.
This means, if the network trains mostly on sweatshirts a test set with mostly pants would not yield an acceptable accuracy, because the network does not know these particular features.

For processing the dataset, a one-hot encoding \cite{Harris2012} of the labels can be performed.
Usually, the labels are categorical data.
This means, they contain label or string values, respectively, instead of numeric values, that the networks needs.
For example, there is a fashion variable with the values "boot", "sweatshirt" and "pants".
The network would not know how to interpret these.
Thus, these values need to be converted to numeric values.
Furthermore, if these label values are outputs of the network, it should be easily possible to convert them back from numeric values.
Hence, they are converted to integers that represent a class.
Referring to the example, this results in the numeric values 0, 1 and 2 for the labels "boot", "sweatshirt" and "pants", respectively.
But numeric values have a natural ordered relationship between each other, that neural networks could exploit.
The index of "pants" is higher than the one of "boot", but neither of these classes is better or worse than the other.
Therefore, the indices are one-hot encoded as well.
This means removing the integer representation and inserting binary variables for simulating existing features.
Applying this to the example results in the label vector $\vec{y}^{(2)} = (0, 1, 0)^T$ for the "sweatshirt" label.
This vector has a length of the number of different classes available, where every element is 0 except the one of the corresponding class which is 1.
\tabref{tab:one-hot-encoding} summarizes this approach.
This encoding needs to be performed on each label in the dataset.
\begin{table}[]
	\caption[One-Hot Encoding of Categorical Data]{One-hot encoding of categorical data. First, categorical label values are transformed to numeric values representing a class index. Then, this is replaced with binary variables to represent features, that removes the natural relationship of numeric values to each other. This vector has a length of the number of different classes, where every element is 0 except for the corresponding class which is 1.}
	\label{tab:one-hot-encoding}
	\centering
	\begin{tabular}{l|l|l}
		Categorical   & Integer & One-Hot                   \\ \hline
		"Boot"       & 0       & $\vec{y}^{(1)} = (1, 0, 0)^T$ \\
		"Sweatshirt" & 1       & $\vec{y}^{(2)} = (0, 1, 0)^T$  \\
		"Pants"      & 2       & $\vec{y}^{(3)} = (0, 0, 1)^T$
	\end{tabular}
\end{table}
\subsubsection{Stochastic Gradient Descent}
\label{sec:training-stochastic-gradient}
\subsubsection{Adam Optimization Algorithm}
\label{sec:training-adam}

Expressing the activation of every node with \thmref{eq:perceptron-sum} would get very complex with only a few nodes.
Hence, a matrix notation is the way to go in the long run.
First, for every layer a vector is build