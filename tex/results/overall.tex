\section{Overall Performance}
\label{sec:results-overall}
For evaluating the overall performances of the network architecture, the choice of hyperparameters is explained.
In \figref{fig:optimal-learning-rate} the increase of the learning rate against the related loss is shown for finding the range of optimal learning rates.
Additionally, the gradient of loss with respect to the learning rate is outlined.
This is performed once on the 0-3 network.
At a learning rate of approximately $\gamma = 10^{-4}$ the loss starts to decrease slightly.
At approximately $\gamma=0.004$ its decrease gets strikingly faster until at approximately $\gamma=0.01$ it starts to increase drastically.
Although for this network, in particular, a learning rate of $\gamma=10^{-2}$ seems to be suited well, a general initial learning rate of $\gamma = 10^{-3}$ is chosen.
On one hand, this is the most basic network, hence, it is supposed, that for more complicated ones, a smaller learning rate is better suited due to the more complex cost function.
Furthermore, interpreting those graphs is time-consuming and for more complicated networks not that easy anymore, because the loss changes more rapidly.
On the other hand, a learning rate of $\gamma=10^{-2}$ is close to the increase.
Hence, if the learning rate is shifted, the parameters of the network would be changed tremendously.
It was actually verified, that a learning rate of $\gamma=10^{-3}$ is a satisfiable choice for more complex networks because it lies close to the upper bound of the optimal learning rates range.
As a default value for all networks, it works as well, though.
\begin{figure}
	\setlength\figureheight{.3\textwidth}
	\setlength\figurewidth{.45\textwidth}
	\centering
	\begin{subfigure}{.5\textwidth}
		\centering
		% This file was created by matplotlib2tikz v0.7.3.
\begin{tikzpicture}

\begin{axis}[
height=\figureheight,
width=\figurewidth,
log basis x={10},
tick align=outside,
tick pos=left,
x grid style={white!69.01960784313725!black},
xlabel={Learning Rate},
xmin=6.93261084814295e-06, xmax=0.0219364693423635,
xmode=log,
xtick style={color=black},
xtick={1e-07,1e-06,1e-05,0.0001,0.001,0.01,0.1,1},
xticklabels={,,\(\displaystyle 10^{-5}\),\(\displaystyle 10^{-4}\),\(\displaystyle 10^{-3}\),\(\displaystyle 10^{-2}\),,},
y grid style={white!69.01960784313725!black},
ylabel={Loss},
ymin=-0.0923433851450682, ymax=3.27405786775053,
ytick style={color=black},
ytick={-0.5,0,0.5,1,1.5,2,2.5,3,3.5},
yticklabels={,\(\displaystyle 0.0\),\(\displaystyle 0.5\),\(\displaystyle 1.0\),\(\displaystyle 1.5\),\(\displaystyle 2.0\),\(\displaystyle 2.5\),\(\displaystyle 3.0\),}
]
\addplot [semithick, green!50.0!black]
table {%
1e-05 1.24936330318451
1.02e-05 1.11771082878113
1.0404e-05 1.25553345680237
1.061208e-05 0.912209510803223
1.08243216e-05 1.17866849899292
1.1040808032e-05 1.14146018028259
1.126162419264e-05 1.00543582439423
1.14868566764928e-05 1.1741509437561
1.17165938100227e-05 0.712578415870667
1.19509256862231e-05 1.09097123146057
1.21899441999476e-05 1.04582691192627
1.24337430839465e-05 1.04490494728088
1.26824179456255e-05 1.12166380882263
1.2936066304538e-05 1.20993518829346
1.31947876306287e-05 1.20642042160034
1.34586833832413e-05 1.25412130355835
1.37278570509061e-05 1.18533754348755
1.40024141919243e-05 1.04806065559387
1.42824624757627e-05 1.03853487968445
1.4568111725278e-05 1.12853026390076
1.48594739597836e-05 1.18146014213562
1.51566634389792e-05 1.21181511878967
1.54597967077588e-05 1.07602608203888
1.5768992641914e-05 1.03872931003571
1.60843724947523e-05 1.13338041305542
1.64060599446473e-05 1.07078385353088
1.67341811435403e-05 1.11643004417419
1.70688647664111e-05 1.11921977996826
1.74102420617393e-05 1.07632839679718
1.77584469029741e-05 1.09956479072571
1.81136158410335e-05 1.01091659069061
1.84758881578542e-05 1.08717322349548
1.88454059210113e-05 1.06513786315918
1.92223140394315e-05 1.04800224304199
1.96067603202202e-05 1.01836133003235
1.99988955266246e-05 1.11461520195007
2.03988734371571e-05 1.03720045089722
2.08068509059002e-05 1.1808967590332
2.12229879240182e-05 1.11440849304199
2.16474476824986e-05 1.08650374412537
2.20803966361485e-05 1.04353654384613
2.25220045688715e-05 1.18413186073303
2.29724446602489e-05 1.03592896461487
2.34318935534539e-05 1.10247671604156
2.3900531424523e-05 1.07771921157837
2.43785420530135e-05 1.14615738391876
2.48661128940737e-05 1.06303918361664
2.53634351519552e-05 0.943422675132751
2.58707038549943e-05 1.03866362571716
2.63881179320942e-05 1.18736612796783
2.69158802907361e-05 1.07216048240662
2.74541978965508e-05 1.00082540512085
2.80032818544818e-05 1.03717625141144
2.85633474915715e-05 1.09771478176117
2.91346144414029e-05 1.09777176380157
2.9717306730231e-05 1.0956357717514
3.03116528648356e-05 1.11595177650452
3.09178859221323e-05 1.04101479053497
3.15362436405749e-05 1.01115465164185
3.21669685133864e-05 1.02764511108398
3.28103078836542e-05 1.24870812892914
3.34665140413272e-05 1.00507831573486
3.41358443221538e-05 1.11056876182556
3.48185612085969e-05 1.08555150032043
3.55149324327688e-05 1.00557243824005
3.62252310814242e-05 0.944729149341583
3.69497357030527e-05 1.11037909984589
3.76887304171137e-05 1.16250884532928
3.8442505025456e-05 0.958189606666565
3.92113551259651e-05 1.08383274078369
3.99955822284844e-05 1.12642049789429
4.07954938730541e-05 1.18159532546997
4.16114037505152e-05 1.12061786651611
4.24436318255255e-05 1.08124768733978
4.3292504462036e-05 1.08979761600494
4.41583545512767e-05 1.03111398220062
4.50415216423022e-05 1.11460852622986
4.59423520751483e-05 1.05480051040649
4.68611991166513e-05 1.09905767440796
4.77984230989843e-05 1.05474781990051
4.8754391560964e-05 1.08756709098816
4.97294793921832e-05 1.07962369918823
5.07240689800269e-05 1.036789894104
5.17385503596274e-05 1.08880829811096
5.277332136682e-05 1.06424152851105
5.38287877941564e-05 1.02296662330627
5.49053635500395e-05 0.785083770751953
5.60034708210403e-05 1.11297142505646
5.71235402374611e-05 1.02365207672119
5.82660110422103e-05 1.05898761749268
5.94313312630546e-05 1.05686950683594
6.06199578883156e-05 1.04965496063232
6.1832357046082e-05 1.01759386062622
6.30690041870036e-05 1.00451576709747
6.43303842707437e-05 1.00840127468109
6.56169919561585e-05 1.19472241401672
6.69293317952817e-05 1.16187214851379
6.82679184311873e-05 0.973621010780334
6.96332767998111e-05 0.987213134765625
7.10259423358073e-05 1.14957332611084
7.24464611825234e-05 1.05067920684814
7.38953904061739e-05 1.10304892063141
7.53732982142974e-05 1.10134148597717
7.68807641785834e-05 1.05486869812012
7.8418379462155e-05 0.917574405670166
7.99867470513981e-05 0.872981131076813
8.15864819924261e-05 1.05475461483002
8.32182116322746e-05 1.11500573158264
8.48825758649201e-05 1.00380003452301
8.65802273822185e-05 0.92658805847168
8.83118319298629e-05 1.10154104232788
9.00780685684601e-05 1.24234402179718
9.18796299398294e-05 1.04951858520508
9.37172225386259e-05 1.05376398563385
9.55915669893985e-05 0.819953501224518
9.75033983291864e-05 1.15897917747498
9.94534662957702e-05 1.09967350959778
0.000101442535621686 0.982080698013306
0.000103471386334119 0.994948983192444
0.000105540814060802 1.03953456878662
0.000107651630342018 1.02697360515594
0.000109804662948858 1.01501739025116
0.000112000756207835 0.987826108932495
0.000114240771331992 1.10533022880554
0.000116525586758632 1.05225586891174
0.000118856098493804 1.02115893363953
0.000121233220463681 1.00868940353394
0.000123657884872954 0.971531629562378
0.000126131042570413 0.957609057426453
0.000128653663421821 0.930508971214294
0.000131226736690258 0.935918867588043
0.000133851271424063 0.999650537967682
0.000136528296852544 0.943161368370056
0.000139258862789595 1.05014622211456
0.000142044040045387 1.02964305877686
0.000144884920846295 1.01328790187836
0.000147782619263221 0.864341855049133
0.000150738271648485 0.996880412101746
0.000153753037081455 1.10514998435974
0.000156828097823084 0.86850917339325
0.000159964659779546 1.12600386142731
0.000163163952975137 1.03035688400269
0.000166427232034639 1.03447246551514
0.000169755776675332 0.958478450775146
0.000173150892208839 1.04999423027039
0.000176613910053016 1.04217743873596
0.000180146188254076 1.10224974155426
0.000183749112019157 0.968043446540833
0.000187424094259541 1.03931498527527
0.000191172576144731 1.02331411838531
0.000194996027667626 1.002032995224
0.000198895948220979 1.02703738212585
0.000202873867185398 0.932914614677429
0.000206931344529106 0.852756261825562
0.000211069971419688 1.02950429916382
0.000215291370848082 0.808874905109406
0.000219597198265044 0.931403756141663
0.000223989142230344 1.01861166954041
0.000228468925074951 1.08745992183685
0.00023303830357645 1.13773846626282
0.000237699069647979 0.985420346260071
0.000242453051040939 1.0529021024704
0.000247302112061758 1.03232717514038
0.000252248154302993 0.925907611846924
0.000257293117389053 0.884053707122803
0.000262438979736834 0.6212078332901
0.000267687759331571 1.06439614295959
0.000273041514518202 1.04348242282867
0.000278502344808566 1.00455856323242
0.000284072391704737 0.966708421707153
0.000289753839538832 0.887700915336609
0.000295548916329609 1.02882313728333
0.000301459894656201 0.837652683258057
0.000307489092549325 1.14834105968475
0.000313638874400312 1.03436255455017
0.000319911651888318 0.980852484703064
0.000326309884926084 0.89490282535553
0.000332836082624606 0.977206707000732
0.000339492804277098 1.0045964717865
0.00034628266036264 0.926256656646729
0.000353208313569893 0.977613031864166
0.000360272479841291 0.992047607898712
0.000367477929438116 0.915925562381744
0.000374827488026879 0.663772523403168
0.000382324037787416 1.08836114406586
0.000389970518543165 0.85827511548996
0.000397769928914028 0.964348614215851
0.000405725327492308 1.0613226890564
0.000413839834042155 0.988701462745667
0.000422116630722998 0.988372802734375
0.000430558963337458 0.946321308612823
0.000439170142604207 0.692878603935242
0.000447953545456291 1.05103445053101
0.000456912616365417 0.987674474716187
0.000466050868692725 0.757144212722778
0.00047537188606658 0.935925245285034
0.000484879323787911 0.821495890617371
0.000494576910263669 1.02408981323242
0.000504468448468943 0.949554920196533
0.000514557817438322 0.886516690254211
0.000524848973787088 0.825913190841675
0.00053534595326283 1.00788497924805
0.000546052872328086 0.739642798900604
0.000556973929774648 0.918576240539551
0.000568113408370141 0.970334768295288
0.000579475676537544 0.802989959716797
0.000591065190068295 0.817724704742432
0.000602886493869661 0.925661981105804
0.000614944223747054 0.849228799343109
0.000627243108221995 0.977990925312042
0.000639787970386435 0.840499103069305
0.000652583729794164 0.969197034835815
0.000665635404390047 0.677034854888916
0.000678948112477848 0.825850367546082
0.000692527074727405 0.810201406478882
0.000706377616221953 0.784642696380615
0.000720505168546392 0.904613196849823
0.00073491527191732 0.919641375541687
0.000749613577355666 0.804116487503052
0.00076460584890278 0.874090015888214
0.000779897965880835 0.523968160152435
0.000795495925198452 0.914509415626526
0.000811405843702421 0.99234414100647
0.00082763396057647 0.960276961326599
0.000844186639787999 0.935146510601044
0.000861070372583759 0.875925540924072
0.000878291780035434 0.951701283454895
0.000895857615636143 0.874415755271912
0.000913774767948866 0.934476494789124
0.000932050263307843 0.917530179023743
0.000950691268574 0.885922908782959
0.00096970509394548 0.84727942943573
0.00098909919582439 0.829551100730896
0.00100888117974088 0.951678514480591
0.0010290588033357 0.925641059875488
0.00104963997940241 0.862579643726349
0.00107063277899046 0.933078765869141
0.00109204543457027 0.795441687107086
0.00111388634326167 0.912636935710907
0.0011361640701269 0.871430814266205
0.00115888735152944 0.896076083183289
0.00118206509856003 0.866928339004517
0.00120570640053123 0.925468623638153
0.00122982052854186 0.759742617607117
0.00125441693911269 0.840595006942749
0.00127950527789495 0.894559741020203
0.00130509538345285 0.89877849817276
0.0013311972911219 0.897607624530792
0.00135782123694434 0.820272028446198
0.00138497766168323 0.951055288314819
0.00141267721491689 0.949528515338898
0.00144093075921523 0.97175008058548
0.00146974937439954 0.885736107826233
0.00149914436188753 0.86873722076416
0.00152912724912528 0.952134788036346
0.00155970979410778 0.75347101688385
0.00159090398998994 0.836939334869385
0.00162272206978974 0.567769050598145
0.00165517651118553 0.886343419551849
0.00168828004140924 0.789159059524536
0.00172204564223743 0.824781596660614
0.00175648655508218 0.72742748260498
0.00179161628618382 0.678035080432892
0.0018274486119075 0.665984451770782
0.00186399758414565 0.763284206390381
0.00190127753582856 0.857166409492493
0.00193930308654513 0.815934062004089
0.00197808914827603 0.924270272254944
0.00201765093124155 0.782942116260529
0.00205800394986639 0.835995197296143
0.00209916402886371 0.704138517379761
0.00214114730944099 0.855893731117249
0.00218397025562981 0.770207464694977
0.0022276496607424 0.90678858757019
0.00227220265395725 0.886987566947937
0.0023176467070364 0.977225720882416
0.00236399964117713 0.831437289714813
0.00241127963400067 0.843997895717621
0.00245950522668068 0.787012100219727
0.00250869533121429 0.816235184669495
0.00255886923783858 0.603344440460205
0.00261004662259535 0.744927525520325
0.00266224755504726 0.63431715965271
0.0027154925061482 0.653486669063568
0.00276980235627117 0.761883080005646
0.00282519840339659 0.78454464673996
0.00288170237146452 0.791627764701843
0.00293933641889381 0.744171023368835
0.00299812314727169 0.822568416595459
0.00305808561021712 0.746788322925568
0.00311924732242147 0.755545496940613
0.0031816322688699 0.761896312236786
0.00324526491424729 0.343267858028412
0.00331017021253224 0.812684655189514
0.00337637361678289 0.82405960559845
0.00344390108911854 0.665336489677429
0.00351277911090091 0.846346139907837
0.00358303469311893 0.673141419887543
0.00365469538698131 0.567845821380615
0.00372778929472094 0.663246750831604
0.00380234508061536 0.344445824623108
0.00387839198222766 0.601183354854584
0.00395595982187222 0.619645893573761
0.00403507901830966 0.540554285049438
0.00411578059867586 0.960864663124084
0.00419809621064937 0.415902137756348
0.00428205813486236 0.75158280134201
0.00436769929755961 0.666090488433838
0.0044550532835108 0.611324548721313
0.00454415434918102 0.420097589492798
0.00463503743616464 0.484608590602875
0.00472773818488793 0.785852193832397
0.00482229294858569 0.721708297729492
0.0049187388075574 0.697919011116028
0.00501711358370855 0.697352528572083
0.00511745585538272 0.614921867847443
0.00521980497249038 0.694017469882965
0.00532420107194018 0.506624698638916
0.00543068509337899 0.189157143235207
0.00553929879524657 0.430464923381805
0.0056500847711515 0.656598508358002
0.00576308646657453 0.917730212211609
0.00587834819590602 0.295652449131012
0.00599591515982414 0.801897764205933
0.00611583346302062 1.3061910867691
0.00623815013228103 0.925675451755524
0.00636291313492665 0.491060376167297
0.00649017139762519 0.261722505092621
0.00661997482557769 0.720268428325653
0.00675237432208924 1.37355244159698
0.00688742180853103 0.84514856338501
0.00702517024470165 0.472819209098816
0.00716567364959568 0.387085288763046
0.0073089871225876 0.555275678634644
0.00745516686503935 0.529918491840363
0.00760427020234014 0.490843087434769
0.00775635560638694 0.192046701908112
0.00791148271851468 0.320494651794434
0.00806971237288497 0.456416308879852
0.00823110662034267 0.379168927669525
0.00839572875274952 0.264919608831406
0.00856364332780452 0.240180894732475
0.00873491619436061 0.258248090744019
0.00890961451824782 0.502431750297546
0.00908780680861278 0.256245642900467
0.00926956294478503 0.414601385593414
0.00945495420368073 0.0695068910717964
0.00964405328775435 0.710331380367279
0.00983693435350943 0.300577878952026
0.0100336730405796 0.141580551862717
0.0102343465013912 0.418671369552612
0.010439033431419 0.359317660331726
0.0106478141000474 0.131026819348335
0.0108607703820484 0.379914164543152
0.0110779857896893 0.367072641849518
0.0112995455054831 0.134323164820671
0.0115255364155928 0.110901422798634
0.0117560471439046 0.267156094312668
0.0119911680867827 0.356623739004135
0.0122309914485184 0.244695216417313
0.0124756112774888 0.104603454470634
0.0127251235030385 0.0606748536229134
0.0129796259730993 0.138397783041
0.0132392184925613 0.135786771774292
0.0135040028624125 0.49864587187767
0.0137740829196608 1.54214537143707
0.014049564578054 0.34622585773468
0.0143305558696151 0.560321569442749
0.0146171669870074 0.166388660669327
0.0149095103267475 1.61452615261078
0.0152077005332825 3.12103962898254
};
\end{axis}

\end{tikzpicture}
	\end{subfigure}%
	\begin{subfigure}{.5\textwidth}
		\centering
		% This file was created by matplotlib2tikz v0.7.3.
\begin{tikzpicture}

\begin{axis}[
height=\figureheight,
width=\figurewidth,
log basis x={10},
tick align=outside,
tick pos=left,
x grid style={white!69.01960784313725!black},
xlabel={Learning Rate},
xmin=6.93261084814295e-06, xmax=0.0219364693423635,
xmode=log,
xtick style={color=black},
xtick={1e-07,1e-06,1e-05,0.0001,0.001,0.01,0.1,1},
xticklabels={,,\(\displaystyle 10^{-5}\),\(\displaystyle 10^{-4}\),\(\displaystyle 10^{-3}\),\(\displaystyle 10^{-2}\),,},
y grid style={white!69.01960784313725!black},
ylabel={\(\displaystyle \partial\)Loss},
ymin=-0.590783169865608, ymax=1.60638474524021,
ytick style={color=black},
ytick={-0.75,-0.5,-0.25,0,0.25,0.5,0.75,1,1.25,1.5,1.75},
yticklabels={,\(\displaystyle -0.50\),\(\displaystyle -0.25\),\(\displaystyle 0.00\),\(\displaystyle 0.25\),\(\displaystyle 0.50\),\(\displaystyle 0.75\),\(\displaystyle 1.00\),\(\displaystyle 1.25\),\(\displaystyle 1.50\),}
]
\addplot [semithick, green!50.0!black]
table {%
1e-05 -0.131652474403381
1.02e-05 0.00308507680892944
1.0404e-05 -0.102750658988953
1.061208e-05 -0.0384324789047241
1.08243216e-05 0.114625334739685
1.1040808032e-05 -0.0866163372993469
1.126162419264e-05 0.0163453817367554
1.14868566764928e-05 -0.14642870426178
1.17165938100227e-05 -0.0415898561477661
1.19509256862231e-05 0.166624248027802
1.21899441999476e-05 -0.0230331420898438
1.24337430839465e-05 0.0379184484481812
1.26824179456255e-05 0.0825151205062866
1.2936066304538e-05 0.042378306388855
1.31947876306287e-05 0.0220930576324463
1.34586833832413e-05 -0.0105414390563965
1.37278570509061e-05 -0.103030323982239
1.40024141919243e-05 -0.0734013319015503
1.42824624757627e-05 0.0402348041534424
1.4568111725278e-05 0.0714626312255859
1.48594739597836e-05 0.041642427444458
1.51566634389792e-05 -0.0527170300483704
1.54597967077588e-05 -0.0865429043769836
1.5768992641914e-05 0.0286771655082703
1.60843724947523e-05 0.0160272717475891
1.64060599446473e-05 -0.00847518444061279
1.67341811435403e-05 0.024217963218689
1.70688647664111e-05 -0.0200508236885071
1.74102420617393e-05 -0.00982749462127686
1.77584469029741e-05 -0.0327059030532837
1.81136158410335e-05 -0.0061957836151123
1.84758881578542e-05 0.0271106362342834
1.88454059210113e-05 -0.0195854902267456
1.92223140394315e-05 -0.0233882665634155
1.96067603202202e-05 0.0333064794540405
1.99988955266246e-05 0.00941956043243408
2.03988734371571e-05 0.0331407785415649
2.08068509059002e-05 0.0386040210723877
2.12229879240182e-05 -0.0471965074539185
2.16474476824986e-05 -0.0354359745979309
2.20803966361485e-05 0.048814058303833
2.25220045688715e-05 -0.0038037896156311
2.29724446602489e-05 -0.0408275723457336
2.34318935534539e-05 0.0208951234817505
2.3900531424523e-05 0.0218403339385986
2.43785420530135e-05 -0.00734001398086548
2.48661128940737e-05 -0.101367354393005
2.53634351519552e-05 -0.0121877789497375
2.58707038549943e-05 0.121971726417542
2.63881179320942e-05 0.0167484283447266
2.69158802907361e-05 -0.0932703614234924
2.74541978965508e-05 -0.0174921154975891
2.80032818544818e-05 0.0484446883201599
2.85633474915715e-05 0.0302977561950684
2.91346144414029e-05 -0.00103950500488281
2.9717306730231e-05 0.00909000635147095
3.03116528648356e-05 -0.0273104906082153
3.09178859221323e-05 -0.0523985624313354
3.15362436405749e-05 -0.00668483972549438
3.21669685133864e-05 0.118776738643646
3.28103078836542e-05 -0.0112833976745605
3.34665140413272e-05 -0.0690696835517883
3.41358443221538e-05 0.0402365922927856
3.48185612085969e-05 -0.0524981617927551
3.55149324327688e-05 -0.0704111754894257
3.62252310814242e-05 0.0524033308029175
3.69497357030527e-05 0.108889847993851
3.76887304171137e-05 -0.0760947465896606
3.8442505025456e-05 -0.0393380522727966
3.92113551259651e-05 0.0841154456138611
3.99955822284844e-05 0.0488812923431396
4.07954938730541e-05 -0.00290131568908691
4.16114037505152e-05 -0.050173819065094
4.24436318255255e-05 -0.0154101252555847
4.3292504462036e-05 -0.0250668525695801
4.41583545512767e-05 0.0124054551124573
4.50415216423022e-05 0.0118432641029358
4.59423520751483e-05 -0.00777542591094971
4.68611991166513e-05 -2.63452529907227e-05
4.77984230989843e-05 -0.0057452917098999
4.8754391560964e-05 0.0124379396438599
4.97294793921832e-05 -0.0253885984420776
5.07240689800269e-05 0.00459229946136475
5.17385503596274e-05 0.0137258172035217
5.277332136682e-05 -0.0329208374023438
5.38287877941564e-05 -0.139578878879547
5.49053635500395e-05 0.0450024008750916
5.60034708210403e-05 0.119284152984619
5.71235402374611e-05 -0.0269919037818909
5.82660110422103e-05 0.016608715057373
5.94313312630546e-05 -0.00466632843017578
6.06199578883156e-05 -0.0196378231048584
6.1832357046082e-05 -0.0225695967674255
6.30690041870036e-05 -0.0045962929725647
6.43303842707437e-05 0.0951033234596252
6.56169919561585e-05 0.0767354369163513
6.69293317952817e-05 -0.110550701618195
6.82679184311873e-05 -0.0873295068740845
6.96332767998111e-05 0.0879761576652527
7.10259423358073e-05 0.0317330360412598
7.24464611825234e-05 -0.0232622027397156
7.38953904061739e-05 0.0253311395645142
7.53732982142974e-05 -0.0240901112556458
7.68807641785834e-05 -0.0918835401535034
7.8418379462155e-05 -0.0909437835216522
7.99867470513981e-05 0.0685901045799255
8.15864819924261e-05 0.121012300252914
8.32182116322746e-05 -0.0254772901535034
8.48825758649201e-05 -0.094208836555481
8.65802273822185e-05 0.0488705039024353
8.83118319298629e-05 0.15787798166275
9.00780685684601e-05 -0.0260112285614014
9.18796299398294e-05 -0.094290018081665
9.37172225386259e-05 -0.11478254199028
9.55915669893985e-05 0.0526075959205627
9.75033983291864e-05 0.13986000418663
9.94534662957702e-05 -0.088449239730835
0.000101442535621686 -0.0523622632026672
0.000103471386334119 0.0287269353866577
0.000105540814060802 0.0160123109817505
0.000107651630342018 -0.0122585892677307
0.000109804662948858 -0.0195737481117249
0.000112000756207835 0.0451564192771912
0.000114240771331992 0.032214879989624
0.000116525586758632 -0.0420856475830078
0.000118856098493804 -0.0217832326889038
0.000121233220463681 -0.0248136520385742
0.000123657884872954 -0.0255401730537415
0.000126131042570413 -0.0205113291740417
0.000128653663421821 -0.0108450949192047
0.000131226736690258 0.0345707833766937
0.000133851271424063 0.00362125039100647
0.000136528296852544 0.0252478420734406
0.000139258862789595 0.0432408452033997
0.000142044040045387 -0.018429160118103
0.000144884920846295 -0.0826506018638611
0.000147782619263221 -0.00820374488830566
0.000150738271648485 0.120404064655304
0.000153753037081455 -0.064185619354248
0.000156828097823084 0.010426938533783
0.000159964659779546 0.080923855304718
0.000163163952975137 -0.0457656979560852
0.000166427232034639 -0.0359392166137695
0.000169755776675332 0.00776088237762451
0.000173150892208839 0.0418494939804077
0.000176613910053016 0.0261277556419373
0.000180146188254076 -0.0370669960975647
0.000183749112019157 -0.0314673781394958
0.000187424094259541 0.0276353359222412
0.000191172576144731 -0.0186409950256348
0.000194996027667626 0.00186163187026978
0.000198895948220979 -0.0345591902732849
0.000202873867185398 -0.0871405601501465
0.000206931344529106 0.0482948422431946
0.000211069971419688 -0.021940678358078
0.000215291370848082 -0.0490502715110779
0.000219597198265044 0.1048683822155
0.000223989142230344 0.0780280828475952
0.000228468925074951 0.0595633983612061
0.00023303830357645 -0.0510197877883911
0.000237699069647979 -0.0424181818962097
0.000242453051040939 0.023453414440155
0.000247302112061758 -0.0634972453117371
0.000252248154302993 -0.0741367340087891
0.000257293117389053 -0.152349889278412
0.000262438979736834 0.090171217918396
0.000267687759331571 0.211137294769287
0.000273041514518202 -0.0299187898635864
0.000278502344808566 -0.0383870005607605
0.000284072391704737 -0.0584288239479065
0.000289753839538832 0.0310573577880859
0.000295548916329609 -0.0250241160392761
0.000301459894656201 0.0597589612007141
0.000307489092549325 0.0983549356460571
0.000313638874400312 -0.0837442874908447
0.000319911651888318 -0.0697298645973206
0.000326309884926084 -0.00182288885116577
0.000332836082624606 0.0548468232154846
0.000339492804277098 -0.025475025177002
0.00034628266036264 -0.0134917199611664
0.000353208313569893 0.0328954756259918
0.000360272479841291 -0.0308437347412109
0.000367477929438116 -0.164137542247772
0.000374827488026879 0.0862177908420563
0.000382324037787416 0.097251296043396
0.000389970518543165 -0.0620062649250031
0.000397769928914028 0.101523786783218
0.000405725327492308 0.0121764242649078
0.000413839834042155 -0.0364749431610107
0.000422116630722998 -0.0211900770664215
0.000430558963337458 -0.147747099399567
0.000439170142604207 0.0523565709590912
0.000447953545456291 0.147397935390472
0.000456912616365417 -0.146945118904114
0.000466050868692725 -0.0258746147155762
0.00047537188606658 0.0321758389472961
0.000484879323787911 0.0440822839736938
0.000494576910263669 0.0640295147895813
0.000504468448468943 -0.0687865614891052
0.000514557817438322 -0.0618208646774292
0.000524848973787088 0.0606841444969177
0.00053534595326283 -0.0431351959705353
0.000546052872328086 -0.044654369354248
0.000556973929774648 0.115345984697342
0.000568113408370141 -0.057793140411377
0.000579475676537544 -0.0763050317764282
0.000591065190068295 0.0613360106945038
0.000602886493869661 0.0157520473003387
0.000614944223747054 0.0261644721031189
0.000627243108221995 -0.00436484813690186
0.000639787970386435 -0.0043969452381134
0.000652583729794164 -0.0817321240901947
0.000665635404390047 -0.0716733336448669
0.000678948112477848 0.0665832757949829
0.000692527074727405 -0.0206038355827332
0.000706377616221953 0.0472058951854706
0.000720505168546392 0.0674993395805359
0.00073491527191732 -0.0502483546733856
0.000749613577355666 -0.0227756798267365
0.00076460584890278 -0.140074163675308
0.000779897965880835 0.0202096998691559
0.000795495925198452 0.234187990427017
0.000811405843702421 0.0228837728500366
0.00082763396057647 -0.028598815202713
0.000844186639787999 -0.0421757102012634
0.000861070372583759 0.00827738642692566
0.000878291780035434 -0.000754892826080322
0.000895857615636143 -0.00861239433288574
0.000913774767948866 0.0215572118759155
0.000932050263307843 -0.0242767930030823
0.000950691268574 -0.0351253747940063
0.00096970509394548 -0.0281859040260315
0.00098909919582439 0.0521995425224304
0.00100888117974088 0.0480449795722961
0.0010290588033357 -0.044549435377121
0.00104963997940241 0.00371885299682617
0.00107063277899046 -0.0335689783096313
0.00109204543457027 -0.0102209150791168
0.00111388634326167 0.0379945635795593
0.0011361640701269 -0.0082804262638092
0.00115888735152944 -0.00225123763084412
0.00118206509856003 0.0146962702274323
0.00120570640053123 -0.0535928606987
0.00122982052854186 -0.042436808347702
0.00125441693911269 0.067408561706543
0.00127950527789495 0.0290917456150055
0.00130509538345285 0.0015239417552948
0.0013311972911219 -0.0392532348632812
0.00135782123694434 0.0267238318920135
0.00138497766168323 0.0646282434463501
0.00141267721491689 0.0103473961353302
0.00144093075921523 -0.0318962037563324
0.00146974937439954 -0.0515064299106598
0.00149914436188753 0.0331993401050568
0.00152912724912528 -0.057633101940155
0.00155970979410778 -0.0575977265834808
0.00159090398998994 -0.0928509831428528
0.00162272206978974 0.0247020423412323
0.00165517651118553 0.110695004463196
0.00168828004140924 -0.0307809114456177
0.00172204564223743 -0.0308657884597778
0.00175648655508218 -0.0733732581138611
0.00179161628618382 -0.030721515417099
0.0018274486119075 0.0426245629787445
0.00186399758414565 0.0955909788608551
0.00190127753582856 0.0263249278068542
0.00193930308654513 0.0335519313812256
0.00197808914827603 -0.0164959728717804
0.00201765093124155 -0.0441375374794006
0.00205800394986639 -0.0394017994403839
0.00209916402886371 0.00994926691055298
0.00214114730944099 0.033034473657608
0.00218397025562981 0.0254474282264709
0.0022276496607424 0.0583900511264801
0.00227220265395725 0.0352185666561127
0.0023176467070364 -0.0277751386165619
0.00236399964117713 -0.0666139125823975
0.00241127963400067 -0.0222125947475433
0.00245950522668068 -0.0138813555240631
0.00250869533121429 -0.0918338298797607
0.00255886923783858 -0.035653829574585
0.00261004662259535 0.0154863595962524
0.00266224755504726 -0.0457204282283783
0.0027154925061482 0.0637829601764679
0.00276980235627117 0.0655289888381958
0.00282519840339659 0.0148723423480988
0.00288170237146452 -0.0201868116855621
0.00293933641889381 0.0154703259468079
0.00299812314727169 0.00130864977836609
0.00305808561021712 -0.0335114598274231
0.00311924732242147 0.00755399465560913
0.0031816322688699 -0.2061388194561
0.00324526491424729 0.0253941714763641
0.00331017021253224 0.240395873785019
0.00337637361678289 -0.0736740827560425
0.00344390108911854 0.0111432671546936
0.00351277911090091 0.00390246510505676
0.00358303469311893 -0.139250159263611
0.00365469538698131 -0.00494733452796936
0.00372778929472094 -0.111699998378754
0.00380234508061536 -0.0310316979885101
0.00387839198222766 0.137600034475327
0.00395595982187222 -0.0303145349025726
0.00403507901830966 0.170609384775162
0.00411578059867586 -0.0623260736465454
0.00419809621064937 -0.104640930891037
0.00428205813486236 0.125094175338745
0.00436769929755961 -0.0701291263103485
0.0044550532835108 -0.12299644947052
0.00454415434918102 -0.0633579790592194
0.00463503743616464 0.1828773021698
0.00472773818488793 0.118549853563309
0.00482229294858569 -0.0439665913581848
0.0049187388075574 -0.0121778845787048
0.00501711358370855 -0.0414985716342926
0.00511745585538272 -0.00166752934455872
0.00521980497249038 -0.0541485846042633
0.00532420107194018 -0.25243017077446
0.00543068509337899 -0.0380798876285553
0.00553929879524657 0.233720690011978
0.0056500847711515 0.243632644414902
0.00576308646657453 -0.180473029613495
0.00587834819590602 -0.0579162240028381
0.00599591515982414 0.505269289016724
0.00611583346302062 0.0618888437747955
0.00623815013228103 -0.407565355300903
0.00636291313492665 -0.331976473331451
0.00649017139762519 0.114604026079178
0.00661997482557769 0.555914998054504
0.00675237432208924 0.0624400675296783
0.00688742180853103 -0.450366616249084
0.00702517024470165 -0.229031637310982
0.00716567364959568 0.0412282347679138
0.0073089871225876 0.0714166015386581
0.00745516686503935 -0.0322162955999374
0.00760427020234014 -0.168935894966125
0.00775635560638694 -0.0851742178201675
0.00791148271851468 0.13218480348587
0.00806971237288497 0.0293371379375458
0.00823110662034267 -0.0957483500242233
0.00839572875274952 -0.0694940164685249
0.00856364332780452 -0.00333575904369354
0.00873491619436061 0.131125420331955
0.00890961451824782 -0.00100122392177582
0.00908780680861278 -0.043915182352066
0.00926956294478503 -0.0933693796396255
0.00945495420368073 0.147864997386932
0.00964405328775435 0.115535497665405
0.00983693435350943 -0.284375429153442
0.0100336730405796 0.059046745300293
0.0102343465013912 0.108868554234505
0.010439033431419 -0.143822282552719
0.0106478141000474 0.0102982521057129
0.0108607703820484 0.118022911250591
0.0110779857896893 -0.12279549986124
0.0112995455054831 -0.128085613250732
0.0115255364155928 0.0664164647459984
0.0117560471439046 0.122861161828041
0.0119911680867827 -0.0112304389476776
0.0122309914485184 -0.126010149717331
0.0124756112774888 -0.0920101851224899
0.0127251235030385 0.016897164285183
0.0129796259730993 0.0375559590756893
0.0132392184925613 0.180124044418335
0.0135040028624125 0.70317929983139
0.0137740829196608 -0.0762100070714951
0.014049564578054 -0.490911900997162
0.0143305558696151 -0.0899185985326767
0.0146171669870074 0.527102291584015
0.0149095103267475 1.47732543945312
0.0152077005332825 1.50651347637177
};
\end{axis}

\end{tikzpicture}
	\end{subfigure}
	\caption[Optimal learning rate for the 0-3 network]{Optimal learning rate for the 0-3 network. Learning rate is initialized with 0.00001 and multiplied by 1.02 every iteration.}
	\label{fig:optimal-learning-rate}
\end{figure}

Furthermore, the decreased filter size in the first convolutional layer from $11 \times 11$ to $7 \times 7$ compared to the original AlexNet configuration is evaluated.
The reason for this decrease is the large filter size of $11 \times 11$ compared to more recent networks.
The filter sizes of the remaining layers match or are only slightly larger compared to those networks, hence, they are not decreased and evaluated.
In \figref{fig:first-conv-filter} the losses of the training process of both filter configurations of the first convolutional layer are shown for a single run.
\begin{figure}
	\setlength\figureheight{.4\textwidth}
	\setlength\figurewidth{.9\textwidth}
	\centering
	% This file was created by matplotlib2tikz v0.7.3.
\begin{tikzpicture}

\definecolor{color0}{rgb}{0.12156862745098,0.466666666666667,0.705882352941177}

\begin{axis}[
height=\figureheight,
legend cell align={left},
legend style={draw=white!80.0!black},
tick align=outside,
tick pos=left,
width=\figurewidth,
x grid style={white!69.01960784313725!black},
xlabel={Epoch},
xmin=-0.95, xmax=19.95,
xtick style={color=black},
y grid style={white!69.01960784313725!black},
ylabel={Loss},
ymin=0.020286288957793, ymax=1.44030722774124,
ytick style={color=black},
ytick={0,0.2,0.4,0.6,0.8,1,1.2,1.4,1.6},
yticklabels={,0.2,0.4,0.6,0.8,1.0,1.2,1.4,}
]
\addplot [semithick, green!50.0!black]
table {%
0 1.13833824724987
1 0.812229750485256
2 0.584070554067349
3 0.653840091721765
4 0.611592817923118
5 0.511397582703623
6 0.505834147334099
7 0.570515671680713
8 0.265623657032847
9 0.215395397486583
10 0.189785755399706
11 0.177184161087819
12 0.167540197935084
13 0.169515813604511
14 0.148586822994824
15 0.131860823274173
16 0.150735137855698
17 0.151333889677924
18 0.116580078892153
19 0.0848326952661315
};
\addlegendentry{7x7}
\addplot [semithick, color0]
table {%
0 1.08473345329022
1 0.674929343420884
2 0.603545965819523
3 0.614335598616764
4 0.611164523609753
5 0.570500518741279
6 1.3757608214329
7 0.27132933920426
8 0.246414459727303
9 0.262517018436358
10 0.259975378112546
11 0.249353414080266
12 0.212313252982908
13 0.255264111120125
14 0.255625974258472
15 0.254195858628072
16 0.252674387382536
17 0.254422357755488
18 0.253335101273039
19 0.252868478603918
};
\addlegendentry{11x11}
\end{axis}

\end{tikzpicture}
	\caption{Comparison of filter sizes of first convolutional layer based on loss}
	\label{fig:first-conv-filter}
\end{figure}
It can be seen, that with the smaller filter the loss decreases over time, while for the other filter the loss saturates after 13 epochs.
The latter presumably got stuck on a saddle point before and would decrease further with more training epochs.
This could have been an unfavorable weight initialization, but based on all shown cost evaluations, the loss with the $7 \times 7$ filter decreases much more and faster.
That means, there were either many saddle points very close to each other, or the performance of the $11 \times 11$ filter is actually worse.
Because more recent convolutional networks tend to use smaller filters, the latter theory is assumed.
Hence, a filter with a size of $7 \times 7$ is chosen for the first convolution.

The overall training losses for all networks are shown in \figref{fig:train-loss} and the test losses in \figref{fig:train-loss}.
One graph shows the moving averaged loss or accuracy, respectively, against the training iterations $\tau$.
For a direct comparison of the networks iterations are more suited than epochs, because due to the steadily increased dataset more iterations per epoch are executed. 
%This is capped at the smallest number of iterations of a network, in particular the 0-3 one, for an optimal comparison of training of all networks.
This is capped at the number of iterations of the 0-3 network for the single-category ones and at the number of iterations of the 4-3 network for the four-category ones for an optimal comparison of training of all networks.
For a full overview of the whole training process of each network the loss is plotted against the epochs.
For a more clear visualization of training and test, they are split into two separate graphs.

As expected, the 0-3 network starts with the smallest loss and keeps it throughout its training compared to all other networks, due to its simplicity.
Most closely to this comes the 0-4 network, however, with more rapid changes that are not visible for the moving averaged values.
Its loss is approximately from the 210-th iteration only slightly worse than the one of the 0-3 network.
This is expected as well because it is just slightly more complicated.
Moreover, it shows that the additional color class is not that challenging to classify.
The 0-5 and 0-6 network have the highest losses of all single-category networks.
This is not surprising, because they are challenged with the double color marks classification.
However, as training proceeds, their cost function is noticeable going to be minimized, it just takes longer due to their complexity compared to the other single-category networks.
It can be seen in \figref{fig:train-loss}, that the losses of 0-4, 0-5 and 0-6 are similar after 20 epochs, showing that the more complex networks just need a longer training, i.\,e. more iterations, for similar results.

The progress of the 4-0 network's loss behaves similar to the one of 0-3.
Furthermore, both losses match almost exactly after 280 iterations, i.\,e. at the end of 0-3's training.
However, the 4-0 loss gets steadily minimized further.
The remaining four-category networks behave as expected and similar to the single-category ones.
The more classes, in particular color marks, are used for classification, the higher the starting loss is.
That the loss of 4-5 and 4-6 is decreased faster at the beginning than the one of 4-3 and 4-4 is presumably due to a favorable weight initializing.
However, at approximately the 450-th iteration the loss of 4-3 and 4-4 is smaller than the one of 4-5 and 4-6, which is valid until the end of training.
This trend shows, that the less complex networks are faster minimized than the more complex ones.
Though, after 20 epochs, all losses of the four-category networks are similar minimized representing the behavior of the single-category networks.
Moreover, it can be seen that the 4-3 network presumably hit a saddle point in the range of approximately 150 to 250 iterations because its loss is consistent and then drops rapidly.
For the sake of completeness, the related accuracies of the training processes are shown in \figref{fig:networks-accuracy}.
Here the same effects can be seen as with the losses.
\begin{figure}
	\setlength\figureheight{.45\textwidth}
	\setlength\figurewidth{.9\textwidth}
	\centering
	\begin{subfigure}{\textwidth}
		\centering
		% This file was created by matplotlib2tikz v0.7.3.
\begin{tikzpicture}

\definecolor{color0}{rgb}{1,0.647058823529412,0}
\definecolor{color1}{rgb}{0.75,0,0.75}

\begin{axis}[
height=\figureheight,
legend cell align={left},
legend columns=3,
legend style={draw=white!80.0!black},
tick align=outside,
tick pos=left,
width=\figurewidth,
x grid style={white!69.01960784313725!black},
xlabel={Iteration},
xmin=-61.95, xmax=1300.95,
xtick style={color=black},
xtick={0,200,400,600,800,1000,1200},
xticklabels={0,200,400,600,800,1000,1200},
y grid style={white!69.01960784313725!black},
ylabel={Loss},
ymin=-0.195486334431916, ymax=3.0,
ytick style={color=black}
]
\addplot [semithick, blue, dotted]
table {%
0 0.942402780056
1 0.964930057525635
2 0.999739110469818
3 1.02169370651245
4 1.02182745933533
5 1.04909646511078
6 1.11752223968506
7 1.15349841117859
8 1.14131021499634
9 1.1316887140274
10 1.1273021697998
11 1.12632322311401
12 1.11867785453796
13 1.12608885765076
14 1.12801909446716
15 1.13562870025635
16 1.137411236763
17 1.13311815261841
18 1.13246655464172
19 1.13057243824005
20 1.12822151184082
21 1.12721681594849
22 1.126340508461
23 1.12387979030609
24 1.12075698375702
25 1.12070405483246
26 1.11697065830231
27 1.11009645462036
28 1.10216856002808
29 1.08815848827362
30 1.08412551879883
31 1.07042908668518
32 1.06300163269043
33 1.05845093727112
34 1.04194283485413
35 1.03575789928436
36 1.02930414676666
37 1.01190876960754
38 0.993944883346558
39 0.99060982465744
40 0.983669996261597
41 0.974830448627472
42 0.961553275585175
43 0.950428545475006
44 0.940548479557037
45 0.932932555675507
46 0.917222619056702
47 0.912576198577881
48 0.916874170303345
49 0.911451935768127
50 0.903677225112915
51 0.892811179161072
52 0.891962945461273
53 0.888035118579865
54 0.877861797809601
55 0.867728888988495
56 0.849223494529724
57 0.847058534622192
58 0.846346795558929
59 0.831901490688324
60 0.825193762779236
61 0.813772201538086
62 0.808964252471924
63 0.798904776573181
64 0.792795419692993
65 0.781738996505737
66 0.768819451332092
67 0.758540868759155
68 0.745782673358917
69 0.742127299308777
70 0.73616635799408
71 0.717912971973419
72 0.709492027759552
73 0.699841022491455
74 0.690233409404755
75 0.679545760154724
76 0.676006376743317
77 0.666987240314484
78 0.661669254302979
79 0.653653264045715
80 0.63988721370697
81 0.638040721416473
82 0.628730833530426
83 0.621432363986969
84 0.628702104091644
85 0.619229197502136
86 0.629921078681946
87 0.629794418811798
88 0.632252752780914
89 0.622435629367828
90 0.620964646339417
91 0.619370639324188
92 0.623154044151306
93 0.622197926044464
94 0.619424879550934
95 0.616393029689789
96 0.623234033584595
97 0.6236412525177
98 0.613219499588013
99 0.614562928676605
100 0.617020547389984
101 0.61227822303772
102 0.602554500102997
103 0.595621824264526
104 0.59029495716095
105 0.588472604751587
106 0.58794629573822
107 0.574661135673523
108 0.566251993179321
109 0.570290684700012
110 0.566151738166809
111 0.56820273399353
112 0.562565982341766
113 0.558670103549957
114 0.561002731323242
115 0.564750075340271
116 0.561479508876801
117 0.559165894985199
118 0.55871570110321
119 0.544221937656403
120 0.53779798746109
121 0.547378361225128
122 0.544886112213135
123 0.540668845176697
124 0.536812365055084
125 0.53867095708847
126 0.530280470848083
127 0.526177048683167
128 0.521993160247803
129 0.523671865463257
130 0.520040571689606
131 0.511646389961243
132 0.507999837398529
133 0.499381452798843
134 0.485318034887314
135 0.480610251426697
136 0.460596770048141
137 0.45507338643074
138 0.446794241666794
139 0.439700096845627
140 0.427175104618073
141 0.426147073507309
142 0.414374440908432
143 0.406095236539841
144 0.398831099271774
145 0.401038199663162
146 0.390504568815231
147 0.376614689826965
148 0.364942044019699
149 0.353370428085327
150 0.346590965986252
151 0.347770810127258
152 0.338801950216293
153 0.328587651252747
154 0.329972922801971
155 0.319476217031479
156 0.314802795648575
157 0.303958386182785
158 0.293118059635162
159 0.283639550209045
160 0.273603469133377
161 0.262247055768967
162 0.253022879362106
163 0.25170573592186
164 0.237822785973549
165 0.227481245994568
166 0.225645288825035
167 0.226466134190559
168 0.228003084659576
169 0.226726949214935
170 0.220248624682426
171 0.208618998527527
172 0.198523342609406
173 0.196921303868294
174 0.190618768334389
175 0.177749156951904
176 0.176778361201286
177 0.185286030173302
178 0.178018718957901
179 0.1719641238451
180 0.172573789954185
181 0.170998707413673
182 0.168815746903419
183 0.177683442831039
184 0.175656452775002
185 0.180230349302292
186 0.180449560284615
187 0.179496377706528
188 0.187192618846893
189 0.188038527965546
190 0.189123019576073
191 0.180404260754585
192 0.18199647963047
193 0.183265626430511
194 0.183796808123589
195 0.173179194331169
196 0.188160821795464
197 0.195427224040031
198 0.202924445271492
199 0.207202106714249
200 0.20162008702755
201 0.196744829416275
202 0.200010031461716
203 0.207457154989243
204 0.20297808945179
205 0.206282272934914
206 0.201498091220856
207 0.205192565917969
208 0.214321538805962
209 0.21428570151329
210 0.214769214391708
211 0.218998000025749
212 0.219112321734428
213 0.211133703589439
214 0.217845350503922
215 0.21135425567627
216 0.206868901848793
217 0.1982601583004
218 0.187921479344368
219 0.19700488448143
220 0.194881081581116
221 0.201083794236183
222 0.207836270332336
223 0.202414259314537
224 0.20156267285347
225 0.201286375522614
226 0.194439426064491
227 0.181090638041496
228 0.18616995215416
229 0.191151335835457
230 0.190557524561882
231 0.190445348620415
232 0.204797700047493
233 0.199528738856316
234 0.205160140991211
235 0.199618220329285
236 0.19371235370636
237 0.19417816400528
238 0.188773825764656
239 0.187989592552185
240 0.187774121761322
241 0.19393002986908
242 0.192744433879852
243 0.197830319404602
244 0.197658136487007
245 0.204090878367424
246 0.189677730202675
247 0.182958543300629
248 0.175775170326233
249 0.169227808713913
250 0.168639525771141
251 0.168759629130363
252 0.171032622456551
253 0.169779151678085
254 0.174766808748245
255 0.170922264456749
256 0.180588603019714
257 0.181757315993309
258 0.188776031136513
259 0.193720012903214
260 0.193778842687607
261 0.189348921179771
262 0.199617028236389
263 0.199420168995857
264 0.192300528287888
265 0.193445295095444
266 0.194599330425262
267 0.200794473290443
268 0.201657950878143
269 0.19010816514492
270 0.190569445490837
271 0.183838933706284
272 0.176520571112633
273 0.176307216286659
274 0.18829733133316
275 0.188838049769402
276 0.188743934035301
277 0.189726069569588
278 0.185152739286423
279 0.179672837257385
280 0.178661748766899
281 0.178625702857971
282 0.169721201062202
283 0.16514328122139
284 0.163463845849037
285 0.162226870656013
286 0.162150338292122
287 0.161349207162857
288 0.159280985593796
289 0.165194049477577
};
\addlegendentry{0-3}
\addplot [semithick, green!50.0!black, dotted]
table {%
0 1.33364462852478
1 1.31262040138245
2 1.31396782398224
3 1.30905699729919
4 1.32385694980621
5 1.3288449048996
6 1.33288538455963
7 1.33640921115875
8 1.33897531032562
9 1.36705279350281
10 1.36407649517059
11 1.37253105640411
12 1.37660527229309
13 1.37789714336395
14 1.37664842605591
15 1.37351512908936
16 1.37295484542847
17 1.3804897069931
18 1.37753117084503
19 1.36564803123474
20 1.38000380992889
21 1.38235402107239
22 1.38235545158386
23 1.38967907428741
24 1.39104402065277
25 1.3918514251709
26 1.39137172698975
27 1.39070773124695
28 1.38979196548462
29 1.38908529281616
30 1.38563060760498
31 1.37983393669128
32 1.37270069122314
33 1.36954140663147
34 1.36408150196075
35 1.35154974460602
36 1.34561431407928
37 1.33148586750031
38 1.32411444187164
39 1.31616711616516
40 1.30445945262909
41 1.29629552364349
42 1.28089892864227
43 1.28456497192383
44 1.26983642578125
45 1.2667373418808
46 1.258136510849
47 1.25596511363983
48 1.26634252071381
49 1.26919388771057
50 1.27142870426178
51 1.27750468254089
52 1.28157842159271
53 1.29016661643982
54 1.29658675193787
55 1.29145359992981
56 1.2904669046402
57 1.28772747516632
58 1.2856730222702
59 1.28249394893646
60 1.27650535106659
61 1.26926064491272
62 1.26506614685059
63 1.25532877445221
64 1.25210273265839
65 1.24391102790833
66 1.23467457294464
67 1.22525358200073
68 1.22114443778992
69 1.21636414527893
70 1.20180714130402
71 1.19148814678192
72 1.189945936203
73 1.17549884319305
74 1.16285002231598
75 1.1500871181488
76 1.14121973514557
77 1.128293633461
78 1.11592817306519
79 1.12383997440338
80 1.12392091751099
81 1.12128460407257
82 1.11329627037048
83 1.10776889324188
84 1.10118544101715
85 1.10380852222443
86 1.09730792045593
87 1.1020336151123
88 1.09632468223572
89 1.08933591842651
90 1.08686602115631
91 1.08685445785522
92 1.09212255477905
93 1.07744967937469
94 1.07933795452118
95 1.06879389286041
96 1.07313466072083
97 1.06513166427612
98 1.04396367073059
99 1.03080654144287
100 1.02644169330597
101 1.01823246479034
102 1.01067531108856
103 0.993041217327118
104 0.980168521404266
105 0.975585103034973
106 0.965760350227356
107 0.962441563606262
108 0.952889025211334
109 0.942351758480072
110 0.938122928142548
111 0.930932760238647
112 0.921424865722656
113 0.920641481876373
114 0.912951707839966
115 0.911344885826111
116 0.911315202713013
117 0.904798150062561
118 0.896325588226318
119 0.895113289356232
120 0.892458438873291
121 0.889239192008972
122 0.87970894575119
123 0.881925761699677
124 0.881804585456848
125 0.882568538188934
126 0.878630220890045
127 0.878422379493713
128 0.876449584960938
129 0.856239378452301
130 0.847394347190857
131 0.846851110458374
132 0.852110683917999
133 0.853076219558716
134 0.853858411312103
135 0.85463148355484
136 0.852938771247864
137 0.8485347032547
138 0.851024806499481
139 0.853426516056061
140 0.859629273414612
141 0.855558753013611
142 0.851873755455017
143 0.850778102874756
144 0.850480675697327
145 0.852414548397064
146 0.845685243606567
147 0.845747530460358
148 0.84701806306839
149 0.848388671875
150 0.842147886753082
151 0.835478723049164
152 0.826922655105591
153 0.827118635177612
154 0.820093631744385
155 0.831872582435608
156 0.831021249294281
157 0.824788451194763
158 0.833909511566162
159 0.834077775478363
160 0.837272942066193
161 0.837088227272034
162 0.838864982128143
163 0.8383908867836
164 0.834377646446228
165 0.830142617225647
166 0.828896045684814
167 0.823586881160736
168 0.817559957504272
169 0.810361623764038
170 0.803675055503845
171 0.800149440765381
172 0.789933800697327
173 0.779431164264679
174 0.773333191871643
175 0.760376751422882
176 0.753178000450134
177 0.751314997673035
178 0.751313090324402
179 0.744404315948486
180 0.737041294574738
181 0.719632506370544
182 0.701814591884613
183 0.683862090110779
184 0.670620679855347
185 0.665740251541138
186 0.658866047859192
187 0.644789576530457
188 0.634205639362335
189 0.62394118309021
190 0.610334694385529
191 0.596876740455627
192 0.588313102722168
193 0.577766180038452
194 0.565249860286713
195 0.553388893604279
196 0.539877653121948
197 0.525435209274292
198 0.51441901922226
199 0.49912440776825
200 0.481137245893478
201 0.465290427207947
202 0.454492181539536
203 0.446351647377014
204 0.433117002248764
205 0.421835899353027
206 0.40718337893486
207 0.40200263261795
208 0.382859587669373
209 0.378311008214951
210 0.359972149133682
211 0.348284363746643
212 0.332917451858521
213 0.32558873295784
214 0.333329558372498
215 0.323560208082199
216 0.315994679927826
217 0.321398496627808
218 0.317087739706039
219 0.311705499887466
220 0.305895537137985
221 0.300641447305679
222 0.295928835868835
223 0.28825968503952
224 0.311213910579681
225 0.313872158527374
226 0.307736665010452
227 0.295924365520477
228 0.290983080863953
229 0.294309318065643
230 0.290031671524048
231 0.294183611869812
232 0.301302582025528
233 0.31270357966423
234 0.312262028455734
235 0.303417563438416
236 0.299693763256073
237 0.302127599716187
238 0.299601644277573
239 0.295525401830673
240 0.289897322654724
241 0.293219089508057
242 0.288618683815002
243 0.289872080087662
244 0.291275382041931
245 0.297538578510284
246 0.296618968248367
247 0.301849663257599
248 0.299731552600861
249 0.299879223108292
250 0.310171782970428
251 0.309791803359985
252 0.317618936300278
253 0.309323817491531
254 0.310006588697433
255 0.301692843437195
256 0.304335564374924
257 0.300922960042953
258 0.303408950567245
259 0.291813224554062
260 0.294224828481674
261 0.297948062419891
262 0.304940402507782
263 0.303270399570465
264 0.286289006471634
265 0.287691354751587
266 0.28683066368103
267 0.280196845531464
268 0.278699666261673
269 0.27543693780899
270 0.272363096475601
271 0.269620388746262
272 0.278049558401108
273 0.278223961591721
274 0.25544136762619
275 0.249975427985191
276 0.255633234977722
277 0.265075087547302
278 0.258077532052994
279 0.25370118021965
280 0.250888794660568
281 0.251581966876984
282 0.245135769248009
283 0.232179373502731
284 0.229630127549171
285 0.226522326469421
286 0.228003427386284
287 0.225821137428284
288 0.228667885065079
289 0.228503793478012
};
\addlegendentry{0-4}
\addplot [semithick, red, dotted]
table {%
0 1.57860207557678
1 1.61110103130341
2 1.61759471893311
3 1.6336350440979
4 1.60109806060791
5 1.57789862155914
6 1.59971296787262
7 1.60683751106262
8 1.60898971557617
9 1.61189770698547
10 1.60346806049347
11 1.60451662540436
12 1.60858798027039
13 1.60934352874756
14 1.60981619358063
15 1.60813939571381
16 1.60878276824951
17 1.60959208011627
18 1.61297738552094
19 1.6126983165741
20 1.61182641983032
21 1.61308145523071
22 1.6131991147995
23 1.61401879787445
24 1.60530734062195
25 1.60868179798126
26 1.61027777194977
27 1.61204516887665
28 1.60844385623932
29 1.60601127147675
30 1.60604274272919
31 1.60965418815613
32 1.60541653633118
33 1.60518932342529
34 1.60676157474518
35 1.60580933094025
36 1.60606122016907
37 1.60619485378265
38 1.60499310493469
39 1.60584855079651
40 1.60543036460876
41 1.60517334938049
42 1.6065661907196
43 1.607905626297
44 1.60494220256805
45 1.60984492301941
46 1.61113739013672
47 1.60971927642822
48 1.61059272289276
49 1.6107462644577
50 1.61238491535187
51 1.61490893363953
52 1.61586713790894
53 1.61589872837067
54 1.61792433261871
55 1.62124729156494
56 1.62015569210052
57 1.61940455436707
58 1.61793947219849
59 1.61830043792725
60 1.61977660655975
61 1.61983859539032
62 1.6187995672226
63 1.6189181804657
64 1.62058651447296
65 1.61987829208374
66 1.62022793292999
67 1.62107181549072
68 1.62006878852844
69 1.61806666851044
70 1.61941075325012
71 1.61847233772278
72 1.61862170696259
73 1.62112963199615
74 1.62574434280396
75 1.62478601932526
76 1.62359631061554
77 1.62285840511322
78 1.62443435192108
79 1.62564611434937
80 1.62569212913513
81 1.62249875068665
82 1.62506091594696
83 1.62315201759338
84 1.61858201026917
85 1.61028075218201
86 1.60495817661285
87 1.5953803062439
88 1.60006880760193
89 1.59587895870209
90 1.59198832511902
91 1.59088397026062
92 1.58584201335907
93 1.5789657831192
94 1.5766966342926
95 1.56253445148468
96 1.56132006645203
97 1.55517482757568
98 1.54706561565399
99 1.53981626033783
100 1.52856779098511
101 1.51257050037384
102 1.50210070610046
103 1.49939560890198
104 1.48951888084412
105 1.47604763507843
106 1.46315014362335
107 1.44879794120789
108 1.4400497674942
109 1.42601120471954
110 1.41168284416199
111 1.39660859107971
112 1.38781833648682
113 1.37469005584717
114 1.36867475509644
115 1.35935246944427
116 1.35471928119659
117 1.34674119949341
118 1.34317791461945
119 1.3417102098465
120 1.32653319835663
121 1.32144498825073
122 1.31140840053558
123 1.30318164825439
124 1.29305601119995
125 1.28586959838867
126 1.2736976146698
127 1.26085245609283
128 1.25262868404388
129 1.246173620224
130 1.23900365829468
131 1.23651659488678
132 1.22238397598267
133 1.21258020401001
134 1.20891714096069
135 1.20733797550201
136 1.20084381103516
137 1.20232307910919
138 1.18872690200806
139 1.1832070350647
140 1.18197846412659
141 1.17058408260345
142 1.17142772674561
143 1.1631668806076
144 1.15527760982513
145 1.15467000007629
146 1.14520072937012
147 1.14383804798126
148 1.13855075836182
149 1.13357579708099
150 1.13101744651794
151 1.13041925430298
152 1.12753283977509
153 1.11718690395355
154 1.11173915863037
155 1.1200510263443
156 1.11831653118134
157 1.11773526668549
158 1.11574602127075
159 1.10630583763123
160 1.10711109638214
161 1.10752820968628
162 1.1020884513855
163 1.09596264362335
164 1.09247326850891
165 1.08851718902588
166 1.07690000534058
167 1.0613044500351
168 1.04156804084778
169 1.0244038105011
170 1.01327061653137
171 0.986525654792786
172 1.00454235076904
173 1.00070512294769
174 0.991093575954437
175 0.979451656341553
176 0.967794597148895
177 0.958182752132416
178 0.959125578403473
179 0.937781512737274
180 0.923045754432678
181 0.909501314163208
182 0.902677893638611
183 0.887310087680817
184 0.871739447116852
185 0.856948494911194
186 0.844617605209351
187 0.828936815261841
188 0.81721830368042
189 0.802278518676758
190 0.793370604515076
191 0.783027112483978
192 0.769574880599976
193 0.76069450378418
194 0.749208390712738
195 0.735463082790375
196 0.720945060253143
197 0.708743512630463
198 0.696528494358063
199 0.686383843421936
200 0.676187038421631
201 0.666449308395386
202 0.653630912303925
203 0.657661974430084
204 0.64600396156311
205 0.623601973056793
206 0.607428312301636
207 0.599759459495544
208 0.590179800987244
209 0.584529042243958
210 0.58539617061615
211 0.582922756671906
212 0.574901342391968
213 0.567738950252533
214 0.554317712783813
215 0.545648574829102
216 0.54432487487793
217 0.542692124843597
218 0.542812168598175
219 0.5341477394104
220 0.528006196022034
221 0.530404090881348
222 0.497786641120911
223 0.4824098944664
224 0.476942747831345
225 0.468611568212509
226 0.464101284742355
227 0.460294723510742
228 0.457935601472855
229 0.487595200538635
230 0.492829144001007
231 0.483654797077179
232 0.484110563993454
233 0.48961666226387
234 0.490451812744141
235 0.491801559925079
236 0.487928003072739
237 0.487719655036926
238 0.484192073345184
239 0.49207130074501
240 0.494190394878387
241 0.491369158029556
242 0.481452971696854
243 0.482303529977798
244 0.482850223779678
245 0.484167903661728
246 0.490105271339417
247 0.490019679069519
248 0.489310175180435
249 0.484520941972733
250 0.486338794231415
251 0.485691338777542
252 0.482408791780472
253 0.464952707290649
254 0.470268696546555
255 0.46714049577713
256 0.480036824941635
257 0.479293018579483
258 0.476318538188934
259 0.479732543230057
260 0.464420020580292
261 0.450711697340012
262 0.456545323133469
263 0.468982934951782
264 0.463501632213593
265 0.457320511341095
266 0.444403886795044
267 0.44194358587265
268 0.444060355424881
269 0.450991988182068
270 0.45923775434494
271 0.467905163764954
272 0.468518614768982
273 0.47531795501709
274 0.475290417671204
275 0.476156681776047
276 0.487027615308762
277 0.492960274219513
278 0.478120803833008
279 0.445724040269852
280 0.434742629528046
281 0.435318171977997
282 0.433852195739746
283 0.427912950515747
284 0.418981462717056
285 0.412215113639832
286 0.410708010196686
287 0.405264139175415
288 0.408878922462463
289 0.402440935373306
};
\addlegendentry{0-5}
\addplot [semithick, color0, dotted]
table {%
0 1.7376401424408
1 1.80418062210083
2 1.83881962299347
3 1.8249431848526
4 1.86649477481842
5 1.92354011535645
6 1.90002954006195
7 1.90172338485718
8 1.8936163187027
9 1.87153661251068
10 1.87314188480377
11 1.87002003192902
12 1.87392807006836
13 1.86591374874115
14 1.86387777328491
15 1.85356283187866
16 1.84899401664734
17 1.84260940551758
18 1.84219622612
19 1.83938694000244
20 1.8389585018158
21 1.83336400985718
22 1.83582866191864
23 1.8333158493042
24 1.82986080646515
25 1.8283417224884
26 1.82540845870972
27 1.82228016853333
28 1.82494819164276
29 1.8248438835144
30 1.8226181268692
31 1.81694746017456
32 1.8149995803833
33 1.81273233890533
34 1.81356239318848
35 1.81028771400452
36 1.80664026737213
37 1.80657434463501
38 1.80085074901581
39 1.79200398921967
40 1.79038405418396
41 1.77562403678894
42 1.7693954706192
43 1.76733541488647
44 1.75898313522339
45 1.74946057796478
46 1.7458348274231
47 1.73724973201752
48 1.73069298267365
49 1.72490429878235
50 1.71474671363831
51 1.70447313785553
52 1.69357514381409
53 1.68482685089111
54 1.67680907249451
55 1.66097271442413
56 1.64987230300903
57 1.63213312625885
58 1.62133502960205
59 1.61722147464752
60 1.60848724842072
61 1.6001079082489
62 1.58651006221771
63 1.57804811000824
64 1.56933927536011
65 1.55804538726807
66 1.54338681697845
67 1.53763163089752
68 1.52617168426514
69 1.51840496063232
70 1.50006973743439
71 1.48873293399811
72 1.47571003437042
73 1.46701884269714
74 1.4492564201355
75 1.43121814727783
76 1.40861415863037
77 1.38760089874268
78 1.36988377571106
79 1.35335958003998
80 1.33494639396667
81 1.32245206832886
82 1.30887269973755
83 1.29363787174225
84 1.28541469573975
85 1.26946020126343
86 1.25356686115265
87 1.2360851764679
88 1.23184478282928
89 1.22097980976105
90 1.20521295070648
91 1.19674825668335
92 1.17789995670319
93 1.16712832450867
94 1.15082478523254
95 1.14913928508759
96 1.13572692871094
97 1.12123155593872
98 1.10924863815308
99 1.10018980503082
100 1.09864711761475
101 1.08230113983154
102 1.07271552085876
103 1.06593000888824
104 1.04685604572296
105 1.03289520740509
106 1.02984070777893
107 1.02166986465454
108 1.01284492015839
109 0.995309889316559
110 0.986144244670868
111 0.982569098472595
112 0.971156537532806
113 0.958554863929749
114 0.941213369369507
115 0.939702272415161
116 0.93286657333374
117 0.917612552642822
118 0.909021437168121
119 0.891973733901978
120 0.883049786090851
121 0.876766920089722
122 0.866675019264221
123 0.871754288673401
124 0.86466771364212
125 0.856162190437317
126 0.860507667064667
127 0.859530866146088
128 0.865104377269745
129 0.857680976390839
130 0.856486737728119
131 0.85646116733551
132 0.84679788351059
133 0.838946163654327
134 0.823425650596619
135 0.818434596061707
136 0.819297194480896
137 0.809816181659698
138 0.793330371379852
139 0.785064518451691
140 0.779555201530457
141 0.796472787857056
142 0.81270170211792
143 0.804990530014038
144 0.801292955875397
145 0.792064428329468
146 0.785466313362122
147 0.784089148044586
148 0.783244073390961
149 0.78596955537796
150 0.778797626495361
151 0.778055727481842
152 0.776435077190399
153 0.775220155715942
154 0.774967551231384
155 0.772865831851959
156 0.764455437660217
157 0.758490204811096
158 0.752897083759308
159 0.755883991718292
160 0.745424866676331
161 0.745322287082672
162 0.737949311733246
163 0.731251299381256
164 0.735912919044495
165 0.728539526462555
166 0.722156047821045
167 0.718590259552002
168 0.715186536312103
169 0.713361263275146
170 0.714713752269745
171 0.715627372264862
172 0.70914101600647
173 0.691683948040009
174 0.689688920974731
175 0.709293067455292
176 0.706877291202545
177 0.701942265033722
178 0.691383063793182
179 0.688634514808655
180 0.688237309455872
181 0.68431031703949
182 0.699376881122589
183 0.710486650466919
184 0.711543202400208
185 0.708652675151825
186 0.70252001285553
187 0.707252502441406
188 0.709908127784729
189 0.709005653858185
190 0.717047333717346
191 0.698187410831451
192 0.684642314910889
193 0.680441737174988
194 0.682222008705139
195 0.676621854305267
196 0.673786520957947
197 0.672040224075317
198 0.663616359233856
199 0.648498177528381
200 0.6480473279953
201 0.646867752075195
202 0.637879312038422
203 0.632664263248444
204 0.625419974327087
205 0.623938918113708
206 0.615820109844208
207 0.622367799282074
208 0.622380971908569
209 0.619437515735626
210 0.614373207092285
211 0.600149869918823
212 0.5978764295578
213 0.597313404083252
214 0.590370416641235
215 0.584558188915253
216 0.596887409687042
217 0.610316634178162
218 0.613005042076111
219 0.625665187835693
220 0.634264647960663
221 0.631075501441956
222 0.637994706630707
223 0.639780282974243
224 0.644627153873444
225 0.629844605922699
226 0.626988112926483
227 0.636392056941986
228 0.632535636425018
229 0.636333167552948
230 0.630577445030212
231 0.624056816101074
232 0.608361542224884
233 0.597747325897217
234 0.595138490200043
235 0.59357887506485
236 0.587124407291412
237 0.583602130413055
238 0.580242455005646
239 0.588835895061493
240 0.578896760940552
241 0.592526912689209
242 0.592266201972961
243 0.59087747335434
244 0.585438191890717
245 0.587212383747101
246 0.589311420917511
247 0.591280400753021
248 0.592670619487762
249 0.601711332798004
250 0.593786597251892
251 0.588807880878448
252 0.59130334854126
253 0.590716600418091
254 0.602006793022156
255 0.607243537902832
256 0.611180245876312
257 0.613331198692322
258 0.609354972839355
259 0.609372675418854
260 0.617631733417511
261 0.624367356300354
262 0.64156174659729
263 0.646289527416229
264 0.649425327777863
265 0.652057826519012
266 0.637540280818939
267 0.621496140956879
268 0.61769562959671
269 0.606144428253174
270 0.593432188034058
271 0.594060063362122
272 0.589614510536194
273 0.585414588451385
274 0.603993237018585
275 0.600483119487762
276 0.603338837623596
277 0.590704202651978
278 0.591112434864044
279 0.590510487556458
280 0.592690765857697
281 0.590149104595184
282 0.580296993255615
283 0.578722178936005
284 0.578701376914978
285 0.579381465911865
286 0.578392386436462
287 0.577888667583466
288 0.577990531921387
289 0.568486332893372
};
\addlegendentry{0-6}
\addplot [semithick, color1]
table {%
0 1.34983015060425
1 1.33534252643585
2 1.37343466281891
3 1.37657570838928
4 1.35168421268463
5 1.35806894302368
6 1.34452521800995
7 1.37781000137329
8 1.38864028453827
9 1.38150131702423
10 1.37101519107819
11 1.36147892475128
12 1.37035310268402
13 1.37911152839661
14 1.3735157251358
15 1.37144613265991
16 1.37131154537201
17 1.36345911026001
18 1.36380910873413
19 1.36026167869568
20 1.36425292491913
21 1.36361765861511
22 1.36314511299133
23 1.36023008823395
24 1.3520839214325
25 1.34536242485046
26 1.33851706981659
27 1.32980525493622
28 1.31734693050385
29 1.30523288249969
30 1.28661239147186
31 1.26615309715271
32 1.24607503414154
33 1.22643935680389
34 1.20988774299622
35 1.193514585495
36 1.16970801353455
37 1.15784239768982
38 1.14486658573151
39 1.13021910190582
40 1.11669743061066
41 1.10001766681671
42 1.09046924114227
43 1.07437205314636
44 1.06196451187134
45 1.05442357063293
46 1.04453063011169
47 1.02407491207123
48 1.0077953338623
49 1.00008070468903
50 0.983068764209747
51 0.972020864486694
52 0.95326292514801
53 0.934740424156189
54 0.922132194042206
55 0.903079211711884
56 0.884505391120911
57 0.863360643386841
58 0.839447319507599
59 0.828777074813843
60 0.812647342681885
61 0.795737743377686
62 0.773192465305328
63 0.757192075252533
64 0.742434442043304
65 0.721984386444092
66 0.705320835113525
67 0.690411508083344
68 0.675214827060699
69 0.656946241855621
70 0.645908653736115
71 0.630768835544586
72 0.621335387229919
73 0.606631219387054
74 0.589960038661957
75 0.575673997402191
76 0.558686256408691
77 0.541200578212738
78 0.525389611721039
79 0.508732676506042
80 0.510640442371368
81 0.506087005138397
82 0.501390099525452
83 0.492394626140594
84 0.494723707437515
85 0.489485651254654
86 0.487454056739807
87 0.483457058668137
88 0.501280725002289
89 0.496116280555725
90 0.489494830369949
91 0.48908856511116
92 0.495594710111618
93 0.497408479452133
94 0.496334910392761
95 0.492366969585419
96 0.483803182840347
97 0.488053351640701
98 0.489343762397766
99 0.486537098884583
100 0.481244325637817
101 0.471080482006073
102 0.472007840871811
103 0.47416552901268
104 0.468473255634308
105 0.470322161912918
106 0.469926387071609
107 0.464040398597717
108 0.463972747325897
109 0.453718155622482
110 0.450393378734589
111 0.451002359390259
112 0.452475398778915
113 0.445430725812912
114 0.438507914543152
115 0.441930115222931
116 0.450597316026688
117 0.44651934504509
118 0.435103565454483
119 0.43302246928215
120 0.418537795543671
121 0.414561152458191
122 0.405003637075424
123 0.398872643709183
124 0.393951535224915
125 0.386660069227219
126 0.385998368263245
127 0.384511798620224
128 0.385610312223434
129 0.387300908565521
130 0.376764059066772
131 0.372288793325424
132 0.365735858678818
133 0.371224015951157
134 0.37733393907547
135 0.370472341775894
136 0.367037653923035
137 0.366755187511444
138 0.340700060129166
139 0.342472076416016
140 0.343701004981995
141 0.343312710523605
142 0.325798690319061
143 0.320555329322815
144 0.319995194673538
145 0.318489044904709
146 0.324332773685455
147 0.323836624622345
148 0.321693658828735
149 0.329219460487366
150 0.335118651390076
151 0.33415150642395
152 0.32707291841507
153 0.320598632097244
154 0.315851122140884
155 0.309961944818497
156 0.307916611433029
157 0.307975739240646
158 0.310197114944458
159 0.309518933296204
160 0.306694626808167
161 0.30021607875824
162 0.293419808149338
163 0.290613651275635
164 0.291043192148209
165 0.283000081777573
166 0.273843139410019
167 0.27305006980896
168 0.273723125457764
169 0.276381373405457
170 0.281070321798325
171 0.287447333335876
172 0.291002660989761
173 0.290117233991623
174 0.294659912586212
175 0.297326683998108
176 0.305214434862137
177 0.309925377368927
178 0.311148166656494
179 0.31269970536232
180 0.316027045249939
181 0.318749040365219
182 0.323364019393921
183 0.329967319965363
184 0.314682364463806
185 0.315435320138931
186 0.318233013153076
187 0.313637644052505
188 0.314516007900238
189 0.310964018106461
190 0.308234065771103
191 0.313363164663315
192 0.325651735067368
193 0.324560314416885
194 0.319507777690887
195 0.317466497421265
196 0.311763912439346
197 0.307018876075745
198 0.320914953947067
199 0.30441877245903
200 0.297366559505463
201 0.294582813978195
202 0.295061230659485
203 0.29597595334053
204 0.299817889928818
205 0.295600920915604
206 0.29771026968956
207 0.301751226186752
208 0.296370625495911
209 0.294946223497391
210 0.294455498456955
211 0.298038810491562
212 0.307115793228149
213 0.307443976402283
214 0.308187633752823
215 0.308444023132324
216 0.299621760845184
217 0.296912342309952
218 0.30083829164505
219 0.301758944988251
220 0.316562205553055
221 0.304391384124756
222 0.2950798869133
223 0.291860312223434
224 0.288262248039246
225 0.289728581905365
226 0.279137492179871
227 0.272973090410233
228 0.279915422201157
229 0.27585768699646
230 0.269677996635437
231 0.268481463193893
232 0.264224141836166
233 0.252981096506119
234 0.247031435370445
235 0.250069439411163
236 0.253411114215851
237 0.250906497240067
238 0.246740967035294
239 0.246392905712128
240 0.247781157493591
241 0.235666185617447
242 0.223030835390091
243 0.221181467175484
244 0.219666346907616
245 0.220342382788658
246 0.218994826078415
247 0.234534218907356
248 0.219746395945549
249 0.227266907691956
250 0.224434942007065
251 0.223535388708115
252 0.228676304221153
253 0.223341405391693
254 0.221789479255676
255 0.228198856115341
256 0.227518856525421
257 0.221864357590675
258 0.223247110843658
259 0.22857041656971
260 0.226867109537125
261 0.221431329846382
262 0.214041650295258
263 0.210780680179596
264 0.207579359412193
265 0.206594243645668
266 0.207180514931679
267 0.20525848865509
268 0.21699246764183
269 0.213504493236542
270 0.193177610635757
271 0.191867396235466
272 0.1917425096035
273 0.191048741340637
274 0.190629124641418
275 0.18484354019165
276 0.182184830307961
277 0.182546198368073
278 0.169725939631462
279 0.171030506491661
280 0.169329360127449
281 0.166307747364044
282 0.169199138879776
283 0.168795436620712
284 0.169819310307503
285 0.166883811354637
286 0.160438150167465
287 0.160826072096825
288 0.159875348210335
289 0.159005552530289
290 0.154507696628571
291 0.166796952486038
292 0.165087148547173
293 0.170931056141853
294 0.170379817485809
295 0.163132384419441
296 0.163804352283478
297 0.150304645299911
298 0.148156687617302
299 0.142644345760345
300 0.143741652369499
301 0.143377333879471
302 0.136150196194649
303 0.136237770318985
304 0.132156804203987
305 0.148717790842056
306 0.143243297934532
307 0.14001789689064
308 0.136611819267273
309 0.12889489531517
310 0.129842609167099
311 0.131141513586044
312 0.138441756367683
313 0.140243321657181
314 0.139389082789421
315 0.13911184668541
316 0.137471139431
317 0.137817576527596
318 0.121214382350445
319 0.119588792324066
320 0.117456264793873
321 0.116848096251488
322 0.115003727376461
323 0.115037001669407
324 0.113973580300808
325 0.120511740446091
326 0.12224967777729
327 0.121028080582619
328 0.122925646603107
329 0.120359376072884
330 0.119746826589108
331 0.118556939065456
332 0.116759277880192
333 0.12085098028183
334 0.120854310691357
335 0.120885521173477
336 0.122177168726921
337 0.135426804423332
338 0.135265573859215
339 0.132487967610359
340 0.136333152651787
341 0.123688548803329
342 0.123349726200104
343 0.116530001163483
344 0.115237191319466
345 0.11607414484024
346 0.113609373569489
347 0.11279571801424
348 0.112188637256622
349 0.11005873978138
350 0.111630983650684
351 0.113487794995308
352 0.111222468316555
353 0.111690104007721
354 0.114909656345844
355 0.0925014615058899
356 0.0961182042956352
357 0.097423255443573
358 0.0970946103334427
359 0.096199743449688
360 0.11288870871067
361 0.111105792224407
362 0.101433001458645
363 0.0991827324032784
364 0.0999032184481621
365 0.0994881168007851
366 0.0996756330132484
367 0.0995417684316635
368 0.09956905990839
369 0.0960053130984306
370 0.0968994870781898
371 0.0981788858771324
372 0.0979500040411949
373 0.097731739282608
374 0.0968924984335899
375 0.0908194482326508
376 0.0890713185071945
377 0.0885981172323227
378 0.0867009162902832
379 0.0870817005634308
380 0.0863237679004669
381 0.0860393941402435
382 0.0837688148021698
383 0.0769779980182648
384 0.0758244544267654
385 0.0747707113623619
386 0.0732480883598328
387 0.0566432103514671
388 0.0564258433878422
389 0.0561881624162197
390 0.0521079413592815
391 0.051975890994072
392 0.0517627708613873
393 0.0515343956649303
394 0.0522651039063931
395 0.0515420623123646
396 0.0513292849063873
397 0.0501937828958035
398 0.0502929873764515
399 0.0498217679560184
400 0.0464591346681118
401 0.0444175340235233
402 0.0444104038178921
403 0.0437727086246014
404 0.0404807776212692
405 0.0463133305311203
406 0.0432997234165668
407 0.0415987819433212
408 0.0425586476922035
409 0.0425300635397434
410 0.0245639011263847
411 0.0245553590357304
412 0.023579016327858
413 0.0232073906809092
414 0.0211637802422047
415 0.020867520943284
416 0.021022729575634
417 0.0206302423030138
418 0.0201246999204159
419 0.0197444837540388
};
\addlegendentry{4-0}
\addplot [semithick, blue]
table {%
0 2.33129477500916
1 2.38665771484375
2 2.42037987709045
3 2.43258810043335
4 2.45723915100098
5 2.4638786315918
6 2.45602989196777
7 2.4748854637146
8 2.48317933082581
9 2.48737168312073
10 2.49040627479553
11 2.48772811889648
12 2.49746084213257
13 2.49147582054138
14 2.49411606788635
15 2.50318384170532
16 2.50943899154663
17 2.50510883331299
18 2.50504899024963
19 2.50619125366211
20 2.50967121124268
21 2.5086977481842
22 2.50158929824829
23 2.50248551368713
24 2.50174379348755
25 2.50355291366577
26 2.50213742256165
27 2.50307869911194
28 2.49955773353577
29 2.49999785423279
30 2.49739336967468
31 2.49679231643677
32 2.49761199951172
33 2.4989800453186
34 2.49923181533813
35 2.49761486053467
36 2.49874377250671
37 2.49727582931519
38 2.49702119827271
39 2.49460029602051
40 2.49387669563293
41 2.49373888969421
42 2.49544501304626
43 2.49539804458618
44 2.49453854560852
45 2.49326205253601
46 2.49286127090454
47 2.49309301376343
48 2.49350047111511
49 2.49212026596069
50 2.49467492103577
51 2.49493908882141
52 2.492356300354
53 2.49020862579346
54 2.48943519592285
55 2.4890570640564
56 2.48687720298767
57 2.48322820663452
58 2.47709536552429
59 2.47560620307922
60 2.4696147441864
61 2.46041202545166
62 2.44629859924316
63 2.43918395042419
64 2.42881083488464
65 2.41184401512146
66 2.40035891532898
67 2.38736319541931
68 2.37625551223755
69 2.36257696151733
70 2.34689354896545
71 2.33381843566895
72 2.32250356674194
73 2.31046724319458
74 2.28995418548584
75 2.27543568611145
76 2.26541709899902
77 2.25003147125244
78 2.23278856277466
79 2.21823048591614
80 2.21388840675354
81 2.20557308197021
82 2.18877172470093
83 2.17310762405396
84 2.16680645942688
85 2.15340900421143
86 2.13515567779541
87 2.12194728851318
88 2.10460186004639
89 2.08284187316895
90 2.06399297714233
91 2.0531210899353
92 2.03315567970276
93 2.01789999008179
94 2.00660276412964
95 1.98194921016693
96 1.96885645389557
97 1.94989883899689
98 1.92943036556244
99 1.91660022735596
100 1.90165102481842
101 1.88702976703644
102 1.8761078119278
103 1.86875677108765
104 1.84938097000122
105 1.83158040046692
106 1.81934022903442
107 1.80540287494659
108 1.78660273551941
109 1.76449406147003
110 1.75672376155853
111 1.74977767467499
112 1.7428492307663
113 1.7358855009079
114 1.72408473491669
115 1.71763956546783
116 1.70636522769928
117 1.70424163341522
118 1.70130050182343
119 1.69406509399414
120 1.69381999969482
121 1.68843734264374
122 1.68495142459869
123 1.67474734783173
124 1.68302881717682
125 1.67928218841553
126 1.66879546642303
127 1.66494309902191
128 1.66693079471588
129 1.66522419452667
130 1.65376281738281
131 1.6373804807663
132 1.63312208652496
133 1.62428462505341
134 1.60998964309692
135 1.60074508190155
136 1.59740400314331
137 1.59757578372955
138 1.59812474250793
139 1.60127604007721
140 1.60487854480743
141 1.59367418289185
142 1.59381496906281
143 1.5891592502594
144 1.5828902721405
145 1.59191834926605
146 1.58650934696198
147 1.58442103862762
148 1.5838235616684
149 1.58000993728638
150 1.57593882083893
151 1.57436192035675
152 1.5689297914505
153 1.56254184246063
154 1.56318140029907
155 1.55724287033081
156 1.56553637981415
157 1.55745136737823
158 1.56637728214264
159 1.56328475475311
160 1.55379641056061
161 1.54946839809418
162 1.54994547367096
163 1.54262435436249
164 1.53744828701019
165 1.54390943050385
166 1.54674053192139
167 1.54405581951141
168 1.5343451499939
169 1.53582835197449
170 1.52211451530457
171 1.52385342121124
172 1.52146100997925
173 1.52216827869415
174 1.51885604858398
175 1.51833164691925
176 1.51469743251801
177 1.51876604557037
178 1.51794052124023
179 1.51385760307312
180 1.51219165325165
181 1.52366077899933
182 1.52544140815735
183 1.53073084354401
184 1.53224682807922
185 1.53769361972809
186 1.53751909732819
187 1.52877926826477
188 1.52626490592957
189 1.52484631538391
190 1.51759946346283
191 1.51756608486176
192 1.51603043079376
193 1.51549851894379
194 1.51711237430573
195 1.50700902938843
196 1.51666855812073
197 1.51765179634094
198 1.52031445503235
199 1.51961505413055
200 1.51945114135742
201 1.51875507831573
202 1.51831257343292
203 1.51212358474731
204 1.50572264194489
205 1.50719773769379
206 1.49127542972565
207 1.4941486120224
208 1.48993134498596
209 1.4929187297821
210 1.49403238296509
211 1.49772500991821
212 1.49172532558441
213 1.49087083339691
214 1.4936980009079
215 1.48494100570679
216 1.49144220352173
217 1.48835849761963
218 1.49158143997192
219 1.49287247657776
220 1.5041629076004
221 1.50045239925385
222 1.49322116374969
223 1.48833954334259
224 1.47911608219147
225 1.47602844238281
226 1.47714114189148
227 1.46796905994415
228 1.45855164527893
229 1.46288514137268
230 1.45679998397827
231 1.45096516609192
232 1.45866298675537
233 1.45685148239136
234 1.45501661300659
235 1.44546473026276
236 1.43808543682098
237 1.43345630168915
238 1.43226408958435
239 1.4259045124054
240 1.41864573955536
241 1.41916036605835
242 1.40255153179169
243 1.40766906738281
244 1.40293121337891
245 1.39974415302277
246 1.38509094715118
247 1.37873113155365
248 1.3645054101944
249 1.35580158233643
250 1.35248506069183
251 1.33293879032135
252 1.32791972160339
253 1.31794583797455
254 1.3088698387146
255 1.30343210697174
256 1.30182099342346
257 1.29517507553101
258 1.27784311771393
259 1.26882123947144
260 1.25345635414124
261 1.23479390144348
262 1.22730243206024
263 1.21479642391205
264 1.20964121818542
265 1.20149064064026
266 1.17738711833954
267 1.16654789447784
268 1.15952062606812
269 1.15210449695587
270 1.1513934135437
271 1.14646136760712
272 1.14952898025513
273 1.14512014389038
274 1.1331889629364
275 1.131587266922
276 1.12254416942596
277 1.11044502258301
278 1.09497058391571
279 1.06986951828003
280 1.05387556552887
281 1.04897212982178
282 1.02116978168488
283 1.01245808601379
284 1.00017368793488
285 0.988133072853088
286 0.97907155752182
287 0.989948093891144
288 0.975592195987701
289 0.969232022762299
290 0.959914922714233
291 0.951411724090576
292 0.950142800807953
293 0.930688083171844
294 0.916055142879486
295 0.905915200710297
296 0.885888874530792
297 0.872867345809937
298 0.866801142692566
299 0.85893052816391
300 0.84284508228302
301 0.835180997848511
302 0.824329435825348
303 0.825981199741364
304 0.828795611858368
305 0.820032358169556
306 0.805013954639435
307 0.786311507225037
308 0.784499645233154
309 0.77556449174881
310 0.768305838108063
311 0.770572483539581
312 0.761676073074341
313 0.764714598655701
314 0.760346293449402
315 0.757917642593384
316 0.752889335155487
317 0.753051400184631
318 0.74730771780014
319 0.733895421028137
320 0.711281299591064
321 0.703983843326569
322 0.684446156024933
323 0.679798305034637
324 0.671868562698364
325 0.66100686788559
326 0.648284554481506
327 0.643210768699646
328 0.642725884914398
329 0.651155769824982
330 0.649598598480225
331 0.632165431976318
332 0.627856254577637
333 0.618902623653412
334 0.630373775959015
335 0.634677171707153
336 0.627642512321472
337 0.601850867271423
338 0.601639032363892
339 0.589225947856903
340 0.583585977554321
341 0.565486907958984
342 0.558054089546204
343 0.547731697559357
344 0.538524091243744
345 0.532804012298584
346 0.528963923454285
347 0.519852161407471
348 0.51866227388382
349 0.515166163444519
350 0.51445597410202
351 0.51697301864624
352 0.507339417934418
353 0.4959497153759
354 0.481106460094452
355 0.492114633321762
356 0.48827251791954
357 0.510482311248779
358 0.513253390789032
359 0.517054915428162
360 0.51488596200943
361 0.509433031082153
362 0.504511058330536
363 0.501874983310699
364 0.488265544176102
365 0.477777987718582
366 0.476348489522934
367 0.469686239957809
368 0.487450689077377
369 0.478630304336548
370 0.47581547498703
371 0.468404948711395
372 0.468857944011688
373 0.467395901679993
374 0.486870586872101
375 0.475224405527115
376 0.488094925880432
377 0.484794527292252
378 0.481145054101944
379 0.463931232690811
380 0.459472566843033
381 0.452331513166428
382 0.447576016187668
383 0.437886536121368
384 0.412877529859543
385 0.401649355888367
386 0.405605435371399
387 0.403112173080444
388 0.395787507295609
389 0.401480853557587
390 0.400527149438858
391 0.400287449359894
392 0.397572964429855
393 0.397302776575089
394 0.39456245303154
395 0.400493979454041
396 0.407704889774323
397 0.431364208459854
398 0.436523139476776
399 0.430124551057816
400 0.424409985542297
401 0.420502156019211
402 0.424545973539352
403 0.431956946849823
404 0.432399302721024
405 0.422156304121017
406 0.422543615102768
407 0.403155237436295
408 0.392052114009857
409 0.37992849946022
410 0.380136519670486
411 0.378744930028915
412 0.375251591205597
413 0.364010035991669
414 0.367169767618179
415 0.363186120986938
416 0.355701893568039
417 0.357481718063354
418 0.323271214962006
419 0.326947033405304
420 0.320276260375977
421 0.317690998315811
422 0.314837694168091
423 0.312744408845901
424 0.300187528133392
425 0.299349367618561
426 0.293781369924545
427 0.298087120056152
428 0.303978949785233
429 0.306329756975174
430 0.311641126871109
431 0.319178283214569
432 0.320301204919815
433 0.322268754243851
434 0.319538176059723
435 0.323445051908493
436 0.316746532917023
437 0.317510694265366
438 0.318771243095398
439 0.317096740007401
440 0.330291658639908
441 0.337811380624771
442 0.337091267108917
443 0.335458159446716
444 0.341351121664047
445 0.349026143550873
446 0.346608817577362
447 0.324264377355576
448 0.313443750143051
449 0.312495797872543
450 0.312232702970505
451 0.309302419424057
452 0.303267866373062
453 0.295312613248825
454 0.298886269330978
455 0.288333356380463
456 0.303461641073227
457 0.300966382026672
458 0.309808641672134
459 0.310688138008118
460 0.30965256690979
461 0.304594129323959
462 0.310332983732224
463 0.332339197397232
464 0.329020410776138
465 0.333606749773026
466 0.334895431995392
467 0.325485527515411
468 0.329407185316086
469 0.333092838525772
470 0.334989368915558
471 0.329853594303131
472 0.327522963285446
473 0.318057239055634
474 0.306751042604446
475 0.303904742002487
476 0.296158909797668
477 0.288542866706848
478 0.282704561948776
479 0.282824039459229
480 0.288229286670685
481 0.287587404251099
482 0.288519203662872
483 0.295817911624908
484 0.304345101118088
485 0.298497796058655
486 0.30112823843956
487 0.301988244056702
488 0.298047810792923
489 0.29993012547493
490 0.288698434829712
491 0.281066298484802
492 0.287681996822357
493 0.291070550680161
494 0.284633249044418
495 0.26857453584671
496 0.26569801568985
497 0.265808194875717
498 0.26208359003067
499 0.260940104722977
500 0.264836192131042
501 0.275702029466629
502 0.273815751075745
503 0.277321308851242
504 0.270799994468689
505 0.28061980009079
506 0.260823845863342
507 0.259612441062927
508 0.257368355989456
509 0.264155507087708
510 0.263271749019623
511 0.261143833398819
512 0.258048385381699
513 0.235734656453133
514 0.233782202005386
515 0.23219607770443
516 0.241733074188232
517 0.239455491304398
518 0.236466184258461
519 0.231674209237099
520 0.237750932574272
521 0.238122120499611
522 0.238836780190468
523 0.237703740596771
524 0.250113517045975
525 0.259698361158371
526 0.255813628435135
527 0.26776984333992
528 0.267179667949677
529 0.266537010669708
530 0.253469079732895
531 0.257340788841248
532 0.262082815170288
533 0.255378991365433
534 0.250660002231598
535 0.250001788139343
536 0.247996404767036
537 0.243962630629539
538 0.243547037243843
539 0.239087522029877
540 0.236183285713196
541 0.235419765114784
542 0.228216603398323
543 0.222419202327728
544 0.221520349383354
545 0.224346041679382
546 0.225951373577118
547 0.224409908056259
548 0.227022632956505
549 0.225313663482666
550 0.219747558236122
551 0.20828253030777
552 0.214599400758743
553 0.204220548272133
554 0.205240979790688
555 0.193055957555771
556 0.192969337105751
557 0.190818324685097
558 0.192962095141411
559 0.186196818947792
560 0.18463534116745
561 0.18551316857338
562 0.190365463495255
563 0.189169600605965
564 0.189758256077766
565 0.185306802392006
566 0.178732544183731
567 0.179940208792686
568 0.180483415722847
569 0.178325086832047
570 0.171884000301361
571 0.172299772500992
572 0.176853463053703
573 0.177249178290367
574 0.179428145289421
575 0.170798569917679
576 0.171249881386757
577 0.157296612858772
578 0.160508617758751
579 0.159069284796715
580 0.161390244960785
581 0.150546804070473
582 0.143500626087189
583 0.142362475395203
584 0.138026252388954
585 0.137691587209702
586 0.134951218962669
587 0.135197207331657
588 0.131570428609848
589 0.137416169047356
590 0.135055869817734
591 0.134625419974327
592 0.137399882078171
593 0.138740986585617
594 0.138337835669518
595 0.134195685386658
596 0.131013929843903
597 0.134466990828514
598 0.131467595696449
599 0.133981689810753
600 0.134067490696907
601 0.143310397863388
602 0.153196796774864
603 0.161470219492912
604 0.161313727498055
605 0.171571999788284
606 0.173000901937485
607 0.175943300127983
608 0.169219270348549
609 0.17597608268261
610 0.177467286586761
611 0.176420092582703
612 0.177103519439697
613 0.176849961280823
614 0.177140101790428
615 0.181490436196327
616 0.176816686987877
617 0.176021531224251
618 0.183807015419006
619 0.180976256728172
620 0.179584488272667
621 0.184665665030479
622 0.183018773794174
623 0.182538285851479
624 0.173832342028618
625 0.183267503976822
626 0.188150271773338
627 0.191963717341423
628 0.191246911883354
629 0.190707594156265
630 0.195215836167336
631 0.196592926979065
632 0.197112783789635
633 0.198415070772171
634 0.207418292760849
635 0.210020363330841
636 0.209922984242439
637 0.20906275510788
638 0.214796632528305
639 0.20759229362011
640 0.208603367209435
641 0.209275603294373
642 0.206888049840927
643 0.211999669671059
644 0.212009280920029
645 0.211965352296829
646 0.211986050009727
647 0.208005085587502
648 0.20869117975235
649 0.209514796733856
650 0.207839086651802
651 0.198537483811378
652 0.190954402089119
653 0.183839172124863
654 0.181398645043373
655 0.17049777507782
656 0.168324828147888
657 0.165348544716835
658 0.165563777089119
659 0.158480882644653
660 0.157734528183937
661 0.156294614076614
662 0.147629514336586
663 0.147727265954018
664 0.147077292203903
665 0.142197921872139
666 0.142068013548851
667 0.145534262061119
668 0.139248311519623
669 0.139505952596664
670 0.139338031411171
671 0.131657510995865
672 0.12686724960804
673 0.136132434010506
674 0.131148338317871
675 0.120008803904057
676 0.114602230489254
677 0.110625445842743
678 0.105907499790192
679 0.108520515263081
680 0.101262710988522
681 0.099707119166851
682 0.0992409139871597
683 0.0963441580533981
684 0.0882310122251511
685 0.0861193388700485
686 0.0907356664538383
687 0.0895541608333588
688 0.090669572353363
689 0.0940496474504471
690 0.0955455005168915
691 0.0961259678006172
692 0.0966421812772751
693 0.0916671305894852
694 0.100530587136745
695 0.101037099957466
696 0.0999447703361511
697 0.101233184337616
698 0.100399829447269
699 0.0969540029764175
700 0.097513921558857
701 0.0983475744724274
702 0.0936366021633148
703 0.0923354029655457
704 0.0937277525663376
705 0.0963004529476166
706 0.103937618434429
707 0.104356691241264
708 0.103758566081524
709 0.104470491409302
710 0.114359341561794
711 0.117098785936832
712 0.120993860065937
713 0.122633315622807
714 0.122866965830326
715 0.12815061211586
716 0.130016654729843
717 0.126738682389259
718 0.123925097286701
719 0.131478354334831
720 0.131758600473404
721 0.132016956806183
722 0.132318615913391
723 0.122444421052933
724 0.121206246316433
725 0.133107483386993
726 0.135643020272255
727 0.138667926192284
728 0.138951927423477
729 0.136740669608116
730 0.136889308691025
731 0.136611863970757
732 0.136037409305573
733 0.139896228909492
734 0.141027510166168
735 0.141902729868889
736 0.139191776514053
737 0.139871880412102
738 0.132913485169411
739 0.127231195569038
740 0.124564416706562
741 0.124095737934113
742 0.123345524072647
743 0.139973551034927
744 0.133140176534653
745 0.133806899189949
746 0.136549904942513
747 0.143869012594223
748 0.146510377526283
749 0.152204364538193
750 0.153595119714737
751 0.155985549092293
752 0.153568372130394
753 0.154500529170036
754 0.155449241399765
755 0.153918698430061
756 0.14642171561718
757 0.147248148918152
758 0.148547649383545
759 0.147001817822456
760 0.136198669672012
761 0.140643939375877
762 0.137368738651276
763 0.14145690202713
764 0.149280592799187
765 0.145365104079247
766 0.146843433380127
767 0.146649643778801
768 0.147212594747543
769 0.144339025020599
770 0.144966021180153
771 0.145057246088982
772 0.144734129309654
773 0.146036773920059
774 0.148248493671417
775 0.13771453499794
776 0.152657315135002
777 0.150108382105827
778 0.152743220329285
779 0.153435811400414
780 0.153384938836098
781 0.154881522059441
782 0.157181113958359
783 0.152163788676262
784 0.152692750096321
785 0.156167447566986
786 0.157497584819794
787 0.158010840415955
788 0.160842165350914
789 0.160611718893051
790 0.16063828766346
791 0.160018593072891
792 0.1599962413311
793 0.152311742305756
794 0.150575384497643
795 0.149373754858971
796 0.15107749402523
797 0.142098098993301
798 0.139325812458992
799 0.132979571819305
800 0.131229892373085
801 0.129214197397232
802 0.129897087812424
803 0.129647672176361
804 0.129487201571465
805 0.130610376596451
806 0.130394369363785
807 0.128552675247192
808 0.122594051063061
809 0.12277002632618
810 0.12343393266201
811 0.116856269538403
812 0.115951187908649
813 0.111788488924503
814 0.104688137769699
815 0.103964492678642
816 0.100828550755978
817 0.100052073597908
818 0.0990055575966835
819 0.093985415995121
820 0.093588151037693
821 0.0933543890714645
822 0.0950532555580139
823 0.0943356603384018
824 0.0923698246479034
825 0.0914097502827644
826 0.0759063884615898
827 0.0810497924685478
828 0.0799665302038193
829 0.0799490287899971
830 0.0805517584085464
831 0.0787246972322464
832 0.0767288058996201
833 0.0774636715650558
834 0.0847287848591805
835 0.0797363445162773
836 0.0762106776237488
837 0.0826324969530106
838 0.0792888626456261
839 0.0793103650212288
840 0.106183812022209
841 0.107177294790745
842 0.109868705272675
843 0.0995732396841049
844 0.102320998907089
845 0.106863670051098
846 0.103123724460602
847 0.104959428310394
848 0.105054467916489
849 0.105044834315777
850 0.111785262823105
851 0.112149484455585
852 0.118733055889606
853 0.11903091520071
854 0.125426709651947
855 0.126828268170357
856 0.128842309117317
857 0.129161074757576
858 0.132011666893959
859 0.136767163872719
860 0.137734442949295
861 0.137200206518173
862 0.141484752297401
863 0.13941265642643
864 0.144629806280136
865 0.145110368728638
866 0.145039618015289
867 0.145263448357582
868 0.145916402339935
869 0.153603091835976
870 0.163447380065918
871 0.163328230381012
872 0.161731660366058
873 0.163467615842819
874 0.163632199168205
875 0.162714302539825
876 0.160759106278419
877 0.163056030869484
878 0.161354959011078
879 0.160640314221382
880 0.160223826766014
881 0.160787314176559
882 0.159506618976593
883 0.160246640443802
884 0.153304859995842
885 0.153490349650383
886 0.153350040316582
887 0.146619737148285
888 0.155039697885513
889 0.164033368229866
890 0.137785285711288
891 0.145707666873932
892 0.145025998353958
893 0.144811898469925
894 0.142027869820595
895 0.13810396194458
896 0.137003988027573
897 0.134949326515198
898 0.134792745113373
899 0.135125279426575
900 0.131670162081718
901 0.130311533808708
902 0.120571404695511
903 0.120086438953876
904 0.114250689744949
905 0.110912434756756
906 0.110614016652107
907 0.110246375203133
908 0.107172787189484
909 0.105635270476341
910 0.113152600824833
911 0.114890977740288
912 0.113969594240189
913 0.119211204349995
914 0.11517071723938
915 0.11530277132988
916 0.115107677876949
917 0.114867106080055
918 0.114732891321182
919 0.107780076563358
920 0.0973461717367172
921 0.0974369943141937
922 0.0974341481924057
923 0.095078743994236
924 0.0949235707521439
925 0.0954554826021194
926 0.0974892154335976
927 0.0906742289662361
928 0.0905633941292763
929 0.0903925970196724
930 0.0958831608295441
931 0.0952451899647713
932 0.0950167775154114
933 0.0923724621534348
934 0.0894909501075745
935 0.094097726047039
936 0.0953395962715149
937 0.0986744090914726
938 0.0903038755059242
939 0.0825363844633102
940 0.0884959250688553
941 0.0796850174665451
942 0.076993852853775
943 0.0769974738359451
944 0.0922065153717995
945 0.0936018079519272
946 0.0930764898657799
947 0.0939157381653786
948 0.0938935577869415
949 0.0939874053001404
950 0.0899478122591972
951 0.0920792743563652
952 0.0923893228173256
953 0.0934360772371292
954 0.091505765914917
955 0.0917521566152573
956 0.0896122679114342
957 0.0894335806369781
958 0.0982369109988213
959 0.0946308523416519
960 0.086094819009304
961 0.0840081200003624
962 0.0809100717306137
963 0.0847381204366684
964 0.0822737812995911
965 0.0815272703766823
966 0.0815967246890068
967 0.0826256573200226
968 0.0834469199180603
969 0.0827040746808052
970 0.084224671125412
971 0.0842737555503845
972 0.0850622206926346
973 0.0853117927908897
974 0.0849290192127228
975 0.0922365486621857
976 0.0902880728244781
977 0.0901361629366875
978 0.0900687053799629
979 0.0904935896396637
980 0.085443027317524
981 0.0853754058480263
982 0.0875523090362549
983 0.0881157368421555
984 0.0879749208688736
985 0.0925106778740883
986 0.0916450023651123
987 0.0867559462785721
988 0.0870375409722328
989 0.0858071148395538
990 0.0790577903389931
991 0.0889875888824463
992 0.0900303721427917
993 0.0898417755961418
994 0.0744632706046104
995 0.0854339227080345
996 0.0864408314228058
997 0.0872078984975815
998 0.0876032188534737
999 0.0925874635577202
1000 0.0927086770534515
1001 0.0923304781317711
1002 0.0942759439349174
1003 0.0988375917077065
1004 0.107719205319881
1005 0.107083819806576
1006 0.107433840632439
1007 0.114624224603176
1008 0.107558943331242
1009 0.107855603098869
1010 0.108451820909977
1011 0.108472675085068
1012 0.109370306134224
1013 0.101104289293289
1014 0.10096101462841
1015 0.101012170314789
1016 0.1045808121562
1017 0.103496067225933
1018 0.102379761636257
1019 0.104422084987164
1020 0.103459671139717
1021 0.103186026215553
1022 0.102262265980244
1023 0.104017712175846
1024 0.104052156209946
1025 0.0970505401492119
1026 0.0975001901388168
1027 0.0967206507921219
1028 0.097065307199955
1029 0.0985395014286041
1030 0.0983845889568329
1031 0.098434753715992
1032 0.0990947186946869
1033 0.100976139307022
1034 0.101159580051899
1035 0.0983273908495903
1036 0.0985770225524902
1037 0.0984438732266426
1038 0.0981074497103691
1039 0.101838357746601
1040 0.102017968893051
1041 0.0991392210125923
1042 0.103055506944656
1043 0.104144886136055
1044 0.104665100574493
1045 0.0921479612588882
1046 0.0912512391805649
1047 0.0901860073208809
1048 0.0901644006371498
1049 0.0855006277561188
1050 0.0854073241353035
1051 0.0837745293974876
1052 0.0815354734659195
1053 0.0795421004295349
1054 0.0703882426023483
1055 0.0703166276216507
1056 0.0699346736073494
1057 0.0627479553222656
1058 0.0609742365777493
1059 0.0606302879750729
1060 0.0596472881734371
1061 0.0598446764051914
1062 0.0586695782840252
1063 0.0576240718364716
1064 0.0571961030364037
1065 0.0576590336859226
1066 0.0542750842869282
1067 0.0594916641712189
1068 0.0592672005295753
1069 0.0570783726871014
1070 0.0563357248902321
1071 0.0565889403223991
1072 0.0566993393003941
1073 0.0550459660589695
1074 0.0552269890904427
1075 0.0553929917514324
1076 0.0633509755134583
1077 0.0643796175718307
1078 0.0669336766004562
1079 0.0688331350684166
1080 0.0703024789690971
1081 0.0741575583815575
1082 0.0716533437371254
1083 0.0734348073601723
1084 0.0742225348949432
1085 0.0688532963395119
1086 0.0690812095999718
1087 0.0702520534396172
1088 0.0702757388353348
1089 0.0665335655212402
1090 0.0740294679999352
1091 0.0721814408898354
1092 0.0676903128623962
1093 0.0703865438699722
1094 0.0700704827904701
1095 0.0709975585341454
1096 0.0709201991558075
1097 0.0712373182177544
1098 0.0721053034067154
1099 0.0713349506258965
1100 0.071537584066391
1101 0.0715178474783897
1102 0.071562334895134
1103 0.0675817430019379
1104 0.075454406440258
1105 0.0755093619227409
1106 0.0755078867077827
1107 0.0754318311810493
1108 0.0870448648929596
1109 0.0874293893575668
1110 0.0928610116243362
1111 0.0948630645871162
1112 0.0947898775339127
1113 0.0945317521691322
1114 0.0951849669218063
1115 0.0941895768046379
1116 0.100676022469997
1117 0.0957202315330505
1118 0.0964416190981865
1119 0.0968752577900887
1120 0.0971720591187477
1121 0.0981351882219315
1122 0.0991549789905548
1123 0.0991274937987328
1124 0.0994855910539627
1125 0.102795422077179
1126 0.0943015664815903
1127 0.092951700091362
1128 0.0981695353984833
1129 0.0941736325621605
1130 0.0929390117526054
1131 0.0908286273479462
1132 0.0904823765158653
1133 0.0865305885672569
1134 0.0865855515003204
1135 0.0858815163373947
1136 0.0849745497107506
1137 0.0838053226470947
1138 0.0843231379985809
1139 0.0840488150715828
1140 0.0784965679049492
1141 0.0764742493629456
1142 0.0784014016389847
1143 0.0756422057747841
1144 0.0753694698214531
1145 0.0817034244537354
1146 0.082318976521492
1147 0.0811943709850311
1148 0.0813645198941231
1149 0.0838886052370071
1150 0.0836841836571693
1151 0.0841564536094666
1152 0.0857444107532501
1153 0.0856302753090858
1154 0.0782116502523422
1155 0.0781198814511299
1156 0.0781703591346741
1157 0.0782065242528915
1158 0.0668919235467911
1159 0.0666489750146866
1160 0.0610032752156258
1161 0.0601364001631737
1162 0.0607776008546352
1163 0.0649277716875076
1164 0.0648685693740845
1165 0.0650695785880089
1166 0.058843445032835
1167 0.0594024211168289
1168 0.0593123435974121
1169 0.0590458400547504
1170 0.058690432459116
1171 0.0576255023479462
1172 0.0564563572406769
1173 0.0560553595423698
1174 0.0581319220364094
1175 0.0538565590977669
1176 0.0585272312164307
1177 0.0589631758630276
1178 0.051429495215416
1179 0.0512381345033646
1180 0.0504474602639675
1181 0.048856183886528
1182 0.0502215847373009
1183 0.0499813891947269
1184 0.0497492216527462
1185 0.0499628446996212
1186 0.0499003641307354
1187 0.0498941093683243
1188 0.0495074242353439
1189 0.049506351351738
1190 0.0474685095250607
1191 0.0440507754683495
1192 0.0458998307585716
1193 0.0449986830353737
1194 0.044986929744482
1195 0.0387359820306301
1196 0.0377114377915859
1197 0.0503069385886192
1198 0.0486799031496048
1199 0.0461326017975807
1200 0.0460965409874916
1201 0.0453466884791851
1202 0.0439932011067867
1203 0.0448119938373566
1204 0.0444434434175491
1205 0.0446652285754681
1206 0.0514292009174824
1207 0.0513990260660648
1208 0.0514115802943707
1209 0.0514495484530926
1210 0.052003089338541
1211 0.052299652248621
1212 0.0521687120199203
1213 0.0497050285339355
1214 0.0493291430175304
1215 0.0490786880254745
1216 0.0485861338675022
1217 0.0480351336300373
1218 0.0474171787500381
1219 0.0472726337611675
1220 0.0472320467233658
1221 0.0475306026637554
1222 0.0476898513734341
1223 0.0477608256042004
1224 0.0450458079576492
1225 0.0454254522919655
1226 0.0411027707159519
1227 0.040572002530098
1228 0.0410161316394806
1229 0.0414069071412086
1230 0.0414285697042942
1231 0.0440778769552708
1232 0.0426816381514072
1233 0.0426541902124882
1234 0.0419077016413212
1235 0.0441917665302753
1236 0.0481306463479996
1237 0.0487882718443871
1238 0.0487094894051552
1239 0.0487259589135647
};
\addlegendentry{4-3}
\addplot [semithick, green!50.0!black]
table {%
0 2.6607460975647
1 2.65746307373047
2 2.69301986694336
3 2.73558187484741
4 2.77220153808594
5 2.75296950340271
6 2.76393127441406
7 2.76610660552979
8 2.75702142715454
9 2.77715706825256
10 2.77666568756104
11 2.78550148010254
12 2.78960132598877
13 2.79148077964783
14 2.7904007434845
15 2.78757643699646
16 2.78801417350769
17 2.7838830947876
18 2.78217315673828
19 2.78733348846436
20 2.78865337371826
21 2.78772711753845
22 2.78629803657532
23 2.7876091003418
24 2.78949856758118
25 2.78413414955139
26 2.78478670120239
27 2.78385710716248
28 2.78483581542969
29 2.78811097145081
30 2.78946590423584
31 2.78882694244385
32 2.7895085811615
33 2.78782033920288
34 2.78766465187073
35 2.78627514839172
36 2.78620600700378
37 2.7866530418396
38 2.78691983222961
39 2.78731513023376
40 2.78766560554504
41 2.78895378112793
42 2.78840517997742
43 2.78720831871033
44 2.78800988197327
45 2.78651285171509
46 2.7853889465332
47 2.78348278999329
48 2.7824821472168
49 2.78219890594482
50 2.78029108047485
51 2.78386068344116
52 2.78140830993652
53 2.77776646614075
54 2.77198314666748
55 2.77140688896179
56 2.76459264755249
57 2.76367855072021
58 2.75487661361694
59 2.74656653404236
60 2.74216246604919
61 2.7367115020752
62 2.72679209709167
63 2.72216439247131
64 2.7073917388916
65 2.69769358634949
66 2.68446040153503
67 2.67772960662842
68 2.67128610610962
69 2.6532621383667
70 2.64535546302795
71 2.63551425933838
72 2.62416410446167
73 2.61902856826782
74 2.59821677207947
75 2.58203148841858
76 2.56719303131104
77 2.55222058296204
78 2.53204441070557
79 2.51230454444885
80 2.50186491012573
81 2.48675107955933
82 2.46896433830261
83 2.45795679092407
84 2.43487596511841
85 2.42410683631897
86 2.4107711315155
87 2.392254114151
88 2.3721022605896
89 2.36155796051025
90 2.33689761161804
91 2.31887173652649
92 2.31103920936584
93 2.2989706993103
94 2.27556467056274
95 2.25867414474487
96 2.24568843841553
97 2.22966170310974
98 2.22106695175171
99 2.20209097862244
100 2.18861794471741
101 2.17848491668701
102 2.1635844707489
103 2.15436601638794
104 2.14444398880005
105 2.13094043731689
106 2.11947226524353
107 2.10301661491394
108 2.09001564979553
109 2.0691819190979
110 2.05877494812012
111 2.04384207725525
112 2.04508090019226
113 2.03214168548584
114 2.03156757354736
115 2.02256560325623
116 2.01508712768555
117 2.0087718963623
118 1.99641942977905
119 1.99130773544312
120 1.98712491989136
121 1.98512125015259
122 1.98209619522095
123 1.9700710773468
124 1.97072672843933
125 1.96820020675659
126 1.96683466434479
127 1.95444226264954
128 1.95541071891785
129 1.95358049869537
130 1.94641125202179
131 1.94784510135651
132 1.94002759456635
133 1.93430423736572
134 1.93187999725342
135 1.9232439994812
136 1.91337907314301
137 1.90745341777802
138 1.90774357318878
139 1.89485335350037
140 1.89359223842621
141 1.8841632604599
142 1.87122344970703
143 1.86450886726379
144 1.86140513420105
145 1.85051584243774
146 1.84041750431061
147 1.83368420600891
148 1.81315076351166
149 1.79817235469818
150 1.80380523204803
151 1.78783237934113
152 1.78172242641449
153 1.77237224578857
154 1.773402094841
155 1.77143383026123
156 1.76700353622437
157 1.76153528690338
158 1.77012777328491
159 1.77591216564178
160 1.77069544792175
161 1.76749110221863
162 1.76118803024292
163 1.76066386699677
164 1.74403715133667
165 1.73535001277924
166 1.72692549228668
167 1.71218168735504
168 1.70592570304871
169 1.70357131958008
170 1.68451809883118
171 1.67258191108704
172 1.66069102287292
173 1.65530061721802
174 1.64959514141083
175 1.63892960548401
176 1.63697421550751
177 1.63644444942474
178 1.62824475765228
179 1.61631894111633
180 1.60566258430481
181 1.58256733417511
182 1.57991647720337
183 1.57594132423401
184 1.58396649360657
185 1.57548940181732
186 1.58391082286835
187 1.57582640647888
188 1.56846821308136
189 1.56483995914459
190 1.57025289535522
191 1.56269145011902
192 1.54942119121552
193 1.53753435611725
194 1.53961420059204
195 1.53491115570068
196 1.52260828018188
197 1.51220917701721
198 1.51113057136536
199 1.51568174362183
200 1.50603556632996
201 1.51719903945923
202 1.51532244682312
203 1.50667643547058
204 1.49493253231049
205 1.48574876785278
206 1.48057246208191
207 1.46586346626282
208 1.44207620620728
209 1.43775677680969
210 1.42667484283447
211 1.42986845970154
212 1.41511905193329
213 1.40688812732697
214 1.40292942523956
215 1.39670979976654
216 1.40618503093719
217 1.41180300712585
218 1.41317963600159
219 1.41987919807434
220 1.42961537837982
221 1.43526577949524
222 1.43094396591187
223 1.42590379714966
224 1.41471934318542
225 1.41628444194794
226 1.41014099121094
227 1.40467798709869
228 1.4003301858902
229 1.40533721446991
230 1.40424084663391
231 1.41275334358215
232 1.40679109096527
233 1.39127838611603
234 1.37339568138123
235 1.36721765995026
236 1.34978640079498
237 1.3582638502121
238 1.35407054424286
239 1.35668706893921
240 1.34007036685944
241 1.34155452251434
242 1.34395611286163
243 1.34128499031067
244 1.33466112613678
245 1.33686006069183
246 1.33910799026489
247 1.34087634086609
248 1.33197259902954
249 1.32148313522339
250 1.31871914863586
251 1.30457377433777
252 1.29347705841064
253 1.29130220413208
254 1.2846075296402
255 1.27910435199738
256 1.26473772525787
257 1.26491725444794
258 1.26997542381287
259 1.2552695274353
260 1.24936199188232
261 1.23569774627686
262 1.2348450422287
263 1.23054707050323
264 1.23697447776794
265 1.23237001895905
266 1.21957492828369
267 1.20001709461212
268 1.18752992153168
269 1.1619176864624
270 1.14166641235352
271 1.11951124668121
272 1.10995578765869
273 1.10272467136383
274 1.11281311511993
275 1.11432003974915
276 1.09797894954681
277 1.08839011192322
278 1.08788669109344
279 1.08017289638519
280 1.07683610916138
281 1.06798589229584
282 1.06528437137604
283 1.06789779663086
284 1.06977999210358
285 1.06414103507996
286 1.04958093166351
287 1.04560089111328
288 1.03106462955475
289 1.02904057502747
290 1.01753878593445
291 1.00871467590332
292 0.993185520172119
293 0.982912302017212
294 0.973694980144501
295 0.972913146018982
296 0.96198707818985
297 0.946750700473785
298 0.945476531982422
299 0.939827024936676
300 0.922705054283142
301 0.905072271823883
302 0.906699776649475
303 0.893231987953186
304 0.88075065612793
305 0.879011750221252
306 0.876028835773468
307 0.863447964191437
308 0.847621083259583
309 0.843056499958038
310 0.828112661838531
311 0.823945760726929
312 0.807567238807678
313 0.794164061546326
314 0.788317501544952
315 0.787758052349091
316 0.772528231143951
317 0.778091669082642
318 0.775095164775848
319 0.772423386573792
320 0.76373565196991
321 0.759519040584564
322 0.752344906330109
323 0.735517263412476
324 0.723655700683594
325 0.717068076133728
326 0.709175646305084
327 0.709663689136505
328 0.69525820016861
329 0.683118999004364
330 0.670455455780029
331 0.662656962871552
332 0.657289862632751
333 0.645365476608276
334 0.640935361385345
335 0.629350066184998
336 0.64174872636795
337 0.66284716129303
338 0.657359898090363
339 0.631067395210266
340 0.63592004776001
341 0.627349615097046
342 0.631044864654541
343 0.626687288284302
344 0.620566487312317
345 0.604954779148102
346 0.599915146827698
347 0.59835958480835
348 0.588316023349762
349 0.581666886806488
350 0.587456881999969
351 0.585145473480225
352 0.576070427894592
353 0.576952993869781
354 0.570246756076813
355 0.566266000270844
356 0.56237268447876
357 0.55975753068924
358 0.575875222682953
359 0.571889579296112
360 0.579927384853363
361 0.575603723526001
362 0.586448907852173
363 0.584764122962952
364 0.572318315505981
365 0.572290301322937
366 0.581371486186981
367 0.571698129177094
368 0.563274919986725
369 0.561265766620636
370 0.586203932762146
371 0.582991778850555
372 0.581371903419495
373 0.600562810897827
374 0.587301433086395
375 0.579713582992554
376 0.586424171924591
377 0.582321047782898
378 0.575705349445343
379 0.56948333978653
380 0.573343217372894
381 0.564748823642731
382 0.554653108119965
383 0.552077233791351
384 0.537737667560577
385 0.540691256523132
386 0.544020473957062
387 0.50907838344574
388 0.519190549850464
389 0.539377570152283
390 0.536193072795868
391 0.543850660324097
392 0.54244601726532
393 0.538962602615356
394 0.533571124076843
395 0.541039884090424
396 0.538527965545654
397 0.56167608499527
398 0.575544655323029
399 0.582240283489227
400 0.568711459636688
401 0.566554248332977
402 0.55592942237854
403 0.563290774822235
404 0.575695276260376
405 0.570451319217682
406 0.573455631732941
407 0.580852806568146
408 0.56549745798111
409 0.559592962265015
410 0.556433737277985
411 0.547632396221161
412 0.529668211936951
413 0.529602706432343
414 0.52241188287735
415 0.522080361843109
416 0.511714160442352
417 0.521346747875214
418 0.517190098762512
419 0.516548335552216
420 0.490060657262802
421 0.487461924552917
422 0.485375434160233
423 0.467049956321716
424 0.465379178524017
425 0.468069165945053
426 0.460062593221664
427 0.457936555147171
428 0.456628382205963
429 0.465474545955658
430 0.461963474750519
431 0.467959672212601
432 0.47091343998909
433 0.474675178527832
434 0.478126376867294
435 0.483551532030106
436 0.466952323913574
437 0.462280184030533
438 0.452399343252182
439 0.432889401912689
440 0.431295484304428
441 0.419596821069717
442 0.416484445333481
443 0.41443145275116
444 0.423016160726547
445 0.411880671977997
446 0.413790464401245
447 0.388252705335617
448 0.374820679426193
449 0.367588073015213
450 0.368901342153549
451 0.366865694522858
452 0.373190194368362
453 0.360344618558884
454 0.345977962017059
455 0.336898565292358
456 0.329724162817001
457 0.327130675315857
458 0.323091238737106
459 0.322307139635086
460 0.318328559398651
461 0.315694868564606
462 0.318596810102463
463 0.32841369509697
464 0.331335753202438
465 0.317016661167145
466 0.315666705369949
467 0.301872849464417
468 0.31786772608757
469 0.314087808132172
470 0.317153066396713
471 0.313175618648529
472 0.31269770860672
473 0.314443290233612
474 0.313121974468231
475 0.301618158817291
476 0.296458572149277
477 0.29234305024147
478 0.296387642621994
479 0.296043157577515
480 0.285096555948257
481 0.280993074178696
482 0.275891810655594
483 0.27178156375885
484 0.267673045396805
485 0.25810706615448
486 0.25140792131424
487 0.244401499629021
488 0.253188997507095
489 0.253954291343689
490 0.251802384853363
491 0.25512957572937
492 0.248878344893456
493 0.252885043621063
494 0.245754048228264
495 0.243736624717712
496 0.245472654700279
497 0.243078961968422
498 0.250555753707886
499 0.249579176306725
500 0.253226339817047
501 0.251158744096756
502 0.245692685246468
503 0.242234587669373
504 0.241033405065536
505 0.240188002586365
506 0.24526035785675
507 0.239360198378563
508 0.237936109304428
509 0.236843943595886
510 0.233567506074905
511 0.232089087367058
512 0.23835776746273
513 0.222597941756248
514 0.237072259187698
515 0.243174642324448
516 0.243158042430878
517 0.241813853383064
518 0.227010026574135
519 0.223834782838821
520 0.217957094311714
521 0.217626631259918
522 0.214478850364685
523 0.208347544074059
524 0.210409298539162
525 0.214124858379364
526 0.219549655914307
527 0.225416123867035
528 0.230439305305481
529 0.234626725316048
530 0.235247284173965
531 0.23217348754406
532 0.231513500213623
533 0.225885525345802
534 0.224756225943565
535 0.223178744316101
536 0.234314724802971
537 0.237853795289993
538 0.229270413517952
539 0.226057678461075
540 0.221432343125343
541 0.220695629715919
542 0.221411421895027
543 0.222551703453064
544 0.230241715908051
545 0.231456562876701
546 0.227642729878426
547 0.22858190536499
548 0.21742644906044
549 0.218120187520981
550 0.210837140679359
551 0.214527428150177
552 0.216675221920013
553 0.214808136224747
554 0.215900018811226
555 0.217664167284966
556 0.227185592055321
557 0.22539621591568
558 0.224956974387169
559 0.22572273015976
560 0.22516918182373
561 0.22318796813488
562 0.212392628192902
563 0.209261417388916
564 0.192053616046906
565 0.186892971396446
566 0.182383865118027
567 0.180199906229973
568 0.181417733430862
569 0.185266315937042
570 0.186389058828354
571 0.187971502542496
572 0.188754424452782
573 0.1898053586483
574 0.188721045851707
575 0.183307692408562
576 0.177368849515915
577 0.173172831535339
578 0.162900328636169
579 0.148291066288948
580 0.149562612175941
581 0.149289801716805
582 0.149075016379356
583 0.150046572089195
584 0.162597730755806
585 0.170998990535736
586 0.158821225166321
587 0.152915045619011
588 0.150380149483681
589 0.150810495018959
590 0.151263192296028
591 0.147218406200409
592 0.145612344145775
593 0.156414672732353
594 0.144550442695618
595 0.155597284436226
596 0.162639021873474
597 0.176962226629257
598 0.184591382741928
599 0.189289450645447
600 0.19642773270607
601 0.195388793945312
602 0.205322071909904
603 0.209740579128265
604 0.207394644618034
605 0.214530378580093
606 0.207286283373833
607 0.210004672408104
608 0.211464524269104
609 0.217328026890755
610 0.227118343114853
611 0.228795036673546
612 0.229926466941833
613 0.229914516210556
614 0.231883928179741
615 0.231262892484665
616 0.232403710484505
617 0.239326566457748
618 0.235805034637451
619 0.232987329363823
620 0.239855036139488
621 0.249433740973473
622 0.249046668410301
623 0.250086486339569
624 0.250465601682663
625 0.25418159365654
626 0.256662279367447
627 0.253896147012711
628 0.256173104047775
629 0.255377382040024
630 0.261936604976654
631 0.26496946811676
632 0.265232443809509
633 0.264095902442932
634 0.25107154250145
635 0.253278821706772
636 0.275346338748932
637 0.27982971072197
638 0.27905011177063
639 0.301006853580475
640 0.303303688764572
641 0.314232110977173
642 0.315527230501175
643 0.300576508045197
644 0.30147060751915
645 0.291197270154953
646 0.284241855144501
647 0.284864157438278
648 0.277417421340942
649 0.271808445453644
650 0.266008138656616
651 0.262887984514236
652 0.257198303937912
653 0.253532856702805
654 0.253266185522079
655 0.243792682886124
656 0.233567297458649
657 0.231913983821869
658 0.229896634817123
659 0.223397925496101
660 0.217602640390396
661 0.223809778690338
662 0.222160637378693
663 0.221306622028351
664 0.21964767575264
665 0.218405649065971
666 0.216383874416351
667 0.209776937961578
668 0.212654113769531
669 0.215411961078644
670 0.21382936835289
671 0.205338954925537
672 0.211461052298546
673 0.208581581711769
674 0.216566041111946
675 0.211688563227654
676 0.209998533129692
677 0.209155365824699
678 0.212976679205894
679 0.215850740671158
680 0.207601949572563
681 0.205662548542023
682 0.206189721822739
683 0.206010401248932
684 0.20562656223774
685 0.199699118733406
686 0.177716240286827
687 0.173481851816177
688 0.172033488750458
689 0.158767431974411
690 0.15891720354557
691 0.14790403842926
692 0.148060649633408
693 0.151767820119858
694 0.151833444833755
695 0.149800375103951
696 0.147176221013069
697 0.131304413080215
698 0.136699080467224
699 0.1368537992239
700 0.13347327709198
701 0.134106129407883
702 0.13225394487381
703 0.132174342870712
704 0.131782546639442
705 0.131761878728867
706 0.136810645461082
707 0.140732243657112
708 0.141261905431747
709 0.141533598303795
710 0.138402208685875
711 0.130917638540268
712 0.13105982542038
713 0.13353443145752
714 0.133622840046883
715 0.133449569344521
716 0.134521633386612
717 0.134093999862671
718 0.131838083267212
719 0.127224504947662
720 0.132029727101326
721 0.12958650290966
722 0.124242044985294
723 0.13744243979454
724 0.126936718821526
725 0.141037985682487
726 0.140925243496895
727 0.141358196735382
728 0.135771870613098
729 0.133256018161774
730 0.149979263544083
731 0.154383987188339
732 0.160742551088333
733 0.16164655983448
734 0.171958491206169
735 0.166941314935684
736 0.174934789538383
737 0.186739385128021
738 0.187856405973434
739 0.178316816687584
740 0.175181061029434
741 0.175013661384583
742 0.177197173237801
743 0.172032684087753
744 0.170408591628075
745 0.176043495535851
746 0.177594795823097
747 0.184578880667686
748 0.182458147406578
749 0.182148337364197
750 0.181473106145859
751 0.182142168283463
752 0.177444279193878
753 0.177510052919388
754 0.176935598254204
755 0.179335996508598
756 0.175268575549126
757 0.175626829266548
758 0.175111159682274
759 0.178975835442543
760 0.178919225931168
761 0.179763406515121
762 0.179791375994682
763 0.177986234426498
764 0.177038460969925
765 0.177691578865051
766 0.177147373557091
767 0.178850561380386
768 0.176818147301674
769 0.176795139908791
770 0.165477573871613
771 0.174929186701775
772 0.173786759376526
773 0.161632627248764
774 0.164368614554405
775 0.150296419858932
776 0.149632170796394
777 0.15076507627964
778 0.152264267206192
779 0.153844177722931
780 0.136944711208344
781 0.131206706166267
782 0.12342644482851
783 0.122954584658146
784 0.112689822912216
785 0.113539092242718
786 0.107262343168259
787 0.0953831747174263
788 0.107539139688015
789 0.106511563062668
790 0.109276428818703
791 0.111215323209763
792 0.109362974762917
793 0.127702996134758
794 0.131267324090004
795 0.12519559264183
796 0.131181627511978
797 0.124743193387985
798 0.121189385652542
799 0.12071381509304
800 0.120005637407303
801 0.118967227637768
802 0.118999660015106
803 0.117624014616013
804 0.117354862391949
805 0.114825457334518
806 0.114836342632771
807 0.109341695904732
808 0.109149768948555
809 0.107783906161785
810 0.107234992086887
811 0.118785746395588
812 0.118462026119232
813 0.128545477986336
814 0.134300425648689
815 0.13488344848156
816 0.134805649518967
817 0.137152135372162
818 0.137540459632874
819 0.14070987701416
820 0.141248866915703
821 0.131207779049873
822 0.139455765485764
823 0.138689428567886
824 0.136232525110245
825 0.136740669608116
826 0.136626303195953
827 0.146673142910004
828 0.144923433661461
829 0.143074661493301
830 0.142762035131454
831 0.142841696739197
832 0.143217593431473
833 0.143095269799232
834 0.143382951617241
835 0.145955950021744
836 0.145837292075157
837 0.145622134208679
838 0.140194609761238
839 0.141213119029999
840 0.143210828304291
841 0.142278984189034
842 0.140673190355301
843 0.122110344469547
844 0.135363310575485
845 0.135711058974266
846 0.128073185682297
847 0.127229198813438
848 0.129479348659515
849 0.130191311240196
850 0.129861176013947
851 0.131493031978607
852 0.13040779531002
853 0.132604792714119
854 0.133920595049858
855 0.137737214565277
856 0.138898074626923
857 0.141766771674156
858 0.144056171178818
859 0.156047701835632
860 0.163042694330215
861 0.160288512706757
862 0.176218941807747
863 0.171794950962067
864 0.176300674676895
865 0.181127116084099
866 0.190563008189201
867 0.190957322716713
868 0.200486481189728
869 0.201786383986473
870 0.204217851161957
871 0.207589328289032
872 0.202743396162987
873 0.207508236169815
874 0.224231630563736
875 0.231868475675583
876 0.241752162575722
877 0.234553143382072
878 0.238094076514244
879 0.245352491736412
880 0.254038065671921
881 0.254102617502213
882 0.255570650100708
883 0.281378388404846
884 0.281083077192307
885 0.279105842113495
886 0.289848178625107
887 0.292433142662048
888 0.286171585321426
889 0.285290747880936
890 0.282665461301804
891 0.287394225597382
892 0.287900179624557
893 0.290928393602371
894 0.279288411140442
895 0.278401970863342
896 0.278967469930649
897 0.283216863870621
898 0.284799009561539
899 0.292623102664948
900 0.293206930160522
901 0.291785150766373
902 0.293630212545395
903 0.296992272138596
904 0.307508170604706
905 0.303012609481812
906 0.301078796386719
907 0.298522382974625
908 0.299123734235764
909 0.287752419710159
910 0.285611659288406
911 0.275580555200577
912 0.265694558620453
913 0.260770678520203
914 0.252457171678543
915 0.247125700116158
916 0.252457141876221
917 0.247976437211037
918 0.239863097667694
919 0.237464547157288
920 0.233461678028107
921 0.23128142952919
922 0.227945312857628
923 0.22847755253315
924 0.212943956255913
925 0.213938027620316
926 0.205766201019287
927 0.201832234859467
928 0.197991162538528
929 0.194023460149765
930 0.188539505004883
931 0.189184859395027
932 0.187783315777779
933 0.16195310652256
934 0.161364421248436
935 0.159582316875458
936 0.147516027092934
937 0.145407751202583
938 0.163239672780037
939 0.16921529173851
940 0.167946085333824
941 0.162591651082039
942 0.162577897310257
943 0.160649701952934
944 0.157012715935707
945 0.156542718410492
946 0.156587272882462
947 0.153118625283241
948 0.152129948139191
949 0.145013749599457
950 0.156275629997253
951 0.155355259776115
952 0.153173610568047
953 0.1485266238451
954 0.137122064828873
955 0.137068793177605
956 0.136730551719666
957 0.136834740638733
958 0.137679517269135
959 0.151437997817993
960 0.14641098678112
961 0.146927878260612
962 0.14110004901886
963 0.144171118736267
964 0.142440870404243
965 0.142092362046242
966 0.127328217029572
967 0.127506271004677
968 0.132214829325676
969 0.130636870861053
970 0.131505936384201
971 0.146186903119087
972 0.153998196125031
973 0.148752257227898
974 0.147663533687592
975 0.146678447723389
976 0.158632442355156
977 0.168352469801903
978 0.171821787953377
979 0.171257674694061
980 0.169170990586281
981 0.170189380645752
982 0.17409111559391
983 0.176560014486313
984 0.177060559391975
985 0.178282529115677
986 0.178014874458313
987 0.183440014719963
988 0.167412474751472
989 0.161330178380013
990 0.160572931170464
991 0.161031544208527
992 0.168912768363953
993 0.168485566973686
994 0.16727115213871
995 0.166951775550842
996 0.166343346238136
997 0.165218845009804
998 0.162549555301666
999 0.160569950938225
1000 0.149028196930885
1001 0.155889630317688
1002 0.159737974405289
1003 0.160631433129311
1004 0.16447140276432
1005 0.1661456823349
1006 0.17068412899971
1007 0.172477170825005
1008 0.170103073120117
1009 0.154374152421951
1010 0.161200314760208
1011 0.163997784256935
1012 0.163641512393951
1013 0.158764779567719
1014 0.158404871821404
1015 0.158396869897842
1016 0.162225812673569
1017 0.170521408319473
1018 0.171479985117912
1019 0.178956240415573
1020 0.178637370467186
1021 0.164887651801109
1022 0.157491460442543
1023 0.158161550760269
1024 0.158746242523193
1025 0.150770857930183
1026 0.138108298182487
1027 0.128717660903931
1028 0.125909030437469
1029 0.124718777835369
1030 0.132394090294838
1031 0.133010163903236
1032 0.129567950963974
1033 0.127776384353638
1034 0.130906209349632
1035 0.129958495497704
1036 0.129942819476128
1037 0.126593604683876
1038 0.126251548528671
1039 0.132363602519035
1040 0.135905995965004
1041 0.135529935359955
1042 0.136024057865143
1043 0.135043531656265
1044 0.134759336709976
1045 0.135616138577461
1046 0.140005946159363
1047 0.140129491686821
1048 0.13917101919651
1049 0.14033667743206
1050 0.141612410545349
1051 0.135090067982674
1052 0.135608926415443
1053 0.133745729923248
1054 0.130691662430763
1055 0.129745811223984
1056 0.126729339361191
1057 0.125764086842537
1058 0.129456907510757
1059 0.128093346953392
1060 0.12069895118475
1061 0.120026327669621
1062 0.120171569287777
1063 0.123110197484493
1064 0.123402334749699
1065 0.124150343239307
1066 0.119849376380444
1067 0.11449770629406
1068 0.107252903282642
1069 0.10283150523901
1070 0.102600209414959
1071 0.101676866412163
1072 0.104243792593479
1073 0.1087601557374
1074 0.109879866242409
1075 0.117740824818611
1076 0.118846520781517
1077 0.118205957114697
1078 0.11737185716629
1079 0.115431174635887
1080 0.110890872776508
1081 0.108770206570625
1082 0.10841716080904
1083 0.108063504099846
1084 0.104998037219048
1085 0.105089329183102
1086 0.115578755736351
1087 0.113115884363651
1088 0.110690981149673
1089 0.104709260165691
1090 0.100727722048759
1091 0.101432122290134
1092 0.0959399715065956
1093 0.0979929715394974
1094 0.0978973656892776
1095 0.0970732122659683
1096 0.0927243232727051
1097 0.0999242812395096
1098 0.100084997713566
1099 0.104294873774052
1100 0.104742251336575
1101 0.106886021792889
1102 0.103514119982719
1103 0.105324313044548
1104 0.109114550054073
1105 0.109134048223495
1106 0.108322322368622
1107 0.110123440623283
1108 0.107953809201717
1109 0.108255930244923
1110 0.109147697687149
1111 0.106789462268353
1112 0.109476543962955
1113 0.107098318636417
1114 0.10751610994339
1115 0.106754571199417
1116 0.106661148369312
1117 0.103812545537949
1118 0.103258848190308
1119 0.100820228457451
1120 0.10021910816431
1121 0.098816767334938
1122 0.0953196585178375
1123 0.0894819349050522
1124 0.0875133723020554
1125 0.0793874561786652
1126 0.0770278498530388
1127 0.0765086561441422
1128 0.0952583700418472
1129 0.101533584296703
1130 0.0972785353660583
1131 0.0969951152801514
1132 0.097367525100708
1133 0.0970968082547188
1134 0.100634671747684
1135 0.100980766117573
1136 0.090890496969223
1137 0.0935883224010468
1138 0.0941379442811012
1139 0.0979970097541809
1140 0.0981408283114433
1141 0.101350329816341
1142 0.0985583886504173
1143 0.0966857150197029
1144 0.09816525131464
1145 0.103151299059391
1146 0.107316061854362
1147 0.0999467968940735
1148 0.100116260349751
1149 0.0960660874843597
1150 0.0946302339434624
1151 0.103616386651993
1152 0.103987589478493
1153 0.105665437877178
1154 0.101508408784866
1155 0.101579390466213
1156 0.100888803601265
1157 0.09903684258461
1158 0.0968358963727951
1159 0.105175733566284
1160 0.105203993618488
1161 0.106000937521458
1162 0.10314904153347
1163 0.102451354265213
1164 0.103321254253387
1165 0.103020764887333
1166 0.104995138943195
1167 0.104758061468601
1168 0.105270206928253
1169 0.104673109948635
1170 0.105252154171467
1171 0.105467617511749
1172 0.106661356985569
1173 0.106693223118782
1174 0.10626819729805
1175 0.107016943395138
1176 0.107650011777878
1177 0.107603959739208
1178 0.0887916386127472
1179 0.0833324268460274
1180 0.0833764001727104
1181 0.0832395851612091
1182 0.0822155177593231
1183 0.0817456841468811
1184 0.0774655938148499
1185 0.0767671540379524
1186 0.0760274454951286
1187 0.073627732694149
1188 0.0724057704210281
1189 0.0758288130164146
1190 0.0761572048068047
1191 0.0719505250453949
1192 0.0725371241569519
1193 0.0759730637073517
1194 0.0733767002820969
1195 0.0688232630491257
1196 0.0649723634123802
1197 0.0653197169303894
1198 0.0657993406057358
1199 0.0667392089962959
1200 0.0660980269312859
1201 0.0542596764862537
1202 0.0529553703963757
1203 0.0497457943856716
1204 0.0490313321352005
1205 0.0634758397936821
1206 0.0631846860051155
1207 0.0752298608422279
1208 0.0747668743133545
1209 0.0658961459994316
1210 0.0649350807070732
1211 0.0662356615066528
1212 0.066621296107769
1213 0.066811315715313
1214 0.0661007240414619
1215 0.0667007192969322
1216 0.0650474280118942
1217 0.0646871104836464
1218 0.0641522109508514
1219 0.0636865347623825
1220 0.0634388253092766
1221 0.063281424343586
1222 0.0622499100863934
1223 0.062452644109726
1224 0.0632162690162659
1225 0.0638835653662682
1226 0.0646988824009895
1227 0.0669960826635361
1228 0.0681373775005341
1229 0.06782016903162
1230 0.0683094188570976
1231 0.0746507719159126
1232 0.0746740028262138
1233 0.0749629214406013
1234 0.0758362039923668
1235 0.077187106013298
1236 0.0772564336657524
1237 0.0771041959524155
1238 0.0803164839744568
1239 0.0738029181957245
};
\addlegendentry{4-4}
\addplot [semithick, red]
table {%
0 2.59678983688354
1 2.59590792655945
2 2.75101375579834
3 2.87062239646912
4 2.88748407363892
5 2.91791534423828
6 2.94017362594604
7 2.93484354019165
8 2.92113184928894
9 2.91582131385803
10 2.90966033935547
11 2.90915584564209
12 2.93164014816284
13 2.9291672706604
14 2.93177127838135
15 2.9377429485321
16 2.92440176010132
17 2.90448117256165
18 2.89795994758606
19 2.90656328201294
20 2.89178919792175
21 2.88842511177063
22 2.8833441734314
23 2.88567924499512
24 2.88421702384949
25 2.86837530136108
26 2.86304974555969
27 2.85072565078735
28 2.83561944961548
29 2.81288766860962
30 2.80491971969604
31 2.80019521713257
32 2.79057860374451
33 2.77708458900452
34 2.77161002159119
35 2.76089572906494
36 2.75549530982971
37 2.75108122825623
38 2.74393248558044
39 2.73983979225159
40 2.73968911170959
41 2.71906280517578
42 2.71385407447815
43 2.70260667800903
44 2.69486737251282
45 2.68814778327942
46 2.67811107635498
47 2.66219115257263
48 2.65316843986511
49 2.63775539398193
50 2.62820887565613
51 2.61667561531067
52 2.59504413604736
53 2.56479501724243
54 2.5414617061615
55 2.51731944084167
56 2.48859477043152
57 2.47150158882141
58 2.4601194858551
59 2.44597291946411
60 2.42486810684204
61 2.41784381866455
62 2.38381242752075
63 2.36618661880493
64 2.33957386016846
65 2.31422448158264
66 2.293781042099
67 2.28141236305237
68 2.25822496414185
69 2.22637152671814
70 2.20401859283447
71 2.19246745109558
72 2.17546319961548
73 2.14603281021118
74 2.13197207450867
75 2.1067214012146
76 2.08929038047791
77 2.06781506538391
78 2.0505952835083
79 2.03631734848022
80 2.02124261856079
81 1.99009156227112
82 1.97300684452057
83 1.96830916404724
84 1.9467511177063
85 1.92063617706299
86 1.90506744384766
87 1.8828399181366
88 1.86664152145386
89 1.84718430042267
90 1.81434524059296
91 1.80884647369385
92 1.79168093204498
93 1.77702438831329
94 1.76470839977264
95 1.74432301521301
96 1.73165202140808
97 1.71900761127472
98 1.69744038581848
99 1.68481707572937
100 1.66607117652893
101 1.64464831352234
102 1.62887728214264
103 1.61254274845123
104 1.59253633022308
105 1.58305835723877
106 1.57423138618469
107 1.54793870449066
108 1.54902160167694
109 1.52643692493439
110 1.51976180076599
111 1.49059784412384
112 1.47877168655396
113 1.46735596656799
114 1.45642328262329
115 1.45090329647064
116 1.44444751739502
117 1.41890072822571
118 1.41784012317657
119 1.40710556507111
120 1.40460479259491
121 1.38372373580933
122 1.37576186656952
123 1.38330495357513
124 1.37287402153015
125 1.37349903583527
126 1.36556303501129
127 1.36192631721497
128 1.34982299804688
129 1.34000682830811
130 1.3188157081604
131 1.33114778995514
132 1.31786441802979
133 1.30278432369232
134 1.29386484622955
135 1.29880237579346
136 1.28509247303009
137 1.27753400802612
138 1.27000689506531
139 1.2566009759903
140 1.24951314926147
141 1.2500102519989
142 1.24531590938568
143 1.24406516551971
144 1.23881757259369
145 1.22403585910797
146 1.21362042427063
147 1.21881532669067
148 1.23347353935242
149 1.24104928970337
150 1.24059998989105
151 1.24412798881531
152 1.2391105890274
153 1.23669910430908
154 1.24290990829468
155 1.23515164852142
156 1.23544025421143
157 1.23799848556519
158 1.2108496427536
159 1.22142910957336
160 1.20918989181519
161 1.21167385578156
162 1.21301913261414
163 1.19919157028198
164 1.19103956222534
165 1.18391072750092
166 1.18507599830627
167 1.19104468822479
168 1.17656290531158
169 1.16591691970825
170 1.1582270860672
171 1.14945149421692
172 1.14853596687317
173 1.13185107707977
174 1.11956357955933
175 1.12930023670197
176 1.12786412239075
177 1.11359083652496
178 1.12289440631866
179 1.12241172790527
180 1.13353991508484
181 1.11840188503265
182 1.1233811378479
183 1.1113703250885
184 1.11149799823761
185 1.10493755340576
186 1.09864711761475
187 1.08977127075195
188 1.09042620658875
189 1.09211874008179
190 1.08859014511108
191 1.09213006496429
192 1.07617783546448
193 1.07964217662811
194 1.07165575027466
195 1.07958197593689
196 1.07555758953094
197 1.05989694595337
198 1.03665316104889
199 1.01261568069458
200 1.01418352127075
201 1.01940548419952
202 1.02713096141815
203 1.03565180301666
204 1.03759121894836
205 1.0345733165741
206 1.03325700759888
207 1.02800977230072
208 1.02119994163513
209 1.00319349765778
210 1.00515520572662
211 1.01232290267944
212 1.00610184669495
213 1.00893878936768
214 1.02900040149689
215 1.03221654891968
216 1.0152063369751
217 1.02231991291046
218 1.02085208892822
219 1.03359723091125
220 1.04075622558594
221 1.0386415719986
222 1.02795886993408
223 1.02682161331177
224 1.02036070823669
225 1.00030064582825
226 0.987675547599792
227 0.988868355751038
228 0.981900095939636
229 0.975939512252808
230 0.95805811882019
231 0.943461835384369
232 0.951706290245056
233 0.962754130363464
234 0.977528810501099
235 0.968666911125183
236 0.968289852142334
237 0.976908802986145
238 0.967238366603851
239 0.976558983325958
240 0.991234600543976
241 0.969929754734039
242 0.985602498054504
243 0.965146481990814
244 0.96525251865387
245 0.959702432155609
246 0.960596919059753
247 0.955322265625
248 0.96356612443924
249 0.969794511795044
250 0.956980347633362
251 0.951214075088501
252 0.940254151821136
253 0.927924811840057
254 0.917701721191406
255 0.909317910671234
256 0.898667931556702
257 0.909234642982483
258 0.903371214866638
259 0.898175477981567
260 0.892753064632416
261 0.869247794151306
262 0.866957247257233
263 0.85139524936676
264 0.82480001449585
265 0.8265580534935
266 0.824792563915253
267 0.819974839687347
268 0.813243865966797
269 0.814299464225769
270 0.797528207302094
271 0.801401674747467
272 0.795710742473602
273 0.796040177345276
274 0.78901594877243
275 0.777634501457214
276 0.775063633918762
277 0.77012300491333
278 0.753753483295441
279 0.765035450458527
280 0.767096757888794
281 0.776657998561859
282 0.763814210891724
283 0.745123744010925
284 0.734030783176422
285 0.746466457843781
286 0.741751194000244
287 0.729719161987305
288 0.725075304508209
289 0.715401887893677
290 0.710131704807281
291 0.704166412353516
292 0.692008435726166
293 0.705504298210144
294 0.697940230369568
295 0.693825662136078
296 0.696765959262848
297 0.698218107223511
298 0.696668326854706
299 0.689159631729126
300 0.694761753082275
301 0.693408906459808
302 0.683873534202576
303 0.683524072170258
304 0.678232550621033
305 0.677457511425018
306 0.679401695728302
307 0.66728401184082
308 0.672220468521118
309 0.678399980068207
310 0.666331350803375
311 0.675841391086578
312 0.667498111724854
313 0.683808386325836
314 0.685750663280487
315 0.674798369407654
316 0.680722653865814
317 0.671781301498413
318 0.67503023147583
319 0.672611773014069
320 0.664632439613342
321 0.657027661800385
322 0.655264973640442
323 0.648194074630737
324 0.647404909133911
325 0.654576778411865
326 0.660909056663513
327 0.657030880451202
328 0.670163094997406
329 0.665215730667114
330 0.668345093727112
331 0.662778079509735
332 0.658287763595581
333 0.65514725446701
334 0.639684438705444
335 0.634044110774994
336 0.627147078514099
337 0.625668346881866
338 0.628358423709869
339 0.620725154876709
340 0.611018896102905
341 0.613542497158051
342 0.610411107540131
343 0.590672373771667
344 0.587502121925354
345 0.585266411304474
346 0.57744425535202
347 0.576790988445282
348 0.564848065376282
349 0.562713384628296
350 0.568242430686951
351 0.567556858062744
352 0.566457509994507
353 0.55884450674057
354 0.582909345626831
355 0.580630362033844
356 0.583325982093811
357 0.582602381706238
358 0.575695276260376
359 0.572914958000183
360 0.577572226524353
361 0.566410183906555
362 0.57895040512085
363 0.5682093501091
364 0.562845468521118
365 0.561423063278198
366 0.556996643543243
367 0.568258047103882
368 0.563437402248383
369 0.560743868350983
370 0.563634097576141
371 0.57193922996521
372 0.569201707839966
373 0.570289433002472
374 0.577151358127594
375 0.575428247451782
376 0.573058009147644
377 0.584021389484406
378 0.572924196720123
379 0.562016665935516
380 0.558976888656616
381 0.564875960350037
382 0.555567681789398
383 0.564675271511078
384 0.57198840379715
385 0.567412197589874
386 0.575898945331573
387 0.579379081726074
388 0.568845272064209
389 0.558307826519012
390 0.558379650115967
391 0.56427937746048
392 0.558204591274261
393 0.558468997478485
394 0.558579325675964
395 0.553581357002258
396 0.542885720729828
397 0.53716242313385
398 0.531862735748291
399 0.557475209236145
400 0.543539643287659
401 0.532419860363007
402 0.536534488201141
403 0.545560240745544
404 0.521053731441498
405 0.520885825157166
406 0.520542025566101
407 0.519568800926208
408 0.526644110679626
409 0.52539199590683
410 0.526282548904419
411 0.524179518222809
412 0.523144841194153
413 0.526824057102203
414 0.531147718429565
415 0.520941972732544
416 0.518409550189972
417 0.501878559589386
418 0.509727537631989
419 0.496408730745316
420 0.497467130422592
421 0.499977231025696
422 0.529457867145538
423 0.521331548690796
424 0.516798913478851
425 0.522356033325195
426 0.518022179603577
427 0.534804701805115
428 0.543055593967438
429 0.554401457309723
430 0.554489433765411
431 0.552714943885803
432 0.552460908889771
433 0.549288153648376
434 0.549603819847107
435 0.544115543365479
436 0.53734827041626
437 0.538735568523407
438 0.541299343109131
439 0.548104465007782
440 0.547117531299591
441 0.537801027297974
442 0.53787636756897
443 0.545280575752258
444 0.543665111064911
445 0.555971264839172
446 0.561529099941254
447 0.561853229999542
448 0.578320682048798
449 0.560425102710724
450 0.559406340122223
451 0.558175265789032
452 0.561510264873505
453 0.561092674732208
454 0.560914695262909
455 0.570018410682678
456 0.56483668088913
457 0.565229058265686
458 0.565376758575439
459 0.557288885116577
460 0.556912183761597
461 0.558433592319489
462 0.555881798267365
463 0.547000408172607
464 0.545860409736633
465 0.545956432819366
466 0.551421403884888
467 0.561468124389648
468 0.55274224281311
469 0.549915254116058
470 0.555827140808105
471 0.540501773357391
472 0.509048223495483
473 0.515046060085297
474 0.519398391246796
475 0.510223567485809
476 0.503298103809357
477 0.475067973136902
478 0.46527835726738
479 0.448936820030212
480 0.448824971914291
481 0.441882818937302
482 0.447440028190613
483 0.444903254508972
484 0.431850612163544
485 0.433641403913498
486 0.429723471403122
487 0.425101786851883
488 0.42957279086113
489 0.438053697347641
490 0.431285500526428
491 0.444652855396271
492 0.443532139062881
493 0.438658535480499
494 0.437042206525803
495 0.435852080583572
496 0.433506846427917
497 0.443060219287872
498 0.427663564682007
499 0.421972662210464
500 0.420177042484283
501 0.433098793029785
502 0.433424264192581
503 0.448688060045242
504 0.452911972999573
505 0.45700278878212
506 0.45426943898201
507 0.45862877368927
508 0.458448231220245
509 0.464954137802124
510 0.463639348745346
511 0.466135144233704
512 0.462708503007889
513 0.464361637830734
514 0.465253710746765
515 0.465032935142517
516 0.455877840518951
517 0.456689387559891
518 0.458966642618179
519 0.467495918273926
520 0.461790055036545
521 0.469905436038971
522 0.472016870975494
523 0.47276359796524
524 0.464110285043716
525 0.467842400074005
526 0.464954406023026
527 0.469577938318253
528 0.469541847705841
529 0.470114171504974
530 0.466095954179764
531 0.47483828663826
532 0.470255434513092
533 0.469545900821686
534 0.478378146886826
535 0.473794162273407
536 0.490935504436493
537 0.483983844518661
538 0.478097796440125
539 0.482729494571686
540 0.483299791812897
541 0.477235823869705
542 0.479847401380539
543 0.477463990449905
544 0.481468886137009
545 0.486140102148056
546 0.481934577226639
547 0.476261019706726
548 0.473969876766205
549 0.472931057214737
550 0.474122494459152
551 0.462436109781265
552 0.45628410577774
553 0.437420189380646
554 0.432967066764832
555 0.421336263418198
556 0.414498418569565
557 0.410012811422348
558 0.407656848430634
559 0.405756592750549
560 0.40166974067688
561 0.39878261089325
562 0.395860224962234
563 0.392816454172134
564 0.404204219579697
565 0.408640325069427
566 0.420007675886154
567 0.404885649681091
568 0.405716627836227
569 0.403450965881348
570 0.399945288896561
571 0.395257711410522
572 0.413751840591431
573 0.417116850614548
574 0.420087397098541
575 0.422352135181427
576 0.427765965461731
577 0.427185595035553
578 0.424268007278442
579 0.438189625740051
580 0.449998319149017
581 0.449910253286362
582 0.444164425134659
583 0.438795208930969
584 0.432350128889084
585 0.440414011478424
586 0.427054762840271
587 0.435325890779495
588 0.442912071943283
589 0.433926999568939
590 0.44232514500618
591 0.434420555830002
592 0.431045353412628
593 0.449607968330383
594 0.457052201032639
595 0.441492050886154
596 0.454688787460327
597 0.460451275110245
598 0.466792345046997
599 0.464778184890747
600 0.458458483219147
601 0.466767907142639
602 0.467339158058167
603 0.468598961830139
604 0.470272868871689
605 0.470150530338287
606 0.472177922725677
607 0.470319360494614
608 0.472667813301086
609 0.475111782550812
610 0.475508689880371
611 0.481533735990524
612 0.479370206594467
613 0.486349016427994
614 0.479027569293976
615 0.476176261901855
616 0.47457230091095
617 0.47584941983223
618 0.476508438587189
619 0.486209332942963
620 0.495712429285049
621 0.493567049503326
622 0.473845243453979
623 0.465346574783325
624 0.465786904096603
625 0.469521284103394
626 0.469069063663483
627 0.467380881309509
628 0.467991560697556
629 0.456183940172195
630 0.442669063806534
631 0.436127662658691
632 0.441261202096939
633 0.442399471998215
634 0.449723690748215
635 0.443434834480286
636 0.451428711414337
637 0.448508858680725
638 0.450288951396942
639 0.439941823482513
640 0.444584280252457
641 0.450215309858322
642 0.457314878702164
643 0.437754452228546
644 0.421808242797852
645 0.430491030216217
646 0.419380068778992
647 0.411733031272888
648 0.41211986541748
649 0.411808580160141
650 0.414199471473694
651 0.40126359462738
652 0.398495823144913
653 0.395738631486893
654 0.411341398954391
655 0.40768700838089
656 0.405396044254303
657 0.402108073234558
658 0.406705766916275
659 0.40708926320076
660 0.406226068735123
661 0.399336993694305
662 0.400813817977905
663 0.396057099103928
664 0.391725957393646
665 0.391180068254471
666 0.381281316280365
667 0.387635916471481
668 0.382114380598068
669 0.38231098651886
670 0.377467066049576
671 0.378955572843552
672 0.385310709476471
673 0.379791140556335
674 0.376053214073181
675 0.364396810531616
676 0.362032741308212
677 0.362285763025284
678 0.385667949914932
679 0.387279540300369
680 0.383970260620117
681 0.387170642614365
682 0.386412024497986
683 0.390785664319992
684 0.379496961832047
685 0.387407183647156
686 0.380515813827515
687 0.378718256950378
688 0.375237643718719
689 0.378292202949524
690 0.367282956838608
691 0.364103734493256
692 0.367003321647644
693 0.364121347665787
694 0.367255389690399
695 0.357270151376724
696 0.357971966266632
697 0.353905558586121
698 0.352483987808228
699 0.358302235603333
700 0.363921314477921
701 0.36850517988205
702 0.367025882005692
703 0.376500248908997
704 0.356507331132889
705 0.354752719402313
706 0.359489589929581
707 0.361483037471771
708 0.357482939958572
709 0.362527430057526
710 0.365177750587463
711 0.363695293664932
712 0.370322108268738
713 0.374608159065247
714 0.371468991041183
715 0.370676010847092
716 0.367061376571655
717 0.369642853736877
718 0.38134303689003
719 0.373073011636734
720 0.37427169084549
721 0.370097160339355
722 0.362978219985962
723 0.366087049245834
724 0.365727037191391
725 0.364725083112717
726 0.364325821399689
727 0.369614750146866
728 0.34482353925705
729 0.343071907758713
730 0.347481161355972
731 0.341488480567932
732 0.345005005598068
733 0.340445280075073
734 0.344577938318253
735 0.342129439115524
736 0.337499886751175
737 0.341989487409592
738 0.334447503089905
739 0.335204988718033
740 0.341419726610184
741 0.341765850782394
742 0.332983106374741
743 0.335510671138763
744 0.331741839647293
745 0.333115428686142
746 0.341685473918915
747 0.339587867259979
748 0.340272337198257
749 0.334843039512634
750 0.331101536750793
751 0.324374616146088
752 0.326713740825653
753 0.315315544605255
754 0.314667612314224
755 0.314284563064575
756 0.314301520586014
757 0.316872268915176
758 0.310293257236481
759 0.301120817661285
760 0.308309555053711
761 0.322708666324615
762 0.316129177808762
763 0.318277657032013
764 0.314563572406769
765 0.331573218107224
766 0.341425746679306
767 0.334380924701691
768 0.322978466749191
769 0.317512810230255
770 0.317860126495361
771 0.317545771598816
772 0.313410848379135
773 0.328127980232239
774 0.328190088272095
775 0.331137537956238
776 0.333084791898727
777 0.342555016279221
778 0.342766851186752
779 0.338147729635239
780 0.340406537055969
781 0.340148955583572
782 0.33227550983429
783 0.333343684673309
784 0.333283185958862
785 0.326847344636917
786 0.332598805427551
787 0.325381368398666
788 0.321858137845993
789 0.317432701587677
790 0.306421399116516
791 0.302934885025024
792 0.309146076440811
793 0.309623688459396
794 0.312190562486649
795 0.310393840074539
796 0.299192994832993
797 0.303856492042542
798 0.298848330974579
799 0.296812295913696
800 0.292376548051834
801 0.300716906785965
802 0.296622633934021
803 0.293275654315948
804 0.29060685634613
805 0.292516380548477
806 0.296693801879883
807 0.306417137384415
808 0.306267380714417
809 0.308419167995453
810 0.309816122055054
811 0.296004086732864
812 0.293676525354385
813 0.296421080827713
814 0.300777971744537
815 0.28930202126503
816 0.280271291732788
817 0.274272322654724
818 0.277120977640152
819 0.277002781629562
820 0.271779149770737
821 0.271081835031509
822 0.282997488975525
823 0.267547696828842
824 0.266123950481415
825 0.271815270185471
826 0.26720729470253
827 0.252468943595886
828 0.259099215269089
829 0.259643524885178
830 0.255074471235275
831 0.252277880907059
832 0.253267854452133
833 0.253506451845169
834 0.25099390745163
835 0.252036392688751
836 0.245704427361488
837 0.244603291153908
838 0.245359823107719
839 0.248666495084763
840 0.25513482093811
841 0.25166791677475
842 0.245218694210052
843 0.242126524448395
844 0.255446225404739
845 0.258302867412567
846 0.264885872602463
847 0.265716463327408
848 0.267922788858414
849 0.270209133625031
850 0.271718531847
851 0.268951952457428
852 0.278391152620316
853 0.283827990293503
854 0.289761751890182
855 0.288861006498337
856 0.283800035715103
857 0.270242512226105
858 0.279115706682205
859 0.277688354253769
860 0.268577307462692
861 0.266201436519623
862 0.277393728494644
863 0.265979528427124
864 0.276314377784729
865 0.267499059438705
866 0.272753685712814
867 0.28326216340065
868 0.281325697898865
869 0.28791069984436
870 0.286851406097412
871 0.291471272706985
872 0.285652339458466
873 0.283572733402252
874 0.283070266246796
875 0.276905983686447
876 0.284407407045364
877 0.281418025493622
878 0.280593991279602
879 0.281834155321121
880 0.279037535190582
881 0.280415028333664
882 0.280734747648239
883 0.281265676021576
884 0.287917882204056
885 0.28343802690506
886 0.285989016294479
887 0.287603288888931
888 0.293691575527191
889 0.304446041584015
890 0.297540098428726
891 0.2985720038414
892 0.302580386400223
893 0.302464663982391
894 0.285451203584671
895 0.288895726203918
896 0.287767231464386
897 0.286468416452408
898 0.28837451338768
899 0.287611335515976
900 0.290112346410751
901 0.294715046882629
902 0.287021666765213
903 0.282534062862396
904 0.276306539773941
905 0.276849985122681
906 0.271796256303787
907 0.279635667800903
908 0.272264510393143
909 0.270521700382233
910 0.268960386514664
911 0.278320819139481
912 0.269797682762146
913 0.272958695888519
914 0.260265916585922
915 0.262096285820007
916 0.260913461446762
917 0.26263427734375
918 0.258173584938049
919 0.256536036729813
920 0.255402207374573
921 0.251640826463699
922 0.247370928525925
923 0.250090718269348
924 0.251248151063919
925 0.248572334647179
926 0.24482598900795
927 0.244464814662933
928 0.240522980690002
929 0.242929339408875
930 0.245494544506073
931 0.243261903524399
932 0.241419672966003
933 0.241629809141159
934 0.236843511462212
935 0.237471908330917
936 0.237517490983009
937 0.239005282521248
938 0.234700202941895
939 0.223395481705666
940 0.227660298347473
941 0.234876483678818
942 0.226441517472267
943 0.225163131952286
944 0.227961450815201
945 0.220441341400146
946 0.21893909573555
947 0.215659275650978
948 0.212660163640976
949 0.211354121565819
950 0.207645639777184
951 0.195816859602928
952 0.198400571942329
953 0.198823809623718
954 0.198977172374725
955 0.204888761043549
956 0.205719619989395
957 0.200095504522324
958 0.196209818124771
959 0.203141286969185
960 0.207470238208771
961 0.201024010777473
962 0.2054722905159
963 0.206730216741562
964 0.217935785651207
965 0.219201385974884
966 0.21870145201683
967 0.210650533437729
968 0.213897436857224
969 0.217977538704872
970 0.221197217702866
971 0.221515670418739
972 0.232278540730476
973 0.229122430086136
974 0.243444427847862
975 0.25059512257576
976 0.246706336736679
977 0.260897636413574
978 0.270359873771667
979 0.268089145421982
980 0.274798840284348
981 0.274535655975342
982 0.279635012149811
983 0.276799470186234
984 0.277019083499908
985 0.285195171833038
986 0.284667134284973
987 0.29046630859375
988 0.293591618537903
989 0.291362166404724
990 0.296856284141541
991 0.292562514543533
992 0.297801107168198
993 0.301314055919647
994 0.300351530313492
995 0.305239737033844
996 0.317045867443085
997 0.31923896074295
998 0.322387427091599
999 0.320581555366516
1000 0.321430325508118
1001 0.324369698762894
1002 0.336310893297195
1003 0.336337447166443
1004 0.352353513240814
1005 0.345134884119034
1006 0.353631615638733
1007 0.348840683698654
1008 0.357439577579498
1009 0.349040597677231
1010 0.344929575920105
1011 0.344218075275421
1012 0.343518614768982
1013 0.348150104284286
1014 0.337283819913864
1015 0.341545552015305
1016 0.340966254472733
1017 0.341567754745483
1018 0.343420475721359
1019 0.336064338684082
1020 0.33524963259697
1021 0.341935843229294
1022 0.339630484580994
1023 0.344955146312714
1024 0.328683406114578
1025 0.323349416255951
1026 0.322672039270401
1027 0.320443242788315
1028 0.309137791395187
1029 0.310054451227188
1030 0.302664041519165
1031 0.313852727413177
1032 0.319240599870682
1033 0.318136185407639
1034 0.317119956016541
1035 0.313199400901794
1036 0.312913924455643
1037 0.315276712179184
1038 0.310923755168915
1039 0.316324919462204
1040 0.312133818864822
1041 0.308340460062027
1042 0.302642226219177
1043 0.300212860107422
1044 0.304615885019302
1045 0.300890535116196
1046 0.286172926425934
1047 0.286540895700455
1048 0.283185213804245
1049 0.283637404441833
1050 0.283934146165848
1051 0.282294183969498
1052 0.265926450490952
1053 0.263138592243195
1054 0.249341785907745
1055 0.253069430589676
1056 0.246152997016907
1057 0.246915623545647
1058 0.241202056407928
1059 0.246554583311081
1060 0.261067032814026
1061 0.268098533153534
1062 0.263009369373322
1063 0.260259509086609
1064 0.259300798177719
1065 0.253278017044067
1066 0.250909060239792
1067 0.247058168053627
1068 0.241798132658005
1069 0.240989491343498
1070 0.238563269376755
1071 0.237511575222015
1072 0.227390587329865
1073 0.2259172052145
1074 0.232033029198647
1075 0.24093970656395
1076 0.244647487998009
1077 0.239125460386276
1078 0.243022531270981
1079 0.24088653922081
1080 0.240321278572083
1081 0.230736181139946
1082 0.233330100774765
1083 0.233645185828209
1084 0.229200303554535
1085 0.226744517683983
1086 0.226074784994125
1087 0.216433808207512
1088 0.216641485691071
1089 0.210054278373718
1090 0.208622053265572
1091 0.207574039697647
1092 0.209114372730255
1093 0.210907071828842
1094 0.205567538738251
1095 0.204517379403114
1096 0.20297184586525
1097 0.204900920391083
1098 0.20121306180954
1099 0.202894032001495
1100 0.202247053384781
1101 0.202268585562706
1102 0.204625383019447
1103 0.205460026860237
1104 0.214772552251816
1105 0.213214322924614
1106 0.216127797961235
1107 0.229275956749916
1108 0.224563658237457
1109 0.221735611557961
1110 0.206140652298927
1111 0.211957514286041
1112 0.212099760770798
1113 0.206018105149269
1114 0.20968422293663
1115 0.209913983941078
1116 0.211695685982704
1117 0.211328908801079
1118 0.211179211735725
1119 0.21276181936264
1120 0.212409317493439
1121 0.218534663319588
1122 0.218327194452286
1123 0.215600341558456
1124 0.211815014481544
1125 0.200975432991982
1126 0.230140432715416
1127 0.222796306014061
1128 0.217564776539803
1129 0.216204628348351
1130 0.220564991235733
1131 0.220921516418457
1132 0.209394544363022
1133 0.210892602801323
1134 0.212774932384491
1135 0.210222721099854
1136 0.210747301578522
1137 0.212077751755714
1138 0.219882875680923
1139 0.225686475634575
1140 0.229255676269531
1141 0.231160029768944
1142 0.233509764075279
1143 0.235130697488785
1144 0.243353098630905
1145 0.247434481978416
1146 0.250445634126663
1147 0.248190581798553
1148 0.24930901825428
1149 0.249835968017578
1150 0.254392027854919
1151 0.264346152544022
1152 0.272270113229752
1153 0.272788345813751
1154 0.263200402259827
1155 0.269933104515076
1156 0.271543622016907
1157 0.261707931756973
1158 0.268400102853775
1159 0.267502635717392
1160 0.270208925008774
1161 0.261174559593201
1162 0.260784536600113
1163 0.268817037343979
1164 0.267142713069916
1165 0.271066784858704
1166 0.269121527671814
1167 0.271849185228348
1168 0.273959398269653
1169 0.269364953041077
1170 0.272175312042236
1171 0.262820422649384
1172 0.261985272169113
1173 0.264150857925415
1174 0.26496422290802
1175 0.267777472734451
1176 0.236048772931099
1177 0.242433965206146
1178 0.249649375677109
1179 0.251047819852829
1180 0.249307587742805
1181 0.25564780831337
1182 0.254352480173111
1183 0.25397002696991
1184 0.260536342859268
1185 0.273680567741394
1186 0.273896247148514
1187 0.273785173892975
1188 0.264718770980835
1189 0.260283291339874
1190 0.252244889736176
1191 0.256597906351089
1192 0.256404489278793
1193 0.27394226193428
1194 0.267926186323166
1195 0.266623824834824
1196 0.270900249481201
1197 0.270013093948364
1198 0.269901484251022
1199 0.270395219326019
1200 0.270420879125595
1201 0.258996844291687
1202 0.255081087350845
1203 0.254302829504013
1204 0.254756242036819
1205 0.246071219444275
1206 0.240121752023697
1207 0.23727211356163
1208 0.232361525297165
1209 0.228591620922089
1210 0.227341175079346
1211 0.225697115063667
1212 0.223288714885712
1213 0.220473945140839
1214 0.217337012290955
1215 0.210953369736671
1216 0.209515437483788
1217 0.208810716867447
1218 0.208633780479431
1219 0.209041193127632
1220 0.213857084512711
1221 0.209362551569939
1222 0.208073392510414
1223 0.203855365514755
1224 0.203906610608101
1225 0.202974930405617
1226 0.202904433012009
1227 0.198479175567627
1228 0.193796426057816
1229 0.192341864109039
1230 0.188092857599258
1231 0.179190784692764
1232 0.180566996335983
1233 0.179591998457909
1234 0.170534685254097
1235 0.159582242369652
1236 0.160688489675522
1237 0.160161241889
1238 0.15999610722065
1239 0.162133976817131
};
\addlegendentry{4-5}
\addplot [semithick, color0]
table {%
0 4.32436084747314
1 3.93828535079956
2 3.86521077156067
3 3.61764430999756
4 3.50971937179565
5 3.41278862953186
6 3.28259587287903
7 3.29786109924316
8 3.29175567626953
9 3.22691297531128
10 3.20382714271545
11 3.20681095123291
12 3.21549606323242
13 3.19032263755798
14 3.17570877075195
15 3.17224192619324
16 3.15745878219604
17 3.14927673339844
18 3.15190386772156
19 3.13440251350403
20 3.12832498550415
21 3.13129043579102
22 3.12948036193848
23 3.12632179260254
24 3.11628484725952
25 3.10948443412781
26 3.11896800994873
27 3.11217737197876
28 3.10688042640686
29 3.10273432731628
30 3.10000491142273
31 3.09025120735168
32 3.08406901359558
33 3.08047413825989
34 3.07213425636292
35 3.058349609375
36 3.06546401977539
37 3.05197596549988
38 3.04472279548645
39 3.03483963012695
40 3.01758360862732
41 3.01059699058533
42 3.01388311386108
43 3.00063729286194
44 2.99952220916748
45 2.99022912979126
46 2.97143983840942
47 2.9678955078125
48 2.94875764846802
49 2.93824362754822
50 2.89975833892822
51 2.87991237640381
52 2.86235380172729
53 2.84305262565613
54 2.81758451461792
55 2.80766844749451
56 2.80655956268311
57 2.78252816200256
58 2.75753498077393
59 2.74235510826111
60 2.72225880622864
61 2.70496082305908
62 2.68639254570007
63 2.67078471183777
64 2.64158153533936
65 2.62247610092163
66 2.60535192489624
67 2.58639883995056
68 2.56226420402527
69 2.5457706451416
70 2.53241276741028
71 2.52754211425781
72 2.50951862335205
73 2.49662041664124
74 2.48680520057678
75 2.47939705848694
76 2.46573781967163
77 2.4566855430603
78 2.43720483779907
79 2.42165231704712
80 2.40287256240845
81 2.38750910758972
82 2.36765623092651
83 2.34898829460144
84 2.32667255401611
85 2.31068849563599
86 2.27534437179565
87 2.26464676856995
88 2.25372982025146
89 2.22529053688049
90 2.21351003646851
91 2.18715500831604
92 2.16662883758545
93 2.14837431907654
94 2.12905478477478
95 2.10775232315063
96 2.09834909439087
97 2.07803869247437
98 2.07442235946655
99 2.08122563362122
100 2.06522607803345
101 2.06094217300415
102 2.04185509681702
103 2.05133128166199
104 2.05515098571777
105 2.04005527496338
106 2.02296900749207
107 2.01499152183533
108 2.00687456130981
109 1.99515807628632
110 1.99712765216827
111 1.97818517684937
112 1.94404363632202
113 1.92640852928162
114 1.92367374897003
115 1.90854227542877
116 1.89708364009857
117 1.89248132705688
118 1.89999556541443
119 1.90402007102966
120 1.88675534725189
121 1.86431229114532
122 1.86316537857056
123 1.84923160076141
124 1.83734095096588
125 1.81630718708038
126 1.80031383037567
127 1.78175616264343
128 1.78034639358521
129 1.76560831069946
130 1.75963628292084
131 1.74434041976929
132 1.74050354957581
133 1.72148656845093
134 1.70394313335419
135 1.70356225967407
136 1.70584762096405
137 1.6976306438446
138 1.68318390846252
139 1.69536817073822
140 1.68671655654907
141 1.69394361972809
142 1.68019068241119
143 1.68335521221161
144 1.65715742111206
145 1.6625953912735
146 1.64823698997498
147 1.64379012584686
148 1.63903141021729
149 1.59932339191437
150 1.59152007102966
151 1.56587553024292
152 1.54703283309937
153 1.52688479423523
154 1.52582788467407
155 1.50739109516144
156 1.5033712387085
157 1.48938369750977
158 1.4858729839325
159 1.4733966588974
160 1.45901799201965
161 1.46450531482697
162 1.47545409202576
163 1.46655762195587
164 1.45993220806122
165 1.46465826034546
166 1.4637953042984
167 1.45578122138977
168 1.43230938911438
169 1.4091465473175
170 1.39957594871521
171 1.39296388626099
172 1.37193369865417
173 1.36537432670593
174 1.35605847835541
175 1.35597324371338
176 1.33730208873749
177 1.32342851161957
178 1.3158472776413
179 1.30433583259583
180 1.28459072113037
181 1.29446506500244
182 1.29462993144989
183 1.30016589164734
184 1.30942368507385
185 1.30134510993958
186 1.28679180145264
187 1.27121365070343
188 1.2539154291153
189 1.23359727859497
190 1.22343492507935
191 1.20780789852142
192 1.21135401725769
193 1.21330499649048
194 1.23407173156738
195 1.22323143482208
196 1.22582030296326
197 1.21823680400848
198 1.20633006095886
199 1.21048200130463
200 1.23285317420959
201 1.24935102462769
202 1.26290464401245
203 1.26068210601807
204 1.24550020694733
205 1.2595294713974
206 1.24492800235748
207 1.26061129570007
208 1.25772535800934
209 1.26812934875488
210 1.25342583656311
211 1.23202776908875
212 1.23247921466827
213 1.23343729972839
214 1.23086667060852
215 1.23409509658813
216 1.23206496238708
217 1.22296464443207
218 1.22497403621674
219 1.22385632991791
220 1.22895681858063
221 1.21241402626038
222 1.22441470623016
223 1.21245288848877
224 1.22311389446259
225 1.20960998535156
226 1.21341133117676
227 1.21944797039032
228 1.22315990924835
229 1.22984075546265
230 1.22568476200104
231 1.23124277591705
232 1.22150659561157
233 1.22480630874634
234 1.21243250370026
235 1.20672273635864
236 1.21161270141602
237 1.20290994644165
238 1.19673347473145
239 1.21031939983368
240 1.20501065254211
241 1.21557307243347
242 1.22046875953674
243 1.19753837585449
244 1.18000149726868
245 1.18094265460968
246 1.18398869037628
247 1.18078136444092
248 1.1806708574295
249 1.18210780620575
250 1.16104865074158
251 1.13401448726654
252 1.12716817855835
253 1.10988402366638
254 1.11262810230255
255 1.10982429981232
256 1.11969578266144
257 1.10703194141388
258 1.10112833976746
259 1.09362292289734
260 1.09556841850281
261 1.11280965805054
262 1.11203408241272
263 1.10856318473816
264 1.10814154148102
265 1.08506095409393
266 1.07006311416626
267 1.07571005821228
268 1.06886303424835
269 1.06667149066925
270 1.06743812561035
271 1.06732296943665
272 1.04712903499603
273 1.04679369926453
274 1.02802419662476
275 1.02752351760864
276 1.0320143699646
277 1.0244163274765
278 1.01537501811981
279 1.00370013713837
280 1.01971197128296
281 1.00443398952484
282 0.992978513240814
283 0.98851203918457
284 0.986077189445496
285 0.991186439990997
286 0.990309298038483
287 0.999189019203186
288 1.01419496536255
289 1.0028680562973
290 1.01105272769928
291 0.993866801261902
292 0.974079966545105
293 0.988656222820282
294 1.00380623340607
295 1.01571750640869
296 1.00381076335907
297 0.998522698879242
298 1.00679779052734
299 0.997916460037231
300 0.991004467010498
301 0.990844547748566
302 0.975879907608032
303 0.975704908370972
304 0.958229660987854
305 0.939976215362549
306 0.930448591709137
307 0.919495642185211
308 0.911844730377197
309 0.91618013381958
310 0.911237776279449
311 0.904591619968414
312 0.889284312725067
313 0.885103225708008
314 0.88383275270462
315 0.89067155122757
316 0.90614914894104
317 0.893259227275848
318 0.887393653392792
319 0.883138716220856
320 0.870128691196442
321 0.867881715297699
322 0.881187498569489
323 0.878512799739838
324 0.875963807106018
325 0.866550922393799
326 0.85930198431015
327 0.874060750007629
328 0.887661337852478
329 0.892142057418823
330 0.894614994525909
331 0.895983576774597
332 0.895161271095276
333 0.886476755142212
334 0.88586950302124
335 0.882128596305847
336 0.875570774078369
337 0.873137354850769
338 0.863704681396484
339 0.869528353214264
340 0.872587144374847
341 0.866730213165283
342 0.868760347366333
343 0.859192967414856
344 0.83998441696167
345 0.821004033088684
346 0.837583541870117
347 0.840828716754913
348 0.826452314853668
349 0.829930901527405
350 0.835344612598419
351 0.837771475315094
352 0.84784871339798
353 0.877181768417358
354 0.881841480731964
355 0.886438190937042
356 0.895219504833221
357 0.896081209182739
358 0.906446218490601
359 0.900347292423248
360 0.902207612991333
361 0.893646419048309
362 0.919306874275208
363 0.916751682758331
364 0.92010110616684
365 0.928577899932861
366 0.914822161197662
367 0.917143046855927
368 0.912083804607391
369 0.914627552032471
370 0.925984621047974
371 0.931846976280212
372 0.925500929355621
373 0.929390847682953
374 0.928160846233368
375 0.942864298820496
376 0.935057044029236
377 0.916571021080017
378 0.893900811672211
379 0.898883819580078
380 0.883271455764771
381 0.875908970832825
382 0.886713743209839
383 0.879848539829254
384 0.890858888626099
385 0.88024115562439
386 0.877873301506042
387 0.874533891677856
388 0.872593820095062
389 0.861987471580505
390 0.85866391658783
391 0.863450407981873
392 0.869705140590668
393 0.86180567741394
394 0.88382226228714
395 0.879257202148438
396 0.859145283699036
397 0.852520346641541
398 0.860540211200714
399 0.863038122653961
400 0.848823547363281
401 0.856457531452179
402 0.852920770645142
403 0.822342216968536
404 0.827133178710938
405 0.828649520874023
406 0.825636684894562
407 0.819645524024963
408 0.807197272777557
409 0.819022059440613
410 0.826749622821808
411 0.827429711818695
412 0.798907339572906
413 0.818274974822998
414 0.804985523223877
415 0.799819111824036
416 0.799619317054749
417 0.809372782707214
418 0.82096916437149
419 0.834146976470947
420 0.830106556415558
421 0.83394593000412
422 0.823239684104919
423 0.844119131565094
424 0.845398485660553
425 0.842697322368622
426 0.845244765281677
427 0.865061938762665
428 0.862782716751099
429 0.858362674713135
430 0.85526168346405
431 0.850125551223755
432 0.83691793680191
433 0.840235233306885
434 0.828659057617188
435 0.841291427612305
436 0.834851324558258
437 0.82648777961731
438 0.826001346111298
439 0.820238053798676
440 0.81071013212204
441 0.836822807788849
442 0.81454598903656
443 0.817196249961853
444 0.800109922885895
445 0.803349018096924
446 0.815687954425812
447 0.814990937709808
448 0.816057980060577
449 0.810615003108978
450 0.814124524593353
451 0.805984675884247
452 0.797118663787842
453 0.811641871929169
454 0.801814436912537
455 0.793738186359406
456 0.780067920684814
457 0.774208426475525
458 0.775902807712555
459 0.754818499088287
460 0.742937326431274
461 0.743462920188904
462 0.755250155925751
463 0.746726810932159
464 0.754533052444458
465 0.753097474575043
466 0.749465882778168
467 0.730903565883636
468 0.7147536277771
469 0.701813340187073
470 0.697221040725708
471 0.693693995475769
472 0.697370290756226
473 0.671178340911865
474 0.674887418746948
475 0.671399295330048
476 0.687350153923035
477 0.674964308738708
478 0.672580540180206
479 0.674121022224426
480 0.672230958938599
481 0.672479927539825
482 0.675803661346436
483 0.674327552318573
484 0.671665132045746
485 0.660494923591614
486 0.656984865665436
487 0.674145102500916
488 0.684811949729919
489 0.695984780788422
490 0.702621459960938
491 0.690747499465942
492 0.705999314785004
493 0.703054904937744
494 0.699734508991241
495 0.700312495231628
496 0.698921978473663
497 0.700286328792572
498 0.688042759895325
499 0.678890824317932
500 0.686698317527771
501 0.686891317367554
502 0.684338927268982
503 0.6710324883461
504 0.686802804470062
505 0.69287770986557
506 0.700315475463867
507 0.708364307880402
508 0.724888443946838
509 0.737965226173401
510 0.736933052539825
511 0.758136689662933
512 0.74981963634491
513 0.743754029273987
514 0.736428737640381
515 0.729298174381256
516 0.727186441421509
517 0.729643940925598
518 0.73699277639389
519 0.726666510105133
520 0.717598676681519
521 0.7122722864151
522 0.705782830715179
523 0.697019755840302
524 0.687728643417358
525 0.679506242275238
526 0.659769892692566
527 0.649886846542358
528 0.648022592067719
529 0.649222493171692
530 0.643494129180908
531 0.643532156944275
532 0.64289402961731
533 0.666905283927917
534 0.675813138484955
535 0.667280793190002
536 0.680776536464691
537 0.669408738613129
538 0.666018843650818
539 0.662063598632812
540 0.656406223773956
541 0.634541451931
542 0.622009336948395
543 0.621220350265503
544 0.626796126365662
545 0.62408047914505
546 0.614111959934235
547 0.617369830608368
548 0.621234536170959
549 0.630360305309296
550 0.624240696430206
551 0.629437983036041
552 0.635275483131409
553 0.630449771881104
554 0.616627752780914
555 0.614097833633423
556 0.604860782623291
557 0.600711226463318
558 0.583170115947723
559 0.572312474250793
560 0.569989860057831
561 0.537309467792511
562 0.540709972381592
563 0.542764127254486
564 0.550848484039307
565 0.548071920871735
566 0.554859459400177
567 0.558203876018524
568 0.565332055091858
569 0.571553528308868
570 0.583487689495087
571 0.59218442440033
572 0.597634196281433
573 0.604292333126068
574 0.606614470481873
575 0.607666611671448
576 0.601982831954956
577 0.596391499042511
578 0.603200733661652
579 0.600111126899719
580 0.602958917617798
581 0.59186863899231
582 0.587769150733948
583 0.562035918235779
584 0.551623106002808
585 0.558540284633636
586 0.54601377248764
587 0.546416401863098
588 0.552244484424591
589 0.54449599981308
590 0.539152443408966
591 0.548231482505798
592 0.55480682849884
593 0.575376331806183
594 0.570701360702515
595 0.572361469268799
596 0.568530797958374
597 0.566851615905762
598 0.571131885051727
599 0.567280769348145
600 0.558659732341766
601 0.555838465690613
602 0.551654040813446
603 0.560782253742218
604 0.554884135723114
605 0.563889861106873
606 0.582661867141724
607 0.58278501033783
608 0.582101404666901
609 0.580959141254425
610 0.588202834129333
611 0.587977766990662
612 0.595424890518188
613 0.586766481399536
614 0.582302749156952
615 0.577879548072815
616 0.582161545753479
617 0.575257539749146
618 0.569544613361359
619 0.570314168930054
620 0.573902368545532
621 0.565642118453979
622 0.56596302986145
623 0.568767488002777
624 0.565443515777588
625 0.567052125930786
626 0.574113368988037
627 0.595101833343506
628 0.591833174228668
629 0.581981301307678
630 0.587809681892395
631 0.595582723617554
632 0.60473507642746
633 0.607420444488525
634 0.607970833778381
635 0.610159277915955
636 0.613376557826996
637 0.620403349399567
638 0.600656867027283
639 0.601860165596008
640 0.613784968852997
641 0.60339617729187
642 0.607855021953583
643 0.588117778301239
644 0.582764685153961
645 0.575228154659271
646 0.576821327209473
647 0.565888583660126
648 0.556169450283051
649 0.553969204425812
650 0.55528461933136
651 0.544800698757172
652 0.540684580802917
653 0.534149706363678
654 0.539557635784149
655 0.528273940086365
656 0.511005699634552
657 0.508201718330383
658 0.509496450424194
659 0.509371042251587
660 0.518760502338409
661 0.523351907730103
662 0.516817152500153
663 0.525293171405792
664 0.525289535522461
665 0.533513307571411
666 0.522647559642792
667 0.523440361022949
668 0.518515408039093
669 0.514768898487091
670 0.506056547164917
671 0.507176578044891
672 0.502204537391663
673 0.496574252843857
674 0.491425484418869
675 0.492573231458664
676 0.483353197574615
677 0.46863055229187
678 0.465193837881088
679 0.463591992855072
680 0.458511501550674
681 0.455878555774689
682 0.444679528474808
683 0.443335264921188
684 0.446974247694016
685 0.444213092327118
686 0.438564717769623
687 0.443053781986237
688 0.45141875743866
689 0.455749869346619
690 0.441185176372528
691 0.453702509403229
692 0.449358642101288
693 0.453910380601883
694 0.456860363483429
695 0.451454818248749
696 0.447020173072815
697 0.452900886535645
698 0.451622992753983
699 0.453991740942001
700 0.452838867902756
701 0.452224761247635
702 0.456694334745407
703 0.45473837852478
704 0.459006577730179
705 0.458776086568832
706 0.459532350301743
707 0.458983719348907
708 0.477751612663269
709 0.475291609764099
710 0.470283359289169
711 0.47356915473938
712 0.464547693729401
713 0.458252668380737
714 0.455478101968765
715 0.44379249215126
716 0.446265757083893
717 0.445127487182617
718 0.448496252298355
719 0.450051009654999
720 0.446787178516388
721 0.442137271165848
722 0.455136060714722
723 0.454041063785553
724 0.453368544578552
725 0.448076903820038
726 0.456910043954849
727 0.459137141704559
728 0.451444506645203
729 0.448483318090439
730 0.447671383619308
731 0.441575080156326
732 0.441342234611511
733 0.444254338741302
734 0.438431084156036
735 0.430606871843338
736 0.43513149023056
737 0.420909702777863
738 0.412273794412613
739 0.407562345266342
740 0.411404639482498
741 0.403856664896011
742 0.403706133365631
743 0.398448646068573
744 0.391089081764221
745 0.393143266439438
746 0.395823329687119
747 0.392540067434311
748 0.389159649610519
749 0.387593924999237
750 0.394452124834061
751 0.398211359977722
752 0.395226359367371
753 0.394191354513168
754 0.39099657535553
755 0.395636945962906
756 0.402416914701462
757 0.405047029256821
758 0.378029823303223
759 0.381334662437439
760 0.372041940689087
761 0.369476199150085
762 0.368163228034973
763 0.37373873591423
764 0.367042571306229
765 0.369390040636063
766 0.379419565200806
767 0.377644389867783
768 0.371107220649719
769 0.371370434761047
770 0.376247644424438
771 0.368958175182343
772 0.355372816324234
773 0.353040724992752
774 0.355315387248993
775 0.354591637849808
776 0.344063311815262
777 0.335707157850266
778 0.345111131668091
779 0.352212995290756
780 0.348771631717682
781 0.355936735868454
782 0.366813898086548
783 0.355420231819153
784 0.363501787185669
785 0.370833516120911
786 0.370160758495331
787 0.370701611042023
788 0.377303302288055
789 0.37752166390419
790 0.375574201345444
791 0.37895479798317
792 0.377389341592789
793 0.380219310522079
794 0.385582387447357
795 0.394619017839432
796 0.400280356407166
797 0.396974474191666
798 0.400292098522186
799 0.401826471090317
800 0.399735450744629
801 0.39671990275383
802 0.396964192390442
803 0.399751126766205
804 0.405222117900848
805 0.39579501748085
806 0.383941411972046
807 0.387823522090912
808 0.39338681101799
809 0.39125582575798
810 0.393185198307037
811 0.390429198741913
812 0.39129826426506
813 0.392969936132431
814 0.395736187696457
815 0.393287569284439
816 0.377472579479218
817 0.399395823478699
818 0.40322744846344
819 0.401775389909744
820 0.392338633537292
821 0.398928910493851
822 0.405094265937805
823 0.404731839895248
824 0.406086385250092
825 0.414060026407242
826 0.419501841068268
827 0.424184888601303
828 0.420056164264679
829 0.413545191287994
830 0.426035076379776
831 0.417756944894791
832 0.422691494226456
833 0.424225956201553
834 0.421357870101929
835 0.416661500930786
836 0.421636462211609
837 0.424861997365952
838 0.425796002149582
839 0.428682923316956
840 0.437027364969254
841 0.432017356157303
842 0.435820668935776
843 0.431241601705551
844 0.427858501672745
845 0.42032390832901
846 0.419929057359695
847 0.431498259305954
848 0.429553627967834
849 0.430472254753113
850 0.43156823515892
851 0.429128617048264
852 0.434942483901978
853 0.435272097587585
854 0.425420194864273
855 0.432729631662369
856 0.436403810977936
857 0.43659982085228
858 0.43011474609375
859 0.429969370365143
860 0.423500239849091
861 0.422304600477219
862 0.435697674751282
863 0.434089660644531
864 0.435923278331757
865 0.433786481618881
866 0.433406054973602
867 0.423319846391678
868 0.42529833316803
869 0.419412851333618
870 0.421224862337112
871 0.432973325252533
872 0.430410355329514
873 0.443693697452545
874 0.441410720348358
875 0.434914320707321
876 0.433928579092026
877 0.43889045715332
878 0.448603510856628
879 0.454704731702805
880 0.443486899137497
881 0.449235498905182
882 0.429141759872437
883 0.434754252433777
884 0.434765785932541
885 0.43685656785965
886 0.439815491437912
887 0.445038884878159
888 0.445346176624298
889 0.445002138614655
890 0.433813512325287
891 0.432946503162384
892 0.421738028526306
893 0.422421365976334
894 0.440195798873901
895 0.442139625549316
896 0.433882027864456
897 0.426729917526245
898 0.42223185300827
899 0.41524139046669
900 0.40898060798645
901 0.419217348098755
902 0.423799127340317
903 0.429901838302612
904 0.429728209972382
905 0.430002331733704
906 0.428630411624908
907 0.424878269433975
908 0.426428377628326
909 0.43210905790329
910 0.433062642812729
911 0.43633097410202
912 0.420901119709015
913 0.411839842796326
914 0.4139564037323
915 0.424997687339783
916 0.425074607133865
917 0.429966926574707
918 0.424255341291428
919 0.426383525133133
920 0.436758756637573
921 0.421112090349197
922 0.416009604930878
923 0.405037760734558
924 0.41661873459816
925 0.41234889626503
926 0.409441649913788
927 0.407876968383789
928 0.398895978927612
929 0.399888426065445
930 0.397326916456223
931 0.393864661455154
932 0.396327048540115
933 0.389769852161407
934 0.385003179311752
935 0.387297123670578
936 0.378454804420471
937 0.368590623140335
938 0.359733819961548
939 0.349331706762314
940 0.352296382188797
941 0.355437457561493
942 0.3558289706707
943 0.370739251375198
944 0.36181977391243
945 0.359606504440308
946 0.362935483455658
947 0.362803339958191
948 0.364414989948273
949 0.370096296072006
950 0.370652049779892
951 0.367964625358582
952 0.356633752584457
953 0.353288471698761
954 0.34930893778801
955 0.342868030071259
956 0.363157510757446
957 0.365999102592468
958 0.364112824201584
959 0.354851275682449
960 0.359206020832062
961 0.353387594223022
962 0.361185252666473
963 0.368913799524307
964 0.372906267642975
965 0.369592487812042
966 0.372052431106567
967 0.370735704898834
968 0.394015491008759
969 0.391412228345871
970 0.384508967399597
971 0.3831667304039
972 0.380517452955246
973 0.381648987531662
974 0.391545712947845
975 0.392517536878586
976 0.392823040485382
977 0.387430846691132
978 0.387192368507385
979 0.385053753852844
980 0.392522126436234
981 0.39469176530838
982 0.399002581834793
983 0.400768578052521
984 0.405987709760666
985 0.397942274808884
986 0.402434498071671
987 0.409610569477081
988 0.41594460606575
989 0.422978520393372
990 0.422800719738007
991 0.429097473621368
992 0.429217338562012
993 0.41152423620224
994 0.399284243583679
995 0.396724551916122
996 0.392789334058762
997 0.404016375541687
998 0.405918717384338
999 0.401175409555435
1000 0.401325464248657
1001 0.398147314786911
1002 0.398707509040833
1003 0.396119117736816
1004 0.397128075361252
1005 0.394923508167267
1006 0.373085260391235
1007 0.366579651832581
1008 0.366841465234756
1009 0.369544059038162
1010 0.363335639238358
1011 0.370071172714233
1012 0.377268075942993
1013 0.372803270816803
1014 0.372935265302658
1015 0.374981582164764
1016 0.374809116125107
1017 0.358109027147293
1018 0.333938449621201
1019 0.333074569702148
1020 0.33550626039505
1021 0.334667801856995
1022 0.337820708751678
1023 0.335194408893585
1024 0.314629822969437
1025 0.314237207174301
1026 0.316716313362122
1027 0.328164875507355
1028 0.324822813272476
1029 0.322397738695145
1030 0.323088496923447
1031 0.323511928319931
1032 0.317550897598267
1033 0.326375693082809
1034 0.321707606315613
1035 0.323424100875854
1036 0.319703489542007
1037 0.314680576324463
1038 0.309795200824738
1039 0.309242874383926
1040 0.306000798940659
1041 0.293169170618057
1042 0.292766928672791
1043 0.288525879383087
1044 0.289922595024109
1045 0.289917290210724
1046 0.294492423534393
1047 0.300443947315216
1048 0.296801686286926
1049 0.293142288923264
1050 0.286551237106323
1051 0.285043329000473
1052 0.287365347146988
1053 0.285559296607971
1054 0.28906711935997
1055 0.295546114444733
1056 0.302495926618576
1057 0.317414402961731
1058 0.318576902151108
1059 0.329318732023239
1060 0.336422801017761
1061 0.336585849523544
1062 0.32257091999054
1063 0.324106514453888
1064 0.319574385881424
1065 0.313894093036652
1066 0.309526354074478
1067 0.311521023511887
1068 0.312889456748962
1069 0.315612316131592
1070 0.308924436569214
1071 0.316064834594727
1072 0.313041597604752
1073 0.310619741678238
1074 0.336055755615234
1075 0.335856288671494
1076 0.338863909244537
1077 0.329002350568771
1078 0.329568266868591
1079 0.329756647348404
1080 0.325839817523956
1081 0.330859363079071
1082 0.331497341394424
1083 0.323373675346375
1084 0.3248330950737
1085 0.327685475349426
1086 0.323676258325577
1087 0.318156063556671
1088 0.319209545850754
1089 0.318539798259735
1090 0.319921523332596
1091 0.337015181779861
1092 0.34209668636322
1093 0.347388118505478
1094 0.343277096748352
1095 0.3408023416996
1096 0.34454008936882
1097 0.327364146709442
1098 0.333383202552795
1099 0.336306273937225
1100 0.340905159711838
1101 0.338874638080597
1102 0.337419509887695
1103 0.33815586566925
1104 0.33752703666687
1105 0.335218846797943
1106 0.333873510360718
1107 0.318891018629074
1108 0.317830502986908
1109 0.308390706777573
1110 0.302973300218582
1111 0.296647131443024
1112 0.294316619634628
1113 0.287160813808441
1114 0.281223088502884
1115 0.291861176490784
1116 0.29228287935257
1117 0.290662497282028
1118 0.287492990493774
1119 0.288245499134064
1120 0.282108455896378
1121 0.276229321956635
1122 0.274071395397186
1123 0.276023477315903
1124 0.246109291911125
1125 0.249388694763184
1126 0.249283716082573
1127 0.243428349494934
1128 0.241086155176163
1129 0.238416060805321
1130 0.236386388540268
1131 0.228550255298615
1132 0.229672774672508
1133 0.230982184410095
1134 0.231396719813347
1135 0.226004600524902
1136 0.230152517557144
1137 0.228645503520966
1138 0.225135415792465
1139 0.220839649438858
1140 0.221378982067108
1141 0.205851912498474
1142 0.202103614807129
1143 0.201150715351105
1144 0.203783467411995
1145 0.202595189213753
1146 0.196328639984131
1147 0.194078162312508
1148 0.189461663365364
1149 0.188323631882668
1150 0.18525867164135
1151 0.184713035821915
1152 0.18487311899662
1153 0.186353623867035
1154 0.18622887134552
1155 0.181085050106049
1156 0.176143661141396
1157 0.181132927536964
1158 0.180821508169174
1159 0.176486134529114
1160 0.184564709663391
1161 0.183256566524506
1162 0.193416938185692
1163 0.200016409158707
1164 0.209380954504013
1165 0.200903907418251
1166 0.200689807534218
1167 0.202018439769745
1168 0.202245086431503
1169 0.206110760569572
1170 0.212525144219398
1171 0.218314856290817
1172 0.220830321311951
1173 0.22463695704937
1174 0.226019784808159
1175 0.225964799523354
1176 0.22165510058403
1177 0.224602311849594
1178 0.226532876491547
1179 0.225733980536461
1180 0.222474113106728
1181 0.227890208363533
1182 0.223937451839447
1183 0.219743475317955
1184 0.220447823405266
1185 0.224098384380341
1186 0.221814408898354
1187 0.228495076298714
1188 0.234287083148956
1189 0.233148828148842
1190 0.238878235220909
1191 0.236378684639931
1192 0.233960852026939
1193 0.23084981739521
1194 0.23284924030304
1195 0.235911220312119
1196 0.247480109333992
1197 0.250715613365173
1198 0.250287711620331
1199 0.251612186431885
1200 0.26264876127243
1201 0.270948559045792
1202 0.268341898918152
1203 0.267273247241974
1204 0.261502176523209
1205 0.262225061655045
1206 0.262665003538132
1207 0.260116249322891
1208 0.265030533075333
1209 0.273026764392853
1210 0.263787746429443
1211 0.266354590654373
1212 0.257230579853058
1213 0.260406047105789
1214 0.256158143281937
1215 0.269209295511246
1216 0.279599905014038
1217 0.289135575294495
1218 0.293021231889725
1219 0.288164019584656
1220 0.285863250494003
1221 0.281361013650894
1222 0.285789042711258
1223 0.283511698246002
1224 0.283188849687576
1225 0.289204746484756
1226 0.286257773637772
1227 0.293573886156082
1228 0.290761291980743
1229 0.298643529415131
1230 0.302357703447342
1231 0.298492163419724
1232 0.299566268920898
1233 0.300793528556824
1234 0.316042900085449
1235 0.321469962596893
1236 0.322765469551086
1237 0.320433855056763
1238 0.32108011841774
1239 0.326197177171707
};
\addlegendentry{4-6}
\end{axis}

\end{tikzpicture}
		\caption{Moving averaged training losses of all networks capped at the number of iterations of the 0-3 network}
		\label{fig:train-loss-iteration}
	\end{subfigure}
	\begin{subfigure}{\textwidth}
		\centering
		% This file was created by matplotlib2tikz v0.7.3.
\begin{tikzpicture}

\definecolor{color0}{rgb}{1,0.647058823529412,0}
\definecolor{color1}{rgb}{0.75,0,0.75}

\begin{axis}[
height=\figureheight,
legend cell align={left},
legend columns=3,
legend style={draw=white!80.0!black},
tick align=outside,
tick pos=left,
width=\figurewidth,
x grid style={white!69.01960784313725!black},
xlabel={Epoch},
xmin=-0.95, xmax=19.95,
xtick style={color=black},
y grid style={white!69.01960784313725!black},
ylabel={Loss},
ymin=-0.0805723423154104, ymax=2.38241177177559,
ytick style={color=black}
]
\addplot [semithick, blue, dotted]
table {%
0 1.13833824724987
1 0.812229750485256
2 0.584070554067349
3 0.653840091721765
4 0.611592817923118
5 0.511397582703623
6 0.505834147334099
7 0.570515671680713
8 0.265623657032847
9 0.215395397486583
10 0.189785755399706
11 0.177184161087819
12 0.167540197935084
13 0.169515813604511
14 0.148586822994824
15 0.131860823274173
16 0.150735137855698
17 0.151333889677924
18 0.116580078892153
19 0.0848326952661315
};
\addlegendentry{0-3}
\addplot [semithick, green!50.0!black, dotted]
table {%
0 1.41628715319511
1 1.06693764069141
2 1.27663794389138
3 0.939248499197838
4 0.922509286648188
5 0.814877930359963
6 0.944846005011827
7 0.837868760793637
8 0.588574065993994
9 0.254980220531042
10 0.246385987370442
11 0.425020346274743
12 0.251233636520994
13 0.273704399283116
14 0.269391394674014
15 0.225501206488563
16 0.142166017674101
17 0.232411613449072
18 0.0985216816457418
19 0.37464939076931
};
\addlegendentry{0-4}
\addplot [semithick, red, dotted]
table {%
0 1.61394396363472
1 1.61241714808406
2 1.61542675933059
3 1.19769875370726
4 1.12012973610236
5 1.07242683975064
6 0.572706997715298
7 0.488326222312694
8 0.365541997764792
9 0.390685512570246
10 0.420321245582736
11 0.296174842027985
12 0.369830398535221
13 0.275235192127982
14 0.431895680117364
15 0.773406562148308
16 0.356734114718072
17 0.310169430259539
18 0.325223492087834
19 0.313252269290388
};
\addlegendentry{0-5}
\addplot [semithick, color0, dotted]
table {%
0 1.79954144872468
1 1.39765593820605
2 0.918116900941421
3 0.811565977746043
4 0.707633372010856
5 0.862650385704534
6 0.575900484261842
7 0.68641593435715
8 0.658841194777653
9 0.546973061201901
10 0.489648585946395
11 0.596296549614133
12 0.474540107466024
13 0.406798694154312
14 0.514070398463257
15 0.308308220500576
16 0.34670556285258
17 0.41068932555359
18 0.531169772405049
19 0.371928190619781
};
\addlegendentry{0-6}
\addplot [semithick, color1]
table {%
0 1.31922038680031
1 0.537356126166525
2 0.509941637161232
3 0.668747946165413
4 0.552789914310865
5 0.335837253502437
6 0.273976282881839
7 0.229415118783003
8 0.252485667488405
9 0.234744345137317
10 0.484040063817139
11 0.443404531736243
12 0.100486980995075
13 0.106039581149595
14 0.0911984494873433
15 0.090028658032506
16 0.159029026881659
17 0.0375711282840279
18 0.0313814810523625
19 0.0501351276997455
};
\addlegendentry{4-0}
\addplot [semithick, blue]
table {%
0 2.27045794840782
1 1.67320595825872
2 1.52473438939741
3 1.22753044962883
4 0.567531871819688
5 0.639054567823487
6 0.250710032309496
7 0.246565743019023
8 0.266477209208871
9 0.157806131616047
10 0.114142085423678
11 0.236958690025213
12 0.102651759707031
13 0.143183809564782
14 0.0727671862002734
15 0.123320500984377
16 0.0510480908447613
17 0.105988740771511
18 0.0522422706475364
19 0.0493501171798515
};
\addlegendentry{4-3}
\addplot [semithick, green!50.0!black]
table {%
0 2.09102133259629
1 1.4918020678289
2 1.26745744762999
3 0.650058149388342
4 0.363539182530208
5 0.291427397902942
6 0.23953185002249
7 0.204724518577992
8 0.160771581341484
9 0.154362564776657
10 0.239738764779847
11 0.151718994489673
12 0.118789408258586
13 0.0819602399104924
14 0.0997377019544894
15 0.109213638975934
16 0.103474246655506
17 0.0601031640727976
18 0.0958307481109814
19 0.0929106862178411
};
\addlegendentry{4-4}
\addplot [semithick, red]
table {%
0 1.43747566949918
1 0.890828702328862
2 0.643040600048315
3 0.484754529804135
4 0.477274538130288
5 0.351468806366486
6 0.323403118142255
7 0.376530914656831
8 0.28521834227036
9 0.311579979600718
10 0.274974094910754
11 0.189092540201295
12 0.331202010257431
13 0.153885170752138
14 0.222422234835508
15 0.172085734412686
16 0.131525352002787
17 0.275105380572397
18 0.114382526963265
19 0.141973072747308
};
\addlegendentry{4-5}
\addplot [semithick, color0]
table {%
0 1.65152212016044
1 1.06966266096119
2 0.927276197340219
3 0.712931053652879
4 0.536401939803674
5 0.451171906214328
6 0.52690852489022
7 0.392871726815018
8 0.321086789220346
9 0.291251453796342
10 0.25394301768674
11 0.255851911665823
12 0.255499456545353
13 0.224904286973704
14 0.287978695735576
15 0.214823928302071
16 0.197295561795228
17 0.156861799005832
18 0.26024220253232
19 0.152284982224603
};
\addlegendentry{4-6}
\end{axis}

\end{tikzpicture}
		\caption{Training losses of all networks}
		\label{fig:train-loss}
	\end{subfigure}
	\begin{subfigure}{\textwidth}
		\centering
		% This file was created by matplotlib2tikz v0.7.3.
\begin{tikzpicture}

\definecolor{color0}{rgb}{1,0.647058823529412,0}
\definecolor{color1}{rgb}{0.75,0,0.75}

\begin{axis}[
height=\figureheight,
legend cell align={left},
legend columns=3,
legend style={draw=white!80.0!black},
tick align=outside,
tick pos=left,
width=\figurewidth,
x grid style={white!69.01960784313725!black},
xlabel={Epoch},
xmin=-0.95, xmax=19.95,
xtick style={color=black},
y grid style={white!69.01960784313725!black},
ylabel={Loss},
ymin=-0.0853052690211269, ymax=2.39284538854327,
ytick style={color=black}
]
\addplot [semithick, blue, dotted]
table {%
0 1.13834580963041
1 0.819701056421539
2 0.529852289476512
3 0.650018291708864
4 0.541065454483032
5 0.477622405982312
6 0.482195921150255
7 0.484079216733391
8 0.240436148461828
9 0.109353313831911
10 0.10457138882743
11 0.0867671773389534
12 0.0680540795495481
13 0.100065112113953
14 0.0623865347401595
15 0.0455729867573138
16 0.090689538989538
17 0.0848291361773456
18 0.0580837203212726
19 0.0273379426863458
};
\addlegendentry{0-3}
\addplot [semithick, green!50.0!black, dotted]
table {%
0 1.41877777046627
1 1.09895885432208
2 1.30189108848572
3 0.852174897988637
4 0.93568006268254
5 0.74748252497779
6 0.904305950359062
7 0.772045022911496
8 0.482929673459795
9 0.17125207323719
10 0.182321731139112
11 0.280397803143219
12 0.212415169510576
13 0.312828022848677
14 0.256048236870103
15 0.147567953224535
16 0.0646619006853413
17 0.227034108368335
18 0.0956500672079899
19 0.319737830095821
};
\addlegendentry{0-4}
\addplot [semithick, red, dotted]
table {%
0 1.61795137634984
1 1.61525387322461
2 1.6126560193521
3 1.17054683190805
4 1.0959413016284
5 1.00886131834101
6 0.465152272471675
7 0.401622267784896
8 0.299568555200541
9 0.32522867973204
10 0.275719350134885
11 0.24882577293449
12 0.2741159816583
13 0.226243400408162
14 0.357824105189906
15 0.839071429658819
16 0.272184612353643
17 0.227931215917623
18 0.350412118269338
19 0.426525664991803
};
\addlegendentry{0-5}
\addplot [semithick, color0, dotted]
table {%
0 1.79808582788632
1 1.39468362301956
2 0.881186917976097
3 0.65552407613507
4 0.677725610173779
5 0.742035320419588
6 0.541102151443929
7 0.565531315626921
8 0.580334145621753
9 0.575805837725416
10 0.467508394777039
11 0.549396163887448
12 0.451250706189944
13 0.422776754991508
14 0.610205575271889
15 0.385673365935131
16 0.400466478027311
17 0.415432715673506
18 0.506099748574657
19 0.388884787574226
};
\addlegendentry{0-6}
\addplot [semithick, color1]
table {%
0 1.31252484851413
1 0.494484464327494
2 0.605050813268732
3 0.667229978712621
4 0.706620792547862
5 0.389387093760349
6 0.355923498808234
7 0.368730988491465
8 0.393539419915113
9 0.42179138544533
10 0.698278293560078
11 0.733306862965778
12 0.285662040804271
13 0.186879033498742
14 0.258718020455153
15 0.201414605759998
16 0.396008350817418
17 0.158963799273436
18 0.145997776799723
19 0.202985017853617
};
\addlegendentry{4-0}
\addplot [semithick, blue]
table {%
0 2.2802021768358
1 1.74962776972924
2 1.59888599242693
3 1.3850210522428
4 0.864513941385128
5 0.856940123769972
6 0.442444276846485
7 0.541102950219755
8 0.586551466948111
9 0.470370907451452
10 0.50805777983372
11 0.577160506244795
12 0.403487660729701
13 0.399087234943484
14 0.350926063989324
15 0.642531458205072
16 0.439764100159933
17 0.592475035216337
18 0.497008767956293
19 0.259522616167633
};
\addlegendentry{4-3}
\addplot [semithick, green!50.0!black]
table {%
0 2.12454695392538
1 1.60750920242733
2 1.32656559679243
3 0.808148795255908
4 0.550891699148687
5 0.351662465612646
6 0.295548688499602
7 0.408893935992469
8 0.294318522395635
9 0.278538173228433
10 0.487413752785263
11 0.294311502775936
12 0.378941870363498
13 0.257949455544197
14 0.402242276903794
15 0.369214037299605
16 0.303164472570643
17 0.392844607465021
18 0.266194019669287
19 0.272122140279626
};
\addlegendentry{4-4}
\addplot [semithick, red]
table {%
0 1.62563294657954
1 1.05228343981284
2 0.795493834658905
3 0.763625848293304
4 0.70652764638265
5 0.575109307026422
6 0.691668875763814
7 0.621096031312589
8 0.616801506129128
9 0.616154017837511
10 0.538304043919952
11 0.561477287444803
12 0.792346257109333
13 0.457350617923118
14 0.526198475600945
15 0.357124763189091
16 0.445093795691651
17 0.610736994731619
18 0.339136986151614
19 0.395979986526072
};
\addlegendentry{4-5}
\addplot [semithick, color0]
table {%
0 1.70052718342364
1 1.20901169843246
2 1.20705273770475
3 1.01885115814725
4 0.740241774635669
5 0.633953313992418
6 0.864576328804909
7 0.679498406429379
8 0.701618151281495
9 0.694343101020206
10 0.643477512209732
11 0.724102863342353
12 0.607141545327406
13 0.70988562650161
14 0.747071208985065
15 0.533761044385536
16 0.607914808129969
17 0.499952533001647
18 0.701931490941431
19 0.581270127452138
};
\addlegendentry{4-6}
\end{axis}

\end{tikzpicture}
		\caption{Test losses of all networks}
		\label{fig:test-loss}
	\end{subfigure}
	\caption{Training and test losses of networks}
	\label{fig:networks-loss}
\end{figure}
\begin{figure}
	\setlength\figureheight{.45\textwidth}
	\setlength\figurewidth{.9\textwidth}
	\centering
	\begin{subfigure}{\textwidth}
		\centering
		% This file was created by matplotlib2tikz v0.7.3.
\begin{tikzpicture}

\definecolor{color0}{rgb}{1,0.647058823529412,0}
\definecolor{color1}{rgb}{0.75,0,0.75}

\begin{axis}[
height=\figureheight,
legend cell align={left},
legend columns=3,
legend style={at={(0.97,0.03)}, anchor=south east, draw=white!80.0!black},
tick align=outside,
tick pos=left,
width=\figurewidth,
x grid style={white!69.01960784313725!black},
xlabel={Iteration},
xmin=-61.95, xmax=1300.95,
xtick style={color=black},
xtick={0,200,400,600,800,1000,1200},
xticklabels={0,200,400,600,800,1000,1200},
y grid style={white!69.01960784313725!black},
ylabel={Accuracy},
ymin=-0.0498750001192093, ymax=1.04737500250339,
ytick style={color=black},
ytick={-0.2,0,0.2,0.4,0.6,0.8,1,1.2},
yticklabels={,0.0,0.2,0.4,0.6,0.8,1.0,}
]
\addplot [semithick, blue, dotted]
table {%
0 0.625
1 0.5625
2 0.541666686534882
3 0.46875
4 0.474999994039536
5 0.458333343267441
6 0.41071429848671
7 0.359375
8 0.361111104488373
9 0.375
10 0.386363625526428
11 0.385416656732559
12 0.384615391492844
13 0.366071432828903
14 0.360714286565781
15 0.345982134342194
16 0.332983195781708
17 0.342261910438538
18 0.337406009435654
19 0.33928570151329
20 0.340986400842667
21 0.336850643157959
22 0.333074539899826
23 0.345238089561462
24 0.346428573131561
25 0.34271976351738
26 0.33928570151329
27 0.345025509595871
28 0.354679793119431
29 0.367857128381729
30 0.376152068376541
31 0.391741067171097
32 0.39502164721489
33 0.3981092274189
34 0.404591828584671
35 0.414186507463455
36 0.41650578379631
37 0.42199245095253
38 0.433608025312424
39 0.435267835855484
40 0.442944228649139
41 0.447278887033463
42 0.454318910837173
43 0.456980496644974
44 0.463492035865784
45 0.4670030772686
46 0.475683867931366
47 0.478794664144516
48 0.479227423667908
49 0.482142865657806
50 0.482142865657806
51 0.487142860889435
52 0.482142865657806
53 0.492142856121063
54 0.497142881155014
55 0.504642844200134
56 0.512142896652222
57 0.519642889499664
58 0.51714289188385
59 0.527142882347107
60 0.532142877578735
61 0.539642870426178
62 0.542142868041992
63 0.552142858505249
64 0.553928554058075
65 0.563928544521332
66 0.573928594589233
67 0.573928594589233
68 0.58392858505249
69 0.581428587436676
70 0.58392858505249
71 0.596428573131561
72 0.600000023841858
73 0.595000028610229
74 0.600000023841858
75 0.605000019073486
76 0.607500016689301
77 0.612500011920929
78 0.615000009536743
79 0.615000009536743
80 0.617500007152557
81 0.605000019073486
82 0.610000014305115
83 0.612500011920929
84 0.612500011920929
85 0.615000009536743
86 0.607500016689301
87 0.610000014305115
88 0.610000014305115
89 0.615000009536743
90 0.612500011920929
91 0.610000014305115
92 0.605000019073486
93 0.611071407794952
94 0.611071407794952
95 0.616071462631226
96 0.613571405410767
97 0.613571405410767
98 0.61857146024704
99 0.61857146024704
100 0.613571405410767
101 0.615714311599731
102 0.623214304447174
103 0.62071430683136
104 0.623214304447174
105 0.618214309215546
106 0.623214304447174
107 0.630714297294617
108 0.640714287757874
109 0.635714292526245
110 0.638214290142059
111 0.628214299678802
112 0.633214294910431
113 0.633214294910431
114 0.638214290142059
115 0.630714297294617
116 0.633214294910431
117 0.640714287757874
118 0.640714287757874
119 0.655714273452759
120 0.660714268684387
121 0.645714282989502
122 0.657142877578735
123 0.667142868041992
124 0.672142863273621
125 0.682142853736877
126 0.692142844200134
127 0.694642841815948
128 0.692142844200134
129 0.692142844200134
130 0.697142839431763
131 0.712142884731293
132 0.714642882347107
133 0.72214287519455
134 0.729642868041992
135 0.732142865657806
136 0.747142851352692
137 0.75214284658432
138 0.754642844200134
139 0.759642839431763
140 0.76714289188385
141 0.774642884731293
142 0.78464287519455
143 0.787142872810364
144 0.792142868041992
145 0.792142868041992
146 0.797142863273621
147 0.804642856121063
148 0.809642851352692
149 0.817142844200134
150 0.827142894268036
151 0.827499985694885
152 0.834999978542328
153 0.842499971389771
154 0.842499971389771
155 0.852500021457672
156 0.855000019073486
157 0.860000014305115
158 0.865000009536743
159 0.870000004768372
160 0.872500002384186
161 0.887499988079071
162 0.892499983310699
163 0.897499978542328
164 0.902499973773956
165 0.915000021457672
166 0.917500019073486
167 0.917500019073486
168 0.917500019073486
169 0.917500019073486
170 0.922500014305115
171 0.939999997615814
172 0.939999997615814
173 0.939999997615814
174 0.942499995231628
175 0.942499995231628
176 0.942499995231628
177 0.939999997615814
178 0.947499990463257
179 0.952499985694885
180 0.952499985694885
181 0.952499985694885
182 0.954999983310699
183 0.952499985694885
184 0.952499985694885
185 0.949999988079071
186 0.949999988079071
187 0.949999988079071
188 0.947142839431763
189 0.947142839431763
190 0.947142839431763
191 0.949642896652222
192 0.949642896652222
193 0.949642896652222
194 0.949642896652222
195 0.952142894268036
196 0.947142839431763
197 0.944642841815948
198 0.942142844200134
199 0.93964284658432
200 0.942142844200134
201 0.944642841815948
202 0.942142844200134
203 0.937142848968506
204 0.93964284658432
205 0.93964284658432
206 0.942142844200134
207 0.93964284658432
208 0.937142848968506
209 0.937142848968506
210 0.937142848968506
211 0.934642851352692
212 0.934642851352692
213 0.937142848968506
214 0.934642851352692
215 0.937142848968506
216 0.93964284658432
217 0.942142844200134
218 0.947142839431763
219 0.944642841815948
220 0.944642841815948
221 0.942142844200134
222 0.93964284658432
223 0.942142844200134
224 0.942142844200134
225 0.942142844200134
226 0.944642841815948
227 0.949642896652222
228 0.947142839431763
229 0.944642841815948
230 0.944642841815948
231 0.944642841815948
232 0.93964284658432
233 0.93964284658432
234 0.937142848968506
235 0.93964284658432
236 0.942142844200134
237 0.942142844200134
238 0.944999992847443
239 0.944999992847443
240 0.944999992847443
241 0.942499995231628
242 0.942499995231628
243 0.939999997615814
244 0.939999997615814
245 0.9375
246 0.942499995231628
247 0.944999992847443
248 0.947499990463257
249 0.949999988079071
250 0.949999988079071
251 0.949999988079071
252 0.949999988079071
253 0.952499985694885
254 0.949999988079071
255 0.949999988079071
256 0.947499990463257
257 0.947499990463257
258 0.942499995231628
259 0.939999997615814
260 0.939999997615814
261 0.942499995231628
262 0.9375
263 0.9375
264 0.939999997615814
265 0.939999997615814
266 0.939999997615814
267 0.9375
268 0.9375
269 0.939999997615814
270 0.939999997615814
271 0.942499995231628
272 0.944999992847443
273 0.944999992847443
274 0.942499995231628
275 0.942499995231628
276 0.942499995231628
277 0.942499995231628
278 0.944999992847443
279 0.947499990463257
280 0.947499990463257
281 0.947499990463257
282 0.949999988079071
283 0.952499985694885
284 0.952499985694885
285 0.952499985694885
286 0.952499985694885
287 0.952499985694885
288 0.952499985694885
289 0.949999988079071
};
\addlegendentry{0-3}
\addplot [semithick, green!50.0!black, dotted]
table {%
0 0.25
1 0.25
2 0.291666656732559
3 0.3125
4 0.300000011920929
5 0.3125
6 0.321428567171097
7 0.3125
8 0.305555552244186
9 0.275000005960464
10 0.272727280855179
11 0.25
12 0.240384608507156
13 0.241071432828903
14 0.241666659712791
15 0.265625
16 0.264705896377563
17 0.25
18 0.256578952074051
19 0.268750011920929
20 0.255952388048172
21 0.255681812763214
22 0.266304343938828
23 0.255208343267441
24 0.25
25 0.245192304253578
26 0.245370373129845
27 0.25
28 0.25
29 0.241666659712791
30 0.241935476660728
31 0.25
32 0.25
33 0.253676474094391
34 0.257142871618271
35 0.267361104488373
36 0.273648649454117
37 0.282894730567932
38 0.288461536169052
39 0.290625005960464
40 0.301829278469086
41 0.306547611951828
42 0.319767445325851
43 0.315340906381607
44 0.319444447755814
45 0.323369562625885
46 0.332446813583374
47 0.3359375
48 0.331632643938065
49 0.330000013113022
50 0.332500010728836
51 0.330000013113022
52 0.327499985694885
53 0.322499990463257
54 0.319999992847443
55 0.324999988079071
56 0.322499990463257
57 0.324999988079071
58 0.33500000834465
59 0.342500001192093
60 0.347499996423721
61 0.360000014305115
62 0.367500007152557
63 0.372500002384186
64 0.375
65 0.372500002384186
66 0.377499997615814
67 0.389999985694885
68 0.389999985694885
69 0.387499988079071
70 0.39750000834465
71 0.402500003576279
72 0.39750000834465
73 0.405000001192093
74 0.41499999165535
75 0.427500009536743
76 0.427500009536743
77 0.432500004768372
78 0.442499995231628
79 0.447499990463257
80 0.452499985694885
81 0.447499990463257
82 0.452499985694885
83 0.452499985694885
84 0.457500010728836
85 0.46000000834465
86 0.462500005960464
87 0.46000000834465
88 0.457500010728836
89 0.465000003576279
90 0.465000003576279
91 0.462500005960464
92 0.457500010728836
93 0.467500001192093
94 0.467500001192093
95 0.472499996423721
96 0.465000003576279
97 0.469999998807907
98 0.482499986886978
99 0.492500007152557
100 0.492500007152557
101 0.5
102 0.5
103 0.509999990463257
104 0.512499988079071
105 0.509999990463257
106 0.512499988079071
107 0.509999990463257
108 0.502499997615814
109 0.502499997615814
110 0.502499997615814
111 0.5
112 0.497500002384186
113 0.497500002384186
114 0.497500002384186
115 0.495000004768372
116 0.492500007152557
117 0.492500007152557
118 0.495000004768372
119 0.492500007152557
120 0.490000009536743
121 0.487500011920929
122 0.495000004768372
123 0.497500002384186
124 0.492500007152557
125 0.482499986886978
126 0.487500011920929
127 0.482499986886978
128 0.485000014305115
129 0.492500007152557
130 0.485000014305115
131 0.492500007152557
132 0.490000009536743
133 0.492500007152557
134 0.482499986886978
135 0.474999994039536
136 0.467500001192093
137 0.469999998807907
138 0.469999998807907
139 0.462500005960464
140 0.449999988079071
141 0.452499985694885
142 0.452499985694885
143 0.455000013113022
144 0.46000000834465
145 0.46000000834465
146 0.462500005960464
147 0.457500010728836
148 0.449999988079071
149 0.439999997615814
150 0.4375
151 0.435000002384186
152 0.442499995231628
153 0.439999997615814
154 0.444999992847443
155 0.439999997615814
156 0.439999997615814
157 0.442499995231628
158 0.439999997615814
159 0.442499995231628
160 0.442499995231628
161 0.442499995231628
162 0.444999992847443
163 0.447499990463257
164 0.457500010728836
165 0.469999998807907
166 0.474999994039536
167 0.482499986886978
168 0.487500011920929
169 0.5
170 0.509999990463257
171 0.514999985694885
172 0.522499978542328
173 0.529999971389771
174 0.535000026226044
175 0.550000011920929
176 0.555000007152557
177 0.564999997615814
178 0.557500004768372
179 0.560000002384186
180 0.572499990463257
181 0.579999983310699
182 0.592499971389771
183 0.602500021457672
184 0.620000004768372
185 0.629999995231628
186 0.639999985694885
187 0.647499978542328
188 0.657500028610229
189 0.667500019073486
190 0.682500004768372
191 0.692499995231628
192 0.697499990463257
193 0.702499985694885
194 0.707499980926514
195 0.712499976158142
196 0.722500026226044
197 0.732500016689301
198 0.742500007152557
199 0.757499992847443
200 0.772499978542328
201 0.785000026226044
202 0.790000021457672
203 0.797500014305115
204 0.807500004768372
205 0.817499995231628
206 0.829999983310699
207 0.834999978542328
208 0.847500026226044
209 0.852500021457672
210 0.862500011920929
211 0.872500002384186
212 0.882499992847443
213 0.884999990463257
214 0.882499992847443
215 0.882499992847443
216 0.884999990463257
217 0.879999995231628
218 0.884999990463257
219 0.887499988079071
220 0.889999985694885
221 0.894999980926514
222 0.894999980926514
223 0.897499978542328
224 0.897499978542328
225 0.894999980926514
226 0.899999976158142
227 0.902499973773956
228 0.910000026226044
229 0.907500028610229
230 0.910000026226044
231 0.907500028610229
232 0.904999971389771
233 0.902499973773956
234 0.902499973773956
235 0.902499973773956
236 0.907500028610229
237 0.904999971389771
238 0.904999971389771
239 0.907500028610229
240 0.910000026226044
241 0.907500028610229
242 0.910000026226044
243 0.910000026226044
244 0.910000026226044
245 0.907500028610229
246 0.907500028610229
247 0.902499973773956
248 0.904999971389771
249 0.904999971389771
250 0.899999976158142
251 0.899999976158142
252 0.897499978542328
253 0.899999976158142
254 0.899999976158142
255 0.902499973773956
256 0.899999976158142
257 0.904999971389771
258 0.904999971389771
259 0.910000026226044
260 0.907500028610229
261 0.904999971389771
262 0.902499973773956
263 0.902499973773956
264 0.907500028610229
265 0.904999971389771
266 0.907500028610229
267 0.910000026226044
268 0.910000026226044
269 0.910000026226044
270 0.910000026226044
271 0.910000026226044
272 0.907500028610229
273 0.907500028610229
274 0.912500023841858
275 0.915000021457672
276 0.912500023841858
277 0.910000026226044
278 0.912500023841858
279 0.917500019073486
280 0.920000016689301
281 0.920000016689301
282 0.922500014305115
283 0.925000011920929
284 0.925000011920929
285 0.925000011920929
286 0.922500014305115
287 0.925000011920929
288 0.925000011920929
289 0.925000011920929
};
\addlegendentry{0-4}
\addplot [semithick, red, dotted]
table {%
0 0.25
1 0.25
2 0.25
3 0.21875
4 0.200000002980232
5 0.22916667163372
6 0.214285716414452
7 0.203125
8 0.222222223877907
9 0.212500005960464
10 0.238636359572411
11 0.22916667163372
12 0.22115384042263
13 0.214285716414452
14 0.216666668653488
15 0.2109375
16 0.220588237047195
17 0.222222223877907
18 0.217105269432068
19 0.21875
20 0.220238089561462
21 0.210227265954018
22 0.206521734595299
23 0.203125
24 0.234999999403954
25 0.235576927661896
26 0.226851850748062
27 0.227678567171097
28 0.237068966031075
29 0.241666659712791
30 0.237903222441673
31 0.23046875
32 0.238636359572411
33 0.242647051811218
34 0.235714286565781
35 0.232638895511627
36 0.229729726910591
37 0.226973682641983
38 0.230769231915474
39 0.228125005960464
40 0.225609749555588
41 0.223214283585548
42 0.220930233597755
43 0.21875
44 0.227777779102325
45 0.222826093435287
46 0.22074468433857
47 0.2265625
48 0.224489793181419
49 0.224999994039536
50 0.222499996423721
51 0.217500001192093
52 0.215000003576279
53 0.215000003576279
54 0.217500001192093
55 0.212500005960464
56 0.209999993443489
57 0.212500005960464
58 0.212500005960464
59 0.209999993443489
60 0.204999998211861
61 0.204999998211861
62 0.209999993443489
63 0.209999993443489
64 0.204999998211861
65 0.204999998211861
66 0.202500000596046
67 0.202500000596046
68 0.204999998211861
69 0.215000003576279
70 0.209999993443489
71 0.215000003576279
72 0.217500001192093
73 0.215000003576279
74 0.194999992847443
75 0.192499995231628
76 0.192499995231628
77 0.194999992847443
78 0.1875
79 0.1875
80 0.194999992847443
81 0.202500000596046
82 0.194999992847443
83 0.194999992847443
84 0.207499995827675
85 0.217500001192093
86 0.219999998807907
87 0.222499996423721
88 0.217500001192093
89 0.222499996423721
90 0.224999994039536
91 0.224999994039536
92 0.232500001788139
93 0.242500007152557
94 0.234999999403954
95 0.245000004768372
96 0.245000004768372
97 0.239999994635582
98 0.242500007152557
99 0.242500007152557
100 0.247500002384186
101 0.257499992847443
102 0.267500013113022
103 0.270000010728836
104 0.27250000834465
105 0.280000001192093
106 0.28999999165535
107 0.294999986886978
108 0.294999986886978
109 0.305000007152557
110 0.3125
111 0.322499990463257
112 0.324999988079071
113 0.330000013113022
114 0.33500000834465
115 0.344999998807907
116 0.349999994039536
117 0.349999994039536
118 0.35249999165535
119 0.340000003576279
120 0.35249999165535
121 0.349999994039536
122 0.365000009536743
123 0.377499997615814
124 0.387499988079071
125 0.389999985694885
126 0.402500003576279
127 0.405000001192093
128 0.407499998807907
129 0.407499998807907
130 0.400000005960464
131 0.395000010728836
132 0.407499998807907
133 0.409999996423721
134 0.407499998807907
135 0.407499998807907
136 0.409999996423721
137 0.412499994039536
138 0.419999986886978
139 0.422500014305115
140 0.417499989271164
141 0.419999986886978
142 0.409999996423721
143 0.409999996423721
144 0.412499994039536
145 0.409999996423721
146 0.412499994039536
147 0.417499989271164
148 0.425000011920929
149 0.427500009536743
150 0.430000007152557
151 0.427500009536743
152 0.427500009536743
153 0.430000007152557
154 0.432500004768372
155 0.430000007152557
156 0.432500004768372
157 0.4375
158 0.442499995231628
159 0.447499990463257
160 0.447499990463257
161 0.444999992847443
162 0.447499990463257
163 0.455000013113022
164 0.457500010728836
165 0.46000000834465
166 0.462500005960464
167 0.474999994039536
168 0.482499986886978
169 0.495000004768372
170 0.5
171 0.517499983310699
172 0.512499988079071
173 0.512499988079071
174 0.519999980926514
175 0.529999971389771
176 0.532500028610229
177 0.537500023841858
178 0.542500019073486
179 0.555000007152557
180 0.567499995231628
181 0.577499985694885
182 0.577499985694885
183 0.584999978542328
184 0.592499971389771
185 0.595000028610229
186 0.602500021457672
187 0.610000014305115
188 0.617500007152557
189 0.625
190 0.637499988079071
191 0.647499978542328
192 0.657500028610229
193 0.662500023841858
194 0.672500014305115
195 0.682500004768372
196 0.694999992847443
197 0.702499985694885
198 0.707499980926514
199 0.714999973773956
200 0.720000028610229
201 0.725000023841858
202 0.730000019073486
203 0.737500011920929
204 0.745000004768372
205 0.757499992847443
206 0.764999985694885
207 0.767499983310699
208 0.769999980926514
209 0.772499978542328
210 0.774999976158142
211 0.777499973773956
212 0.779999971389771
213 0.782500028610229
214 0.787500023841858
215 0.787500023841858
216 0.785000026226044
217 0.785000026226044
218 0.785000026226044
219 0.790000021457672
220 0.792500019073486
221 0.792500019073486
222 0.797500014305115
223 0.802500009536743
224 0.802500009536743
225 0.805000007152557
226 0.810000002384186
227 0.8125
228 0.817499995231628
229 0.807500004768372
230 0.802500009536743
231 0.807500004768372
232 0.807500004768372
233 0.805000007152557
234 0.802500009536743
235 0.805000007152557
236 0.807500004768372
237 0.807500004768372
238 0.807500004768372
239 0.805000007152557
240 0.807500004768372
241 0.810000002384186
242 0.817499995231628
243 0.8125
244 0.8125
245 0.810000002384186
246 0.802500009536743
247 0.800000011920929
248 0.802500009536743
249 0.805000007152557
250 0.807500004768372
251 0.807500004768372
252 0.810000002384186
253 0.810000002384186
254 0.807500004768372
255 0.807500004768372
256 0.802500009536743
257 0.797500014305115
258 0.800000011920929
259 0.800000011920929
260 0.805000007152557
261 0.8125
262 0.810000002384186
263 0.810000002384186
264 0.814999997615814
265 0.819999992847443
266 0.829999983310699
267 0.829999983310699
268 0.832499980926514
269 0.832499980926514
270 0.829999983310699
271 0.824999988079071
272 0.822499990463257
273 0.817499995231628
274 0.817499995231628
275 0.817499995231628
276 0.8125
277 0.807500004768372
278 0.807500004768372
279 0.817499995231628
280 0.824999988079071
281 0.824999988079071
282 0.827499985694885
283 0.832499980926514
284 0.837499976158142
285 0.839999973773956
286 0.842499971389771
287 0.847500026226044
288 0.845000028610229
289 0.847500026226044
};
\addlegendentry{0-5}
\addplot [semithick, color0, dotted]
table {%
0 0.25
1 0.1875
2 0.125
3 0.15625
4 0.150000005960464
5 0.125
6 0.125
7 0.109375
8 0.111111111938953
9 0.137500002980232
10 0.136363640427589
11 0.13541667163372
12 0.125
13 0.133928567171097
14 0.133333340287209
15 0.140625
16 0.139705881476402
17 0.14583332836628
18 0.144736841320992
19 0.150000005960464
20 0.154761910438538
21 0.159090906381607
22 0.157608702778816
23 0.16666667163372
24 0.165000006556511
25 0.168269231915474
26 0.171296298503876
27 0.169642850756645
28 0.163793101906776
29 0.162499994039536
30 0.157258063554764
31 0.1640625
32 0.170454546809196
33 0.169117644429207
34 0.164285719394684
35 0.173611104488373
36 0.182432428002357
37 0.180921047925949
38 0.182692304253578
39 0.1875
40 0.189024388790131
41 0.202380955219269
42 0.209302321076393
43 0.207386359572411
44 0.20833332836628
45 0.20923912525177
46 0.212765961885452
47 0.21614582836628
48 0.216836735606194
49 0.215000003576279
50 0.219999998807907
51 0.219999998807907
52 0.230000004172325
53 0.232500001788139
54 0.230000004172325
55 0.232500001788139
56 0.239999994635582
57 0.25
58 0.254999995231628
59 0.254999995231628
60 0.257499992847443
61 0.257499992847443
62 0.264999985694885
63 0.259999990463257
64 0.262499988079071
65 0.270000010728836
66 0.282499998807907
67 0.287499994039536
68 0.294999986886978
69 0.292499989271164
70 0.300000011920929
71 0.307500004768372
72 0.314999997615814
73 0.314999997615814
74 0.324999988079071
75 0.337500005960464
76 0.347499996423721
77 0.362500011920929
78 0.375
79 0.382499992847443
80 0.39750000834465
81 0.39750000834465
82 0.400000005960464
83 0.407499998807907
84 0.41499999165535
85 0.41499999165535
86 0.419999986886978
87 0.427500009536743
88 0.430000007152557
89 0.4375
90 0.442499995231628
91 0.442499995231628
92 0.449999988079071
93 0.46000000834465
94 0.467500001192093
95 0.469999998807907
96 0.474999994039536
97 0.482499986886978
98 0.490000009536743
99 0.5
100 0.495000004768372
101 0.507499992847443
102 0.509999990463257
103 0.509999990463257
104 0.524999976158142
105 0.537500023841858
106 0.535000026226044
107 0.540000021457672
108 0.545000016689301
109 0.555000007152557
110 0.560000002384186
111 0.564999997615814
112 0.572499990463257
113 0.584999978542328
114 0.597500026226044
115 0.592499971389771
116 0.589999973773956
117 0.592499971389771
118 0.597500026226044
119 0.610000014305115
120 0.612500011920929
121 0.612500011920929
122 0.612500011920929
123 0.610000014305115
124 0.612500011920929
125 0.612500011920929
126 0.607500016689301
127 0.605000019073486
128 0.602500021457672
129 0.605000019073486
130 0.602500021457672
131 0.602500021457672
132 0.605000019073486
133 0.612500011920929
134 0.617500007152557
135 0.622500002384186
136 0.620000004768372
137 0.629999995231628
138 0.634999990463257
139 0.634999990463257
140 0.639999985694885
141 0.632499992847443
142 0.622500002384186
143 0.622500002384186
144 0.627499997615814
145 0.632499992847443
146 0.632499992847443
147 0.634999990463257
148 0.634999990463257
149 0.629999995231628
150 0.639999985694885
151 0.642499983310699
152 0.642499983310699
153 0.647499978542328
154 0.644999980926514
155 0.642499983310699
156 0.649999976158142
157 0.652499973773956
158 0.654999971389771
159 0.652499973773956
160 0.660000026226044
161 0.667500019073486
162 0.672500014305115
163 0.675000011920929
164 0.667500019073486
165 0.677500009536743
166 0.680000007152557
167 0.682500004768372
168 0.677500009536743
169 0.680000007152557
170 0.680000007152557
171 0.682500004768372
172 0.689999997615814
173 0.699999988079071
174 0.702499985694885
175 0.692499995231628
176 0.694999992847443
177 0.692499995231628
178 0.697499990463257
179 0.697499990463257
180 0.697499990463257
181 0.704999983310699
182 0.702499985694885
183 0.694999992847443
184 0.697499990463257
185 0.697499990463257
186 0.699999988079071
187 0.694999992847443
188 0.699999988079071
189 0.702499985694885
190 0.699999988079071
191 0.707499980926514
192 0.714999973773956
193 0.714999973773956
194 0.712499976158142
195 0.717499971389771
196 0.720000028610229
197 0.720000028610229
198 0.725000023841858
199 0.735000014305115
200 0.732500016689301
201 0.732500016689301
202 0.737500011920929
203 0.737500011920929
204 0.740000009536743
205 0.742500007152557
206 0.747500002384186
207 0.742500007152557
208 0.742500007152557
209 0.742500007152557
210 0.742500007152557
211 0.742500007152557
212 0.742500007152557
213 0.747500002384186
214 0.754999995231628
215 0.752499997615814
216 0.747500002384186
217 0.745000004768372
218 0.747500002384186
219 0.737500011920929
220 0.735000014305115
221 0.737500011920929
222 0.732500016689301
223 0.730000019073486
224 0.727500021457672
225 0.732500016689301
226 0.732500016689301
227 0.735000014305115
228 0.737500011920929
229 0.737500011920929
230 0.737500011920929
231 0.737500011920929
232 0.740000009536743
233 0.747500002384186
234 0.745000004768372
235 0.745000004768372
236 0.747500002384186
237 0.75
238 0.747500002384186
239 0.737500011920929
240 0.737500011920929
241 0.730000019073486
242 0.727500021457672
243 0.732500016689301
244 0.735000014305115
245 0.732500016689301
246 0.732500016689301
247 0.730000019073486
248 0.727500021457672
249 0.725000023841858
250 0.727500021457672
251 0.730000019073486
252 0.730000019073486
253 0.732500016689301
254 0.722500026226044
255 0.717499971389771
256 0.712499976158142
257 0.712499976158142
258 0.714999973773956
259 0.712499976158142
260 0.707499980926514
261 0.702499985694885
262 0.692499995231628
263 0.685000002384186
264 0.685000002384186
265 0.685000002384186
266 0.694999992847443
267 0.699999988079071
268 0.697499990463257
269 0.707499980926514
270 0.712499976158142
271 0.709999978542328
272 0.712499976158142
273 0.709999978542328
274 0.704999983310699
275 0.712499976158142
276 0.709999978542328
277 0.712499976158142
278 0.709999978542328
279 0.714999973773956
280 0.720000028610229
281 0.722500026226044
282 0.730000019073486
283 0.725000023841858
284 0.722500026226044
285 0.722500026226044
286 0.725000023841858
287 0.720000028610229
288 0.717499971389771
289 0.727500021457672
};
\addlegendentry{0-6}
\addplot [semithick, color1]
table {%
0 0.25
1 0.3125
2 0.291666656732559
3 0.34375
4 0.349999994039536
5 0.3125
6 0.321428567171097
7 0.3125
8 0.305555552244186
9 0.337500005960464
10 0.363636374473572
11 0.375
12 0.355769217014313
13 0.330357134342194
14 0.341666668653488
15 0.3515625
16 0.352941185235977
17 0.368055552244186
18 0.361842095851898
19 0.375
20 0.357142865657806
21 0.363636374473572
22 0.35326087474823
23 0.354166656732559
24 0.370000004768372
25 0.379807680845261
26 0.384259253740311
27 0.383928567171097
28 0.387931048870087
29 0.395833343267441
30 0.403225809335709
31 0.41015625
32 0.424242436885834
33 0.430147051811218
34 0.439285725355148
35 0.447916656732559
36 0.456081092357635
37 0.463815778493881
38 0.474358975887299
39 0.484375
40 0.484756112098694
41 0.494047611951828
42 0.497093021869659
43 0.505681812763214
44 0.513888895511627
45 0.51902174949646
46 0.523936152458191
47 0.533854186534882
48 0.540816307067871
49 0.542500019073486
50 0.550000011920929
51 0.555000007152557
52 0.564999997615814
53 0.572499990463257
54 0.574999988079071
55 0.589999973773956
56 0.600000023841858
57 0.607500016689301
58 0.620000004768372
59 0.620000004768372
60 0.625
61 0.629999995231628
62 0.644999980926514
63 0.662500023841858
64 0.665000021457672
65 0.672500014305115
66 0.677500009536743
67 0.682500004768372
68 0.692499995231628
69 0.697499990463257
70 0.712499976158142
71 0.720000028610229
72 0.725000023841858
73 0.727500021457672
74 0.727500021457672
75 0.730000019073486
76 0.737500011920929
77 0.747500002384186
78 0.757499992847443
79 0.764999985694885
80 0.767499983310699
81 0.767499983310699
82 0.767499983310699
83 0.774999976158142
84 0.772499978542328
85 0.772499978542328
86 0.774999976158142
87 0.774999976158142
88 0.764999985694885
89 0.764999985694885
90 0.769999980926514
91 0.767499983310699
92 0.767499983310699
93 0.764999985694885
94 0.762499988079071
95 0.762499988079071
96 0.767499983310699
97 0.764999985694885
98 0.764999985694885
99 0.769999980926514
100 0.777499973773956
101 0.782500028610229
102 0.777499973773956
103 0.774999976158142
104 0.779999971389771
105 0.779999971389771
106 0.779999971389771
107 0.785000026226044
108 0.785000026226044
109 0.787500023841858
110 0.785000026226044
111 0.785000026226044
112 0.785000026226044
113 0.785000026226044
114 0.790000021457672
115 0.787500023841858
116 0.790000021457672
117 0.790000021457672
118 0.795000016689301
119 0.795000016689301
120 0.797500014305115
121 0.795000016689301
122 0.802500009536743
123 0.810000002384186
124 0.814999997615814
125 0.819999992847443
126 0.819999992847443
127 0.819999992847443
128 0.817499995231628
129 0.817499995231628
130 0.822499990463257
131 0.829999983310699
132 0.832499980926514
133 0.824999988079071
134 0.824999988079071
135 0.829999983310699
136 0.832499980926514
137 0.832499980926514
138 0.845000028610229
139 0.845000028610229
140 0.850000023841858
141 0.850000023841858
142 0.857500016689301
143 0.860000014305115
144 0.857500016689301
145 0.855000019073486
146 0.850000023841858
147 0.852500021457672
148 0.852500021457672
149 0.845000028610229
150 0.839999973773956
151 0.842499971389771
152 0.852500021457672
153 0.855000019073486
154 0.860000014305115
155 0.862500011920929
156 0.862500011920929
157 0.860000014305115
158 0.862500011920929
159 0.865000009536743
160 0.870000004768372
161 0.875
162 0.877499997615814
163 0.879999995231628
164 0.879999995231628
165 0.884999990463257
166 0.884999990463257
167 0.884999990463257
168 0.884999990463257
169 0.884999990463257
170 0.884999990463257
171 0.884999990463257
172 0.884999990463257
173 0.887499988079071
174 0.887499988079071
175 0.887499988079071
176 0.879999995231628
177 0.879999995231628
178 0.879999995231628
179 0.879999995231628
180 0.875
181 0.872500002384186
182 0.872500002384186
183 0.870000004768372
184 0.872500002384186
185 0.872500002384186
186 0.872500002384186
187 0.875
188 0.875
189 0.875
190 0.875
191 0.872500002384186
192 0.865000009536743
193 0.865000009536743
194 0.872500002384186
195 0.877499997615814
196 0.882499992847443
197 0.882499992847443
198 0.877499997615814
199 0.887499988079071
200 0.889999985694885
201 0.889999985694885
202 0.887499988079071
203 0.884999990463257
204 0.879999995231628
205 0.879999995231628
206 0.879999995231628
207 0.879999995231628
208 0.877499997615814
209 0.877499997615814
210 0.877499997615814
211 0.875
212 0.870000004768372
213 0.867500007152557
214 0.867500007152557
215 0.867500007152557
216 0.872500002384186
217 0.875
218 0.872500002384186
219 0.872500002384186
220 0.872500002384186
221 0.877499997615814
222 0.882499992847443
223 0.882499992847443
224 0.882499992847443
225 0.879999995231628
226 0.887499988079071
227 0.889999985694885
228 0.887499988079071
229 0.887499988079071
230 0.889999985694885
231 0.889999985694885
232 0.889999985694885
233 0.899999976158142
234 0.904999971389771
235 0.902499973773956
236 0.899999976158142
237 0.902499973773956
238 0.902499973773956
239 0.904999971389771
240 0.902499973773956
241 0.910000026226044
242 0.917500019073486
243 0.920000016689301
244 0.920000016689301
245 0.917500019073486
246 0.917500019073486
247 0.910000026226044
248 0.917500019073486
249 0.915000021457672
250 0.917500019073486
251 0.917500019073486
252 0.917500019073486
253 0.922500014305115
254 0.925000011920929
255 0.917500019073486
256 0.917500019073486
257 0.922500014305115
258 0.925000011920929
259 0.925000011920929
260 0.925000011920929
261 0.927500009536743
262 0.930000007152557
263 0.932500004768372
264 0.935000002384186
265 0.935000002384186
266 0.935000002384186
267 0.935000002384186
268 0.935000002384186
269 0.935000002384186
270 0.9375
271 0.9375
272 0.9375
273 0.9375
274 0.9375
275 0.939999997615814
276 0.942499995231628
277 0.942499995231628
278 0.947499990463257
279 0.944999992847443
280 0.947499990463257
281 0.947499990463257
282 0.944999992847443
283 0.942499995231628
284 0.942499995231628
285 0.944999992847443
286 0.947499990463257
287 0.944999992847443
288 0.944999992847443
289 0.942499995231628
290 0.944999992847443
291 0.939999997615814
292 0.939999997615814
293 0.935000002384186
294 0.935000002384186
295 0.939999997615814
296 0.939999997615814
297 0.947499990463257
298 0.947499990463257
299 0.949999988079071
300 0.949999988079071
301 0.949999988079071
302 0.949999988079071
303 0.949999988079071
304 0.952499985694885
305 0.957499980926514
306 0.959999978542328
307 0.959999978542328
308 0.959999978542328
309 0.962499976158142
310 0.962499976158142
311 0.962499976158142
312 0.957499980926514
313 0.957499980926514
314 0.957499980926514
315 0.957499980926514
316 0.957499980926514
317 0.957499980926514
318 0.959999978542328
319 0.959999978542328
320 0.959999978542328
321 0.959999978542328
322 0.959999978542328
323 0.959999978542328
324 0.959999978542328
325 0.957499980926514
326 0.957499980926514
327 0.957499980926514
328 0.957499980926514
329 0.959999978542328
330 0.959999978542328
331 0.962499976158142
332 0.964999973773956
333 0.959999978542328
334 0.959999978542328
335 0.959999978542328
336 0.959999978542328
337 0.957499980926514
338 0.957499980926514
339 0.959999978542328
340 0.957499980926514
341 0.962499976158142
342 0.962499976158142
343 0.967499971389771
344 0.967499971389771
345 0.967499971389771
346 0.967499971389771
347 0.967499971389771
348 0.967499971389771
349 0.967499971389771
350 0.964999973773956
351 0.964999973773956
352 0.967499971389771
353 0.967499971389771
354 0.964999973773956
355 0.967499971389771
356 0.964999973773956
357 0.964999973773956
358 0.964999973773956
359 0.964999973773956
360 0.959999978542328
361 0.959999978542328
362 0.967499971389771
363 0.967499971389771
364 0.967499971389771
365 0.967499971389771
366 0.967499971389771
367 0.967499971389771
368 0.967499971389771
369 0.970000028610229
370 0.970000028610229
371 0.970000028610229
372 0.970000028610229
373 0.970000028610229
374 0.970000028610229
375 0.972500026226044
376 0.972500026226044
377 0.972500026226044
378 0.972500026226044
379 0.972500026226044
380 0.972500026226044
381 0.972500026226044
382 0.972500026226044
383 0.980000019073486
384 0.980000019073486
385 0.980000019073486
386 0.980000019073486
387 0.985000014305115
388 0.985000014305115
389 0.985000014305115
390 0.987500011920929
391 0.987500011920929
392 0.987500011920929
393 0.987500011920929
394 0.987500011920929
395 0.987500011920929
396 0.987500011920929
397 0.987500011920929
398 0.987500011920929
399 0.987500011920929
400 0.990000009536743
401 0.990000009536743
402 0.990000009536743
403 0.990000009536743
404 0.992500007152557
405 0.990000009536743
406 0.992500007152557
407 0.992500007152557
408 0.992500007152557
409 0.992500007152557
410 0.997500002384186
411 0.997500002384186
412 0.997500002384186
413 0.997500002384186
414 0.997500002384186
415 0.997500002384186
416 0.997500002384186
417 0.997500002384186
418 0.997500002384186
419 0.997500002384186
};
\addlegendentry{4-0}
\addplot [semithick, blue]
table {%
0 0.125
1 0.125
2 0.0833333358168602
3 0.125
4 0.150000005960464
5 0.14583332836628
6 0.125
7 0.109375
8 0.0972222238779068
9 0.0874999985098839
10 0.0795454531908035
11 0.0833333358168602
12 0.0769230797886848
13 0.0803571417927742
14 0.0833333358168602
15 0.0859375
16 0.0808823555707932
17 0.0763888880610466
18 0.0789473652839661
19 0.0812499970197678
20 0.0833333358168602
21 0.0795454531908035
22 0.0869565233588219
23 0.0885416641831398
24 0.0900000035762787
25 0.0865384638309479
26 0.0879629626870155
27 0.0892857164144516
28 0.0905172377824783
29 0.0916666686534882
30 0.0927419364452362
31 0.09375
32 0.094696968793869
33 0.0919117629528046
34 0.0928571447730064
35 0.09375
36 0.0912162140011787
37 0.0953947380185127
38 0.0929487198591232
39 0.09375
40 0.0914634168148041
41 0.095238097012043
42 0.0959302335977554
43 0.0965909063816071
44 0.0972222238779068
45 0.0978260859847069
46 0.101063832640648
47 0.0989583358168602
48 0.0994897931814194
49 0.10249999910593
50 0.10249999910593
51 0.104999996721745
52 0.107500001788139
53 0.104999996721745
54 0.100000001490116
55 0.100000001490116
56 0.107500001788139
57 0.112499997019768
58 0.117499999701977
59 0.119999997317791
60 0.122500002384186
61 0.129999995231628
62 0.140000000596046
63 0.142499998211861
64 0.147499993443489
65 0.150000005960464
66 0.152500003576279
67 0.162499994039536
68 0.165000006556511
69 0.170000001788139
70 0.177499994635582
71 0.182500004768372
72 0.185000002384186
73 0.189999997615814
74 0.200000002980232
75 0.204999998211861
76 0.202500000596046
77 0.202500000596046
78 0.204999998211861
79 0.207499995827675
80 0.207499995827675
81 0.209999993443489
82 0.209999993443489
83 0.217500001192093
84 0.215000003576279
85 0.215000003576279
86 0.222499996423721
87 0.219999998807907
88 0.227500006556511
89 0.234999999403954
90 0.242500007152557
91 0.242500007152557
92 0.247500002384186
93 0.252499997615814
94 0.254999995231628
95 0.262499988079071
96 0.259999990463257
97 0.264999985694885
98 0.27250000834465
99 0.27250000834465
100 0.275000005960464
101 0.27250000834465
102 0.270000010728836
103 0.270000010728836
104 0.275000005960464
105 0.280000001192093
106 0.277500003576279
107 0.280000001192093
108 0.282499998807907
109 0.287499994039536
110 0.28999999165535
111 0.282499998807907
112 0.277500003576279
113 0.27250000834465
114 0.270000010728836
115 0.270000010728836
116 0.275000005960464
117 0.275000005960464
118 0.27250000834465
119 0.275000005960464
120 0.27250000834465
121 0.275000005960464
122 0.27250000834465
123 0.275000005960464
124 0.267500013113022
125 0.264999985694885
126 0.27250000834465
127 0.277500003576279
128 0.277500003576279
129 0.275000005960464
130 0.282499998807907
131 0.287499994039536
132 0.294999986886978
133 0.294999986886978
134 0.297500014305115
135 0.307500004768372
136 0.310000002384186
137 0.310000002384186
138 0.305000007152557
139 0.305000007152557
140 0.297500014305115
141 0.297500014305115
142 0.297500014305115
143 0.297500014305115
144 0.302500009536743
145 0.297500014305115
146 0.297500014305115
147 0.297500014305115
148 0.294999986886978
149 0.297500014305115
150 0.300000011920929
151 0.300000011920929
152 0.307500004768372
153 0.3125
154 0.3125
155 0.310000002384186
156 0.307500004768372
157 0.310000002384186
158 0.310000002384186
159 0.310000002384186
160 0.310000002384186
161 0.314999997615814
162 0.3125
163 0.322499990463257
164 0.324999988079071
165 0.324999988079071
166 0.319999992847443
167 0.319999992847443
168 0.327499985694885
169 0.319999992847443
170 0.324999988079071
171 0.319999992847443
172 0.317499995231628
173 0.310000002384186
174 0.307500004768372
175 0.307500004768372
176 0.307500004768372
177 0.305000007152557
178 0.302500009536743
179 0.302500009536743
180 0.297500014305115
181 0.292499989271164
182 0.284999996423721
183 0.282499998807907
184 0.284999996423721
185 0.277500003576279
186 0.275000005960464
187 0.277500003576279
188 0.280000001192093
189 0.280000001192093
190 0.28999999165535
191 0.292499989271164
192 0.28999999165535
193 0.287499994039536
194 0.282499998807907
195 0.284999996423721
196 0.287499994039536
197 0.28999999165535
198 0.284999996423721
199 0.282499998807907
200 0.277500003576279
201 0.277500003576279
202 0.27250000834465
203 0.27250000834465
204 0.277500003576279
205 0.282499998807907
206 0.284999996423721
207 0.282499998807907
208 0.280000001192093
209 0.27250000834465
210 0.27250000834465
211 0.267500013113022
212 0.275000005960464
213 0.270000010728836
214 0.275000005960464
215 0.282499998807907
216 0.282499998807907
217 0.280000001192093
218 0.275000005960464
219 0.277500003576279
220 0.270000010728836
221 0.275000005960464
222 0.282499998807907
223 0.287499994039536
224 0.300000011920929
225 0.302500009536743
226 0.300000011920929
227 0.300000011920929
228 0.302500009536743
229 0.300000011920929
230 0.305000007152557
231 0.310000002384186
232 0.307500004768372
233 0.3125
234 0.317499995231628
235 0.324999988079071
236 0.327499985694885
237 0.33500000834465
238 0.33500000834465
239 0.337500005960464
240 0.340000003576279
241 0.337500005960464
242 0.349999994039536
243 0.35249999165535
244 0.354999989271164
245 0.360000014305115
246 0.362500011920929
247 0.367500007152557
248 0.377499997615814
249 0.382499992847443
250 0.387499988079071
251 0.402500003576279
252 0.407499998807907
253 0.412499994039536
254 0.417499989271164
255 0.419999986886978
256 0.422500014305115
257 0.425000011920929
258 0.435000002384186
259 0.444999992847443
260 0.457500010728836
261 0.469999998807907
262 0.472499996423721
263 0.485000014305115
264 0.479999989271164
265 0.479999989271164
266 0.492500007152557
267 0.5
268 0.509999990463257
269 0.514999985694885
270 0.517499983310699
271 0.522499978542328
272 0.522499978542328
273 0.527499973773956
274 0.532500028610229
275 0.529999971389771
276 0.540000021457672
277 0.547500014305115
278 0.5625
279 0.579999983310699
280 0.589999973773956
281 0.595000028610229
282 0.612500011920929
283 0.615000009536743
284 0.617500007152557
285 0.622500002384186
286 0.625
287 0.625
288 0.634999990463257
289 0.637499988079071
290 0.644999980926514
291 0.657500028610229
292 0.657500028610229
293 0.665000021457672
294 0.672500014305115
295 0.677500009536743
296 0.6875
297 0.689999997615814
298 0.694999992847443
299 0.697499990463257
300 0.704999983310699
301 0.704999983310699
302 0.709999978542328
303 0.707499980926514
304 0.704999983310699
305 0.704999983310699
306 0.712499976158142
307 0.722500026226044
308 0.725000023841858
309 0.730000019073486
310 0.730000019073486
311 0.730000019073486
312 0.732500016689301
313 0.730000019073486
314 0.737500011920929
315 0.740000009536743
316 0.742500007152557
317 0.742500007152557
318 0.742500007152557
319 0.75
320 0.757499992847443
321 0.757499992847443
322 0.767499983310699
323 0.772499978542328
324 0.769999980926514
325 0.782500028610229
326 0.785000026226044
327 0.790000021457672
328 0.785000026226044
329 0.785000026226044
330 0.779999971389771
331 0.782500028610229
332 0.782500028610229
333 0.785000026226044
334 0.787500023841858
335 0.787500023841858
336 0.792500019073486
337 0.800000011920929
338 0.800000011920929
339 0.805000007152557
340 0.805000007152557
341 0.807500004768372
342 0.807500004768372
343 0.8125
344 0.814999997615814
345 0.814999997615814
346 0.817499995231628
347 0.822499990463257
348 0.822499990463257
349 0.822499990463257
350 0.822499990463257
351 0.822499990463257
352 0.827499985694885
353 0.834999978542328
354 0.842499971389771
355 0.839999973773956
356 0.842499971389771
357 0.834999978542328
358 0.832499980926514
359 0.834999978542328
360 0.837499976158142
361 0.839999973773956
362 0.842499971389771
363 0.842499971389771
364 0.847500026226044
365 0.850000023841858
366 0.850000023841858
367 0.852500021457672
368 0.850000023841858
369 0.852500021457672
370 0.855000019073486
371 0.860000014305115
372 0.857500016689301
373 0.855000019073486
374 0.850000023841858
375 0.855000019073486
376 0.852500021457672
377 0.852500021457672
378 0.857500016689301
379 0.860000014305115
380 0.865000009536743
381 0.867500007152557
382 0.870000004768372
383 0.875
384 0.877499997615814
385 0.879999995231628
386 0.879999995231628
387 0.879999995231628
388 0.882499992847443
389 0.879999995231628
390 0.877499997615814
391 0.877499997615814
392 0.879999995231628
393 0.879999995231628
394 0.882499992847443
395 0.879999995231628
396 0.877499997615814
397 0.872500002384186
398 0.870000004768372
399 0.877499997615814
400 0.882499992847443
401 0.884999990463257
402 0.884999990463257
403 0.879999995231628
404 0.879999995231628
405 0.884999990463257
406 0.884999990463257
407 0.889999985694885
408 0.894999980926514
409 0.897499978542328
410 0.894999980926514
411 0.894999980926514
412 0.897499978542328
413 0.902499973773956
414 0.899999976158142
415 0.902499973773956
416 0.904999971389771
417 0.904999971389771
418 0.912500023841858
419 0.910000026226044
420 0.912500023841858
421 0.912500023841858
422 0.915000021457672
423 0.917500019073486
424 0.922500014305115
425 0.922500014305115
426 0.922500014305115
427 0.922500014305115
428 0.920000016689301
429 0.920000016689301
430 0.917500019073486
431 0.915000021457672
432 0.915000021457672
433 0.912500023841858
434 0.915000021457672
435 0.910000026226044
436 0.912500023841858
437 0.910000026226044
438 0.910000026226044
439 0.910000026226044
440 0.904999971389771
441 0.902499973773956
442 0.902499973773956
443 0.902499973773956
444 0.899999976158142
445 0.899999976158142
446 0.899999976158142
447 0.904999971389771
448 0.907500028610229
449 0.904999971389771
450 0.902499973773956
451 0.902499973773956
452 0.904999971389771
453 0.910000026226044
454 0.907500028610229
455 0.912500023841858
456 0.904999971389771
457 0.904999971389771
458 0.899999976158142
459 0.899999976158142
460 0.902499973773956
461 0.902499973773956
462 0.899999976158142
463 0.894999980926514
464 0.894999980926514
465 0.892499983310699
466 0.892499983310699
467 0.894999980926514
468 0.892499983310699
469 0.889999985694885
470 0.889999985694885
471 0.892499983310699
472 0.892499983310699
473 0.894999980926514
474 0.897499978542328
475 0.897499978542328
476 0.899999976158142
477 0.899999976158142
478 0.902499973773956
479 0.902499973773956
480 0.899999976158142
481 0.897499978542328
482 0.897499978542328
483 0.897499978542328
484 0.897499978542328
485 0.902499973773956
486 0.899999976158142
487 0.899999976158142
488 0.899999976158142
489 0.899999976158142
490 0.902499973773956
491 0.904999971389771
492 0.902499973773956
493 0.899999976158142
494 0.902499973773956
495 0.907500028610229
496 0.907500028610229
497 0.904999971389771
498 0.907500028610229
499 0.910000026226044
500 0.907500028610229
501 0.902499973773956
502 0.902499973773956
503 0.899999976158142
504 0.902499973773956
505 0.897499978542328
506 0.907500028610229
507 0.910000026226044
508 0.912500023841858
509 0.907500028610229
510 0.907500028610229
511 0.910000026226044
512 0.912500023841858
513 0.917500019073486
514 0.920000016689301
515 0.920000016689301
516 0.917500019073486
517 0.917500019073486
518 0.920000016689301
519 0.922500014305115
520 0.917500019073486
521 0.917500019073486
522 0.917500019073486
523 0.917500019073486
524 0.912500023841858
525 0.910000026226044
526 0.912500023841858
527 0.910000026226044
528 0.910000026226044
529 0.910000026226044
530 0.915000021457672
531 0.917500019073486
532 0.915000021457672
533 0.915000021457672
534 0.915000021457672
535 0.915000021457672
536 0.917500019073486
537 0.920000016689301
538 0.922500014305115
539 0.922500014305115
540 0.927500009536743
541 0.927500009536743
542 0.930000007152557
543 0.932500004768372
544 0.932500004768372
545 0.930000007152557
546 0.930000007152557
547 0.932500004768372
548 0.930000007152557
549 0.930000007152557
550 0.935000002384186
551 0.939999997615814
552 0.9375
553 0.942499995231628
554 0.942499995231628
555 0.947499990463257
556 0.947499990463257
557 0.947499990463257
558 0.947499990463257
559 0.952499985694885
560 0.952499985694885
561 0.949999988079071
562 0.947499990463257
563 0.947499990463257
564 0.947499990463257
565 0.949999988079071
566 0.949999988079071
567 0.949999988079071
568 0.949999988079071
569 0.952499985694885
570 0.957499980926514
571 0.957499980926514
572 0.954999983310699
573 0.954999983310699
574 0.954999983310699
575 0.957499980926514
576 0.957499980926514
577 0.962499976158142
578 0.959999978542328
579 0.959999978542328
580 0.957499980926514
581 0.959999978542328
582 0.962499976158142
583 0.962499976158142
584 0.962499976158142
585 0.962499976158142
586 0.962499976158142
587 0.959999978542328
588 0.959999978542328
589 0.959999978542328
590 0.959999978542328
591 0.959999978542328
592 0.957499980926514
593 0.957499980926514
594 0.957499980926514
595 0.959999978542328
596 0.962499976158142
597 0.959999978542328
598 0.962499976158142
599 0.962499976158142
600 0.962499976158142
601 0.959999978542328
602 0.957499980926514
603 0.954999983310699
604 0.954999983310699
605 0.952499985694885
606 0.952499985694885
607 0.949999988079071
608 0.952499985694885
609 0.949999988079071
610 0.949999988079071
611 0.952499985694885
612 0.949999988079071
613 0.949999988079071
614 0.949999988079071
615 0.947499990463257
616 0.949999988079071
617 0.949999988079071
618 0.947499990463257
619 0.947499990463257
620 0.947499990463257
621 0.944999992847443
622 0.944999992847443
623 0.944999992847443
624 0.947499990463257
625 0.942499995231628
626 0.939999997615814
627 0.9375
628 0.9375
629 0.9375
630 0.935000002384186
631 0.935000002384186
632 0.935000002384186
633 0.935000002384186
634 0.932500004768372
635 0.932500004768372
636 0.932500004768372
637 0.935000002384186
638 0.932500004768372
639 0.932500004768372
640 0.932500004768372
641 0.932500004768372
642 0.935000002384186
643 0.932500004768372
644 0.932500004768372
645 0.932500004768372
646 0.932500004768372
647 0.935000002384186
648 0.935000002384186
649 0.932500004768372
650 0.932500004768372
651 0.935000002384186
652 0.9375
653 0.939999997615814
654 0.939999997615814
655 0.942499995231628
656 0.942499995231628
657 0.944999992847443
658 0.942499995231628
659 0.944999992847443
660 0.944999992847443
661 0.944999992847443
662 0.949999988079071
663 0.949999988079071
664 0.949999988079071
665 0.952499985694885
666 0.952499985694885
667 0.949999988079071
668 0.949999988079071
669 0.949999988079071
670 0.949999988079071
671 0.952499985694885
672 0.954999983310699
673 0.952499985694885
674 0.954999983310699
675 0.959999978542328
676 0.962499976158142
677 0.964999973773956
678 0.967499971389771
679 0.967499971389771
680 0.972500026226044
681 0.972500026226044
682 0.972500026226044
683 0.975000023841858
684 0.977500021457672
685 0.977500021457672
686 0.975000023841858
687 0.975000023841858
688 0.975000023841858
689 0.975000023841858
690 0.975000023841858
691 0.975000023841858
692 0.975000023841858
693 0.977500021457672
694 0.975000023841858
695 0.975000023841858
696 0.975000023841858
697 0.975000023841858
698 0.975000023841858
699 0.977500021457672
700 0.977500021457672
701 0.977500021457672
702 0.977500021457672
703 0.977500021457672
704 0.977500021457672
705 0.975000023841858
706 0.972500026226044
707 0.972500026226044
708 0.972500026226044
709 0.972500026226044
710 0.967499971389771
711 0.967499971389771
712 0.967499971389771
713 0.967499971389771
714 0.967499971389771
715 0.964999973773956
716 0.964999973773956
717 0.967499971389771
718 0.970000028610229
719 0.967499971389771
720 0.967499971389771
721 0.967499971389771
722 0.967499971389771
723 0.970000028610229
724 0.970000028610229
725 0.967499971389771
726 0.964999973773956
727 0.962499976158142
728 0.962499976158142
729 0.962499976158142
730 0.962499976158142
731 0.962499976158142
732 0.962499976158142
733 0.959999978542328
734 0.959999978542328
735 0.959999978542328
736 0.962499976158142
737 0.962499976158142
738 0.964999973773956
739 0.967499971389771
740 0.967499971389771
741 0.967499971389771
742 0.967499971389771
743 0.962499976158142
744 0.964999973773956
745 0.964999973773956
746 0.962499976158142
747 0.957499980926514
748 0.957499980926514
749 0.957499980926514
750 0.954999983310699
751 0.954999983310699
752 0.957499980926514
753 0.957499980926514
754 0.957499980926514
755 0.959999978542328
756 0.962499976158142
757 0.962499976158142
758 0.962499976158142
759 0.962499976158142
760 0.967499971389771
761 0.962499976158142
762 0.962499976158142
763 0.959999978542328
764 0.957499980926514
765 0.959999978542328
766 0.957499980926514
767 0.957499980926514
768 0.957499980926514
769 0.957499980926514
770 0.957499980926514
771 0.957499980926514
772 0.957499980926514
773 0.957499980926514
774 0.954999983310699
775 0.957499980926514
776 0.957499980926514
777 0.959999978542328
778 0.959999978542328
779 0.959999978542328
780 0.959999978542328
781 0.959999978542328
782 0.957499980926514
783 0.959999978542328
784 0.959999978542328
785 0.957499980926514
786 0.957499980926514
787 0.957499980926514
788 0.954999983310699
789 0.954999983310699
790 0.954999983310699
791 0.954999983310699
792 0.954999983310699
793 0.957499980926514
794 0.957499980926514
795 0.957499980926514
796 0.957499980926514
797 0.962499976158142
798 0.962499976158142
799 0.962499976158142
800 0.964999973773956
801 0.964999973773956
802 0.962499976158142
803 0.962499976158142
804 0.962499976158142
805 0.962499976158142
806 0.962499976158142
807 0.962499976158142
808 0.964999973773956
809 0.964999973773956
810 0.964999973773956
811 0.970000028610229
812 0.970000028610229
813 0.970000028610229
814 0.972500026226044
815 0.972500026226044
816 0.975000023841858
817 0.975000023841858
818 0.975000023841858
819 0.977500021457672
820 0.977500021457672
821 0.977500021457672
822 0.977500021457672
823 0.977500021457672
824 0.980000019073486
825 0.980000019073486
826 0.982500016689301
827 0.980000019073486
828 0.980000019073486
829 0.980000019073486
830 0.980000019073486
831 0.980000019073486
832 0.982500016689301
833 0.982500016689301
834 0.980000019073486
835 0.982500016689301
836 0.982500016689301
837 0.980000019073486
838 0.982500016689301
839 0.982500016689301
840 0.975000023841858
841 0.975000023841858
842 0.972500026226044
843 0.975000023841858
844 0.972500026226044
845 0.970000028610229
846 0.972500026226044
847 0.972500026226044
848 0.972500026226044
849 0.972500026226044
850 0.970000028610229
851 0.970000028610229
852 0.967499971389771
853 0.967499971389771
854 0.964999973773956
855 0.962499976158142
856 0.962499976158142
857 0.962499976158142
858 0.962499976158142
859 0.962499976158142
860 0.962499976158142
861 0.962499976158142
862 0.959999978542328
863 0.962499976158142
864 0.957499980926514
865 0.957499980926514
866 0.957499980926514
867 0.957499980926514
868 0.957499980926514
869 0.954999983310699
870 0.952499985694885
871 0.952499985694885
872 0.952499985694885
873 0.949999988079071
874 0.949999988079071
875 0.949999988079071
876 0.949999988079071
877 0.949999988079071
878 0.949999988079071
879 0.949999988079071
880 0.949999988079071
881 0.949999988079071
882 0.949999988079071
883 0.947499990463257
884 0.949999988079071
885 0.949999988079071
886 0.949999988079071
887 0.952499985694885
888 0.949999988079071
889 0.947499990463257
890 0.954999983310699
891 0.952499985694885
892 0.954999983310699
893 0.954999983310699
894 0.957499980926514
895 0.959999978542328
896 0.959999978542328
897 0.959999978542328
898 0.959999978542328
899 0.959999978542328
900 0.959999978542328
901 0.959999978542328
902 0.964999973773956
903 0.964999973773956
904 0.967499971389771
905 0.970000028610229
906 0.967499971389771
907 0.967499971389771
908 0.967499971389771
909 0.964999973773956
910 0.962499976158142
911 0.962499976158142
912 0.962499976158142
913 0.959999978542328
914 0.962499976158142
915 0.962499976158142
916 0.962499976158142
917 0.962499976158142
918 0.962499976158142
919 0.964999973773956
920 0.967499971389771
921 0.967499971389771
922 0.967499971389771
923 0.970000028610229
924 0.970000028610229
925 0.970000028610229
926 0.970000028610229
927 0.972500026226044
928 0.972500026226044
929 0.972500026226044
930 0.970000028610229
931 0.970000028610229
932 0.970000028610229
933 0.972500026226044
934 0.972500026226044
935 0.970000028610229
936 0.970000028610229
937 0.967499971389771
938 0.970000028610229
939 0.972500026226044
940 0.970000028610229
941 0.972500026226044
942 0.972500026226044
943 0.972500026226044
944 0.970000028610229
945 0.970000028610229
946 0.970000028610229
947 0.970000028610229
948 0.970000028610229
949 0.970000028610229
950 0.972500026226044
951 0.972500026226044
952 0.972500026226044
953 0.972500026226044
954 0.972500026226044
955 0.972500026226044
956 0.975000023841858
957 0.975000023841858
958 0.972500026226044
959 0.975000023841858
960 0.977500021457672
961 0.977500021457672
962 0.980000019073486
963 0.980000019073486
964 0.982500016689301
965 0.982500016689301
966 0.982500016689301
967 0.982500016689301
968 0.982500016689301
969 0.982500016689301
970 0.982500016689301
971 0.982500016689301
972 0.982500016689301
973 0.982500016689301
974 0.982500016689301
975 0.980000019073486
976 0.980000019073486
977 0.980000019073486
978 0.980000019073486
979 0.980000019073486
980 0.982500016689301
981 0.982500016689301
982 0.982500016689301
983 0.982500016689301
984 0.982500016689301
985 0.982500016689301
986 0.982500016689301
987 0.985000014305115
988 0.985000014305115
989 0.985000014305115
990 0.987500011920929
991 0.985000014305115
992 0.985000014305115
993 0.985000014305115
994 0.987500011920929
995 0.985000014305115
996 0.985000014305115
997 0.985000014305115
998 0.985000014305115
999 0.982500016689301
1000 0.982500016689301
1001 0.982500016689301
1002 0.980000019073486
1003 0.977500021457672
1004 0.975000023841858
1005 0.975000023841858
1006 0.975000023841858
1007 0.972500026226044
1008 0.975000023841858
1009 0.975000023841858
1010 0.975000023841858
1011 0.975000023841858
1012 0.975000023841858
1013 0.977500021457672
1014 0.977500021457672
1015 0.977500021457672
1016 0.975000023841858
1017 0.975000023841858
1018 0.975000023841858
1019 0.975000023841858
1020 0.975000023841858
1021 0.975000023841858
1022 0.975000023841858
1023 0.975000023841858
1024 0.975000023841858
1025 0.977500021457672
1026 0.977500021457672
1027 0.977500021457672
1028 0.977500021457672
1029 0.977500021457672
1030 0.977500021457672
1031 0.977500021457672
1032 0.975000023841858
1033 0.972500026226044
1034 0.972500026226044
1035 0.972500026226044
1036 0.972500026226044
1037 0.972500026226044
1038 0.972500026226044
1039 0.970000028610229
1040 0.970000028610229
1041 0.970000028610229
1042 0.967499971389771
1043 0.967499971389771
1044 0.967499971389771
1045 0.970000028610229
1046 0.970000028610229
1047 0.970000028610229
1048 0.970000028610229
1049 0.972500026226044
1050 0.972500026226044
1051 0.972500026226044
1052 0.975000023841858
1053 0.975000023841858
1054 0.977500021457672
1055 0.977500021457672
1056 0.977500021457672
1057 0.980000019073486
1058 0.980000019073486
1059 0.980000019073486
1060 0.980000019073486
1061 0.980000019073486
1062 0.980000019073486
1063 0.980000019073486
1064 0.980000019073486
1065 0.980000019073486
1066 0.982500016689301
1067 0.980000019073486
1068 0.980000019073486
1069 0.980000019073486
1070 0.980000019073486
1071 0.980000019073486
1072 0.980000019073486
1073 0.980000019073486
1074 0.980000019073486
1075 0.980000019073486
1076 0.977500021457672
1077 0.977500021457672
1078 0.977500021457672
1079 0.977500021457672
1080 0.977500021457672
1081 0.975000023841858
1082 0.977500021457672
1083 0.977500021457672
1084 0.977500021457672
1085 0.980000019073486
1086 0.980000019073486
1087 0.980000019073486
1088 0.980000019073486
1089 0.982500016689301
1090 0.977500021457672
1091 0.977500021457672
1092 0.980000019073486
1093 0.980000019073486
1094 0.980000019073486
1095 0.980000019073486
1096 0.980000019073486
1097 0.980000019073486
1098 0.980000019073486
1099 0.980000019073486
1100 0.980000019073486
1101 0.980000019073486
1102 0.980000019073486
1103 0.982500016689301
1104 0.980000019073486
1105 0.980000019073486
1106 0.980000019073486
1107 0.980000019073486
1108 0.977500021457672
1109 0.977500021457672
1110 0.975000023841858
1111 0.972500026226044
1112 0.972500026226044
1113 0.972500026226044
1114 0.972500026226044
1115 0.972500026226044
1116 0.970000028610229
1117 0.972500026226044
1118 0.972500026226044
1119 0.972500026226044
1120 0.972500026226044
1121 0.972500026226044
1122 0.972500026226044
1123 0.972500026226044
1124 0.972500026226044
1125 0.970000028610229
1126 0.972500026226044
1127 0.972500026226044
1128 0.970000028610229
1129 0.970000028610229
1130 0.970000028610229
1131 0.972500026226044
1132 0.972500026226044
1133 0.975000023841858
1134 0.975000023841858
1135 0.975000023841858
1136 0.975000023841858
1137 0.975000023841858
1138 0.975000023841858
1139 0.975000023841858
1140 0.980000019073486
1141 0.980000019073486
1142 0.980000019073486
1143 0.980000019073486
1144 0.980000019073486
1145 0.977500021457672
1146 0.977500021457672
1147 0.977500021457672
1148 0.977500021457672
1149 0.977500021457672
1150 0.977500021457672
1151 0.977500021457672
1152 0.977500021457672
1153 0.977500021457672
1154 0.980000019073486
1155 0.980000019073486
1156 0.980000019073486
1157 0.980000019073486
1158 0.982500016689301
1159 0.982500016689301
1160 0.985000014305115
1161 0.987500011920929
1162 0.987500011920929
1163 0.985000014305115
1164 0.985000014305115
1165 0.985000014305115
1166 0.987500011920929
1167 0.987500011920929
1168 0.987500011920929
1169 0.987500011920929
1170 0.987500011920929
1171 0.987500011920929
1172 0.987500011920929
1173 0.987500011920929
1174 0.987500011920929
1175 0.990000009536743
1176 0.987500011920929
1177 0.987500011920929
1178 0.990000009536743
1179 0.990000009536743
1180 0.990000009536743
1181 0.990000009536743
1182 0.990000009536743
1183 0.990000009536743
1184 0.990000009536743
1185 0.990000009536743
1186 0.990000009536743
1187 0.990000009536743
1188 0.990000009536743
1189 0.990000009536743
1190 0.990000009536743
1191 0.992500007152557
1192 0.990000009536743
1193 0.990000009536743
1194 0.990000009536743
1195 0.992500007152557
1196 0.992500007152557
1197 0.990000009536743
1198 0.990000009536743
1199 0.990000009536743
1200 0.990000009536743
1201 0.990000009536743
1202 0.990000009536743
1203 0.990000009536743
1204 0.990000009536743
1205 0.990000009536743
1206 0.987500011920929
1207 0.987500011920929
1208 0.987500011920929
1209 0.987500011920929
1210 0.987500011920929
1211 0.987500011920929
1212 0.987500011920929
1213 0.990000009536743
1214 0.990000009536743
1215 0.990000009536743
1216 0.990000009536743
1217 0.990000009536743
1218 0.990000009536743
1219 0.990000009536743
1220 0.990000009536743
1221 0.990000009536743
1222 0.990000009536743
1223 0.990000009536743
1224 0.990000009536743
1225 0.990000009536743
1226 0.992500007152557
1227 0.992500007152557
1228 0.992500007152557
1229 0.992500007152557
1230 0.992500007152557
1231 0.990000009536743
1232 0.990000009536743
1233 0.990000009536743
1234 0.990000009536743
1235 0.987500011920929
1236 0.985000014305115
1237 0.985000014305115
1238 0.985000014305115
1239 0.985000014305115
};
\addlegendentry{4-3}
\addplot [semithick, green!50.0!black]
table {%
0 0
1 0.125
2 0.0833333358168602
3 0.0625
4 0.0750000029802322
5 0.0833333358168602
6 0.0714285746216774
7 0.0625
8 0.0555555559694767
9 0.0500000007450581
10 0.0568181835114956
11 0.0520833320915699
12 0.0480769239366055
13 0.0446428582072258
14 0.0416666679084301
15 0.046875
16 0.0514705888926983
17 0.0625
18 0.0592105276882648
19 0.0562499985098839
20 0.0595238097012043
21 0.0568181835114956
22 0.0543478243052959
23 0.0520833320915699
24 0.0500000007450581
25 0.0625
26 0.0601851865649223
27 0.0625
28 0.0603448264300823
29 0.058333333581686
30 0.0564516112208366
31 0.05859375
32 0.0568181835114956
33 0.0551470592617989
34 0.0535714291036129
35 0.0520833320915699
36 0.0506756752729416
37 0.0493421070277691
38 0.0512820519506931
39 0.0500000007450581
40 0.0518292672932148
41 0.0505952388048172
42 0.049418605864048
43 0.0482954531908035
44 0.0472222231328487
45 0.0489130429923534
46 0.0531914904713631
47 0.0546875
48 0.0535714291036129
49 0.0549999997019768
50 0.0575000010430813
51 0.0575000010430813
52 0.0599999986588955
53 0.0649999976158142
54 0.0649999976158142
55 0.0625
56 0.067500002682209
57 0.067500002682209
58 0.0750000029802322
59 0.0799999982118607
60 0.0825000032782555
61 0.0850000008940697
62 0.0874999985098839
63 0.0900000035762787
64 0.100000001490116
65 0.100000001490116
66 0.104999996721745
67 0.10249999910593
68 0.10249999910593
69 0.107500001788139
70 0.107500001788139
71 0.115000002086163
72 0.119999997317791
73 0.125
74 0.132499992847443
75 0.127499997615814
76 0.129999995231628
77 0.135000005364418
78 0.140000000596046
79 0.142499998211861
80 0.142499998211861
81 0.142499998211861
82 0.150000005960464
83 0.152500003576279
84 0.159999996423721
85 0.159999996423721
86 0.159999996423721
87 0.170000001788139
88 0.172499999403954
89 0.174999997019768
90 0.177499994635582
91 0.185000002384186
92 0.185000002384186
93 0.1875
94 0.189999997615814
95 0.189999997615814
96 0.1875
97 0.1875
98 0.192499995231628
99 0.197500005364418
100 0.200000002980232
101 0.200000002980232
102 0.202500000596046
103 0.197500005364418
104 0.197500005364418
105 0.202500000596046
106 0.200000002980232
107 0.204999998211861
108 0.207499995827675
109 0.212500005960464
110 0.209999993443489
111 0.209999993443489
112 0.209999993443489
113 0.215000003576279
114 0.207499995827675
115 0.207499995827675
116 0.204999998211861
117 0.202500000596046
118 0.209999993443489
119 0.209999993443489
120 0.209999993443489
121 0.204999998211861
122 0.202500000596046
123 0.204999998211861
124 0.200000002980232
125 0.202500000596046
126 0.200000002980232
127 0.202500000596046
128 0.200000002980232
129 0.202500000596046
130 0.207499995827675
131 0.204999998211861
132 0.200000002980232
133 0.202500000596046
134 0.194999992847443
135 0.200000002980232
136 0.204999998211861
137 0.204999998211861
138 0.209999993443489
139 0.209999993443489
140 0.215000003576279
141 0.215000003576279
142 0.217500001192093
143 0.219999998807907
144 0.219999998807907
145 0.227500006556511
146 0.232500001788139
147 0.237499997019768
148 0.242500007152557
149 0.242500007152557
150 0.237499997019768
151 0.237499997019768
152 0.234999999403954
153 0.239999994635582
154 0.239999994635582
155 0.237499997019768
156 0.245000004768372
157 0.252499997615814
158 0.247500002384186
159 0.239999994635582
160 0.242500007152557
161 0.242500007152557
162 0.247500002384186
163 0.247500002384186
164 0.259999990463257
165 0.264999985694885
166 0.270000010728836
167 0.280000001192093
168 0.277500003576279
169 0.277500003576279
170 0.284999996423721
171 0.287499994039536
172 0.297500014305115
173 0.297500014305115
174 0.305000007152557
175 0.310000002384186
176 0.317499995231628
177 0.3125
178 0.314999997615814
179 0.322499990463257
180 0.330000013113022
181 0.342500001192093
182 0.347499996423721
183 0.344999998807907
184 0.35249999165535
185 0.354999989271164
186 0.349999994039536
187 0.35249999165535
188 0.35249999165535
189 0.357499986886978
190 0.35249999165535
191 0.35249999165535
192 0.362500011920929
193 0.367500007152557
194 0.367500007152557
195 0.365000009536743
196 0.365000009536743
197 0.365000009536743
198 0.360000014305115
199 0.360000014305115
200 0.367500007152557
201 0.370000004768372
202 0.372500002384186
203 0.372500002384186
204 0.379999995231628
205 0.387499988079071
206 0.382499992847443
207 0.379999995231628
208 0.384999990463257
209 0.389999985694885
210 0.395000010728836
211 0.395000010728836
212 0.400000005960464
213 0.39750000834465
214 0.392500013113022
215 0.39750000834465
216 0.392500013113022
217 0.392500013113022
218 0.39750000834465
219 0.39750000834465
220 0.395000010728836
221 0.392500013113022
222 0.382499992847443
223 0.384999990463257
224 0.382499992847443
225 0.384999990463257
226 0.389999985694885
227 0.395000010728836
228 0.39750000834465
229 0.392500013113022
230 0.387499988079071
231 0.379999995231628
232 0.382499992847443
233 0.387499988079071
234 0.389999985694885
235 0.392500013113022
236 0.402500003576279
237 0.395000010728836
238 0.39750000834465
239 0.395000010728836
240 0.400000005960464
241 0.400000005960464
242 0.392500013113022
243 0.384999990463257
244 0.395000010728836
245 0.400000005960464
246 0.39750000834465
247 0.400000005960464
248 0.407499998807907
249 0.417499989271164
250 0.41499999165535
251 0.41499999165535
252 0.425000011920929
253 0.432500004768372
254 0.430000007152557
255 0.427500009536743
256 0.435000002384186
257 0.439999997615814
258 0.439999997615814
259 0.444999992847443
260 0.442499995231628
261 0.452499985694885
262 0.449999988079071
263 0.449999988079071
264 0.449999988079071
265 0.447499990463257
266 0.452499985694885
267 0.455000013113022
268 0.46000000834465
269 0.469999998807907
270 0.47749999165535
271 0.492500007152557
272 0.504999995231628
273 0.509999990463257
274 0.507499992847443
275 0.507499992847443
276 0.512499988079071
277 0.519999980926514
278 0.524999976158142
279 0.529999971389771
280 0.535000026226044
281 0.545000016689301
282 0.550000011920929
283 0.552500009536743
284 0.552500009536743
285 0.555000007152557
286 0.5625
287 0.572499990463257
288 0.577499985694885
289 0.577499985694885
290 0.587499976158142
291 0.597500026226044
292 0.610000014305115
293 0.625
294 0.627499997615814
295 0.622500002384186
296 0.629999995231628
297 0.637499988079071
298 0.639999985694885
299 0.637499988079071
300 0.647499978542328
301 0.657500028610229
302 0.654999971389771
303 0.660000026226044
304 0.670000016689301
305 0.675000011920929
306 0.677500009536743
307 0.680000007152557
308 0.689999997615814
309 0.694999992847443
310 0.707499980926514
311 0.704999983310699
312 0.712499976158142
313 0.722500026226044
314 0.725000023841858
315 0.730000019073486
316 0.737500011920929
317 0.737500011920929
318 0.737500011920929
319 0.742500007152557
320 0.747500002384186
321 0.75
322 0.752499997615814
323 0.757499992847443
324 0.764999985694885
325 0.764999985694885
326 0.764999985694885
327 0.764999985694885
328 0.769999980926514
329 0.774999976158142
330 0.779999971389771
331 0.777499973773956
332 0.774999976158142
333 0.779999971389771
334 0.779999971389771
335 0.787500023841858
336 0.782500028610229
337 0.779999971389771
338 0.779999971389771
339 0.795000016689301
340 0.790000021457672
341 0.792500019073486
342 0.790000021457672
343 0.792500019073486
344 0.795000016689301
345 0.805000007152557
346 0.810000002384186
347 0.810000002384186
348 0.8125
349 0.817499995231628
350 0.817499995231628
351 0.819999992847443
352 0.822499990463257
353 0.819999992847443
354 0.819999992847443
355 0.822499990463257
356 0.824999988079071
357 0.827499985694885
358 0.819999992847443
359 0.819999992847443
360 0.8125
361 0.817499995231628
362 0.810000002384186
363 0.810000002384186
364 0.814999997615814
365 0.814999997615814
366 0.814999997615814
367 0.819999992847443
368 0.824999988079071
369 0.822499990463257
370 0.814999997615814
371 0.8125
372 0.814999997615814
373 0.802500009536743
374 0.810000002384186
375 0.814999997615814
376 0.814999997615814
377 0.814999997615814
378 0.817499995231628
379 0.817499995231628
380 0.817499995231628
381 0.824999988079071
382 0.832499980926514
383 0.834999978542328
384 0.842499971389771
385 0.842499971389771
386 0.845000028610229
387 0.847500026226044
388 0.845000028610229
389 0.839999973773956
390 0.839999973773956
391 0.837499976158142
392 0.842499971389771
393 0.842499971389771
394 0.845000028610229
395 0.839999973773956
396 0.842499971389771
397 0.837499976158142
398 0.832499980926514
399 0.829999983310699
400 0.834999978542328
401 0.834999978542328
402 0.839999973773956
403 0.839999973773956
404 0.837499976158142
405 0.839999973773956
406 0.837499976158142
407 0.832499980926514
408 0.839999973773956
409 0.842499971389771
410 0.845000028610229
411 0.847500026226044
412 0.857500016689301
413 0.860000014305115
414 0.862500011920929
415 0.862500011920929
416 0.862500011920929
417 0.860000014305115
418 0.860000014305115
419 0.860000014305115
420 0.865000009536743
421 0.865000009536743
422 0.862500011920929
423 0.872500002384186
424 0.872500002384186
425 0.870000004768372
426 0.870000004768372
427 0.872500002384186
428 0.872500002384186
429 0.875
430 0.875
431 0.872500002384186
432 0.870000004768372
433 0.867500007152557
434 0.870000004768372
435 0.865000009536743
436 0.867500007152557
437 0.867500007152557
438 0.872500002384186
439 0.877499997615814
440 0.875
441 0.877499997615814
442 0.875
443 0.875
444 0.872500002384186
445 0.879999995231628
446 0.877499997615814
447 0.884999990463257
448 0.889999985694885
449 0.892499983310699
450 0.889999985694885
451 0.889999985694885
452 0.887499988079071
453 0.889999985694885
454 0.894999980926514
455 0.897499978542328
456 0.899999976158142
457 0.902499973773956
458 0.902499973773956
459 0.902499973773956
460 0.904999971389771
461 0.907500028610229
462 0.904999971389771
463 0.902499973773956
464 0.902499973773956
465 0.907500028610229
466 0.907500028610229
467 0.910000026226044
468 0.904999971389771
469 0.904999971389771
470 0.904999971389771
471 0.907500028610229
472 0.907500028610229
473 0.907500028610229
474 0.907500028610229
475 0.910000026226044
476 0.912500023841858
477 0.912500023841858
478 0.907500028610229
479 0.904999971389771
480 0.907500028610229
481 0.910000026226044
482 0.912500023841858
483 0.917500019073486
484 0.917500019073486
485 0.922500014305115
486 0.925000011920929
487 0.930000007152557
488 0.925000011920929
489 0.922500014305115
490 0.927500009536743
491 0.925000011920929
492 0.927500009536743
493 0.925000011920929
494 0.925000011920929
495 0.925000011920929
496 0.925000011920929
497 0.925000011920929
498 0.922500014305115
499 0.922500014305115
500 0.922500014305115
501 0.922500014305115
502 0.925000011920929
503 0.927500009536743
504 0.927500009536743
505 0.927500009536743
506 0.927500009536743
507 0.930000007152557
508 0.930000007152557
509 0.930000007152557
510 0.932500004768372
511 0.932500004768372
512 0.932500004768372
513 0.935000002384186
514 0.932500004768372
515 0.927500009536743
516 0.930000007152557
517 0.932500004768372
518 0.935000002384186
519 0.9375
520 0.939999997615814
521 0.939999997615814
522 0.942499995231628
523 0.944999992847443
524 0.944999992847443
525 0.942499995231628
526 0.939999997615814
527 0.9375
528 0.9375
529 0.935000002384186
530 0.935000002384186
531 0.935000002384186
532 0.935000002384186
533 0.935000002384186
534 0.935000002384186
535 0.935000002384186
536 0.930000007152557
537 0.927500009536743
538 0.927500009536743
539 0.930000007152557
540 0.932500004768372
541 0.932500004768372
542 0.932500004768372
543 0.932500004768372
544 0.932500004768372
545 0.930000007152557
546 0.932500004768372
547 0.932500004768372
548 0.9375
549 0.9375
550 0.939999997615814
551 0.939999997615814
552 0.939999997615814
553 0.939999997615814
554 0.9375
555 0.9375
556 0.9375
557 0.9375
558 0.9375
559 0.9375
560 0.9375
561 0.9375
562 0.939999997615814
563 0.942499995231628
564 0.944999992847443
565 0.949999988079071
566 0.949999988079071
567 0.949999988079071
568 0.949999988079071
569 0.947499990463257
570 0.944999992847443
571 0.944999992847443
572 0.944999992847443
573 0.944999992847443
574 0.944999992847443
575 0.949999988079071
576 0.952499985694885
577 0.952499985694885
578 0.957499980926514
579 0.962499976158142
580 0.962499976158142
581 0.962499976158142
582 0.962499976158142
583 0.962499976158142
584 0.959999978542328
585 0.957499980926514
586 0.962499976158142
587 0.964999973773956
588 0.967499971389771
589 0.967499971389771
590 0.967499971389771
591 0.970000028610229
592 0.970000028610229
593 0.970000028610229
594 0.972500026226044
595 0.970000028610229
596 0.967499971389771
597 0.964999973773956
598 0.962499976158142
599 0.959999978542328
600 0.954999983310699
601 0.954999983310699
602 0.952499985694885
603 0.952499985694885
604 0.954999983310699
605 0.952499985694885
606 0.949999988079071
607 0.949999988079071
608 0.949999988079071
609 0.947499990463257
610 0.944999992847443
611 0.944999992847443
612 0.944999992847443
613 0.944999992847443
614 0.944999992847443
615 0.944999992847443
616 0.944999992847443
617 0.942499995231628
618 0.944999992847443
619 0.947499990463257
620 0.947499990463257
621 0.944999992847443
622 0.944999992847443
623 0.944999992847443
624 0.944999992847443
625 0.942499995231628
626 0.942499995231628
627 0.944999992847443
628 0.944999992847443
629 0.944999992847443
630 0.939999997615814
631 0.939999997615814
632 0.939999997615814
633 0.939999997615814
634 0.942499995231628
635 0.942499995231628
636 0.9375
637 0.935000002384186
638 0.9375
639 0.932500004768372
640 0.932500004768372
641 0.927500009536743
642 0.927500009536743
643 0.930000007152557
644 0.930000007152557
645 0.932500004768372
646 0.932500004768372
647 0.930000007152557
648 0.932500004768372
649 0.935000002384186
650 0.939999997615814
651 0.939999997615814
652 0.939999997615814
653 0.939999997615814
654 0.939999997615814
655 0.942499995231628
656 0.947499990463257
657 0.947499990463257
658 0.947499990463257
659 0.949999988079071
660 0.949999988079071
661 0.947499990463257
662 0.947499990463257
663 0.947499990463257
664 0.947499990463257
665 0.947499990463257
666 0.947499990463257
667 0.949999988079071
668 0.947499990463257
669 0.944999992847443
670 0.944999992847443
671 0.944999992847443
672 0.939999997615814
673 0.939999997615814
674 0.9375
675 0.939999997615814
676 0.939999997615814
677 0.939999997615814
678 0.935000002384186
679 0.935000002384186
680 0.939999997615814
681 0.939999997615814
682 0.939999997615814
683 0.939999997615814
684 0.939999997615814
685 0.939999997615814
686 0.944999992847443
687 0.947499990463257
688 0.947499990463257
689 0.949999988079071
690 0.949999988079071
691 0.954999983310699
692 0.954999983310699
693 0.952499985694885
694 0.952499985694885
695 0.954999983310699
696 0.957499980926514
697 0.962499976158142
698 0.959999978542328
699 0.959999978542328
700 0.959999978542328
701 0.959999978542328
702 0.959999978542328
703 0.959999978542328
704 0.959999978542328
705 0.959999978542328
706 0.957499980926514
707 0.954999983310699
708 0.954999983310699
709 0.954999983310699
710 0.957499980926514
711 0.959999978542328
712 0.959999978542328
713 0.957499980926514
714 0.957499980926514
715 0.957499980926514
716 0.957499980926514
717 0.957499980926514
718 0.959999978542328
719 0.962499976158142
720 0.962499976158142
721 0.964999973773956
722 0.970000028610229
723 0.964999973773956
724 0.967499971389771
725 0.964999973773956
726 0.964999973773956
727 0.964999973773956
728 0.970000028610229
729 0.970000028610229
730 0.967499971389771
731 0.964999973773956
732 0.959999978542328
733 0.959999978542328
734 0.957499980926514
735 0.959999978542328
736 0.957499980926514
737 0.952499985694885
738 0.952499985694885
739 0.954999983310699
740 0.954999983310699
741 0.954999983310699
742 0.952499985694885
743 0.954999983310699
744 0.954999983310699
745 0.952499985694885
746 0.952499985694885
747 0.949999988079071
748 0.949999988079071
749 0.949999988079071
750 0.949999988079071
751 0.949999988079071
752 0.952499985694885
753 0.952499985694885
754 0.952499985694885
755 0.952499985694885
756 0.954999983310699
757 0.954999983310699
758 0.954999983310699
759 0.952499985694885
760 0.952499985694885
761 0.952499985694885
762 0.952499985694885
763 0.954999983310699
764 0.954999983310699
765 0.954999983310699
766 0.954999983310699
767 0.954999983310699
768 0.954999983310699
769 0.954999983310699
770 0.957499980926514
771 0.954999983310699
772 0.954999983310699
773 0.959999978542328
774 0.957499980926514
775 0.959999978542328
776 0.959999978542328
777 0.957499980926514
778 0.954999983310699
779 0.952499985694885
780 0.954999983310699
781 0.957499980926514
782 0.962499976158142
783 0.962499976158142
784 0.964999973773956
785 0.964999973773956
786 0.967499971389771
787 0.972500026226044
788 0.967499971389771
789 0.967499971389771
790 0.964999973773956
791 0.964999973773956
792 0.967499971389771
793 0.959999978542328
794 0.957499980926514
795 0.959999978542328
796 0.957499980926514
797 0.959999978542328
798 0.962499976158142
799 0.962499976158142
800 0.962499976158142
801 0.962499976158142
802 0.962499976158142
803 0.962499976158142
804 0.962499976158142
805 0.962499976158142
806 0.962499976158142
807 0.964999973773956
808 0.964999973773956
809 0.967499971389771
810 0.967499971389771
811 0.964999973773956
812 0.964999973773956
813 0.962499976158142
814 0.959999978542328
815 0.959999978542328
816 0.959999978542328
817 0.957499980926514
818 0.957499980926514
819 0.954999983310699
820 0.954999983310699
821 0.957499980926514
822 0.952499985694885
823 0.952499985694885
824 0.954999983310699
825 0.954999983310699
826 0.954999983310699
827 0.954999983310699
828 0.957499980926514
829 0.959999978542328
830 0.959999978542328
831 0.959999978542328
832 0.959999978542328
833 0.959999978542328
834 0.959999978542328
835 0.957499980926514
836 0.954999983310699
837 0.954999983310699
838 0.957499980926514
839 0.957499980926514
840 0.957499980926514
841 0.957499980926514
842 0.957499980926514
843 0.964999973773956
844 0.962499976158142
845 0.962499976158142
846 0.964999973773956
847 0.964999973773956
848 0.962499976158142
849 0.962499976158142
850 0.962499976158142
851 0.959999978542328
852 0.959999978542328
853 0.959999978542328
854 0.959999978542328
855 0.957499980926514
856 0.954999983310699
857 0.954999983310699
858 0.952499985694885
859 0.947499990463257
860 0.942499995231628
861 0.942499995231628
862 0.939999997615814
863 0.942499995231628
864 0.942499995231628
865 0.942499995231628
866 0.939999997615814
867 0.942499995231628
868 0.932500004768372
869 0.932500004768372
870 0.930000007152557
871 0.930000007152557
872 0.935000002384186
873 0.935000002384186
874 0.930000007152557
875 0.925000011920929
876 0.920000016689301
877 0.922500014305115
878 0.920000016689301
879 0.917500019073486
880 0.912500023841858
881 0.912500023841858
882 0.912500023841858
883 0.907500028610229
884 0.907500028610229
885 0.910000026226044
886 0.904999971389771
887 0.904999971389771
888 0.907500028610229
889 0.907500028610229
890 0.907500028610229
891 0.904999971389771
892 0.904999971389771
893 0.902499973773956
894 0.904999971389771
895 0.904999971389771
896 0.904999971389771
897 0.902499973773956
898 0.902499973773956
899 0.899999976158142
900 0.899999976158142
901 0.902499973773956
902 0.899999976158142
903 0.897499978542328
904 0.894999980926514
905 0.897499978542328
906 0.899999976158142
907 0.899999976158142
908 0.899999976158142
909 0.902499973773956
910 0.904999971389771
911 0.907500028610229
912 0.907500028610229
913 0.907500028610229
914 0.910000026226044
915 0.910000026226044
916 0.904999971389771
917 0.904999971389771
918 0.915000021457672
919 0.915000021457672
920 0.917500019073486
921 0.917500019073486
922 0.917500019073486
923 0.915000021457672
924 0.920000016689301
925 0.920000016689301
926 0.925000011920929
927 0.925000011920929
928 0.927500009536743
929 0.927500009536743
930 0.932500004768372
931 0.932500004768372
932 0.932500004768372
933 0.9375
934 0.9375
935 0.9375
936 0.944999992847443
937 0.944999992847443
938 0.942499995231628
939 0.939999997615814
940 0.942499995231628
941 0.944999992847443
942 0.944999992847443
943 0.947499990463257
944 0.947499990463257
945 0.947499990463257
946 0.947499990463257
947 0.949999988079071
948 0.949999988079071
949 0.952499985694885
950 0.947499990463257
951 0.947499990463257
952 0.949999988079071
953 0.952499985694885
954 0.954999983310699
955 0.954999983310699
956 0.954999983310699
957 0.954999983310699
958 0.954999983310699
959 0.954999983310699
960 0.957499980926514
961 0.957499980926514
962 0.959999978542328
963 0.957499980926514
964 0.957499980926514
965 0.957499980926514
966 0.964999973773956
967 0.964999973773956
968 0.962499976158142
969 0.964999973773956
970 0.964999973773956
971 0.959999978542328
972 0.954999983310699
973 0.957499980926514
974 0.957499980926514
975 0.959999978542328
976 0.954999983310699
977 0.952499985694885
978 0.949999988079071
979 0.952499985694885
980 0.952499985694885
981 0.952499985694885
982 0.949999988079071
983 0.949999988079071
984 0.949999988079071
985 0.949999988079071
986 0.949999988079071
987 0.947499990463257
988 0.947499990463257
989 0.949999988079071
990 0.949999988079071
991 0.949999988079071
992 0.947499990463257
993 0.947499990463257
994 0.949999988079071
995 0.949999988079071
996 0.949999988079071
997 0.949999988079071
998 0.952499985694885
999 0.952499985694885
1000 0.957499980926514
1001 0.954999983310699
1002 0.952499985694885
1003 0.952499985694885
1004 0.949999988079071
1005 0.949999988079071
1006 0.947499990463257
1007 0.947499990463257
1008 0.949999988079071
1009 0.952499985694885
1010 0.949999988079071
1011 0.947499990463257
1012 0.947499990463257
1013 0.949999988079071
1014 0.949999988079071
1015 0.949999988079071
1016 0.947499990463257
1017 0.944999992847443
1018 0.944999992847443
1019 0.942499995231628
1020 0.942499995231628
1021 0.947499990463257
1022 0.952499985694885
1023 0.952499985694885
1024 0.952499985694885
1025 0.954999983310699
1026 0.959999978542328
1027 0.962499976158142
1028 0.964999973773956
1029 0.964999973773956
1030 0.959999978542328
1031 0.959999978542328
1032 0.962499976158142
1033 0.962499976158142
1034 0.959999978542328
1035 0.959999978542328
1036 0.959999978542328
1037 0.962499976158142
1038 0.962499976158142
1039 0.959999978542328
1040 0.959999978542328
1041 0.959999978542328
1042 0.959999978542328
1043 0.959999978542328
1044 0.959999978542328
1045 0.959999978542328
1046 0.957499980926514
1047 0.957499980926514
1048 0.957499980926514
1049 0.957499980926514
1050 0.957499980926514
1051 0.959999978542328
1052 0.959999978542328
1053 0.959999978542328
1054 0.962499976158142
1055 0.962499976158142
1056 0.964999973773956
1057 0.964999973773956
1058 0.962499976158142
1059 0.962499976158142
1060 0.964999973773956
1061 0.964999973773956
1062 0.964999973773956
1063 0.964999973773956
1064 0.964999973773956
1065 0.964999973773956
1066 0.967499971389771
1067 0.967499971389771
1068 0.970000028610229
1069 0.970000028610229
1070 0.970000028610229
1071 0.970000028610229
1072 0.967499971389771
1073 0.964999973773956
1074 0.964999973773956
1075 0.959999978542328
1076 0.959999978542328
1077 0.959999978542328
1078 0.959999978542328
1079 0.959999978542328
1080 0.962499976158142
1081 0.962499976158142
1082 0.962499976158142
1083 0.962499976158142
1084 0.964999973773956
1085 0.964999973773956
1086 0.962499976158142
1087 0.962499976158142
1088 0.964999973773956
1089 0.967499971389771
1090 0.967499971389771
1091 0.967499971389771
1092 0.967499971389771
1093 0.967499971389771
1094 0.967499971389771
1095 0.967499971389771
1096 0.970000028610229
1097 0.967499971389771
1098 0.967499971389771
1099 0.964999973773956
1100 0.964999973773956
1101 0.962499976158142
1102 0.964999973773956
1103 0.964999973773956
1104 0.962499976158142
1105 0.962499976158142
1106 0.962499976158142
1107 0.959999978542328
1108 0.962499976158142
1109 0.962499976158142
1110 0.962499976158142
1111 0.964999973773956
1112 0.962499976158142
1113 0.962499976158142
1114 0.962499976158142
1115 0.962499976158142
1116 0.962499976158142
1117 0.964999973773956
1118 0.964999973773956
1119 0.967499971389771
1120 0.967499971389771
1121 0.967499971389771
1122 0.970000028610229
1123 0.972500026226044
1124 0.972500026226044
1125 0.977500021457672
1126 0.977500021457672
1127 0.977500021457672
1128 0.972500026226044
1129 0.970000028610229
1130 0.972500026226044
1131 0.972500026226044
1132 0.972500026226044
1133 0.972500026226044
1134 0.970000028610229
1135 0.970000028610229
1136 0.972500026226044
1137 0.972500026226044
1138 0.972500026226044
1139 0.970000028610229
1140 0.970000028610229
1141 0.967499971389771
1142 0.970000028610229
1143 0.970000028610229
1144 0.967499971389771
1145 0.964999973773956
1146 0.962499976158142
1147 0.964999973773956
1148 0.964999973773956
1149 0.967499971389771
1150 0.967499971389771
1151 0.964999973773956
1152 0.964999973773956
1153 0.964999973773956
1154 0.967499971389771
1155 0.967499971389771
1156 0.967499971389771
1157 0.970000028610229
1158 0.970000028610229
1159 0.967499971389771
1160 0.967499971389771
1161 0.967499971389771
1162 0.970000028610229
1163 0.970000028610229
1164 0.970000028610229
1165 0.970000028610229
1166 0.967499971389771
1167 0.967499971389771
1168 0.967499971389771
1169 0.967499971389771
1170 0.967499971389771
1171 0.967499971389771
1172 0.967499971389771
1173 0.967499971389771
1174 0.967499971389771
1175 0.967499971389771
1176 0.967499971389771
1177 0.967499971389771
1178 0.972500026226044
1179 0.975000023841858
1180 0.975000023841858
1181 0.975000023841858
1182 0.975000023841858
1183 0.975000023841858
1184 0.977500021457672
1185 0.977500021457672
1186 0.977500021457672
1187 0.977500021457672
1188 0.977500021457672
1189 0.977500021457672
1190 0.977500021457672
1191 0.980000019073486
1192 0.980000019073486
1193 0.977500021457672
1194 0.980000019073486
1195 0.982500016689301
1196 0.985000014305115
1197 0.985000014305115
1198 0.985000014305115
1199 0.982500016689301
1200 0.982500016689301
1201 0.987500011920929
1202 0.987500011920929
1203 0.987500011920929
1204 0.987500011920929
1205 0.985000014305115
1206 0.985000014305115
1207 0.982500016689301
1208 0.982500016689301
1209 0.985000014305115
1210 0.985000014305115
1211 0.985000014305115
1212 0.985000014305115
1213 0.985000014305115
1214 0.985000014305115
1215 0.985000014305115
1216 0.987500011920929
1217 0.987500011920929
1218 0.987500011920929
1219 0.987500011920929
1220 0.987500011920929
1221 0.987500011920929
1222 0.987500011920929
1223 0.987500011920929
1224 0.987500011920929
1225 0.985000014305115
1226 0.985000014305115
1227 0.982500016689301
1228 0.982500016689301
1229 0.982500016689301
1230 0.982500016689301
1231 0.980000019073486
1232 0.980000019073486
1233 0.980000019073486
1234 0.980000019073486
1235 0.980000019073486
1236 0.980000019073486
1237 0.980000019073486
1238 0.980000019073486
1239 0.982500016689301
};
\addlegendentry{4-4}
\addplot [semithick, red]
table {%
0 0.25
1 0.3125
2 0.25
3 0.1875
4 0.174999997019768
5 0.14583332836628
6 0.125
7 0.140625
8 0.125
9 0.112499997019768
10 0.125
11 0.11458333581686
12 0.105769231915474
13 0.107142858207226
14 0.108333334326744
15 0.1015625
16 0.125
17 0.131944447755814
18 0.131578952074051
19 0.125
20 0.136904761195183
21 0.130681812763214
22 0.135869562625885
23 0.13541667163372
24 0.129999995231628
25 0.134615391492844
26 0.129629626870155
27 0.129464283585548
28 0.133620694279671
29 0.137500002980232
30 0.145161285996437
31 0.1484375
32 0.147727265954018
33 0.154411762952805
34 0.153571426868439
35 0.159722223877907
36 0.155405402183533
37 0.151315793395042
38 0.157051280140877
39 0.162499994039536
40 0.161585360765457
41 0.163690477609634
42 0.159883722662926
43 0.167613640427589
44 0.169444441795349
45 0.171195656061172
46 0.175531908869743
47 0.18229167163372
48 0.181122452020645
49 0.180000007152557
50 0.180000007152557
51 0.174999997019768
52 0.180000007152557
53 0.1875
54 0.1875
55 0.194999992847443
56 0.202500000596046
57 0.202500000596046
58 0.202500000596046
59 0.209999993443489
60 0.217500001192093
61 0.219999998807907
62 0.234999999403954
63 0.237499997019768
64 0.247500002384186
65 0.254999995231628
66 0.25
67 0.25
68 0.254999995231628
69 0.262499988079071
70 0.262499988079071
71 0.267500013113022
72 0.267500013113022
73 0.270000010728836
74 0.27250000834465
75 0.280000001192093
76 0.282499998807907
77 0.284999996423721
78 0.28999999165535
79 0.28999999165535
80 0.287499994039536
81 0.297500014305115
82 0.300000011920929
83 0.297500014305115
84 0.305000007152557
85 0.310000002384186
86 0.314999997615814
87 0.322499990463257
88 0.324999988079071
89 0.324999988079071
90 0.337500005960464
91 0.342500001192093
92 0.347499996423721
93 0.349999994039536
94 0.349999994039536
95 0.357499986886978
96 0.360000014305115
97 0.357499986886978
98 0.370000004768372
99 0.379999995231628
100 0.384999990463257
101 0.400000005960464
102 0.400000005960464
103 0.402500003576279
104 0.409999996423721
105 0.412499994039536
106 0.412499994039536
107 0.422500014305115
108 0.430000007152557
109 0.435000002384186
110 0.430000007152557
111 0.439999997615814
112 0.4375
113 0.444999992847443
114 0.447499990463257
115 0.452499985694885
116 0.457500010728836
117 0.469999998807907
118 0.472499996423721
119 0.482499986886978
120 0.490000009536743
121 0.495000004768372
122 0.5
123 0.502499997615814
124 0.509999990463257
125 0.507499992847443
126 0.517499983310699
127 0.524999976158142
128 0.527499973773956
129 0.537500023841858
130 0.547500014305115
131 0.540000021457672
132 0.550000011920929
133 0.555000007152557
134 0.555000007152557
135 0.550000011920929
136 0.560000002384186
137 0.564999997615814
138 0.564999997615814
139 0.569999992847443
140 0.572499990463257
141 0.569999992847443
142 0.574999988079071
143 0.577499985694885
144 0.579999983310699
145 0.584999978542328
146 0.584999978542328
147 0.582499980926514
148 0.572499990463257
149 0.567499995231628
150 0.567499995231628
151 0.5625
152 0.567499995231628
153 0.574999988079071
154 0.582499980926514
155 0.587499976158142
156 0.595000028610229
157 0.592499971389771
158 0.595000028610229
159 0.589999973773956
160 0.597500026226044
161 0.595000028610229
162 0.595000028610229
163 0.595000028610229
164 0.589999973773956
165 0.587499976158142
166 0.587499976158142
167 0.579999983310699
168 0.579999983310699
169 0.579999983310699
170 0.577499985694885
171 0.579999983310699
172 0.577499985694885
173 0.582499980926514
174 0.584999978542328
175 0.582499980926514
176 0.579999983310699
177 0.584999978542328
178 0.577499985694885
179 0.572499990463257
180 0.569999992847443
181 0.579999983310699
182 0.574999988079071
183 0.582499980926514
184 0.582499980926514
185 0.587499976158142
186 0.587499976158142
187 0.589999973773956
188 0.587499976158142
189 0.589999973773956
190 0.589999973773956
191 0.592499971389771
192 0.600000023841858
193 0.589999973773956
194 0.595000028610229
195 0.587499976158142
196 0.589999973773956
197 0.600000023841858
198 0.610000014305115
199 0.620000004768372
200 0.620000004768372
201 0.615000009536743
202 0.612500011920929
203 0.605000019073486
204 0.595000028610229
205 0.592499971389771
206 0.589999973773956
207 0.595000028610229
208 0.600000023841858
209 0.610000014305115
210 0.605000019073486
211 0.605000019073486
212 0.607500016689301
213 0.607500016689301
214 0.607500016689301
215 0.610000014305115
216 0.615000009536743
217 0.615000009536743
218 0.620000004768372
219 0.617500007152557
220 0.615000009536743
221 0.617500007152557
222 0.622500002384186
223 0.625
224 0.627499997615814
225 0.634999990463257
226 0.639999985694885
227 0.637499988079071
228 0.647499978542328
229 0.649999976158142
230 0.657500028610229
231 0.660000026226044
232 0.665000021457672
233 0.660000026226044
234 0.657500028610229
235 0.660000026226044
236 0.660000026226044
237 0.657500028610229
238 0.667500019073486
239 0.662500023841858
240 0.652499973773956
241 0.657500028610229
242 0.649999976158142
243 0.662500023841858
244 0.662500023841858
245 0.667500019073486
246 0.667500019073486
247 0.667500019073486
248 0.665000021457672
249 0.660000026226044
250 0.667500019073486
251 0.672500014305115
252 0.677500009536743
253 0.682500004768372
254 0.6875
255 0.689999997615814
256 0.692499995231628
257 0.685000002384186
258 0.689999997615814
259 0.689999997615814
260 0.694999992847443
261 0.704999983310699
262 0.704999983310699
263 0.712499976158142
264 0.720000028610229
265 0.720000028610229
266 0.720000028610229
267 0.722500026226044
268 0.722500026226044
269 0.720000028610229
270 0.725000023841858
271 0.725000023841858
272 0.727500021457672
273 0.725000023841858
274 0.730000019073486
275 0.732500016689301
276 0.732500016689301
277 0.735000014305115
278 0.737500011920929
279 0.735000014305115
280 0.730000019073486
281 0.727500021457672
282 0.727500021457672
283 0.732500016689301
284 0.732500016689301
285 0.727500021457672
286 0.727500021457672
287 0.732500016689301
288 0.732500016689301
289 0.737500011920929
290 0.742500007152557
291 0.747500002384186
292 0.75
293 0.740000009536743
294 0.742500007152557
295 0.745000004768372
296 0.742500007152557
297 0.745000004768372
298 0.745000004768372
299 0.75
300 0.747500002384186
301 0.75
302 0.754999995231628
303 0.754999995231628
304 0.762499988079071
305 0.764999985694885
306 0.767499983310699
307 0.774999976158142
308 0.769999980926514
309 0.767499983310699
310 0.772499978542328
311 0.769999980926514
312 0.774999976158142
313 0.769999980926514
314 0.767499983310699
315 0.767499983310699
316 0.767499983310699
317 0.772499978542328
318 0.772499978542328
319 0.772499978542328
320 0.777499973773956
321 0.779999971389771
322 0.779999971389771
323 0.782500028610229
324 0.779999971389771
325 0.777499973773956
326 0.774999976158142
327 0.774999976158142
328 0.772499978542328
329 0.774999976158142
330 0.774999976158142
331 0.772499978542328
332 0.772499978542328
333 0.774999976158142
334 0.782500028610229
335 0.785000026226044
336 0.790000021457672
337 0.790000021457672
338 0.787500023841858
339 0.787500023841858
340 0.790000021457672
341 0.787500023841858
342 0.790000021457672
343 0.800000011920929
344 0.802500009536743
345 0.802500009536743
346 0.807500004768372
347 0.807500004768372
348 0.807500004768372
349 0.807500004768372
350 0.802500009536743
351 0.797500014305115
352 0.797500014305115
353 0.802500009536743
354 0.792500019073486
355 0.795000016689301
356 0.792500019073486
357 0.792500019073486
358 0.795000016689301
359 0.795000016689301
360 0.790000021457672
361 0.790000021457672
362 0.782500028610229
363 0.782500028610229
364 0.785000026226044
365 0.785000026226044
366 0.785000026226044
367 0.782500028610229
368 0.785000026226044
369 0.790000021457672
370 0.787500023841858
371 0.779999971389771
372 0.779999971389771
373 0.777499973773956
374 0.774999976158142
375 0.777499973773956
376 0.779999971389771
377 0.777499973773956
378 0.777499973773956
379 0.782500028610229
380 0.782500028610229
381 0.782500028610229
382 0.787500023841858
383 0.782500028610229
384 0.779999971389771
385 0.782500028610229
386 0.779999971389771
387 0.777499973773956
388 0.779999971389771
389 0.785000026226044
390 0.785000026226044
391 0.782500028610229
392 0.787500023841858
393 0.787500023841858
394 0.785000026226044
395 0.787500023841858
396 0.792500019073486
397 0.795000016689301
398 0.802500009536743
399 0.797500014305115
400 0.802500009536743
401 0.810000002384186
402 0.807500004768372
403 0.805000007152557
404 0.8125
405 0.8125
406 0.814999997615814
407 0.814999997615814
408 0.8125
409 0.814999997615814
410 0.819999992847443
411 0.822499990463257
412 0.824999988079071
413 0.824999988079071
414 0.824999988079071
415 0.829999983310699
416 0.832499980926514
417 0.837499976158142
418 0.837499976158142
419 0.839999973773956
420 0.837499976158142
421 0.839999973773956
422 0.839999973773956
423 0.847500026226044
424 0.850000023841858
425 0.847500026226044
426 0.850000023841858
427 0.842499971389771
428 0.839999973773956
429 0.837499976158142
430 0.834999978542328
431 0.834999978542328
432 0.834999978542328
433 0.837499976158142
434 0.837499976158142
435 0.839999973773956
436 0.842499971389771
437 0.842499971389771
438 0.842499971389771
439 0.839999973773956
440 0.839999973773956
441 0.842499971389771
442 0.839999973773956
443 0.837499976158142
444 0.837499976158142
445 0.829999983310699
446 0.827499985694885
447 0.824999988079071
448 0.819999992847443
449 0.822499990463257
450 0.822499990463257
451 0.822499990463257
452 0.819999992847443
453 0.814999997615814
454 0.817499995231628
455 0.8125
456 0.8125
457 0.810000002384186
458 0.810000002384186
459 0.8125
460 0.810000002384186
461 0.807500004768372
462 0.805000007152557
463 0.810000002384186
464 0.810000002384186
465 0.8125
466 0.810000002384186
467 0.800000011920929
468 0.802500009536743
469 0.802500009536743
470 0.805000007152557
471 0.810000002384186
472 0.8125
473 0.807500004768372
474 0.805000007152557
475 0.807500004768372
476 0.810000002384186
477 0.822499990463257
478 0.827499985694885
479 0.832499980926514
480 0.834999978542328
481 0.837499976158142
482 0.834999978542328
483 0.832499980926514
484 0.839999973773956
485 0.839999973773956
486 0.839999973773956
487 0.839999973773956
488 0.837499976158142
489 0.829999983310699
490 0.834999978542328
491 0.832499980926514
492 0.832499980926514
493 0.832499980926514
494 0.832499980926514
495 0.829999983310699
496 0.829999983310699
497 0.829999983310699
498 0.832499980926514
499 0.834999978542328
500 0.837499976158142
501 0.832499980926514
502 0.834999978542328
503 0.834999978542328
504 0.827499985694885
505 0.824999988079071
506 0.824999988079071
507 0.819999992847443
508 0.822499990463257
509 0.819999992847443
510 0.819999992847443
511 0.819999992847443
512 0.824999988079071
513 0.824999988079071
514 0.824999988079071
515 0.824999988079071
516 0.827499985694885
517 0.827499985694885
518 0.822499990463257
519 0.819999992847443
520 0.819999992847443
521 0.814999997615814
522 0.814999997615814
523 0.814999997615814
524 0.819999992847443
525 0.819999992847443
526 0.819999992847443
527 0.814999997615814
528 0.814999997615814
529 0.814999997615814
530 0.814999997615814
531 0.810000002384186
532 0.810000002384186
533 0.8125
534 0.807500004768372
535 0.810000002384186
536 0.802500009536743
537 0.807500004768372
538 0.810000002384186
539 0.8125
540 0.8125
541 0.810000002384186
542 0.810000002384186
543 0.814999997615814
544 0.814999997615814
545 0.819999992847443
546 0.822499990463257
547 0.822499990463257
548 0.824999988079071
549 0.827499985694885
550 0.827499985694885
551 0.829999983310699
552 0.829999983310699
553 0.834999978542328
554 0.834999978542328
555 0.837499976158142
556 0.839999973773956
557 0.845000028610229
558 0.842499971389771
559 0.842499971389771
560 0.845000028610229
561 0.845000028610229
562 0.847500026226044
563 0.847500026226044
564 0.842499971389771
565 0.839999973773956
566 0.834999978542328
567 0.845000028610229
568 0.845000028610229
569 0.847500026226044
570 0.850000023841858
571 0.852500021457672
572 0.850000023841858
573 0.847500026226044
574 0.845000028610229
575 0.845000028610229
576 0.839999973773956
577 0.842499971389771
578 0.845000028610229
579 0.839999973773956
580 0.837499976158142
581 0.839999973773956
582 0.842499971389771
583 0.845000028610229
584 0.847500026226044
585 0.845000028610229
586 0.850000023841858
587 0.847500026226044
588 0.842499971389771
589 0.842499971389771
590 0.834999978542328
591 0.842499971389771
592 0.845000028610229
593 0.837499976158142
594 0.834999978542328
595 0.839999973773956
596 0.832499980926514
597 0.829999983310699
598 0.824999988079071
599 0.824999988079071
600 0.827499985694885
601 0.824999988079071
602 0.822499990463257
603 0.822499990463257
604 0.827499985694885
605 0.829999983310699
606 0.829999983310699
607 0.832499980926514
608 0.834999978542328
609 0.834999978542328
610 0.834999978542328
611 0.832499980926514
612 0.829999983310699
613 0.827499985694885
614 0.827499985694885
615 0.824999988079071
616 0.824999988079071
617 0.822499990463257
618 0.824999988079071
619 0.819999992847443
620 0.814999997615814
621 0.817499995231628
622 0.817499995231628
623 0.822499990463257
624 0.822499990463257
625 0.819999992847443
626 0.824999988079071
627 0.827499985694885
628 0.827499985694885
629 0.829999983310699
630 0.834999978542328
631 0.837499976158142
632 0.834999978542328
633 0.834999978542328
634 0.834999978542328
635 0.834999978542328
636 0.832499980926514
637 0.834999978542328
638 0.837499976158142
639 0.845000028610229
640 0.842499971389771
641 0.839999973773956
642 0.837499976158142
643 0.842499971389771
644 0.850000023841858
645 0.845000028610229
646 0.852500021457672
647 0.855000019073486
648 0.855000019073486
649 0.855000019073486
650 0.855000019073486
651 0.860000014305115
652 0.862500011920929
653 0.862500011920929
654 0.855000019073486
655 0.857500016689301
656 0.857500016689301
657 0.857500016689301
658 0.852500021457672
659 0.850000023841858
660 0.850000023841858
661 0.855000019073486
662 0.855000019073486
663 0.857500016689301
664 0.862500011920929
665 0.865000009536743
666 0.872500002384186
667 0.870000004768372
668 0.870000004768372
669 0.872500002384186
670 0.872500002384186
671 0.872500002384186
672 0.870000004768372
673 0.872500002384186
674 0.875
675 0.877499997615814
676 0.875
677 0.875
678 0.867500007152557
679 0.865000009536743
680 0.867500007152557
681 0.865000009536743
682 0.865000009536743
683 0.862500011920929
684 0.865000009536743
685 0.857500016689301
686 0.857500016689301
687 0.857500016689301
688 0.860000014305115
689 0.860000014305115
690 0.867500007152557
691 0.865000009536743
692 0.862500011920929
693 0.865000009536743
694 0.860000014305115
695 0.865000009536743
696 0.862500011920929
697 0.865000009536743
698 0.867500007152557
699 0.865000009536743
700 0.862500011920929
701 0.860000014305115
702 0.860000014305115
703 0.852500021457672
704 0.860000014305115
705 0.860000014305115
706 0.857500016689301
707 0.860000014305115
708 0.862500011920929
709 0.862500011920929
710 0.862500011920929
711 0.860000014305115
712 0.857500016689301
713 0.852500021457672
714 0.855000019073486
715 0.857500016689301
716 0.857500016689301
717 0.857500016689301
718 0.855000019073486
719 0.852500021457672
720 0.855000019073486
721 0.857500016689301
722 0.862500011920929
723 0.860000014305115
724 0.860000014305115
725 0.862500011920929
726 0.862500011920929
727 0.855000019073486
728 0.862500011920929
729 0.865000009536743
730 0.862500011920929
731 0.865000009536743
732 0.865000009536743
733 0.867500007152557
734 0.865000009536743
735 0.870000004768372
736 0.875
737 0.872500002384186
738 0.875
739 0.875
740 0.875
741 0.872500002384186
742 0.875
743 0.872500002384186
744 0.877499997615814
745 0.877499997615814
746 0.875
747 0.875
748 0.875
749 0.875
750 0.875
751 0.879999995231628
752 0.882499992847443
753 0.892499983310699
754 0.894999980926514
755 0.894999980926514
756 0.894999980926514
757 0.889999985694885
758 0.892499983310699
759 0.894999980926514
760 0.889999985694885
761 0.884999990463257
762 0.887499988079071
763 0.889999985694885
764 0.889999985694885
765 0.882499992847443
766 0.875
767 0.877499997615814
768 0.879999995231628
769 0.884999990463257
770 0.882499992847443
771 0.882499992847443
772 0.884999990463257
773 0.882499992847443
774 0.882499992847443
775 0.879999995231628
776 0.879999995231628
777 0.882499992847443
778 0.882499992847443
779 0.884999990463257
780 0.882499992847443
781 0.882499992847443
782 0.884999990463257
783 0.884999990463257
784 0.884999990463257
785 0.887499988079071
786 0.884999990463257
787 0.887499988079071
788 0.887499988079071
789 0.887499988079071
790 0.889999985694885
791 0.894999980926514
792 0.889999985694885
793 0.889999985694885
794 0.887499988079071
795 0.887499988079071
796 0.892499983310699
797 0.889999985694885
798 0.892499983310699
799 0.894999980926514
800 0.897499978542328
801 0.892499983310699
802 0.892499983310699
803 0.892499983310699
804 0.892499983310699
805 0.892499983310699
806 0.889999985694885
807 0.889999985694885
808 0.889999985694885
809 0.889999985694885
810 0.887499988079071
811 0.894999980926514
812 0.897499978542328
813 0.894999980926514
814 0.892499983310699
815 0.894999980926514
816 0.902499973773956
817 0.904999971389771
818 0.902499973773956
819 0.902499973773956
820 0.907500028610229
821 0.907500028610229
822 0.902499973773956
823 0.904999971389771
824 0.904999971389771
825 0.899999976158142
826 0.902499973773956
827 0.904999971389771
828 0.902499973773956
829 0.902499973773956
830 0.907500028610229
831 0.910000026226044
832 0.910000026226044
833 0.907500028610229
834 0.910000026226044
835 0.910000026226044
836 0.912500023841858
837 0.912500023841858
838 0.912500023841858
839 0.910000026226044
840 0.907500028610229
841 0.910000026226044
842 0.915000021457672
843 0.917500019073486
844 0.915000021457672
845 0.915000021457672
846 0.912500023841858
847 0.915000021457672
848 0.912500023841858
849 0.912500023841858
850 0.912500023841858
851 0.912500023841858
852 0.907500028610229
853 0.902499973773956
854 0.899999976158142
855 0.899999976158142
856 0.902499973773956
857 0.904999971389771
858 0.904999971389771
859 0.902499973773956
860 0.910000026226044
861 0.910000026226044
862 0.904999971389771
863 0.910000026226044
864 0.910000026226044
865 0.915000021457672
866 0.912500023841858
867 0.907500028610229
868 0.912500023841858
869 0.904999971389771
870 0.904999971389771
871 0.902499973773956
872 0.902499973773956
873 0.904999971389771
874 0.904999971389771
875 0.912500023841858
876 0.910000026226044
877 0.912500023841858
878 0.912500023841858
879 0.912500023841858
880 0.912500023841858
881 0.912500023841858
882 0.912500023841858
883 0.915000021457672
884 0.912500023841858
885 0.915000021457672
886 0.912500023841858
887 0.910000026226044
888 0.907500028610229
889 0.904999971389771
890 0.907500028610229
891 0.904999971389771
892 0.899999976158142
893 0.899999976158142
894 0.904999971389771
895 0.899999976158142
896 0.899999976158142
897 0.897499978542328
898 0.897499978542328
899 0.894999980926514
900 0.892499983310699
901 0.889999985694885
902 0.892499983310699
903 0.897499978542328
904 0.899999976158142
905 0.899999976158142
906 0.902499973773956
907 0.899999976158142
908 0.897499978542328
909 0.899999976158142
910 0.899999976158142
911 0.894999980926514
912 0.897499978542328
913 0.894999980926514
914 0.897499978542328
915 0.897499978542328
916 0.897499978542328
917 0.899999976158142
918 0.899999976158142
919 0.904999971389771
920 0.904999971389771
921 0.907500028610229
922 0.910000026226044
923 0.907500028610229
924 0.907500028610229
925 0.907500028610229
926 0.904999971389771
927 0.904999971389771
928 0.907500028610229
929 0.904999971389771
930 0.904999971389771
931 0.904999971389771
932 0.904999971389771
933 0.902499973773956
934 0.902499973773956
935 0.902499973773956
936 0.904999971389771
937 0.904999971389771
938 0.907500028610229
939 0.910000026226044
940 0.907500028610229
941 0.907500028610229
942 0.915000021457672
943 0.915000021457672
944 0.912500023841858
945 0.917500019073486
946 0.920000016689301
947 0.922500014305115
948 0.922500014305115
949 0.925000011920929
950 0.927500009536743
951 0.935000002384186
952 0.935000002384186
953 0.935000002384186
954 0.935000002384186
955 0.932500004768372
956 0.932500004768372
957 0.935000002384186
958 0.939999997615814
959 0.9375
960 0.935000002384186
961 0.9375
962 0.9375
963 0.939999997615814
964 0.9375
965 0.9375
966 0.935000002384186
967 0.935000002384186
968 0.932500004768372
969 0.930000007152557
970 0.927500009536743
971 0.927500009536743
972 0.925000011920929
973 0.927500009536743
974 0.922500014305115
975 0.915000021457672
976 0.920000016689301
977 0.915000021457672
978 0.907500028610229
979 0.910000026226044
980 0.904999971389771
981 0.904999971389771
982 0.902499973773956
983 0.904999971389771
984 0.907500028610229
985 0.902499973773956
986 0.899999976158142
987 0.894999980926514
988 0.892499983310699
989 0.894999980926514
990 0.892499983310699
991 0.892499983310699
992 0.889999985694885
993 0.889999985694885
994 0.889999985694885
995 0.884999990463257
996 0.879999995231628
997 0.877499997615814
998 0.877499997615814
999 0.877499997615814
1000 0.877499997615814
1001 0.875
1002 0.872500002384186
1003 0.870000004768372
1004 0.867500007152557
1005 0.870000004768372
1006 0.867500007152557
1007 0.870000004768372
1008 0.867500007152557
1009 0.872500002384186
1010 0.875
1011 0.877499997615814
1012 0.875
1013 0.867500007152557
1014 0.870000004768372
1015 0.865000009536743
1016 0.870000004768372
1017 0.872500002384186
1018 0.870000004768372
1019 0.875
1020 0.877499997615814
1021 0.875
1022 0.875
1023 0.872500002384186
1024 0.877499997615814
1025 0.882499992847443
1026 0.882499992847443
1027 0.882499992847443
1028 0.889999985694885
1029 0.889999985694885
1030 0.894999980926514
1031 0.889999985694885
1032 0.887499988079071
1033 0.887499988079071
1034 0.884999990463257
1035 0.887499988079071
1036 0.889999985694885
1037 0.892499983310699
1038 0.894999980926514
1039 0.892499983310699
1040 0.894999980926514
1041 0.897499978542328
1042 0.899999976158142
1043 0.899999976158142
1044 0.897499978542328
1045 0.902499973773956
1046 0.904999971389771
1047 0.904999971389771
1048 0.904999971389771
1049 0.904999971389771
1050 0.904999971389771
1051 0.907500028610229
1052 0.912500023841858
1053 0.915000021457672
1054 0.915000021457672
1055 0.912500023841858
1056 0.915000021457672
1057 0.912500023841858
1058 0.912500023841858
1059 0.907500028610229
1060 0.902499973773956
1061 0.897499978542328
1062 0.902499973773956
1063 0.907500028610229
1064 0.907500028610229
1065 0.912500023841858
1066 0.912500023841858
1067 0.912500023841858
1068 0.917500019073486
1069 0.915000021457672
1070 0.915000021457672
1071 0.915000021457672
1072 0.920000016689301
1073 0.917500019073486
1074 0.915000021457672
1075 0.912500023841858
1076 0.912500023841858
1077 0.915000021457672
1078 0.912500023841858
1079 0.912500023841858
1080 0.912500023841858
1081 0.917500019073486
1082 0.917500019073486
1083 0.917500019073486
1084 0.920000016689301
1085 0.922500014305115
1086 0.922500014305115
1087 0.927500009536743
1088 0.927500009536743
1089 0.930000007152557
1090 0.930000007152557
1091 0.927500009536743
1092 0.927500009536743
1093 0.925000011920929
1094 0.927500009536743
1095 0.927500009536743
1096 0.930000007152557
1097 0.927500009536743
1098 0.930000007152557
1099 0.930000007152557
1100 0.930000007152557
1101 0.930000007152557
1102 0.930000007152557
1103 0.930000007152557
1104 0.927500009536743
1105 0.927500009536743
1106 0.925000011920929
1107 0.925000011920929
1108 0.927500009536743
1109 0.930000007152557
1110 0.935000002384186
1111 0.935000002384186
1112 0.932500004768372
1113 0.935000002384186
1114 0.932500004768372
1115 0.932500004768372
1116 0.930000007152557
1117 0.930000007152557
1118 0.930000007152557
1119 0.930000007152557
1120 0.930000007152557
1121 0.927500009536743
1122 0.927500009536743
1123 0.932500004768372
1124 0.935000002384186
1125 0.939999997615814
1126 0.935000002384186
1127 0.9375
1128 0.939999997615814
1129 0.939999997615814
1130 0.9375
1131 0.935000002384186
1132 0.939999997615814
1133 0.939999997615814
1134 0.9375
1135 0.9375
1136 0.9375
1137 0.9375
1138 0.927500009536743
1139 0.927500009536743
1140 0.927500009536743
1141 0.930000007152557
1142 0.927500009536743
1143 0.927500009536743
1144 0.927500009536743
1145 0.925000011920929
1146 0.925000011920929
1147 0.927500009536743
1148 0.927500009536743
1149 0.925000011920929
1150 0.922500014305115
1151 0.915000021457672
1152 0.910000026226044
1153 0.910000026226044
1154 0.915000021457672
1155 0.912500023841858
1156 0.912500023841858
1157 0.912500023841858
1158 0.907500028610229
1159 0.907500028610229
1160 0.907500028610229
1161 0.910000026226044
1162 0.912500023841858
1163 0.907500028610229
1164 0.907500028610229
1165 0.904999971389771
1166 0.907500028610229
1167 0.907500028610229
1168 0.907500028610229
1169 0.910000026226044
1170 0.907500028610229
1171 0.910000026226044
1172 0.910000026226044
1173 0.907500028610229
1174 0.904999971389771
1175 0.902499973773956
1176 0.907500028610229
1177 0.904999971389771
1178 0.902499973773956
1179 0.902499973773956
1180 0.902499973773956
1181 0.902499973773956
1182 0.902499973773956
1183 0.902499973773956
1184 0.899999976158142
1185 0.897499978542328
1186 0.897499978542328
1187 0.894999980926514
1188 0.904999971389771
1189 0.904999971389771
1190 0.907500028610229
1191 0.902499973773956
1192 0.904999971389771
1193 0.894999980926514
1194 0.894999980926514
1195 0.894999980926514
1196 0.892499983310699
1197 0.894999980926514
1198 0.894999980926514
1199 0.897499978542328
1200 0.899999976158142
1201 0.907500028610229
1202 0.910000026226044
1203 0.910000026226044
1204 0.907500028610229
1205 0.912500023841858
1206 0.915000021457672
1207 0.917500019073486
1208 0.922500014305115
1209 0.925000011920929
1210 0.925000011920929
1211 0.925000011920929
1212 0.925000011920929
1213 0.925000011920929
1214 0.927500009536743
1215 0.930000007152557
1216 0.930000007152557
1217 0.927500009536743
1218 0.927500009536743
1219 0.927500009536743
1220 0.925000011920929
1221 0.927500009536743
1222 0.927500009536743
1223 0.930000007152557
1224 0.930000007152557
1225 0.932500004768372
1226 0.932500004768372
1227 0.935000002384186
1228 0.935000002384186
1229 0.935000002384186
1230 0.9375
1231 0.939999997615814
1232 0.939999997615814
1233 0.939999997615814
1234 0.944999992847443
1235 0.947499990463257
1236 0.944999992847443
1237 0.947499990463257
1238 0.947499990463257
1239 0.944999992847443
};
\addlegendentry{4-5}
\addplot [semithick, color0]
table {%
0 0.25
1 0.25
2 0.20833332836628
3 0.1875
4 0.174999997019768
5 0.1875
6 0.214285716414452
7 0.1875
8 0.180555552244186
9 0.1875
10 0.170454546809196
11 0.15625
12 0.144230768084526
13 0.151785716414452
14 0.150000005960464
15 0.140625
16 0.139705881476402
17 0.131944447755814
18 0.125
19 0.131249994039536
20 0.130952388048172
21 0.125
22 0.125
23 0.11979166418314
24 0.119999997317791
25 0.125
26 0.120370373129845
27 0.125
28 0.125
29 0.120833329856396
30 0.116935484111309
31 0.1171875
32 0.121212124824524
33 0.117647059261799
34 0.11785714328289
35 0.121527776122093
36 0.118243239820004
37 0.11842105537653
38 0.115384615957737
39 0.125
40 0.131097555160522
41 0.133928567171097
42 0.133720934391022
43 0.136363640427589
44 0.136111110448837
45 0.144021734595299
46 0.156914889812469
47 0.15625
48 0.160714283585548
49 0.165000006556511
50 0.165000006556511
51 0.162499994039536
52 0.162499994039536
53 0.167500004172325
54 0.172499999403954
55 0.170000001788139
56 0.167500004172325
57 0.174999997019768
58 0.180000007152557
59 0.180000007152557
60 0.192499995231628
61 0.194999992847443
62 0.204999998211861
63 0.202500000596046
64 0.212500005960464
65 0.219999998807907
66 0.227500006556511
67 0.232500001788139
68 0.239999994635582
69 0.242500007152557
70 0.245000004768372
71 0.247500002384186
72 0.252499997615814
73 0.259999990463257
74 0.259999990463257
75 0.257499992847443
76 0.262499988079071
77 0.262499988079071
78 0.264999985694885
79 0.270000010728836
80 0.275000005960464
81 0.280000001192093
82 0.284999996423721
83 0.292499989271164
84 0.300000011920929
85 0.300000011920929
86 0.307500004768372
87 0.314999997615814
88 0.319999992847443
89 0.319999992847443
90 0.322499990463257
91 0.330000013113022
92 0.33500000834465
93 0.340000003576279
94 0.347499996423721
95 0.347499996423721
96 0.340000003576279
97 0.347499996423721
98 0.349999994039536
99 0.347499996423721
100 0.354999989271164
101 0.354999989271164
102 0.360000014305115
103 0.362500011920929
104 0.360000014305115
105 0.367500007152557
106 0.372500002384186
107 0.372500002384186
108 0.377499997615814
109 0.379999995231628
110 0.372500002384186
111 0.384999990463257
112 0.392500013113022
113 0.400000005960464
114 0.39750000834465
115 0.39750000834465
116 0.39750000834465
117 0.400000005960464
118 0.402500003576279
119 0.402500003576279
120 0.407499998807907
121 0.412499994039536
122 0.407499998807907
123 0.409999996423721
124 0.417499989271164
125 0.425000011920929
126 0.430000007152557
127 0.435000002384186
128 0.432500004768372
129 0.439999997615814
130 0.442499995231628
131 0.444999992847443
132 0.439999997615814
133 0.447499990463257
134 0.449999988079071
135 0.449999988079071
136 0.457500010728836
137 0.455000013113022
138 0.455000013113022
139 0.452499985694885
140 0.457500010728836
141 0.452499985694885
142 0.452499985694885
143 0.449999988079071
144 0.455000013113022
145 0.455000013113022
146 0.46000000834465
147 0.455000013113022
148 0.449999988079071
149 0.457500010728836
150 0.457500010728836
151 0.469999998807907
152 0.482499986886978
153 0.487500011920929
154 0.487500011920929
155 0.495000004768372
156 0.497500002384186
157 0.504999995231628
158 0.5
159 0.509999990463257
160 0.514999985694885
161 0.504999995231628
162 0.497500002384186
163 0.504999995231628
164 0.507499992847443
165 0.512499988079071
166 0.509999990463257
167 0.509999990463257
168 0.512499988079071
169 0.514999985694885
170 0.517499983310699
171 0.522499978542328
172 0.535000026226044
173 0.537500023841858
174 0.537500023841858
175 0.537500023841858
176 0.542500019073486
177 0.547500014305115
178 0.555000007152557
179 0.557500004768372
180 0.567499995231628
181 0.567499995231628
182 0.564999997615814
183 0.560000002384186
184 0.557500004768372
185 0.564999997615814
186 0.564999997615814
187 0.569999992847443
188 0.579999983310699
189 0.584999978542328
190 0.582499980926514
191 0.587499976158142
192 0.584999978542328
193 0.584999978542328
194 0.574999988079071
195 0.577499985694885
196 0.579999983310699
197 0.584999978542328
198 0.592499971389771
199 0.592499971389771
200 0.587499976158142
201 0.577499985694885
202 0.567499995231628
203 0.569999992847443
204 0.582499980926514
205 0.574999988079071
206 0.579999983310699
207 0.569999992847443
208 0.572499990463257
209 0.564999997615814
210 0.567499995231628
211 0.577499985694885
212 0.574999988079071
213 0.572499990463257
214 0.574999988079071
215 0.567499995231628
216 0.567499995231628
217 0.574999988079071
218 0.574999988079071
219 0.574999988079071
220 0.574999988079071
221 0.577499985694885
222 0.572499990463257
223 0.574999988079071
224 0.572499990463257
225 0.577499985694885
226 0.569999992847443
227 0.564999997615814
228 0.5625
229 0.560000002384186
230 0.5625
231 0.560000002384186
232 0.567499995231628
233 0.567499995231628
234 0.574999988079071
235 0.574999988079071
236 0.569999992847443
237 0.574999988079071
238 0.579999983310699
239 0.577499985694885
240 0.579999983310699
241 0.577499985694885
242 0.582499980926514
243 0.589999973773956
244 0.595000028610229
245 0.595000028610229
246 0.592499971389771
247 0.592499971389771
248 0.592499971389771
249 0.592499971389771
250 0.597500026226044
251 0.612500011920929
252 0.615000009536743
253 0.615000009536743
254 0.610000014305115
255 0.607500016689301
256 0.600000023841858
257 0.605000019073486
258 0.605000019073486
259 0.607500016689301
260 0.610000014305115
261 0.602500021457672
262 0.607500016689301
263 0.607500016689301
264 0.605000019073486
265 0.615000009536743
266 0.622500002384186
267 0.615000009536743
268 0.615000009536743
269 0.617500007152557
270 0.615000009536743
271 0.617500007152557
272 0.620000004768372
273 0.622500002384186
274 0.627499997615814
275 0.625
276 0.627499997615814
277 0.632499992847443
278 0.642499983310699
279 0.647499978542328
280 0.639999985694885
281 0.639999985694885
282 0.647499978542328
283 0.647499978542328
284 0.644999980926514
285 0.644999980926514
286 0.644999980926514
287 0.639999985694885
288 0.632499992847443
289 0.634999990463257
290 0.632499992847443
291 0.642499983310699
292 0.644999980926514
293 0.637499988079071
294 0.637499988079071
295 0.632499992847443
296 0.634999990463257
297 0.639999985694885
298 0.637499988079071
299 0.642499983310699
300 0.642499983310699
301 0.637499988079071
302 0.642499983310699
303 0.642499983310699
304 0.647499978542328
305 0.660000026226044
306 0.665000021457672
307 0.670000016689301
308 0.672500014305115
309 0.675000011920929
310 0.677500009536743
311 0.682500004768372
312 0.6875
313 0.689999997615814
314 0.692499995231628
315 0.689999997615814
316 0.685000002384186
317 0.694999992847443
318 0.699999988079071
319 0.699999988079071
320 0.707499980926514
321 0.707499980926514
322 0.707499980926514
323 0.704999983310699
324 0.704999983310699
325 0.712499976158142
326 0.717499971389771
327 0.714999973773956
328 0.699999988079071
329 0.697499990463257
330 0.694999992847443
331 0.697499990463257
332 0.697499990463257
333 0.697499990463257
334 0.697499990463257
335 0.694999992847443
336 0.699999988079071
337 0.702499985694885
338 0.704999983310699
339 0.702499985694885
340 0.704999983310699
341 0.702499985694885
342 0.702499985694885
343 0.707499980926514
344 0.712499976158142
345 0.720000028610229
346 0.720000028610229
347 0.717499971389771
348 0.722500026226044
349 0.722500026226044
350 0.722500026226044
351 0.725000023841858
352 0.720000028610229
353 0.704999983310699
354 0.699999988079071
355 0.699999988079071
356 0.699999988079071
357 0.697499990463257
358 0.694999992847443
359 0.694999992847443
360 0.694999992847443
361 0.697499990463257
362 0.689999997615814
363 0.692499995231628
364 0.6875
365 0.6875
366 0.694999992847443
367 0.689999997615814
368 0.692499995231628
369 0.692499995231628
370 0.692499995231628
371 0.689999997615814
372 0.6875
373 0.6875
374 0.689999997615814
375 0.682500004768372
376 0.685000002384186
377 0.6875
378 0.699999988079071
379 0.694999992847443
380 0.699999988079071
381 0.704999983310699
382 0.699999988079071
383 0.704999983310699
384 0.702499985694885
385 0.709999978542328
386 0.712499976158142
387 0.712499976158142
388 0.714999973773956
389 0.717499971389771
390 0.717499971389771
391 0.712499976158142
392 0.712499976158142
393 0.714999973773956
394 0.709999978542328
395 0.709999978542328
396 0.712499976158142
397 0.717499971389771
398 0.712499976158142
399 0.707499980926514
400 0.712499976158142
401 0.704999983310699
402 0.707499980926514
403 0.725000023841858
404 0.722500026226044
405 0.717499971389771
406 0.714999973773956
407 0.720000028610229
408 0.725000023841858
409 0.725000023841858
410 0.714999973773956
411 0.709999978542328
412 0.720000028610229
413 0.712499976158142
414 0.717499971389771
415 0.714999973773956
416 0.714999973773956
417 0.717499971389771
418 0.709999978542328
419 0.702499985694885
420 0.697499990463257
421 0.694999992847443
422 0.704999983310699
423 0.699999988079071
424 0.694999992847443
425 0.692499995231628
426 0.692499995231628
427 0.689999997615814
428 0.689999997615814
429 0.692499995231628
430 0.692499995231628
431 0.692499995231628
432 0.697499990463257
433 0.697499990463257
434 0.702499985694885
435 0.699999988079071
436 0.697499990463257
437 0.702499985694885
438 0.702499985694885
439 0.704999983310699
440 0.707499980926514
441 0.702499985694885
442 0.709999978542328
443 0.712499976158142
444 0.714999973773956
445 0.712499976158142
446 0.707499980926514
447 0.704999983310699
448 0.707499980926514
449 0.707499980926514
450 0.704999983310699
451 0.709999978542328
452 0.712499976158142
453 0.704999983310699
454 0.712499976158142
455 0.714999973773956
456 0.722500026226044
457 0.725000023841858
458 0.727500021457672
459 0.732500016689301
460 0.745000004768372
461 0.75
462 0.747500002384186
463 0.75
464 0.747500002384186
465 0.75
466 0.747500002384186
467 0.752499997615814
468 0.759999990463257
469 0.769999980926514
470 0.769999980926514
471 0.769999980926514
472 0.762499988079071
473 0.767499983310699
474 0.769999980926514
475 0.774999976158142
476 0.762499988079071
477 0.759999990463257
478 0.762499988079071
479 0.759999990463257
480 0.762499988079071
481 0.759999990463257
482 0.757499992847443
483 0.757499992847443
484 0.757499992847443
485 0.754999995231628
486 0.757499992847443
487 0.747500002384186
488 0.740000009536743
489 0.740000009536743
490 0.737500011920929
491 0.737500011920929
492 0.727500021457672
493 0.722500026226044
494 0.725000023841858
495 0.727500021457672
496 0.725000023841858
497 0.725000023841858
498 0.730000019073486
499 0.737500011920929
500 0.737500011920929
501 0.740000009536743
502 0.742500007152557
503 0.75
504 0.740000009536743
505 0.740000009536743
506 0.735000014305115
507 0.732500016689301
508 0.725000023841858
509 0.720000028610229
510 0.717499971389771
511 0.712499976158142
512 0.712499976158142
513 0.712499976158142
514 0.717499971389771
515 0.722500026226044
516 0.725000023841858
517 0.722500026226044
518 0.717499971389771
519 0.720000028610229
520 0.725000023841858
521 0.722500026226044
522 0.727500021457672
523 0.732500016689301
524 0.737500011920929
525 0.740000009536743
526 0.752499997615814
527 0.759999990463257
528 0.759999990463257
529 0.764999985694885
530 0.767499983310699
531 0.769999980926514
532 0.772499978542328
533 0.767499983310699
534 0.762499988079071
535 0.769999980926514
536 0.769999980926514
537 0.777499973773956
538 0.777499973773956
539 0.777499973773956
540 0.777499973773956
541 0.790000021457672
542 0.800000011920929
543 0.805000007152557
544 0.797500014305115
545 0.797500014305115
546 0.805000007152557
547 0.805000007152557
548 0.800000011920929
549 0.792500019073486
550 0.792500019073486
551 0.790000021457672
552 0.787500023841858
553 0.787500023841858
554 0.795000016689301
555 0.795000016689301
556 0.797500014305115
557 0.800000011920929
558 0.807500004768372
559 0.807500004768372
560 0.810000002384186
561 0.819999992847443
562 0.817499995231628
563 0.817499995231628
564 0.814999997615814
565 0.814999997615814
566 0.814999997615814
567 0.814999997615814
568 0.810000002384186
569 0.807500004768372
570 0.800000011920929
571 0.800000011920929
572 0.800000011920929
573 0.797500014305115
574 0.795000016689301
575 0.795000016689301
576 0.797500014305115
577 0.800000011920929
578 0.792500019073486
579 0.790000021457672
580 0.787500023841858
581 0.792500019073486
582 0.792500019073486
583 0.797500014305115
584 0.802500009536743
585 0.797500014305115
586 0.800000011920929
587 0.797500014305115
588 0.800000011920929
589 0.802500009536743
590 0.807500004768372
591 0.802500009536743
592 0.797500014305115
593 0.792500019073486
594 0.797500014305115
595 0.797500014305115
596 0.797500014305115
597 0.795000016689301
598 0.797500014305115
599 0.802500009536743
600 0.805000007152557
601 0.800000011920929
602 0.800000011920929
603 0.797500014305115
604 0.802500009536743
605 0.795000016689301
606 0.790000021457672
607 0.787500023841858
608 0.785000026226044
609 0.787500023841858
610 0.785000026226044
611 0.785000026226044
612 0.785000026226044
613 0.787500023841858
614 0.785000026226044
615 0.785000026226044
616 0.782500028610229
617 0.782500028610229
618 0.787500023841858
619 0.787500023841858
620 0.792500019073486
621 0.797500014305115
622 0.795000016689301
623 0.790000021457672
624 0.790000021457672
625 0.785000026226044
626 0.779999971389771
627 0.772499978542328
628 0.777499973773956
629 0.779999971389771
630 0.777499973773956
631 0.774999976158142
632 0.769999980926514
633 0.769999980926514
634 0.769999980926514
635 0.767499983310699
636 0.762499988079071
637 0.759999990463257
638 0.764999985694885
639 0.764999985694885
640 0.757499992847443
641 0.762499988079071
642 0.757499992847443
643 0.754999995231628
644 0.759999990463257
645 0.762499988079071
646 0.762499988079071
647 0.769999980926514
648 0.772499978542328
649 0.772499978542328
650 0.769999980926514
651 0.777499973773956
652 0.779999971389771
653 0.779999971389771
654 0.774999976158142
655 0.782500028610229
656 0.787500023841858
657 0.787500023841858
658 0.785000026226044
659 0.785000026226044
660 0.777499973773956
661 0.774999976158142
662 0.772499978542328
663 0.772499978542328
664 0.774999976158142
665 0.769999980926514
666 0.772499978542328
667 0.772499978542328
668 0.774999976158142
669 0.777499973773956
670 0.777499973773956
671 0.777499973773956
672 0.777499973773956
673 0.782500028610229
674 0.787500023841858
675 0.790000021457672
676 0.795000016689301
677 0.797500014305115
678 0.797500014305115
679 0.800000011920929
680 0.800000011920929
681 0.800000011920929
682 0.807500004768372
683 0.807500004768372
684 0.807500004768372
685 0.810000002384186
686 0.8125
687 0.8125
688 0.810000002384186
689 0.805000007152557
690 0.8125
691 0.805000007152557
692 0.810000002384186
693 0.814999997615814
694 0.810000002384186
695 0.810000002384186
696 0.8125
697 0.810000002384186
698 0.810000002384186
699 0.810000002384186
700 0.8125
701 0.814999997615814
702 0.8125
703 0.814999997615814
704 0.8125
705 0.8125
706 0.810000002384186
707 0.8125
708 0.8125
709 0.814999997615814
710 0.814999997615814
711 0.814999997615814
712 0.819999992847443
713 0.819999992847443
714 0.822499990463257
715 0.829999983310699
716 0.829999983310699
717 0.832499980926514
718 0.829999983310699
719 0.827499985694885
720 0.827499985694885
721 0.829999983310699
722 0.829999983310699
723 0.827499985694885
724 0.824999988079071
725 0.827499985694885
726 0.819999992847443
727 0.819999992847443
728 0.824999988079071
729 0.827499985694885
730 0.829999983310699
731 0.832499980926514
732 0.829999983310699
733 0.829999983310699
734 0.832499980926514
735 0.834999978542328
736 0.837499976158142
737 0.842499971389771
738 0.847500026226044
739 0.847500026226044
740 0.845000028610229
741 0.850000023841858
742 0.852500021457672
743 0.855000019073486
744 0.860000014305115
745 0.857500016689301
746 0.857500016689301
747 0.860000014305115
748 0.862500011920929
749 0.860000014305115
750 0.857500016689301
751 0.855000019073486
752 0.855000019073486
753 0.855000019073486
754 0.857500016689301
755 0.855000019073486
756 0.850000023841858
757 0.847500026226044
758 0.855000019073486
759 0.852500021457672
760 0.860000014305115
761 0.857500016689301
762 0.857500016689301
763 0.852500021457672
764 0.855000019073486
765 0.852500021457672
766 0.847500026226044
767 0.847500026226044
768 0.850000023841858
769 0.847500026226044
770 0.847500026226044
771 0.850000023841858
772 0.850000023841858
773 0.855000019073486
774 0.855000019073486
775 0.855000019073486
776 0.862500011920929
777 0.867500007152557
778 0.860000014305115
779 0.855000019073486
780 0.857500016689301
781 0.852500021457672
782 0.850000023841858
783 0.855000019073486
784 0.847500026226044
785 0.842499971389771
786 0.839999973773956
787 0.839999973773956
788 0.834999978542328
789 0.839999973773956
790 0.842499971389771
791 0.842499971389771
792 0.839999973773956
793 0.837499976158142
794 0.832499980926514
795 0.832499980926514
796 0.824999988079071
797 0.824999988079071
798 0.819999992847443
799 0.822499990463257
800 0.822499990463257
801 0.824999988079071
802 0.827499985694885
803 0.824999988079071
804 0.819999992847443
805 0.824999988079071
806 0.834999978542328
807 0.832499980926514
808 0.827499985694885
809 0.829999983310699
810 0.829999983310699
811 0.832499980926514
812 0.834999978542328
813 0.834999978542328
814 0.832499980926514
815 0.832499980926514
816 0.837499976158142
817 0.829999983310699
818 0.829999983310699
819 0.832499980926514
820 0.832499980926514
821 0.827499985694885
822 0.827499985694885
823 0.827499985694885
824 0.827499985694885
825 0.824999988079071
826 0.819999992847443
827 0.817499995231628
828 0.819999992847443
829 0.822499990463257
830 0.814999997615814
831 0.819999992847443
832 0.817499995231628
833 0.814999997615814
834 0.819999992847443
835 0.824999988079071
836 0.822499990463257
837 0.819999992847443
838 0.817499995231628
839 0.814999997615814
840 0.807500004768372
841 0.805000007152557
842 0.807500004768372
843 0.810000002384186
844 0.814999997615814
845 0.819999992847443
846 0.824999988079071
847 0.814999997615814
848 0.819999992847443
849 0.817499995231628
850 0.817499995231628
851 0.817499995231628
852 0.8125
853 0.8125
854 0.819999992847443
855 0.814999997615814
856 0.8125
857 0.8125
858 0.817499995231628
859 0.814999997615814
860 0.817499995231628
861 0.817499995231628
862 0.814999997615814
863 0.817499995231628
864 0.817499995231628
865 0.819999992847443
866 0.819999992847443
867 0.822499990463257
868 0.822499990463257
869 0.824999988079071
870 0.824999988079071
871 0.819999992847443
872 0.819999992847443
873 0.810000002384186
874 0.8125
875 0.814999997615814
876 0.817499995231628
877 0.814999997615814
878 0.810000002384186
879 0.807500004768372
880 0.8125
881 0.807500004768372
882 0.814999997615814
883 0.8125
884 0.810000002384186
885 0.807500004768372
886 0.805000007152557
887 0.805000007152557
888 0.805000007152557
889 0.805000007152557
890 0.8125
891 0.817499995231628
892 0.819999992847443
893 0.817499995231628
894 0.8125
895 0.807500004768372
896 0.810000002384186
897 0.817499995231628
898 0.817499995231628
899 0.822499990463257
900 0.824999988079071
901 0.819999992847443
902 0.819999992847443
903 0.814999997615814
904 0.814999997615814
905 0.814999997615814
906 0.814999997615814
907 0.819999992847443
908 0.819999992847443
909 0.817499995231628
910 0.817499995231628
911 0.817499995231628
912 0.819999992847443
913 0.822499990463257
914 0.819999992847443
915 0.814999997615814
916 0.817499995231628
917 0.817499995231628
918 0.819999992847443
919 0.819999992847443
920 0.817499995231628
921 0.827499985694885
922 0.832499980926514
923 0.839999973773956
924 0.834999978542328
925 0.837499976158142
926 0.839999973773956
927 0.837499976158142
928 0.842499971389771
929 0.845000028610229
930 0.847500026226044
931 0.850000023841858
932 0.850000023841858
933 0.852500021457672
934 0.857500016689301
935 0.857500016689301
936 0.862500011920929
937 0.862500011920929
938 0.870000004768372
939 0.875
940 0.872500002384186
941 0.867500007152557
942 0.865000009536743
943 0.860000014305115
944 0.862500011920929
945 0.865000009536743
946 0.862500011920929
947 0.862500011920929
948 0.862500011920929
949 0.857500016689301
950 0.857500016689301
951 0.857500016689301
952 0.860000014305115
953 0.862500011920929
954 0.865000009536743
955 0.870000004768372
956 0.867500007152557
957 0.865000009536743
958 0.865000009536743
959 0.870000004768372
960 0.867500007152557
961 0.870000004768372
962 0.865000009536743
963 0.865000009536743
964 0.865000009536743
965 0.867500007152557
966 0.865000009536743
967 0.862500011920929
968 0.855000019073486
969 0.857500016689301
970 0.860000014305115
971 0.860000014305115
972 0.860000014305115
973 0.860000014305115
974 0.857500016689301
975 0.857500016689301
976 0.857500016689301
977 0.862500011920929
978 0.865000009536743
979 0.865000009536743
980 0.862500011920929
981 0.862500011920929
982 0.860000014305115
983 0.857500016689301
984 0.850000023841858
985 0.855000019073486
986 0.855000019073486
987 0.850000023841858
988 0.847500026226044
989 0.842499971389771
990 0.845000028610229
991 0.845000028610229
992 0.847500026226044
993 0.857500016689301
994 0.862500011920929
995 0.862500011920929
996 0.865000009536743
997 0.857500016689301
998 0.855000019073486
999 0.857500016689301
1000 0.857500016689301
1001 0.860000014305115
1002 0.860000014305115
1003 0.862500011920929
1004 0.862500011920929
1005 0.862500011920929
1006 0.867500007152557
1007 0.870000004768372
1008 0.870000004768372
1009 0.867500007152557
1010 0.870000004768372
1011 0.865000009536743
1012 0.862500011920929
1013 0.860000014305115
1014 0.862500011920929
1015 0.860000014305115
1016 0.860000014305115
1017 0.867500007152557
1018 0.875
1019 0.875
1020 0.875
1021 0.875
1022 0.872500002384186
1023 0.875
1024 0.879999995231628
1025 0.877499997615814
1026 0.872500002384186
1027 0.865000009536743
1028 0.865000009536743
1029 0.865000009536743
1030 0.862500011920929
1031 0.862500011920929
1032 0.865000009536743
1033 0.867500007152557
1034 0.872500002384186
1035 0.872500002384186
1036 0.872500002384186
1037 0.877499997615814
1038 0.879999995231628
1039 0.882499992847443
1040 0.882499992847443
1041 0.887499988079071
1042 0.884999990463257
1043 0.884999990463257
1044 0.882499992847443
1045 0.882499992847443
1046 0.879999995231628
1047 0.882499992847443
1048 0.884999990463257
1049 0.887499988079071
1050 0.889999985694885
1051 0.889999985694885
1052 0.892499983310699
1053 0.892499983310699
1054 0.889999985694885
1055 0.884999990463257
1056 0.882499992847443
1057 0.877499997615814
1058 0.877499997615814
1059 0.875
1060 0.872500002384186
1061 0.872500002384186
1062 0.879999995231628
1063 0.879999995231628
1064 0.877499997615814
1065 0.879999995231628
1066 0.882499992847443
1067 0.882499992847443
1068 0.879999995231628
1069 0.879999995231628
1070 0.882499992847443
1071 0.877499997615814
1072 0.877499997615814
1073 0.877499997615814
1074 0.867500007152557
1075 0.870000004768372
1076 0.870000004768372
1077 0.875
1078 0.875
1079 0.875
1080 0.877499997615814
1081 0.875
1082 0.875
1083 0.875
1084 0.877499997615814
1085 0.872500002384186
1086 0.875
1087 0.879999995231628
1088 0.879999995231628
1089 0.879999995231628
1090 0.879999995231628
1091 0.875
1092 0.875
1093 0.872500002384186
1094 0.875
1095 0.877499997615814
1096 0.872500002384186
1097 0.875
1098 0.870000004768372
1099 0.870000004768372
1100 0.867500007152557
1101 0.870000004768372
1102 0.870000004768372
1103 0.870000004768372
1104 0.867500007152557
1105 0.872500002384186
1106 0.872500002384186
1107 0.877499997615814
1108 0.877499997615814
1109 0.879999995231628
1110 0.882499992847443
1111 0.884999990463257
1112 0.884999990463257
1113 0.889999985694885
1114 0.894999980926514
1115 0.892499983310699
1116 0.889999985694885
1117 0.889999985694885
1118 0.892499983310699
1119 0.889999985694885
1120 0.892499983310699
1121 0.894999980926514
1122 0.897499978542328
1123 0.894999980926514
1124 0.907500028610229
1125 0.904999971389771
1126 0.907500028610229
1127 0.912500023841858
1128 0.915000021457672
1129 0.915000021457672
1130 0.915000021457672
1131 0.920000016689301
1132 0.917500019073486
1133 0.912500023841858
1134 0.910000026226044
1135 0.915000021457672
1136 0.912500023841858
1137 0.912500023841858
1138 0.912500023841858
1139 0.915000021457672
1140 0.912500023841858
1141 0.915000021457672
1142 0.915000021457672
1143 0.915000021457672
1144 0.915000021457672
1145 0.915000021457672
1146 0.920000016689301
1147 0.925000011920929
1148 0.927500009536743
1149 0.927500009536743
1150 0.930000007152557
1151 0.930000007152557
1152 0.927500009536743
1153 0.925000011920929
1154 0.927500009536743
1155 0.927500009536743
1156 0.930000007152557
1157 0.925000011920929
1158 0.925000011920929
1159 0.927500009536743
1160 0.922500014305115
1161 0.922500014305115
1162 0.917500019073486
1163 0.912500023841858
1164 0.907500028610229
1165 0.907500028610229
1166 0.907500028610229
1167 0.904999971389771
1168 0.904999971389771
1169 0.902499973773956
1170 0.899999976158142
1171 0.899999976158142
1172 0.899999976158142
1173 0.899999976158142
1174 0.899999976158142
1175 0.899999976158142
1176 0.899999976158142
1177 0.897499978542328
1178 0.894999980926514
1179 0.897499978542328
1180 0.899999976158142
1181 0.897499978542328
1182 0.899999976158142
1183 0.907500028610229
1184 0.907500028610229
1185 0.904999971389771
1186 0.907500028610229
1187 0.904999971389771
1188 0.902499973773956
1189 0.902499973773956
1190 0.899999976158142
1191 0.902499973773956
1192 0.904999971389771
1193 0.907500028610229
1194 0.904999971389771
1195 0.902499973773956
1196 0.897499978542328
1197 0.892499983310699
1198 0.894999980926514
1199 0.894999980926514
1200 0.892499983310699
1201 0.887499988079071
1202 0.889999985694885
1203 0.892499983310699
1204 0.894999980926514
1205 0.894999980926514
1206 0.894999980926514
1207 0.897499978542328
1208 0.894999980926514
1209 0.892499983310699
1210 0.897499978542328
1211 0.897499978542328
1212 0.902499973773956
1213 0.902499973773956
1214 0.902499973773956
1215 0.894999980926514
1216 0.892499983310699
1217 0.887499988079071
1218 0.884999990463257
1219 0.889999985694885
1220 0.892499983310699
1221 0.894999980926514
1222 0.889999985694885
1223 0.892499983310699
1224 0.892499983310699
1225 0.889999985694885
1226 0.892499983310699
1227 0.889999985694885
1228 0.892499983310699
1229 0.889999985694885
1230 0.889999985694885
1231 0.889999985694885
1232 0.889999985694885
1233 0.887499988079071
1234 0.884999990463257
1235 0.879999995231628
1236 0.875
1237 0.875
1238 0.872500002384186
1239 0.870000004768372
};
\addlegendentry{4-6}
\end{axis}

\end{tikzpicture}
		\caption{Moving averaged training accuracies of all networks capped at the number of iterations of the 0-3 network}
		\label{fig:train-accuracy-iteration}
	\end{subfigure}
	\begin{subfigure}{\textwidth}
		\centering
		% This file was created by matplotlib2tikz v0.7.3.
\begin{tikzpicture}

\definecolor{color0}{rgb}{0.12156862745098,0.466666666666667,0.705882352941177}
\definecolor{color1}{rgb}{1,0.498039215686275,0.0549019607843137}
\definecolor{color2}{rgb}{0.172549019607843,0.627450980392157,0.172549019607843}
\definecolor{color3}{rgb}{0.83921568627451,0.152941176470588,0.156862745098039}
\definecolor{color4}{rgb}{0.580392156862745,0.403921568627451,0.741176470588235}
\definecolor{color5}{rgb}{0.549019607843137,0.337254901960784,0.294117647058824}
\definecolor{color6}{rgb}{0.890196078431372,0.466666666666667,0.76078431372549}
\definecolor{color7}{rgb}{0.737254901960784,0.741176470588235,0.133333333333333}

\begin{axis}[
height=\figureheight,
legend cell align={left},
legend columns=3,
legend style={at={(0.97,0.03)}, anchor=south east, draw=white!80.0!black},
tick align=outside,
tick pos=left,
width=\figurewidth,
x grid style={white!69.01960784313725!black},
xlabel={Epoch},
xmin=-0.95, xmax=19.95,
xtick style={color=black},
y grid style={white!69.01960784313725!black},
ylabel={Accuracy},
ymin=0.124543308702791, ymax=1.0354525862069,
ytick style={color=black},
ytick={0,0.2,0.4,0.6,0.8,1,1.2},
yticklabels={,0.2,0.4,0.6,0.8,1.0,}
]
\addplot [semithick, color0, dotted]
table {%
0 0.333128079258162
1 0.631773399895635
2 0.640394089550808
3 0.619458128666056
4 0.626847290787204
5 0.732142857436476
6 0.658866995367511
7 0.635467980442376
8 0.974137931034483
9 0.952586206896552
10 0.943965517241379
11 0.948275862068966
12 0.951970443643373
13 0.952586206896552
14 0.956896551724138
15 0.961206896551724
16 0.956896551724138
17 0.956896551724138
18 0.96551724137931
19 0.978448275862069
};
\addlegendentry{0-3}
\addplot [semithick, color1, dotted]
table {%
0 0.253205128205128
1 0.439102564102564
2 0.416666666666667
3 0.471153846153846
4 0.467948717948718
5 0.483974358974359
6 0.474358974358974
7 0.471153846153846
8 0.644230769230769
9 0.926282051282051
10 0.92948717948718
11 0.897435897435897
12 0.923076923076923
13 0.919871794871795
14 0.91025641025641
15 0.926282051282051
16 0.951923076923077
17 0.923076923076923
18 0.980769230769231
19 0.875
};
\addlegendentry{0-4}
\addplot [semithick, color2, dotted]
table {%
0 0.214285714285714
1 0.214285714285714
2 0.196428571428571
3 0.392857142857143
4 0.497448979591837
5 0.438775510204082
6 0.813775510204082
7 0.798469387755102
8 0.823979591836735
9 0.844387755102041
10 0.823979591836735
11 0.923469387755102
12 0.816326530612245
13 0.910714285714286
14 0.793367346938776
15 0.625
16 0.895408163265306
17 0.903061224489796
18 0.887755102040816
19 0.875
};
\addlegendentry{0-5}
\addplot [semithick, color3, dotted]
table {%
0 0.165948275862069
1 0.294540230056335
2 0.599856322181636
3 0.602011494595429
4 0.650862068965517
5 0.637212643335605
6 0.704022988163192
7 0.708333332990778
8 0.665948275862069
9 0.680316091611468
10 0.732040229542502
11 0.701149425629912
12 0.790948275862069
13 0.81106321873336
14 0.739942529078188
15 0.882902298508019
16 0.84698275862069
17 0.808908046319567
18 0.747844827586207
19 0.843390804940257
};
\addlegendentry{0-6}
\addplot [semithick, color4]
table {%
0 0.446428571428571
1 0.80952380952381
2 0.770833333333333
3 0.720238095238095
4 0.75
5 0.863095238095238
6 0.901785714285714
7 0.907738095238095
8 0.910714285714286
9 0.913690476190476
10 0.803571428571429
11 0.845238095238095
12 0.976190476190476
13 0.955357142857143
14 0.976190476190476
15 0.988095238095238
16 0.93452380952381
17 0.994047619047619
18 0.991071428571429
19 0.985119047619048
};
\addlegendentry{4-0}
\addplot [semithick, color5]
table {%
0 0.181787634448659
1 0.277217741935484
2 0.272849462445705
3 0.560483870967742
4 0.820228494463428
5 0.719422042850525
6 0.929435483870968
7 0.92741935483871
8 0.921370967741935
9 0.953629032258065
10 0.960013441020443
11 0.92305107510859
12 0.971774193548387
13 0.953629032258065
14 0.975806451612903
15 0.964717741935484
16 0.981854838709677
17 0.963709677419355
18 0.976814516129032
19 0.983870967741935
};
\addlegendentry{4-3}
\addplot [semithick, color6]
table {%
0 0.175757575757576
1 0.377272727272727
2 0.407575757575758
3 0.777272727272727
4 0.893939393939394
5 0.915909090909091
6 0.928030303030303
7 0.938636363636364
8 0.94469696969697
9 0.956060606060606
10 0.931060606060606
11 0.95530303030303
12 0.962878787878788
13 0.975
14 0.973484848484849
15 0.964393939393939
16 0.970454545454545
17 0.98030303030303
18 0.967424242424242
19 0.970454545454545
};
\addlegendentry{4-4}
\addplot [semithick, white!49.80392156862745!black]
table {%
0 0.493357487922705
1 0.719202898550725
2 0.775966183574879
3 0.821859903381642
4 0.815821256038647
5 0.870772946859903
6 0.879830917874396
7 0.861111111111111
8 0.894927536231884
9 0.878019323671498
10 0.88707729468599
11 0.919685990338164
12 0.876207729468599
13 0.935386473429952
14 0.916666666666667
15 0.93719806763285
16 0.952898550724638
17 0.89975845410628
18 0.954710144927536
19 0.95048309178744
};
\addlegendentry{4-5}
\addplot [semithick, color7]
table {%
0 0.47883064516129
1 0.626008064516129
2 0.647681451612903
3 0.729334677419355
4 0.795866935483871
5 0.807963709677419
6 0.792842741935484
7 0.84375
8 0.876008064516129
9 0.901713709677419
10 0.90070564516129
11 0.901209677419355
12 0.902217741935484
13 0.916834677419355
14 0.890625
15 0.922883064516129
16 0.930947580645161
17 0.941028225806452
18 0.904737903225806
19 0.9375
};
\addlegendentry{4-6}
\end{axis}

\end{tikzpicture}
		\caption{Training accuracies of all networks}
		\label{fig:train-accuracy}
	\end{subfigure}
	\begin{subfigure}{\textwidth}
		\centering
		% This file was created by matplotlib2tikz v0.7.3.
\begin{tikzpicture}

\definecolor{color0}{rgb}{1,0.647058823529412,0}
\definecolor{color1}{rgb}{0.75,0,0.75}

\begin{axis}[
height=\figureheight,
legend cell align={left},
legend columns=3,
legend style={at={(0.97,0.03)}, anchor=south east, draw=white!80.0!black},
tick align=outside,
tick pos=left,
width=\figurewidth,
x grid style={white!69.01960784313725!black},
xlabel={Epoch},
xmin=-0.95, xmax=19.95,
xtick style={color=black},
y grid style={white!69.01960784313725!black},
ylabel={Accuracy},
ymin=0.125, ymax=1.04166666666667,
ytick style={color=black},
ytick={0,0.2,0.4,0.6,0.8,1,1.2},
yticklabels={,0.2,0.4,0.6,0.8,1.0,}
]
\addplot [semithick, blue, dotted]
table {%
0 0.333333333333333
1 0.654320987654321
2 0.666666666666667
3 0.617283950617284
4 0.654320987654321
5 0.740740740740741
6 0.666666666666667
7 0.654320987654321
8 1
9 0.975308641975309
10 0.975308641975309
11 0.975308641975309
12 0.987654320987654
13 0.987654320987654
14 0.987654320987654
15 1
16 0.975308641975309
17 0.987654320987654
18 0.987654320987654
19 1
};
\addlegendentry{0-3}
\addplot [semithick, green!50.0!black, dotted]
table {%
0 0.25
1 0.425925925925926
2 0.398148148148148
3 0.490740740740741
4 0.472222222222222
5 0.481481481481481
6 0.490740740740741
7 0.481481481481481
8 0.657407407407407
9 0.953703703703704
10 0.944444444444444
11 0.944444444444444
12 0.916666666666667
13 0.916666666666667
14 0.898148148148148
15 0.953703703703704
16 0.990740740740741
17 0.925925925925926
18 0.981481481481482
19 0.87037037037037
};
\addlegendentry{0-4}
\addplot [semithick, red, dotted]
table {%
0 0.200000000331137
1 0.2
2 0.2
3 0.385185185516322
4 0.488888890213437
5 0.459259259921533
6 0.881481481922997
7 0.851851852293368
8 0.859259259259259
9 0.851851852293368
10 0.903703703703704
11 0.948148148148148
12 0.896296296737812
13 0.925925925925926
14 0.814814814814815
15 0.555555556438587
16 0.918518518960034
17 0.940740741182257
18 0.881481481481482
19 0.851851851851852
};
\addlegendentry{0-5}
\addplot [semithick, color0, dotted]
table {%
0 0.166666666666667
1 0.296296296296296
2 0.62962962962963
3 0.648148148148148
4 0.697530864197531
5 0.679012345679012
6 0.709876543209877
7 0.728395061728395
8 0.660493827160494
9 0.691358024691358
10 0.728395061728395
11 0.740740740740741
12 0.740740740740741
13 0.753086419753086
14 0.740740740740741
15 0.827160493827161
16 0.802469135802469
17 0.796296296296296
18 0.771604938271605
19 0.839506172839506
};
\addlegendentry{0-6}
\addplot [semithick, color1]
table {%
0 0.472222222222222
1 0.75
2 0.712962962962963
3 0.75
4 0.712962962962963
5 0.87037037037037
6 0.888888888888889
7 0.87962962962963
8 0.898148148148148
9 0.87962962962963
10 0.768518518518518
11 0.796296296296296
12 0.898148148148148
13 0.962962962962963
14 0.916666666666667
15 0.953703703703704
16 0.851851851851852
17 0.972222222222222
18 0.962962962962963
19 0.925925925925926
};
\addlegendentry{4-0}
\addplot [semithick, blue]
table {%
0 0.179012345679012
1 0.240740740740741
2 0.274691358024691
3 0.493827160493827
4 0.728395061728395
5 0.657407407407407
6 0.864197530864197
7 0.864197530864197
8 0.854938271604938
9 0.888888888888889
10 0.858024691358025
11 0.858024691358025
12 0.904320987654321
13 0.898148148148148
14 0.901234567901235
15 0.882716049382716
16 0.904320987654321
17 0.891975308641975
18 0.895061728395062
19 0.925925925925926
};
\addlegendentry{4-3}
\addplot [semithick, green!50.0!black]
table {%
0 0.175925925925926
1 0.361111111111111
2 0.405092592592593
3 0.736111111111111
4 0.847222222222222
5 0.912037037037037
6 0.909722222222222
7 0.881944444444444
8 0.912037037037037
9 0.914351851851852
10 0.881944444444444
11 0.921296296296296
12 0.895833333333333
13 0.921296296296296
14 0.905092592592593
15 0.902777777777778
16 0.918981481481482
17 0.902777777777778
18 0.930555555555556
19 0.935185185185185
};
\addlegendentry{4-4}
\addplot [semithick, red]
table {%
0 0.446296296296296
1 0.67037037037037
2 0.725925925925926
3 0.748148148148148
4 0.755555555555556
5 0.811111111111111
6 0.805555555555556
7 0.798148148148148
8 0.824074074074074
9 0.777777777777778
10 0.807407407407407
11 0.82037037037037
12 0.798148148148148
13 0.851851851851852
14 0.833333333333333
15 0.874074074074074
16 0.87037037037037
17 0.812962962962963
18 0.888888888888889
19 0.885185185185185
};
\addlegendentry{4-5}
\addplot [semithick, color0]
table {%
0 0.431221020092736
1 0.585780525778692
2 0.601236476135401
3 0.649149922996621
4 0.731066460725513
5 0.752704791436792
6 0.723338485501096
7 0.758887171607113
8 0.772797527140038
9 0.774343122194134
10 0.780525502594766
11 0.788253477634934
12 0.81298299854653
13 0.789799072735092
14 0.791344667697063
15 0.839258114558283
16 0.819165378670788
17 0.856259659969088
18 0.808346213476367
19 0.84234930457435
};
\addlegendentry{4-6}
\end{axis}

\end{tikzpicture}
		\caption{Test accuracies of all networks}
		\label{fig:test-accuracy}
	\end{subfigure}
	\caption{Training and test accuracies of networks}
	\label{fig:networks-accuracy}
\end{figure}
In either the losses or accuracies no indicator of overfitting is present, hence, all networks could be trained longer for achieving better results.
This is not performed in this work, though, due to a lack of time.
The minimum losses and maximum accuracies during training and testing of all networks are noted in \tabref{tab:network-performances} for reasons of comparability.
\begin{table}
	\centering
	\caption{Performance of all networks}
	\label{tab:network-performances}
	\begin{tabular}{c|c|c|c|c|c|c|c|c|c}
		& 0-3 & 0-4 & 0-5 & 0-6 & 4-0 & 4-3 & 4-4 & 4-5 & 4-6 \\ \hline
		Train Loss & 0.085 & 0.099 & 0.275 & 0.308 & 0.031 & 0.049 & 0.06 & 0.114 & 0.152 \\
		Test Loss & 0.027 & 0.065 & 0.226 & 0.386 & 0.146 & 0.26 & 0.258 & 0.339 & 0.5 \\ \hline
		Train Accuracy & 0.978 & 0.981 & 0.923 & 0.883 & 0.994 & 0.984 & 0.98 & 0.955 & 0.941 \\
		Test Accuracy & 1.0 & 0.991 & 0.948 & 0.84 & 0.972 & 0.926 & 0.935 & 0.889 & 0.856 \\
	\end{tabular}
\end{table}