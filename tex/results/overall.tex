\section{Overall Performance}
\label{sec:results-overall}
For evaluating the overall performances of the network architecture, the choice of hyperparameters is explained.
In \figref{fig:optimal-learning-rate} the increasing of the learning rate against the related loss is shown for finding the range of optimal learning rates.
Additionally, the change in loss is outlined.
This is performed on the 0-3 network.
At around $10^{-4}$ the loss starts to decrease slightly.
At around 0.004 its decrease gets strikingly faster until at around 0.01 it starts to increase drastically.
Although for this network, in particular, a learning rate of $10^{-2}$ seems to be suited well, a general initial learning rate of $10^{-3}$ is chosen.
On one hand, this is the most basic network, hence, it is supposed, that for more complicated ones, a smaller learning rate is better suited due to the more complex cost function.
Furthermore, interpreting those graphs is time-consuming and for more complicated networks not that easy anymore, because the loss changes more rapidly.
On the other hand, a learning rate of $10^{-2}$ is close to the increase.
Hence, if the learning rate is shifted, the parameters of the network would be changed tremendously.
It was actually verified, that a learning rate of $10^{-3}$ is a satisfiable choice for more complex networks because it lies close to the upper bound of the optimal learning rates range.
As a default value for all networks, it works as well, though.
\begin{figure}
	\setlength\figureheight{.3\textwidth}
	\setlength\figurewidth{.45\textwidth}
	\centering
	\begin{subfigure}{.5\textwidth}
		\centering
		% This file was created by matplotlib2tikz v0.7.3.
\begin{tikzpicture}

\begin{axis}[
height=\figureheight,
width=\figurewidth,
log basis x={10},
tick align=outside,
tick pos=left,
x grid style={white!69.01960784313725!black},
xlabel={Learning Rate},
xmin=6.93261084814295e-06, xmax=0.0219364693423635,
xmode=log,
xtick style={color=black},
xtick={1e-07,1e-06,1e-05,0.0001,0.001,0.01,0.1,1},
xticklabels={,,\(\displaystyle 10^{-5}\),\(\displaystyle 10^{-4}\),\(\displaystyle 10^{-3}\),\(\displaystyle 10^{-2}\),,},
y grid style={white!69.01960784313725!black},
ylabel={Loss},
ymin=-0.0923433851450682, ymax=3.27405786775053,
ytick style={color=black},
ytick={-0.5,0,0.5,1,1.5,2,2.5,3,3.5},
yticklabels={,\(\displaystyle 0.0\),\(\displaystyle 0.5\),\(\displaystyle 1.0\),\(\displaystyle 1.5\),\(\displaystyle 2.0\),\(\displaystyle 2.5\),\(\displaystyle 3.0\),}
]
\addplot [semithick, green!50.0!black]
table {%
1e-05 1.24936330318451
1.02e-05 1.11771082878113
1.0404e-05 1.25553345680237
1.061208e-05 0.912209510803223
1.08243216e-05 1.17866849899292
1.1040808032e-05 1.14146018028259
1.126162419264e-05 1.00543582439423
1.14868566764928e-05 1.1741509437561
1.17165938100227e-05 0.712578415870667
1.19509256862231e-05 1.09097123146057
1.21899441999476e-05 1.04582691192627
1.24337430839465e-05 1.04490494728088
1.26824179456255e-05 1.12166380882263
1.2936066304538e-05 1.20993518829346
1.31947876306287e-05 1.20642042160034
1.34586833832413e-05 1.25412130355835
1.37278570509061e-05 1.18533754348755
1.40024141919243e-05 1.04806065559387
1.42824624757627e-05 1.03853487968445
1.4568111725278e-05 1.12853026390076
1.48594739597836e-05 1.18146014213562
1.51566634389792e-05 1.21181511878967
1.54597967077588e-05 1.07602608203888
1.5768992641914e-05 1.03872931003571
1.60843724947523e-05 1.13338041305542
1.64060599446473e-05 1.07078385353088
1.67341811435403e-05 1.11643004417419
1.70688647664111e-05 1.11921977996826
1.74102420617393e-05 1.07632839679718
1.77584469029741e-05 1.09956479072571
1.81136158410335e-05 1.01091659069061
1.84758881578542e-05 1.08717322349548
1.88454059210113e-05 1.06513786315918
1.92223140394315e-05 1.04800224304199
1.96067603202202e-05 1.01836133003235
1.99988955266246e-05 1.11461520195007
2.03988734371571e-05 1.03720045089722
2.08068509059002e-05 1.1808967590332
2.12229879240182e-05 1.11440849304199
2.16474476824986e-05 1.08650374412537
2.20803966361485e-05 1.04353654384613
2.25220045688715e-05 1.18413186073303
2.29724446602489e-05 1.03592896461487
2.34318935534539e-05 1.10247671604156
2.3900531424523e-05 1.07771921157837
2.43785420530135e-05 1.14615738391876
2.48661128940737e-05 1.06303918361664
2.53634351519552e-05 0.943422675132751
2.58707038549943e-05 1.03866362571716
2.63881179320942e-05 1.18736612796783
2.69158802907361e-05 1.07216048240662
2.74541978965508e-05 1.00082540512085
2.80032818544818e-05 1.03717625141144
2.85633474915715e-05 1.09771478176117
2.91346144414029e-05 1.09777176380157
2.9717306730231e-05 1.0956357717514
3.03116528648356e-05 1.11595177650452
3.09178859221323e-05 1.04101479053497
3.15362436405749e-05 1.01115465164185
3.21669685133864e-05 1.02764511108398
3.28103078836542e-05 1.24870812892914
3.34665140413272e-05 1.00507831573486
3.41358443221538e-05 1.11056876182556
3.48185612085969e-05 1.08555150032043
3.55149324327688e-05 1.00557243824005
3.62252310814242e-05 0.944729149341583
3.69497357030527e-05 1.11037909984589
3.76887304171137e-05 1.16250884532928
3.8442505025456e-05 0.958189606666565
3.92113551259651e-05 1.08383274078369
3.99955822284844e-05 1.12642049789429
4.07954938730541e-05 1.18159532546997
4.16114037505152e-05 1.12061786651611
4.24436318255255e-05 1.08124768733978
4.3292504462036e-05 1.08979761600494
4.41583545512767e-05 1.03111398220062
4.50415216423022e-05 1.11460852622986
4.59423520751483e-05 1.05480051040649
4.68611991166513e-05 1.09905767440796
4.77984230989843e-05 1.05474781990051
4.8754391560964e-05 1.08756709098816
4.97294793921832e-05 1.07962369918823
5.07240689800269e-05 1.036789894104
5.17385503596274e-05 1.08880829811096
5.277332136682e-05 1.06424152851105
5.38287877941564e-05 1.02296662330627
5.49053635500395e-05 0.785083770751953
5.60034708210403e-05 1.11297142505646
5.71235402374611e-05 1.02365207672119
5.82660110422103e-05 1.05898761749268
5.94313312630546e-05 1.05686950683594
6.06199578883156e-05 1.04965496063232
6.1832357046082e-05 1.01759386062622
6.30690041870036e-05 1.00451576709747
6.43303842707437e-05 1.00840127468109
6.56169919561585e-05 1.19472241401672
6.69293317952817e-05 1.16187214851379
6.82679184311873e-05 0.973621010780334
6.96332767998111e-05 0.987213134765625
7.10259423358073e-05 1.14957332611084
7.24464611825234e-05 1.05067920684814
7.38953904061739e-05 1.10304892063141
7.53732982142974e-05 1.10134148597717
7.68807641785834e-05 1.05486869812012
7.8418379462155e-05 0.917574405670166
7.99867470513981e-05 0.872981131076813
8.15864819924261e-05 1.05475461483002
8.32182116322746e-05 1.11500573158264
8.48825758649201e-05 1.00380003452301
8.65802273822185e-05 0.92658805847168
8.83118319298629e-05 1.10154104232788
9.00780685684601e-05 1.24234402179718
9.18796299398294e-05 1.04951858520508
9.37172225386259e-05 1.05376398563385
9.55915669893985e-05 0.819953501224518
9.75033983291864e-05 1.15897917747498
9.94534662957702e-05 1.09967350959778
0.000101442535621686 0.982080698013306
0.000103471386334119 0.994948983192444
0.000105540814060802 1.03953456878662
0.000107651630342018 1.02697360515594
0.000109804662948858 1.01501739025116
0.000112000756207835 0.987826108932495
0.000114240771331992 1.10533022880554
0.000116525586758632 1.05225586891174
0.000118856098493804 1.02115893363953
0.000121233220463681 1.00868940353394
0.000123657884872954 0.971531629562378
0.000126131042570413 0.957609057426453
0.000128653663421821 0.930508971214294
0.000131226736690258 0.935918867588043
0.000133851271424063 0.999650537967682
0.000136528296852544 0.943161368370056
0.000139258862789595 1.05014622211456
0.000142044040045387 1.02964305877686
0.000144884920846295 1.01328790187836
0.000147782619263221 0.864341855049133
0.000150738271648485 0.996880412101746
0.000153753037081455 1.10514998435974
0.000156828097823084 0.86850917339325
0.000159964659779546 1.12600386142731
0.000163163952975137 1.03035688400269
0.000166427232034639 1.03447246551514
0.000169755776675332 0.958478450775146
0.000173150892208839 1.04999423027039
0.000176613910053016 1.04217743873596
0.000180146188254076 1.10224974155426
0.000183749112019157 0.968043446540833
0.000187424094259541 1.03931498527527
0.000191172576144731 1.02331411838531
0.000194996027667626 1.002032995224
0.000198895948220979 1.02703738212585
0.000202873867185398 0.932914614677429
0.000206931344529106 0.852756261825562
0.000211069971419688 1.02950429916382
0.000215291370848082 0.808874905109406
0.000219597198265044 0.931403756141663
0.000223989142230344 1.01861166954041
0.000228468925074951 1.08745992183685
0.00023303830357645 1.13773846626282
0.000237699069647979 0.985420346260071
0.000242453051040939 1.0529021024704
0.000247302112061758 1.03232717514038
0.000252248154302993 0.925907611846924
0.000257293117389053 0.884053707122803
0.000262438979736834 0.6212078332901
0.000267687759331571 1.06439614295959
0.000273041514518202 1.04348242282867
0.000278502344808566 1.00455856323242
0.000284072391704737 0.966708421707153
0.000289753839538832 0.887700915336609
0.000295548916329609 1.02882313728333
0.000301459894656201 0.837652683258057
0.000307489092549325 1.14834105968475
0.000313638874400312 1.03436255455017
0.000319911651888318 0.980852484703064
0.000326309884926084 0.89490282535553
0.000332836082624606 0.977206707000732
0.000339492804277098 1.0045964717865
0.00034628266036264 0.926256656646729
0.000353208313569893 0.977613031864166
0.000360272479841291 0.992047607898712
0.000367477929438116 0.915925562381744
0.000374827488026879 0.663772523403168
0.000382324037787416 1.08836114406586
0.000389970518543165 0.85827511548996
0.000397769928914028 0.964348614215851
0.000405725327492308 1.0613226890564
0.000413839834042155 0.988701462745667
0.000422116630722998 0.988372802734375
0.000430558963337458 0.946321308612823
0.000439170142604207 0.692878603935242
0.000447953545456291 1.05103445053101
0.000456912616365417 0.987674474716187
0.000466050868692725 0.757144212722778
0.00047537188606658 0.935925245285034
0.000484879323787911 0.821495890617371
0.000494576910263669 1.02408981323242
0.000504468448468943 0.949554920196533
0.000514557817438322 0.886516690254211
0.000524848973787088 0.825913190841675
0.00053534595326283 1.00788497924805
0.000546052872328086 0.739642798900604
0.000556973929774648 0.918576240539551
0.000568113408370141 0.970334768295288
0.000579475676537544 0.802989959716797
0.000591065190068295 0.817724704742432
0.000602886493869661 0.925661981105804
0.000614944223747054 0.849228799343109
0.000627243108221995 0.977990925312042
0.000639787970386435 0.840499103069305
0.000652583729794164 0.969197034835815
0.000665635404390047 0.677034854888916
0.000678948112477848 0.825850367546082
0.000692527074727405 0.810201406478882
0.000706377616221953 0.784642696380615
0.000720505168546392 0.904613196849823
0.00073491527191732 0.919641375541687
0.000749613577355666 0.804116487503052
0.00076460584890278 0.874090015888214
0.000779897965880835 0.523968160152435
0.000795495925198452 0.914509415626526
0.000811405843702421 0.99234414100647
0.00082763396057647 0.960276961326599
0.000844186639787999 0.935146510601044
0.000861070372583759 0.875925540924072
0.000878291780035434 0.951701283454895
0.000895857615636143 0.874415755271912
0.000913774767948866 0.934476494789124
0.000932050263307843 0.917530179023743
0.000950691268574 0.885922908782959
0.00096970509394548 0.84727942943573
0.00098909919582439 0.829551100730896
0.00100888117974088 0.951678514480591
0.0010290588033357 0.925641059875488
0.00104963997940241 0.862579643726349
0.00107063277899046 0.933078765869141
0.00109204543457027 0.795441687107086
0.00111388634326167 0.912636935710907
0.0011361640701269 0.871430814266205
0.00115888735152944 0.896076083183289
0.00118206509856003 0.866928339004517
0.00120570640053123 0.925468623638153
0.00122982052854186 0.759742617607117
0.00125441693911269 0.840595006942749
0.00127950527789495 0.894559741020203
0.00130509538345285 0.89877849817276
0.0013311972911219 0.897607624530792
0.00135782123694434 0.820272028446198
0.00138497766168323 0.951055288314819
0.00141267721491689 0.949528515338898
0.00144093075921523 0.97175008058548
0.00146974937439954 0.885736107826233
0.00149914436188753 0.86873722076416
0.00152912724912528 0.952134788036346
0.00155970979410778 0.75347101688385
0.00159090398998994 0.836939334869385
0.00162272206978974 0.567769050598145
0.00165517651118553 0.886343419551849
0.00168828004140924 0.789159059524536
0.00172204564223743 0.824781596660614
0.00175648655508218 0.72742748260498
0.00179161628618382 0.678035080432892
0.0018274486119075 0.665984451770782
0.00186399758414565 0.763284206390381
0.00190127753582856 0.857166409492493
0.00193930308654513 0.815934062004089
0.00197808914827603 0.924270272254944
0.00201765093124155 0.782942116260529
0.00205800394986639 0.835995197296143
0.00209916402886371 0.704138517379761
0.00214114730944099 0.855893731117249
0.00218397025562981 0.770207464694977
0.0022276496607424 0.90678858757019
0.00227220265395725 0.886987566947937
0.0023176467070364 0.977225720882416
0.00236399964117713 0.831437289714813
0.00241127963400067 0.843997895717621
0.00245950522668068 0.787012100219727
0.00250869533121429 0.816235184669495
0.00255886923783858 0.603344440460205
0.00261004662259535 0.744927525520325
0.00266224755504726 0.63431715965271
0.0027154925061482 0.653486669063568
0.00276980235627117 0.761883080005646
0.00282519840339659 0.78454464673996
0.00288170237146452 0.791627764701843
0.00293933641889381 0.744171023368835
0.00299812314727169 0.822568416595459
0.00305808561021712 0.746788322925568
0.00311924732242147 0.755545496940613
0.0031816322688699 0.761896312236786
0.00324526491424729 0.343267858028412
0.00331017021253224 0.812684655189514
0.00337637361678289 0.82405960559845
0.00344390108911854 0.665336489677429
0.00351277911090091 0.846346139907837
0.00358303469311893 0.673141419887543
0.00365469538698131 0.567845821380615
0.00372778929472094 0.663246750831604
0.00380234508061536 0.344445824623108
0.00387839198222766 0.601183354854584
0.00395595982187222 0.619645893573761
0.00403507901830966 0.540554285049438
0.00411578059867586 0.960864663124084
0.00419809621064937 0.415902137756348
0.00428205813486236 0.75158280134201
0.00436769929755961 0.666090488433838
0.0044550532835108 0.611324548721313
0.00454415434918102 0.420097589492798
0.00463503743616464 0.484608590602875
0.00472773818488793 0.785852193832397
0.00482229294858569 0.721708297729492
0.0049187388075574 0.697919011116028
0.00501711358370855 0.697352528572083
0.00511745585538272 0.614921867847443
0.00521980497249038 0.694017469882965
0.00532420107194018 0.506624698638916
0.00543068509337899 0.189157143235207
0.00553929879524657 0.430464923381805
0.0056500847711515 0.656598508358002
0.00576308646657453 0.917730212211609
0.00587834819590602 0.295652449131012
0.00599591515982414 0.801897764205933
0.00611583346302062 1.3061910867691
0.00623815013228103 0.925675451755524
0.00636291313492665 0.491060376167297
0.00649017139762519 0.261722505092621
0.00661997482557769 0.720268428325653
0.00675237432208924 1.37355244159698
0.00688742180853103 0.84514856338501
0.00702517024470165 0.472819209098816
0.00716567364959568 0.387085288763046
0.0073089871225876 0.555275678634644
0.00745516686503935 0.529918491840363
0.00760427020234014 0.490843087434769
0.00775635560638694 0.192046701908112
0.00791148271851468 0.320494651794434
0.00806971237288497 0.456416308879852
0.00823110662034267 0.379168927669525
0.00839572875274952 0.264919608831406
0.00856364332780452 0.240180894732475
0.00873491619436061 0.258248090744019
0.00890961451824782 0.502431750297546
0.00908780680861278 0.256245642900467
0.00926956294478503 0.414601385593414
0.00945495420368073 0.0695068910717964
0.00964405328775435 0.710331380367279
0.00983693435350943 0.300577878952026
0.0100336730405796 0.141580551862717
0.0102343465013912 0.418671369552612
0.010439033431419 0.359317660331726
0.0106478141000474 0.131026819348335
0.0108607703820484 0.379914164543152
0.0110779857896893 0.367072641849518
0.0112995455054831 0.134323164820671
0.0115255364155928 0.110901422798634
0.0117560471439046 0.267156094312668
0.0119911680867827 0.356623739004135
0.0122309914485184 0.244695216417313
0.0124756112774888 0.104603454470634
0.0127251235030385 0.0606748536229134
0.0129796259730993 0.138397783041
0.0132392184925613 0.135786771774292
0.0135040028624125 0.49864587187767
0.0137740829196608 1.54214537143707
0.014049564578054 0.34622585773468
0.0143305558696151 0.560321569442749
0.0146171669870074 0.166388660669327
0.0149095103267475 1.61452615261078
0.0152077005332825 3.12103962898254
};
\end{axis}

\end{tikzpicture}
	\end{subfigure}%
	\begin{subfigure}{.5\textwidth}
		\centering
		% This file was created by matplotlib2tikz v0.7.3.
\begin{tikzpicture}

\begin{axis}[
height=\figureheight,
width=\figurewidth,
log basis x={10},
tick align=outside,
tick pos=left,
x grid style={white!69.01960784313725!black},
xlabel={Learning Rate},
xmin=6.93261084814295e-06, xmax=0.0219364693423635,
xmode=log,
xtick style={color=black},
xtick={1e-07,1e-06,1e-05,0.0001,0.001,0.01,0.1,1},
xticklabels={,,\(\displaystyle 10^{-5}\),\(\displaystyle 10^{-4}\),\(\displaystyle 10^{-3}\),\(\displaystyle 10^{-2}\),,},
y grid style={white!69.01960784313725!black},
ylabel={\(\displaystyle \partial\)Loss},
ymin=-0.590783169865608, ymax=1.60638474524021,
ytick style={color=black},
ytick={-0.75,-0.5,-0.25,0,0.25,0.5,0.75,1,1.25,1.5,1.75},
yticklabels={,\(\displaystyle -0.50\),\(\displaystyle -0.25\),\(\displaystyle 0.00\),\(\displaystyle 0.25\),\(\displaystyle 0.50\),\(\displaystyle 0.75\),\(\displaystyle 1.00\),\(\displaystyle 1.25\),\(\displaystyle 1.50\),}
]
\addplot [semithick, green!50.0!black]
table {%
1e-05 -0.131652474403381
1.02e-05 0.00308507680892944
1.0404e-05 -0.102750658988953
1.061208e-05 -0.0384324789047241
1.08243216e-05 0.114625334739685
1.1040808032e-05 -0.0866163372993469
1.126162419264e-05 0.0163453817367554
1.14868566764928e-05 -0.14642870426178
1.17165938100227e-05 -0.0415898561477661
1.19509256862231e-05 0.166624248027802
1.21899441999476e-05 -0.0230331420898438
1.24337430839465e-05 0.0379184484481812
1.26824179456255e-05 0.0825151205062866
1.2936066304538e-05 0.042378306388855
1.31947876306287e-05 0.0220930576324463
1.34586833832413e-05 -0.0105414390563965
1.37278570509061e-05 -0.103030323982239
1.40024141919243e-05 -0.0734013319015503
1.42824624757627e-05 0.0402348041534424
1.4568111725278e-05 0.0714626312255859
1.48594739597836e-05 0.041642427444458
1.51566634389792e-05 -0.0527170300483704
1.54597967077588e-05 -0.0865429043769836
1.5768992641914e-05 0.0286771655082703
1.60843724947523e-05 0.0160272717475891
1.64060599446473e-05 -0.00847518444061279
1.67341811435403e-05 0.024217963218689
1.70688647664111e-05 -0.0200508236885071
1.74102420617393e-05 -0.00982749462127686
1.77584469029741e-05 -0.0327059030532837
1.81136158410335e-05 -0.0061957836151123
1.84758881578542e-05 0.0271106362342834
1.88454059210113e-05 -0.0195854902267456
1.92223140394315e-05 -0.0233882665634155
1.96067603202202e-05 0.0333064794540405
1.99988955266246e-05 0.00941956043243408
2.03988734371571e-05 0.0331407785415649
2.08068509059002e-05 0.0386040210723877
2.12229879240182e-05 -0.0471965074539185
2.16474476824986e-05 -0.0354359745979309
2.20803966361485e-05 0.048814058303833
2.25220045688715e-05 -0.0038037896156311
2.29724446602489e-05 -0.0408275723457336
2.34318935534539e-05 0.0208951234817505
2.3900531424523e-05 0.0218403339385986
2.43785420530135e-05 -0.00734001398086548
2.48661128940737e-05 -0.101367354393005
2.53634351519552e-05 -0.0121877789497375
2.58707038549943e-05 0.121971726417542
2.63881179320942e-05 0.0167484283447266
2.69158802907361e-05 -0.0932703614234924
2.74541978965508e-05 -0.0174921154975891
2.80032818544818e-05 0.0484446883201599
2.85633474915715e-05 0.0302977561950684
2.91346144414029e-05 -0.00103950500488281
2.9717306730231e-05 0.00909000635147095
3.03116528648356e-05 -0.0273104906082153
3.09178859221323e-05 -0.0523985624313354
3.15362436405749e-05 -0.00668483972549438
3.21669685133864e-05 0.118776738643646
3.28103078836542e-05 -0.0112833976745605
3.34665140413272e-05 -0.0690696835517883
3.41358443221538e-05 0.0402365922927856
3.48185612085969e-05 -0.0524981617927551
3.55149324327688e-05 -0.0704111754894257
3.62252310814242e-05 0.0524033308029175
3.69497357030527e-05 0.108889847993851
3.76887304171137e-05 -0.0760947465896606
3.8442505025456e-05 -0.0393380522727966
3.92113551259651e-05 0.0841154456138611
3.99955822284844e-05 0.0488812923431396
4.07954938730541e-05 -0.00290131568908691
4.16114037505152e-05 -0.050173819065094
4.24436318255255e-05 -0.0154101252555847
4.3292504462036e-05 -0.0250668525695801
4.41583545512767e-05 0.0124054551124573
4.50415216423022e-05 0.0118432641029358
4.59423520751483e-05 -0.00777542591094971
4.68611991166513e-05 -2.63452529907227e-05
4.77984230989843e-05 -0.0057452917098999
4.8754391560964e-05 0.0124379396438599
4.97294793921832e-05 -0.0253885984420776
5.07240689800269e-05 0.00459229946136475
5.17385503596274e-05 0.0137258172035217
5.277332136682e-05 -0.0329208374023438
5.38287877941564e-05 -0.139578878879547
5.49053635500395e-05 0.0450024008750916
5.60034708210403e-05 0.119284152984619
5.71235402374611e-05 -0.0269919037818909
5.82660110422103e-05 0.016608715057373
5.94313312630546e-05 -0.00466632843017578
6.06199578883156e-05 -0.0196378231048584
6.1832357046082e-05 -0.0225695967674255
6.30690041870036e-05 -0.0045962929725647
6.43303842707437e-05 0.0951033234596252
6.56169919561585e-05 0.0767354369163513
6.69293317952817e-05 -0.110550701618195
6.82679184311873e-05 -0.0873295068740845
6.96332767998111e-05 0.0879761576652527
7.10259423358073e-05 0.0317330360412598
7.24464611825234e-05 -0.0232622027397156
7.38953904061739e-05 0.0253311395645142
7.53732982142974e-05 -0.0240901112556458
7.68807641785834e-05 -0.0918835401535034
7.8418379462155e-05 -0.0909437835216522
7.99867470513981e-05 0.0685901045799255
8.15864819924261e-05 0.121012300252914
8.32182116322746e-05 -0.0254772901535034
8.48825758649201e-05 -0.094208836555481
8.65802273822185e-05 0.0488705039024353
8.83118319298629e-05 0.15787798166275
9.00780685684601e-05 -0.0260112285614014
9.18796299398294e-05 -0.094290018081665
9.37172225386259e-05 -0.11478254199028
9.55915669893985e-05 0.0526075959205627
9.75033983291864e-05 0.13986000418663
9.94534662957702e-05 -0.088449239730835
0.000101442535621686 -0.0523622632026672
0.000103471386334119 0.0287269353866577
0.000105540814060802 0.0160123109817505
0.000107651630342018 -0.0122585892677307
0.000109804662948858 -0.0195737481117249
0.000112000756207835 0.0451564192771912
0.000114240771331992 0.032214879989624
0.000116525586758632 -0.0420856475830078
0.000118856098493804 -0.0217832326889038
0.000121233220463681 -0.0248136520385742
0.000123657884872954 -0.0255401730537415
0.000126131042570413 -0.0205113291740417
0.000128653663421821 -0.0108450949192047
0.000131226736690258 0.0345707833766937
0.000133851271424063 0.00362125039100647
0.000136528296852544 0.0252478420734406
0.000139258862789595 0.0432408452033997
0.000142044040045387 -0.018429160118103
0.000144884920846295 -0.0826506018638611
0.000147782619263221 -0.00820374488830566
0.000150738271648485 0.120404064655304
0.000153753037081455 -0.064185619354248
0.000156828097823084 0.010426938533783
0.000159964659779546 0.080923855304718
0.000163163952975137 -0.0457656979560852
0.000166427232034639 -0.0359392166137695
0.000169755776675332 0.00776088237762451
0.000173150892208839 0.0418494939804077
0.000176613910053016 0.0261277556419373
0.000180146188254076 -0.0370669960975647
0.000183749112019157 -0.0314673781394958
0.000187424094259541 0.0276353359222412
0.000191172576144731 -0.0186409950256348
0.000194996027667626 0.00186163187026978
0.000198895948220979 -0.0345591902732849
0.000202873867185398 -0.0871405601501465
0.000206931344529106 0.0482948422431946
0.000211069971419688 -0.021940678358078
0.000215291370848082 -0.0490502715110779
0.000219597198265044 0.1048683822155
0.000223989142230344 0.0780280828475952
0.000228468925074951 0.0595633983612061
0.00023303830357645 -0.0510197877883911
0.000237699069647979 -0.0424181818962097
0.000242453051040939 0.023453414440155
0.000247302112061758 -0.0634972453117371
0.000252248154302993 -0.0741367340087891
0.000257293117389053 -0.152349889278412
0.000262438979736834 0.090171217918396
0.000267687759331571 0.211137294769287
0.000273041514518202 -0.0299187898635864
0.000278502344808566 -0.0383870005607605
0.000284072391704737 -0.0584288239479065
0.000289753839538832 0.0310573577880859
0.000295548916329609 -0.0250241160392761
0.000301459894656201 0.0597589612007141
0.000307489092549325 0.0983549356460571
0.000313638874400312 -0.0837442874908447
0.000319911651888318 -0.0697298645973206
0.000326309884926084 -0.00182288885116577
0.000332836082624606 0.0548468232154846
0.000339492804277098 -0.025475025177002
0.00034628266036264 -0.0134917199611664
0.000353208313569893 0.0328954756259918
0.000360272479841291 -0.0308437347412109
0.000367477929438116 -0.164137542247772
0.000374827488026879 0.0862177908420563
0.000382324037787416 0.097251296043396
0.000389970518543165 -0.0620062649250031
0.000397769928914028 0.101523786783218
0.000405725327492308 0.0121764242649078
0.000413839834042155 -0.0364749431610107
0.000422116630722998 -0.0211900770664215
0.000430558963337458 -0.147747099399567
0.000439170142604207 0.0523565709590912
0.000447953545456291 0.147397935390472
0.000456912616365417 -0.146945118904114
0.000466050868692725 -0.0258746147155762
0.00047537188606658 0.0321758389472961
0.000484879323787911 0.0440822839736938
0.000494576910263669 0.0640295147895813
0.000504468448468943 -0.0687865614891052
0.000514557817438322 -0.0618208646774292
0.000524848973787088 0.0606841444969177
0.00053534595326283 -0.0431351959705353
0.000546052872328086 -0.044654369354248
0.000556973929774648 0.115345984697342
0.000568113408370141 -0.057793140411377
0.000579475676537544 -0.0763050317764282
0.000591065190068295 0.0613360106945038
0.000602886493869661 0.0157520473003387
0.000614944223747054 0.0261644721031189
0.000627243108221995 -0.00436484813690186
0.000639787970386435 -0.0043969452381134
0.000652583729794164 -0.0817321240901947
0.000665635404390047 -0.0716733336448669
0.000678948112477848 0.0665832757949829
0.000692527074727405 -0.0206038355827332
0.000706377616221953 0.0472058951854706
0.000720505168546392 0.0674993395805359
0.00073491527191732 -0.0502483546733856
0.000749613577355666 -0.0227756798267365
0.00076460584890278 -0.140074163675308
0.000779897965880835 0.0202096998691559
0.000795495925198452 0.234187990427017
0.000811405843702421 0.0228837728500366
0.00082763396057647 -0.028598815202713
0.000844186639787999 -0.0421757102012634
0.000861070372583759 0.00827738642692566
0.000878291780035434 -0.000754892826080322
0.000895857615636143 -0.00861239433288574
0.000913774767948866 0.0215572118759155
0.000932050263307843 -0.0242767930030823
0.000950691268574 -0.0351253747940063
0.00096970509394548 -0.0281859040260315
0.00098909919582439 0.0521995425224304
0.00100888117974088 0.0480449795722961
0.0010290588033357 -0.044549435377121
0.00104963997940241 0.00371885299682617
0.00107063277899046 -0.0335689783096313
0.00109204543457027 -0.0102209150791168
0.00111388634326167 0.0379945635795593
0.0011361640701269 -0.0082804262638092
0.00115888735152944 -0.00225123763084412
0.00118206509856003 0.0146962702274323
0.00120570640053123 -0.0535928606987
0.00122982052854186 -0.042436808347702
0.00125441693911269 0.067408561706543
0.00127950527789495 0.0290917456150055
0.00130509538345285 0.0015239417552948
0.0013311972911219 -0.0392532348632812
0.00135782123694434 0.0267238318920135
0.00138497766168323 0.0646282434463501
0.00141267721491689 0.0103473961353302
0.00144093075921523 -0.0318962037563324
0.00146974937439954 -0.0515064299106598
0.00149914436188753 0.0331993401050568
0.00152912724912528 -0.057633101940155
0.00155970979410778 -0.0575977265834808
0.00159090398998994 -0.0928509831428528
0.00162272206978974 0.0247020423412323
0.00165517651118553 0.110695004463196
0.00168828004140924 -0.0307809114456177
0.00172204564223743 -0.0308657884597778
0.00175648655508218 -0.0733732581138611
0.00179161628618382 -0.030721515417099
0.0018274486119075 0.0426245629787445
0.00186399758414565 0.0955909788608551
0.00190127753582856 0.0263249278068542
0.00193930308654513 0.0335519313812256
0.00197808914827603 -0.0164959728717804
0.00201765093124155 -0.0441375374794006
0.00205800394986639 -0.0394017994403839
0.00209916402886371 0.00994926691055298
0.00214114730944099 0.033034473657608
0.00218397025562981 0.0254474282264709
0.0022276496607424 0.0583900511264801
0.00227220265395725 0.0352185666561127
0.0023176467070364 -0.0277751386165619
0.00236399964117713 -0.0666139125823975
0.00241127963400067 -0.0222125947475433
0.00245950522668068 -0.0138813555240631
0.00250869533121429 -0.0918338298797607
0.00255886923783858 -0.035653829574585
0.00261004662259535 0.0154863595962524
0.00266224755504726 -0.0457204282283783
0.0027154925061482 0.0637829601764679
0.00276980235627117 0.0655289888381958
0.00282519840339659 0.0148723423480988
0.00288170237146452 -0.0201868116855621
0.00293933641889381 0.0154703259468079
0.00299812314727169 0.00130864977836609
0.00305808561021712 -0.0335114598274231
0.00311924732242147 0.00755399465560913
0.0031816322688699 -0.2061388194561
0.00324526491424729 0.0253941714763641
0.00331017021253224 0.240395873785019
0.00337637361678289 -0.0736740827560425
0.00344390108911854 0.0111432671546936
0.00351277911090091 0.00390246510505676
0.00358303469311893 -0.139250159263611
0.00365469538698131 -0.00494733452796936
0.00372778929472094 -0.111699998378754
0.00380234508061536 -0.0310316979885101
0.00387839198222766 0.137600034475327
0.00395595982187222 -0.0303145349025726
0.00403507901830966 0.170609384775162
0.00411578059867586 -0.0623260736465454
0.00419809621064937 -0.104640930891037
0.00428205813486236 0.125094175338745
0.00436769929755961 -0.0701291263103485
0.0044550532835108 -0.12299644947052
0.00454415434918102 -0.0633579790592194
0.00463503743616464 0.1828773021698
0.00472773818488793 0.118549853563309
0.00482229294858569 -0.0439665913581848
0.0049187388075574 -0.0121778845787048
0.00501711358370855 -0.0414985716342926
0.00511745585538272 -0.00166752934455872
0.00521980497249038 -0.0541485846042633
0.00532420107194018 -0.25243017077446
0.00543068509337899 -0.0380798876285553
0.00553929879524657 0.233720690011978
0.0056500847711515 0.243632644414902
0.00576308646657453 -0.180473029613495
0.00587834819590602 -0.0579162240028381
0.00599591515982414 0.505269289016724
0.00611583346302062 0.0618888437747955
0.00623815013228103 -0.407565355300903
0.00636291313492665 -0.331976473331451
0.00649017139762519 0.114604026079178
0.00661997482557769 0.555914998054504
0.00675237432208924 0.0624400675296783
0.00688742180853103 -0.450366616249084
0.00702517024470165 -0.229031637310982
0.00716567364959568 0.0412282347679138
0.0073089871225876 0.0714166015386581
0.00745516686503935 -0.0322162955999374
0.00760427020234014 -0.168935894966125
0.00775635560638694 -0.0851742178201675
0.00791148271851468 0.13218480348587
0.00806971237288497 0.0293371379375458
0.00823110662034267 -0.0957483500242233
0.00839572875274952 -0.0694940164685249
0.00856364332780452 -0.00333575904369354
0.00873491619436061 0.131125420331955
0.00890961451824782 -0.00100122392177582
0.00908780680861278 -0.043915182352066
0.00926956294478503 -0.0933693796396255
0.00945495420368073 0.147864997386932
0.00964405328775435 0.115535497665405
0.00983693435350943 -0.284375429153442
0.0100336730405796 0.059046745300293
0.0102343465013912 0.108868554234505
0.010439033431419 -0.143822282552719
0.0106478141000474 0.0102982521057129
0.0108607703820484 0.118022911250591
0.0110779857896893 -0.12279549986124
0.0112995455054831 -0.128085613250732
0.0115255364155928 0.0664164647459984
0.0117560471439046 0.122861161828041
0.0119911680867827 -0.0112304389476776
0.0122309914485184 -0.126010149717331
0.0124756112774888 -0.0920101851224899
0.0127251235030385 0.016897164285183
0.0129796259730993 0.0375559590756893
0.0132392184925613 0.180124044418335
0.0135040028624125 0.70317929983139
0.0137740829196608 -0.0762100070714951
0.014049564578054 -0.490911900997162
0.0143305558696151 -0.0899185985326767
0.0146171669870074 0.527102291584015
0.0149095103267475 1.47732543945312
0.0152077005332825 1.50651347637177
};
\end{axis}

\end{tikzpicture}
	\end{subfigure}
	\caption[Optimal learning rate for the 0-3 network]{Optimal learning rate for the 0-3 network. Learning rate is initialized with 0.00001 and multiplied by 1.02 every iteration.}
	\label{fig:optimal-learning-rate}
\end{figure}

Furthermore, the decreased filter size in the first convolutional layer from $11 \times 11$ to $7 \times 7$ compared to the original AlexNet configuration is evaluated.
In \figref{fig:first-conv-filter} the losses of the training process of both configurations are shown.
\begin{figure}
	\setlength\figureheight{.4\textwidth}
	\setlength\figurewidth{.9\textwidth}
	\centering
	% This file was created by matplotlib2tikz v0.7.3.
\begin{tikzpicture}

\definecolor{color0}{rgb}{0.12156862745098,0.466666666666667,0.705882352941177}

\begin{axis}[
height=\figureheight,
legend cell align={left},
legend style={draw=white!80.0!black},
tick align=outside,
tick pos=left,
width=\figurewidth,
x grid style={white!69.01960784313725!black},
xlabel={Epoch},
xmin=-0.95, xmax=19.95,
xtick style={color=black},
y grid style={white!69.01960784313725!black},
ylabel={Loss},
ymin=0.020286288957793, ymax=1.44030722774124,
ytick style={color=black},
ytick={0,0.2,0.4,0.6,0.8,1,1.2,1.4,1.6},
yticklabels={,0.2,0.4,0.6,0.8,1.0,1.2,1.4,}
]
\addplot [semithick, green!50.0!black]
table {%
0 1.13833824724987
1 0.812229750485256
2 0.584070554067349
3 0.653840091721765
4 0.611592817923118
5 0.511397582703623
6 0.505834147334099
7 0.570515671680713
8 0.265623657032847
9 0.215395397486583
10 0.189785755399706
11 0.177184161087819
12 0.167540197935084
13 0.169515813604511
14 0.148586822994824
15 0.131860823274173
16 0.150735137855698
17 0.151333889677924
18 0.116580078892153
19 0.0848326952661315
};
\addlegendentry{7x7}
\addplot [semithick, color0]
table {%
0 1.08473345329022
1 0.674929343420884
2 0.603545965819523
3 0.614335598616764
4 0.611164523609753
5 0.570500518741279
6 1.3757608214329
7 0.27132933920426
8 0.246414459727303
9 0.262517018436358
10 0.259975378112546
11 0.249353414080266
12 0.212313252982908
13 0.255264111120125
14 0.255625974258472
15 0.254195858628072
16 0.252674387382536
17 0.254422357755488
18 0.253335101273039
19 0.252868478603918
};
\addlegendentry{11x11}
\end{axis}

\end{tikzpicture}
	\caption{Comparison of filter sizes of first convolutional layer based on loss}
	\label{fig:first-conv-filter}
\end{figure}
It can be seen, that with the smaller filter the loss decreases over time, while for the other filter the loss saturates after 13 epochs.
The latter presumably got stuck on a saddle point before and would decrease further with more training epochs.
This could have been an unfavorable weight initialization, but based on all cost evaluations, the loss with the $7 \times 7$ filter decreases much more and faster.
That means, there were either many saddle points very close to each other, or the performance of the $11 \times 11$ filter is actually worse.
Because more recent convolutional networks tend to use smaller filters, the latter theory is assumed.
Hence, a filter with a size of $7 \times 7$ is chosen for the first convolution.

The overall training losses for all networks are shown in \figref{fig:train-loss} and the testing losses in \figref{fig:train-loss}.
For a more compact visualization of training and testing, they are only split into two separate graphs.
As expected, the 0-3 network starts with the smallest loss and proceeds the most smoothly compared to all other networks, due to its simplicity.
During training most closely to this comes the 0-4 network, however, with more rapid changes.
This is expected as well because it is just slightly more complicated.
The 0-5 and 0-6 network have since the 12th epoch the highest losses of all single category networks.
This is not surprising, because they are challenged with the double material features.
However, as the training proceeds, their cost function is noticeable going to be minimized, it just takes longer due to their complexity compared to the other single category networks.
It is surprising, though, that the remaining networks are part of the ones with the smallest losses.
Based on those, they can stick with the 0-3 model.
Moreover, since the 13th epoch, they change considerably small in loss compared to the single category ones.
However, this is difficult to explain, because the cost function is unknown.
It could be, that they are on a plateau with only a small slope.
Though it is unlikely that this happens to all of them in the same epoch when every network has different initialized weights, hence are located on different spots at the cost function.
If there is only small progress, because the parameters are close to a very small local minimum or the actual global one of each cost function, an indicator of overfitting could be noticeable in the testing losses.
However, there is no obvious increase in loss visible.
Not even in the direct comparison of each network's training and testing losses.
The only visible increase is for the 4-3 network after the 14th epoch, but it decreases after the 18th again.
So it was presumably only on a bad location for generalization.
Hence, the training of the networks can be continued for more epochs for trying to achieve a smaller loss and better generalization.
It is not surprising, that the 0-3 network has the smallest loss again.
The other networks are similar to each other.
However, it is noticeable, that the single-category networks have more rapid changes at the beginning and the remaining networks later in the training process.
This is presumably because of the different number of iterations per epoch.
The more complex networks have larger datasets, due to the additional number of material features and categories, hence, more batches with a parameter adaption after each.
This way the less complex networks need more epochs for having processed the same number of batches than a complex network.
That also explains why the four-categories networks have less noticeable changes after several epochs than the single-category ones.
For the sake of completeness, the related accuracies of the training processes are shown in \figref{fig:networks-accuracy}.
Here the same effects can be seen as with the losses.
\begin{figure}
	\setlength\figureheight{.35\textwidth}
	\setlength\figurewidth{.9\textwidth}
	\centering
	\begin{subfigure}{\textwidth}
		\centering
		% This file was created by matplotlib2tikz v0.7.3.
\begin{tikzpicture}

\definecolor{color0}{rgb}{1,0.647058823529412,0}
\definecolor{color1}{rgb}{0.75,0,0.75}

\begin{axis}[
height=\figureheight,
legend cell align={left},
legend columns=3,
legend style={draw=white!80.0!black},
tick align=outside,
tick pos=left,
width=\figurewidth,
x grid style={white!69.01960784313725!black},
xlabel={Epoch},
xmin=-0.95, xmax=19.95,
xtick style={color=black},
y grid style={white!69.01960784313725!black},
ylabel={Loss},
ymin=-0.0805723423154104, ymax=2.38241177177559,
ytick style={color=black}
]
\addplot [semithick, blue, dotted]
table {%
0 1.13833824724987
1 0.812229750485256
2 0.584070554067349
3 0.653840091721765
4 0.611592817923118
5 0.511397582703623
6 0.505834147334099
7 0.570515671680713
8 0.265623657032847
9 0.215395397486583
10 0.189785755399706
11 0.177184161087819
12 0.167540197935084
13 0.169515813604511
14 0.148586822994824
15 0.131860823274173
16 0.150735137855698
17 0.151333889677924
18 0.116580078892153
19 0.0848326952661315
};
\addlegendentry{0-3}
\addplot [semithick, green!50.0!black, dotted]
table {%
0 1.41628715319511
1 1.06693764069141
2 1.27663794389138
3 0.939248499197838
4 0.922509286648188
5 0.814877930359963
6 0.944846005011827
7 0.837868760793637
8 0.588574065993994
9 0.254980220531042
10 0.246385987370442
11 0.425020346274743
12 0.251233636520994
13 0.273704399283116
14 0.269391394674014
15 0.225501206488563
16 0.142166017674101
17 0.232411613449072
18 0.0985216816457418
19 0.37464939076931
};
\addlegendentry{0-4}
\addplot [semithick, red, dotted]
table {%
0 1.61394396363472
1 1.61241714808406
2 1.61542675933059
3 1.19769875370726
4 1.12012973610236
5 1.07242683975064
6 0.572706997715298
7 0.488326222312694
8 0.365541997764792
9 0.390685512570246
10 0.420321245582736
11 0.296174842027985
12 0.369830398535221
13 0.275235192127982
14 0.431895680117364
15 0.773406562148308
16 0.356734114718072
17 0.310169430259539
18 0.325223492087834
19 0.313252269290388
};
\addlegendentry{0-5}
\addplot [semithick, color0, dotted]
table {%
0 1.79954144872468
1 1.39765593820605
2 0.918116900941421
3 0.811565977746043
4 0.707633372010856
5 0.862650385704534
6 0.575900484261842
7 0.68641593435715
8 0.658841194777653
9 0.546973061201901
10 0.489648585946395
11 0.596296549614133
12 0.474540107466024
13 0.406798694154312
14 0.514070398463257
15 0.308308220500576
16 0.34670556285258
17 0.41068932555359
18 0.531169772405049
19 0.371928190619781
};
\addlegendentry{0-6}
\addplot [semithick, color1]
table {%
0 1.31922038680031
1 0.537356126166525
2 0.509941637161232
3 0.668747946165413
4 0.552789914310865
5 0.335837253502437
6 0.273976282881839
7 0.229415118783003
8 0.252485667488405
9 0.234744345137317
10 0.484040063817139
11 0.443404531736243
12 0.100486980995075
13 0.106039581149595
14 0.0911984494873433
15 0.090028658032506
16 0.159029026881659
17 0.0375711282840279
18 0.0313814810523625
19 0.0501351276997455
};
\addlegendentry{4-0}
\addplot [semithick, blue]
table {%
0 2.27045794840782
1 1.67320595825872
2 1.52473438939741
3 1.22753044962883
4 0.567531871819688
5 0.639054567823487
6 0.250710032309496
7 0.246565743019023
8 0.266477209208871
9 0.157806131616047
10 0.114142085423678
11 0.236958690025213
12 0.102651759707031
13 0.143183809564782
14 0.0727671862002734
15 0.123320500984377
16 0.0510480908447613
17 0.105988740771511
18 0.0522422706475364
19 0.0493501171798515
};
\addlegendentry{4-3}
\addplot [semithick, green!50.0!black]
table {%
0 2.09102133259629
1 1.4918020678289
2 1.26745744762999
3 0.650058149388342
4 0.363539182530208
5 0.291427397902942
6 0.23953185002249
7 0.204724518577992
8 0.160771581341484
9 0.154362564776657
10 0.239738764779847
11 0.151718994489673
12 0.118789408258586
13 0.0819602399104924
14 0.0997377019544894
15 0.109213638975934
16 0.103474246655506
17 0.0601031640727976
18 0.0958307481109814
19 0.0929106862178411
};
\addlegendentry{4-4}
\addplot [semithick, red]
table {%
0 1.43747566949918
1 0.890828702328862
2 0.643040600048315
3 0.484754529804135
4 0.477274538130288
5 0.351468806366486
6 0.323403118142255
7 0.376530914656831
8 0.28521834227036
9 0.311579979600718
10 0.274974094910754
11 0.189092540201295
12 0.331202010257431
13 0.153885170752138
14 0.222422234835508
15 0.172085734412686
16 0.131525352002787
17 0.275105380572397
18 0.114382526963265
19 0.141973072747308
};
\addlegendentry{4-5}
\addplot [semithick, color0]
table {%
0 1.65152212016044
1 1.06966266096119
2 0.927276197340219
3 0.712931053652879
4 0.536401939803674
5 0.451171906214328
6 0.52690852489022
7 0.392871726815018
8 0.321086789220346
9 0.291251453796342
10 0.25394301768674
11 0.255851911665823
12 0.255499456545353
13 0.224904286973704
14 0.287978695735576
15 0.214823928302071
16 0.197295561795228
17 0.156861799005832
18 0.26024220253232
19 0.152284982224603
};
\addlegendentry{4-6}
\end{axis}

\end{tikzpicture}
		\caption{Training losses of all networks}
		\label{fig:train-loss}
	\end{subfigure}
	\begin{subfigure}{\textwidth}
		\centering
		% This file was created by matplotlib2tikz v0.7.3.
\begin{tikzpicture}

\definecolor{color0}{rgb}{1,0.647058823529412,0}
\definecolor{color1}{rgb}{0.75,0,0.75}

\begin{axis}[
height=\figureheight,
legend cell align={left},
legend columns=3,
legend style={draw=white!80.0!black},
tick align=outside,
tick pos=left,
width=\figurewidth,
x grid style={white!69.01960784313725!black},
xlabel={Epoch},
xmin=-0.95, xmax=19.95,
xtick style={color=black},
y grid style={white!69.01960784313725!black},
ylabel={Loss},
ymin=-0.0853052690211269, ymax=2.39284538854327,
ytick style={color=black}
]
\addplot [semithick, blue, dotted]
table {%
0 1.13834580963041
1 0.819701056421539
2 0.529852289476512
3 0.650018291708864
4 0.541065454483032
5 0.477622405982312
6 0.482195921150255
7 0.484079216733391
8 0.240436148461828
9 0.109353313831911
10 0.10457138882743
11 0.0867671773389534
12 0.0680540795495481
13 0.100065112113953
14 0.0623865347401595
15 0.0455729867573138
16 0.090689538989538
17 0.0848291361773456
18 0.0580837203212726
19 0.0273379426863458
};
\addlegendentry{0-3}
\addplot [semithick, green!50.0!black, dotted]
table {%
0 1.41877777046627
1 1.09895885432208
2 1.30189108848572
3 0.852174897988637
4 0.93568006268254
5 0.74748252497779
6 0.904305950359062
7 0.772045022911496
8 0.482929673459795
9 0.17125207323719
10 0.182321731139112
11 0.280397803143219
12 0.212415169510576
13 0.312828022848677
14 0.256048236870103
15 0.147567953224535
16 0.0646619006853413
17 0.227034108368335
18 0.0956500672079899
19 0.319737830095821
};
\addlegendentry{0-4}
\addplot [semithick, red, dotted]
table {%
0 1.61795137634984
1 1.61525387322461
2 1.6126560193521
3 1.17054683190805
4 1.0959413016284
5 1.00886131834101
6 0.465152272471675
7 0.401622267784896
8 0.299568555200541
9 0.32522867973204
10 0.275719350134885
11 0.24882577293449
12 0.2741159816583
13 0.226243400408162
14 0.357824105189906
15 0.839071429658819
16 0.272184612353643
17 0.227931215917623
18 0.350412118269338
19 0.426525664991803
};
\addlegendentry{0-5}
\addplot [semithick, color0, dotted]
table {%
0 1.79808582788632
1 1.39468362301956
2 0.881186917976097
3 0.65552407613507
4 0.677725610173779
5 0.742035320419588
6 0.541102151443929
7 0.565531315626921
8 0.580334145621753
9 0.575805837725416
10 0.467508394777039
11 0.549396163887448
12 0.451250706189944
13 0.422776754991508
14 0.610205575271889
15 0.385673365935131
16 0.400466478027311
17 0.415432715673506
18 0.506099748574657
19 0.388884787574226
};
\addlegendentry{0-6}
\addplot [semithick, color1]
table {%
0 1.31252484851413
1 0.494484464327494
2 0.605050813268732
3 0.667229978712621
4 0.706620792547862
5 0.389387093760349
6 0.355923498808234
7 0.368730988491465
8 0.393539419915113
9 0.42179138544533
10 0.698278293560078
11 0.733306862965778
12 0.285662040804271
13 0.186879033498742
14 0.258718020455153
15 0.201414605759998
16 0.396008350817418
17 0.158963799273436
18 0.145997776799723
19 0.202985017853617
};
\addlegendentry{4-0}
\addplot [semithick, blue]
table {%
0 2.2802021768358
1 1.74962776972924
2 1.59888599242693
3 1.3850210522428
4 0.864513941385128
5 0.856940123769972
6 0.442444276846485
7 0.541102950219755
8 0.586551466948111
9 0.470370907451452
10 0.50805777983372
11 0.577160506244795
12 0.403487660729701
13 0.399087234943484
14 0.350926063989324
15 0.642531458205072
16 0.439764100159933
17 0.592475035216337
18 0.497008767956293
19 0.259522616167633
};
\addlegendentry{4-3}
\addplot [semithick, green!50.0!black]
table {%
0 2.12454695392538
1 1.60750920242733
2 1.32656559679243
3 0.808148795255908
4 0.550891699148687
5 0.351662465612646
6 0.295548688499602
7 0.408893935992469
8 0.294318522395635
9 0.278538173228433
10 0.487413752785263
11 0.294311502775936
12 0.378941870363498
13 0.257949455544197
14 0.402242276903794
15 0.369214037299605
16 0.303164472570643
17 0.392844607465021
18 0.266194019669287
19 0.272122140279626
};
\addlegendentry{4-4}
\addplot [semithick, red]
table {%
0 1.62563294657954
1 1.05228343981284
2 0.795493834658905
3 0.763625848293304
4 0.70652764638265
5 0.575109307026422
6 0.691668875763814
7 0.621096031312589
8 0.616801506129128
9 0.616154017837511
10 0.538304043919952
11 0.561477287444803
12 0.792346257109333
13 0.457350617923118
14 0.526198475600945
15 0.357124763189091
16 0.445093795691651
17 0.610736994731619
18 0.339136986151614
19 0.395979986526072
};
\addlegendentry{4-5}
\addplot [semithick, color0]
table {%
0 1.70052718342364
1 1.20901169843246
2 1.20705273770475
3 1.01885115814725
4 0.740241774635669
5 0.633953313992418
6 0.864576328804909
7 0.679498406429379
8 0.701618151281495
9 0.694343101020206
10 0.643477512209732
11 0.724102863342353
12 0.607141545327406
13 0.70988562650161
14 0.747071208985065
15 0.533761044385536
16 0.607914808129969
17 0.499952533001647
18 0.701931490941431
19 0.581270127452138
};
\addlegendentry{4-6}
\end{axis}

\end{tikzpicture}
		\caption{Test losses of all networks}
		\label{fig:test-loss}
	\end{subfigure}
	\caption{Training and test losses of networks}
	\label{fig:networks-loss}
\end{figure}
\begin{figure}
	\setlength\figureheight{.35\textwidth}
	\setlength\figurewidth{.9\textwidth}
	\centering
	\begin{subfigure}{\textwidth}
		\centering
		% This file was created by matplotlib2tikz v0.7.3.
\begin{tikzpicture}

\definecolor{color0}{rgb}{0.12156862745098,0.466666666666667,0.705882352941177}
\definecolor{color1}{rgb}{1,0.498039215686275,0.0549019607843137}
\definecolor{color2}{rgb}{0.172549019607843,0.627450980392157,0.172549019607843}
\definecolor{color3}{rgb}{0.83921568627451,0.152941176470588,0.156862745098039}
\definecolor{color4}{rgb}{0.580392156862745,0.403921568627451,0.741176470588235}
\definecolor{color5}{rgb}{0.549019607843137,0.337254901960784,0.294117647058824}
\definecolor{color6}{rgb}{0.890196078431372,0.466666666666667,0.76078431372549}
\definecolor{color7}{rgb}{0.737254901960784,0.741176470588235,0.133333333333333}

\begin{axis}[
height=\figureheight,
legend cell align={left},
legend columns=3,
legend style={at={(0.97,0.03)}, anchor=south east, draw=white!80.0!black},
tick align=outside,
tick pos=left,
width=\figurewidth,
x grid style={white!69.01960784313725!black},
xlabel={Epoch},
xmin=-0.95, xmax=19.95,
xtick style={color=black},
y grid style={white!69.01960784313725!black},
ylabel={Accuracy},
ymin=0.124543308702791, ymax=1.0354525862069,
ytick style={color=black},
ytick={0,0.2,0.4,0.6,0.8,1,1.2},
yticklabels={,0.2,0.4,0.6,0.8,1.0,}
]
\addplot [semithick, color0, dotted]
table {%
0 0.333128079258162
1 0.631773399895635
2 0.640394089550808
3 0.619458128666056
4 0.626847290787204
5 0.732142857436476
6 0.658866995367511
7 0.635467980442376
8 0.974137931034483
9 0.952586206896552
10 0.943965517241379
11 0.948275862068966
12 0.951970443643373
13 0.952586206896552
14 0.956896551724138
15 0.961206896551724
16 0.956896551724138
17 0.956896551724138
18 0.96551724137931
19 0.978448275862069
};
\addlegendentry{0-3}
\addplot [semithick, color1, dotted]
table {%
0 0.253205128205128
1 0.439102564102564
2 0.416666666666667
3 0.471153846153846
4 0.467948717948718
5 0.483974358974359
6 0.474358974358974
7 0.471153846153846
8 0.644230769230769
9 0.926282051282051
10 0.92948717948718
11 0.897435897435897
12 0.923076923076923
13 0.919871794871795
14 0.91025641025641
15 0.926282051282051
16 0.951923076923077
17 0.923076923076923
18 0.980769230769231
19 0.875
};
\addlegendentry{0-4}
\addplot [semithick, color2, dotted]
table {%
0 0.214285714285714
1 0.214285714285714
2 0.196428571428571
3 0.392857142857143
4 0.497448979591837
5 0.438775510204082
6 0.813775510204082
7 0.798469387755102
8 0.823979591836735
9 0.844387755102041
10 0.823979591836735
11 0.923469387755102
12 0.816326530612245
13 0.910714285714286
14 0.793367346938776
15 0.625
16 0.895408163265306
17 0.903061224489796
18 0.887755102040816
19 0.875
};
\addlegendentry{0-5}
\addplot [semithick, color3, dotted]
table {%
0 0.165948275862069
1 0.294540230056335
2 0.599856322181636
3 0.602011494595429
4 0.650862068965517
5 0.637212643335605
6 0.704022988163192
7 0.708333332990778
8 0.665948275862069
9 0.680316091611468
10 0.732040229542502
11 0.701149425629912
12 0.790948275862069
13 0.81106321873336
14 0.739942529078188
15 0.882902298508019
16 0.84698275862069
17 0.808908046319567
18 0.747844827586207
19 0.843390804940257
};
\addlegendentry{0-6}
\addplot [semithick, color4]
table {%
0 0.446428571428571
1 0.80952380952381
2 0.770833333333333
3 0.720238095238095
4 0.75
5 0.863095238095238
6 0.901785714285714
7 0.907738095238095
8 0.910714285714286
9 0.913690476190476
10 0.803571428571429
11 0.845238095238095
12 0.976190476190476
13 0.955357142857143
14 0.976190476190476
15 0.988095238095238
16 0.93452380952381
17 0.994047619047619
18 0.991071428571429
19 0.985119047619048
};
\addlegendentry{4-0}
\addplot [semithick, color5]
table {%
0 0.181787634448659
1 0.277217741935484
2 0.272849462445705
3 0.560483870967742
4 0.820228494463428
5 0.719422042850525
6 0.929435483870968
7 0.92741935483871
8 0.921370967741935
9 0.953629032258065
10 0.960013441020443
11 0.92305107510859
12 0.971774193548387
13 0.953629032258065
14 0.975806451612903
15 0.964717741935484
16 0.981854838709677
17 0.963709677419355
18 0.976814516129032
19 0.983870967741935
};
\addlegendentry{4-3}
\addplot [semithick, color6]
table {%
0 0.175757575757576
1 0.377272727272727
2 0.407575757575758
3 0.777272727272727
4 0.893939393939394
5 0.915909090909091
6 0.928030303030303
7 0.938636363636364
8 0.94469696969697
9 0.956060606060606
10 0.931060606060606
11 0.95530303030303
12 0.962878787878788
13 0.975
14 0.973484848484849
15 0.964393939393939
16 0.970454545454545
17 0.98030303030303
18 0.967424242424242
19 0.970454545454545
};
\addlegendentry{4-4}
\addplot [semithick, white!49.80392156862745!black]
table {%
0 0.493357487922705
1 0.719202898550725
2 0.775966183574879
3 0.821859903381642
4 0.815821256038647
5 0.870772946859903
6 0.879830917874396
7 0.861111111111111
8 0.894927536231884
9 0.878019323671498
10 0.88707729468599
11 0.919685990338164
12 0.876207729468599
13 0.935386473429952
14 0.916666666666667
15 0.93719806763285
16 0.952898550724638
17 0.89975845410628
18 0.954710144927536
19 0.95048309178744
};
\addlegendentry{4-5}
\addplot [semithick, color7]
table {%
0 0.47883064516129
1 0.626008064516129
2 0.647681451612903
3 0.729334677419355
4 0.795866935483871
5 0.807963709677419
6 0.792842741935484
7 0.84375
8 0.876008064516129
9 0.901713709677419
10 0.90070564516129
11 0.901209677419355
12 0.902217741935484
13 0.916834677419355
14 0.890625
15 0.922883064516129
16 0.930947580645161
17 0.941028225806452
18 0.904737903225806
19 0.9375
};
\addlegendentry{4-6}
\end{axis}

\end{tikzpicture}
		\caption{Training accuracies of all networks}
		\label{fig:train-accuracy}
	\end{subfigure}
	\begin{subfigure}{\textwidth}
		\centering
		% This file was created by matplotlib2tikz v0.7.3.
\begin{tikzpicture}

\definecolor{color0}{rgb}{1,0.647058823529412,0}
\definecolor{color1}{rgb}{0.75,0,0.75}

\begin{axis}[
height=\figureheight,
legend cell align={left},
legend columns=3,
legend style={at={(0.97,0.03)}, anchor=south east, draw=white!80.0!black},
tick align=outside,
tick pos=left,
width=\figurewidth,
x grid style={white!69.01960784313725!black},
xlabel={Epoch},
xmin=-0.95, xmax=19.95,
xtick style={color=black},
y grid style={white!69.01960784313725!black},
ylabel={Accuracy},
ymin=0.125, ymax=1.04166666666667,
ytick style={color=black},
ytick={0,0.2,0.4,0.6,0.8,1,1.2},
yticklabels={,0.2,0.4,0.6,0.8,1.0,}
]
\addplot [semithick, blue, dotted]
table {%
0 0.333333333333333
1 0.654320987654321
2 0.666666666666667
3 0.617283950617284
4 0.654320987654321
5 0.740740740740741
6 0.666666666666667
7 0.654320987654321
8 1
9 0.975308641975309
10 0.975308641975309
11 0.975308641975309
12 0.987654320987654
13 0.987654320987654
14 0.987654320987654
15 1
16 0.975308641975309
17 0.987654320987654
18 0.987654320987654
19 1
};
\addlegendentry{0-3}
\addplot [semithick, green!50.0!black, dotted]
table {%
0 0.25
1 0.425925925925926
2 0.398148148148148
3 0.490740740740741
4 0.472222222222222
5 0.481481481481481
6 0.490740740740741
7 0.481481481481481
8 0.657407407407407
9 0.953703703703704
10 0.944444444444444
11 0.944444444444444
12 0.916666666666667
13 0.916666666666667
14 0.898148148148148
15 0.953703703703704
16 0.990740740740741
17 0.925925925925926
18 0.981481481481482
19 0.87037037037037
};
\addlegendentry{0-4}
\addplot [semithick, red, dotted]
table {%
0 0.200000000331137
1 0.2
2 0.2
3 0.385185185516322
4 0.488888890213437
5 0.459259259921533
6 0.881481481922997
7 0.851851852293368
8 0.859259259259259
9 0.851851852293368
10 0.903703703703704
11 0.948148148148148
12 0.896296296737812
13 0.925925925925926
14 0.814814814814815
15 0.555555556438587
16 0.918518518960034
17 0.940740741182257
18 0.881481481481482
19 0.851851851851852
};
\addlegendentry{0-5}
\addplot [semithick, color0, dotted]
table {%
0 0.166666666666667
1 0.296296296296296
2 0.62962962962963
3 0.648148148148148
4 0.697530864197531
5 0.679012345679012
6 0.709876543209877
7 0.728395061728395
8 0.660493827160494
9 0.691358024691358
10 0.728395061728395
11 0.740740740740741
12 0.740740740740741
13 0.753086419753086
14 0.740740740740741
15 0.827160493827161
16 0.802469135802469
17 0.796296296296296
18 0.771604938271605
19 0.839506172839506
};
\addlegendentry{0-6}
\addplot [semithick, color1]
table {%
0 0.472222222222222
1 0.75
2 0.712962962962963
3 0.75
4 0.712962962962963
5 0.87037037037037
6 0.888888888888889
7 0.87962962962963
8 0.898148148148148
9 0.87962962962963
10 0.768518518518518
11 0.796296296296296
12 0.898148148148148
13 0.962962962962963
14 0.916666666666667
15 0.953703703703704
16 0.851851851851852
17 0.972222222222222
18 0.962962962962963
19 0.925925925925926
};
\addlegendentry{4-0}
\addplot [semithick, blue]
table {%
0 0.179012345679012
1 0.240740740740741
2 0.274691358024691
3 0.493827160493827
4 0.728395061728395
5 0.657407407407407
6 0.864197530864197
7 0.864197530864197
8 0.854938271604938
9 0.888888888888889
10 0.858024691358025
11 0.858024691358025
12 0.904320987654321
13 0.898148148148148
14 0.901234567901235
15 0.882716049382716
16 0.904320987654321
17 0.891975308641975
18 0.895061728395062
19 0.925925925925926
};
\addlegendentry{4-3}
\addplot [semithick, green!50.0!black]
table {%
0 0.175925925925926
1 0.361111111111111
2 0.405092592592593
3 0.736111111111111
4 0.847222222222222
5 0.912037037037037
6 0.909722222222222
7 0.881944444444444
8 0.912037037037037
9 0.914351851851852
10 0.881944444444444
11 0.921296296296296
12 0.895833333333333
13 0.921296296296296
14 0.905092592592593
15 0.902777777777778
16 0.918981481481482
17 0.902777777777778
18 0.930555555555556
19 0.935185185185185
};
\addlegendentry{4-4}
\addplot [semithick, red]
table {%
0 0.446296296296296
1 0.67037037037037
2 0.725925925925926
3 0.748148148148148
4 0.755555555555556
5 0.811111111111111
6 0.805555555555556
7 0.798148148148148
8 0.824074074074074
9 0.777777777777778
10 0.807407407407407
11 0.82037037037037
12 0.798148148148148
13 0.851851851851852
14 0.833333333333333
15 0.874074074074074
16 0.87037037037037
17 0.812962962962963
18 0.888888888888889
19 0.885185185185185
};
\addlegendentry{4-5}
\addplot [semithick, color0]
table {%
0 0.431221020092736
1 0.585780525778692
2 0.601236476135401
3 0.649149922996621
4 0.731066460725513
5 0.752704791436792
6 0.723338485501096
7 0.758887171607113
8 0.772797527140038
9 0.774343122194134
10 0.780525502594766
11 0.788253477634934
12 0.81298299854653
13 0.789799072735092
14 0.791344667697063
15 0.839258114558283
16 0.819165378670788
17 0.856259659969088
18 0.808346213476367
19 0.84234930457435
};
\addlegendentry{4-6}
\end{axis}

\end{tikzpicture}
		\caption{Test accuracies of all networks}
		\label{fig:test-accuracy}
	\end{subfigure}
	\caption{Training and test accuracies of networks}
	\label{fig:networks-accuracy}
\end{figure}