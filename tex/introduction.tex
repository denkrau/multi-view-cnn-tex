\chapter{Introduction}
\label{sec:introduction}
%\section{Overview}
\label{sec:overview}
This chapter presents an outline with a followed motivation on how and why this work presented in this thesis is relevant for computer vision tasks.
The second chapter summarizes recent researches building the fundamentals for this work and supplying the knowledge for being able to choose an approach for this work.
In the third chapter the fundamentals are explained.
They cover the general idea and development of artificial neural networks, followed by the concept of convolutional neural networks, that are more suited for image processing tasks.
Furthermore, it is stated what data networks use, how it is propagated through it and how the actual learning process works.
Moreover, it introduces hyperparameters and how they need to be chosen for achieving a satisfiable network performance and continues with metrics that examine that performance.
It finishes with a brief overview of the used software and framework.
The fourth chapter presents how everything is implemented.
This includes the creation of the dataset, the applying of material features and the conversion from single-views to multi-views.
Furthermore, the network architecture is explained detailed by dividing it into modules.
It continues with how hyperparameters are chosen and finishes with how the network is evaluated.
The fifth chapter presents all results divided into the grouping mechanism and the overall performance of the networks and discusses why wrong predictions happen.
This work finishes with the sixth chapter that summarizes all results and gives an outlook.
%\section{Motivation}
Researches showed that handcrafted 3D descriptors of objects are outperformed by using views of an object and generating 2D descriptors with the help of convolutional neural networks.
Hence, this work follows this approach for classifying objects by collecting multiple views of it and building a multi-view image from those single view images.
That multi-view discretizes the 3D object. 
However, not all of those views are equally relevant to the actual classification task.
Hence, a score per view is calculated that describes its discrimination and its weight in the classification process.
All views are divided into groups depending on their score.
Then for each group, a group descriptor is calculated by averaging the group's views.
Each group gets a weight assigned with the mean of its views discrimination scores.
Finally, those groups are weighted averaged depending on their weights for building a compact single shape descriptor that describes the object.
With this descriptor, the final class is predicted.
The grouping mechanism represents the core functionality of this work.
This is extended with applying color features to each object so that an object exists with its blank views and additionally with its colored ones.
A real-world example could be a robot driving through a scene and needs to classify the same objects, that only differ in a color feature.
As it progresses more views of each object become visible.
Thanks to the grouping system it knows if a view is discriminative enough for a desirable classification or if it needs to collect more for being sure.
Hence, this work examines how each view is treated and what are features the network looks for.
If this is known it could be manipulated for the certain use case.

In the following the outline of this work is presented.
In \secref{sec:fundamentals}, the fundamentals are explained.
They cover the general idea and development of artificial neural networks, followed by the concept of convolutional neural networks, that are more suited for image processing tasks.
Furthermore, it is stated what data networks use, how it is propagated through it and how the actual learning process works.
Moreover, it introduces hyperparameters and how they need to be chosen for achieving a satisfiable network performance and continues with metrics that examine that performance.
It finishes with a brief overview of the used software and framework.
The third chapter \secref{sec:related-work} summarizes recent researches building the fundamentals for this work and supplying the knowledge for being able to choose an approach for this work.
\secref{sec:methods} presents how everything is implemented.
This includes the creation of the dataset, the applying of face material manipulations and the conversion from single-views to multi-views.
Furthermore, the network architecture is explained detailed by dividing it into modules.
It continues with how hyperparameters are chosen and finishes with how the network is evaluated.
\secref{sec:results} presents all results divided into the overall performance of the networks and the grouping mechanism and discusses why wrong predictions happen.
This work finishes with \secref{sec:discussion} that summarizes all results and gives an outlook.