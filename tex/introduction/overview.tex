\section{Overview}
\label{sec:overview}
This chapter presents an outline with a followed motivation on how and why this work presented in this thesis is relevant for computer vision tasks.
The second chapter summarizes recent researches building the fundamentals for this work and supplying the knowledge for being able to choose an approach for this work.
In the third chapter, the fundamentals are explained.
They cover the general idea and development of artificial neural networks, followed by the concept of convolutional neural networks, that are more suited for image processing tasks.
Furthermore, it is stated what data networks use, how it is propagated through it and how the actual learning process works.
Moreover, it introduces hyperparameters and how they need to be chosen for achieving a satisfiable network performance and continues with metrics that examine that performance.
It finishes with a brief overview of the used software and framework.
The fourth chapter presents how everything is implemented.
This includes the creation of the dataset, the applying of material features and the conversion from single-views to multi-views.
Furthermore, the network architecture is explained detailed by dividing it into modules.
It continues with how hyperparameters are chosen and finishes with how the network is evaluated.
The fifth chapter presents all results divided into the grouping mechanism and the overall performance of the networks and discusses why wrong predictions happen.
This work finishes with the sixth chapter that summarizes all results and gives an outlook.