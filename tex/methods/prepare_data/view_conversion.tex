\subsection{Single-View to Multi-View Conversion}
\label{sec:prepare-data-view-conversion}
The views created in \secref{sec:dataset-material-feature} exist in a single view representation.
Thus, a multi-view classification should be performed by the network.
This means, each input represents a model with all its corresponding views.
Hence, each model's single views need to be converted into a related multi-view representation.
For achieving this the single views needs to be collected first.
From a given custom file path to the dataset all model views are collected recursively in a sorted order.
This is necessary for keeping related views in the order by which they are created.
Due to the self-explanatory folder structure, models for the training and testing set can be handled independently.
To each of the sets belongs a list $\vec{\tilde{X}}_{set}$ for storing all related views in RGB color representation.
Now, each view's pixel values are read, normalized to a range between 0 and 1 and then stored in a matrix or array $\vec{X}_{set}^{(i)}$, respectively, which is added to one of the just mentioned lists.
Simultaneously to the view gathering each label needs to be created as well and put into a list $\vec{\tilde{Y}}_{set}$ similar to the views.
Getting the category is quite simple, because the file path of the view is known.
Hence, the second to last element of the view's file path represents the category label.
The file name of a view is build like \textit{category\_object-id\_material-id\_view-id.ext}.
Here is the material index label extracted by splitting the file name by "\textit{\_}" first and then taking the third split element.
Depending on the classification task of the model, those two labels possibly need to be combined.
The single-label classification case embraces the following configurations.
If categories and materials are classified, both labels are appended to each other.
This results in $n_l = n_c \cdot n_m$ labels, where $n_c$ is the number of categories and $n_m$ the number of materials.
If only categories or materials are classified, the final label is the category label or the material label, respectively.
Logically, the number of labels is either $n_l = n_c$ or $n_l = n_m$.
In the multi-label classification case both labels are used independently.
Hence, the number of labels equals $n_l = n_c + n_m$.
An example of those configurations is shown in \tabref{tab:label-generation}.
\begin{table}[]
	\centering
	\caption[Label Generation]{Label Generation with example categories "bathtub" and "sofa" and materials "0" and "1" for different cases of classifications.}
	\label{tab:label-generation}
	\begin{tabular}{l|l|l}
		Classification      & Single-Label                                                                        & Multi-Label                                                    \\ \hline
		Category + Material & \begin{tabular}[c]{@{}l@{}}bathtub\_0\\ bathtub\_1\\ sofa\_0\\ sofa\_1\end{tabular} & \begin{tabular}[c]{@{}l@{}}bathtub\\ sofa\\ 0\\ 1\end{tabular} \\ \hline
		Category            & \begin{tabular}[c]{@{}l@{}}bathtub\\ sofa\end{tabular}                              & n/a                                                            \\ \hline
		Material            & \begin{tabular}[c]{@{}l@{}}0\\ 1\end{tabular}                                       & n/a                                                           
	\end{tabular}
\end{table}
When the number of labels is known, a one-hot encoding is performed on each label.
In conclusion, this process results in four lists
\begin{subequations}
	\begin{align}
		\label{eq:list-x-train}
		\vec{\tilde{X}}_{train} &=
		\begin{pmatrix}
			\vec{X}_{train}^{(1)} & \vec{X}_{train}^{(2)} & \cdots & \vec{X}_{train}^{(m_{train})}
		\end{pmatrix}
		\\
		\label{eq:list-x-test}
		\vec{\tilde{X}}_{test} &=
		\begin{pmatrix}
			\vec{X}_{test}^{(1)} & \vec{X}_{test}^{(2)} & \cdots & \vec{X}_{test}^{(m_{test})}
		\end{pmatrix}
		\\
		\label{eq:list-y-train}
		\vec{\tilde{Y}}_{train} &=
		\begin{pmatrix}
			\vec{y}_{train}^{(1)} & \vec{y}_{train}^{(2)} & \cdots & \vec{y}_{train}^{(m_{train})}
		\end{pmatrix}
		\\
		\label{eq:list-y-test}
		\vec{\tilde{Y}}_{test} &=
		\begin{pmatrix}
			\vec{y}_{test}^{(1)} & \vec{y}_{test}^{(2)} & \cdots & \vec{y}_{test}^{(m_{test})}
		\end{pmatrix}
	\end{align}
\end{subequations}
where $\vec{X}^{(i)}$ and $\vec{Y}^{(i)}$ of the same set are building an input-output pair.

Those single view and label representations need to be converted to a multi-view representation.
Because sorted data was read in, it is known that each 12 elements belong together and need to put together somehow.
Looking at common definitions yields, that images are a three-dimensional matrix with the shape definition $Height \times Width \times Channels$.
Furthermore, Tensorflow mostly uses the shape definition $Batch \times Height \times Width \times Channels$ for tensors.
Hence, assuming each view as a batch element is a reasonable approach.
If an actual batch dimension becomes necessary it is inserted as a new first dimension.
This yields a reduction of \thmref{eq:list-x-train} and \thmref{eq:list-x-test} to a $n_v$-th of its size, where $n_v$ is the number of views.
The labels can be processed similar, however, much easier.
Because each $n_v$ labels are identical, it is sufficient to just keep every $n_v$-th label.
For a later lookup of the labels they are saved to the disk as a text file.