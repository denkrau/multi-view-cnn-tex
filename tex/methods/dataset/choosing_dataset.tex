\subsection{Choosing a Dataset}
\label{sec:dataset-choosing}
One requirement of the dataset is, that there are multiple viewing perspectives of the same object available.
Optimally, these views can be arbitrarily chosen for having as much freedom as possible for training and evaluating the network.
Hence, three-dimensional objects are necessary.
The related object classes are preferably discriminative to each other to focus more on a classification of the color than on the category.
Thus, supporting the objective of this work.
This means the dataset should not contain only flowers, for example, but flowers and birds.
For this work an existing dataset is chosen for being competitive to other researches that probably used the same as a benchmark for their architecture.
The most popular dataset containing CAD objects is ModelNet \cite{conf/cvpr/WuSKYZTX15}.
It contains 127,915 CAD models divided into 662 object classes for now.
For convenience, there are a 10-class and 40-class subset containing 10 or 40 popular classes, respectively.
Both are cleaned in respect to a wrong class sorting and then split into a training and test set.
Furthermore, the orientations of the models of the first one are aligned as well.
Hence, the four classes bathtub, dresser, monitor, and sofa are extracted from ModelNet10.
For this work, this subset is sufficient because it provides enough information for the execution of its task.

Each object in this dataset is modeled using triangles, so-called faces, where each one is defined by the position of its vertices.
A vertex is a three-dimensional point defined by a $x$-, $y$- and $z$-coordinate.
This object representation is called polygon mesh.
Additionally, each face has a normal vector and other information like a color and a texture.
All properties of an object in this representation can be manipulated as desired.
An alternative representation for three-dimensional objects in general is a point cloud, where each point is represented by a vertex.
Such a cloud is often created by 3D scanners that measure a large number of points in a scene, like distances and sometimes color, for digitalizing it.
However, this yields not an as smooth surface as polygon meshes do, because between samples an interpolation is necessary.
Moreover, due to measuring errors, holes or artifacts can be introduced.